\documentclass[12pt]{article} 

\usepackage{epsfig} 

%\usepackage{rotating} 

\usepackage{graphicx} 


\begin{document}
\section{Lead Glass Shower Counters}

\subsection{Overview}

 The Hall A electron spectrometer is equipped with a 2-layer,
segmented shower counter.  The first layer, the so-called
``pre-shower'' counter is made of 48 blocks of TF1 lead glass. Each
block is nominally 10 cm by 10 cm by 35 cm long. The blocks are
assembled in 24 modules with two blocks in each module. PMTs located
outside of the beam envelope view the lead glass from opposite
sides. The second layer, the so-called ``total absorber'' counter is
an assembly of 96 blocks of SF5 lead glass, arranged in a 6-by-16
rectanglular grid.  The total absorber shower blocks are nominally 15
cm by 15 cm by 35 cm and viewed by a PMT from the back side.  The
pre-shower block PMTs are 3'' Hamamatsu 3036, and the shower block
PMTs are 5'' Phillips XP2050. The hadron spectrometer has a single
layer of 15 cm thick shower counter, the so-called ``pion
rejector'. It is made of 32 blocks of SF5 lead glass. Each block has
nominal dimensions of 15 cm by 15 cm by 30 cm. 5'' Phillips XP1050 are
also used in the pion rejector. The HA shower counter is installed
behind the focal plane polarimeter at the same position as can be used
for the S3 scintillator plane. \\


The pre-shower counter is assembled on a 1/2 inch-thick Al plate
bolted to the Detector Frame, meaning removal of any detector takes a
lot of time. However the PMTs are easily accessible from the sides.
Total absorber blocks are installed in a strong box made of 5/8 inch
Al plates.  The PMTs and HV dividers are accessible from the back. The
box can be lifted by the hall crane when the Detector Frame is outside
of Shielding Hut. The lead glass blocks of the total absorber should
be repaired or replaced only when the support box has horizontal
orientation.\\


Pion rejector blocks are supported by a 1/2'' Al plate. The frame is
bolted to the Detector Frame.\\
 
As charged particles pass through the lead glass of the shower
counter, they produce high energy photons and electron-positron pairs.
These particles in turn produce additional trees of new particles: a
so-called electro-magnetic shower.  Cherenkov light is emitted inside
the lead glass when the electron or positron has an energy above ~ 2
MeV, and this light is then detected in the photo tubes. The anode
current pulses are then amplified, delayed and sent to ADCs, which are
gated by the overall event trigger. The ADCs are read out by the {\it
CODA} acquisition software.  The data are histogramed online by the
{\it DHIST} software.  In-depth off-line data analysis requires the
{\it ESPACE} software.

\subsection{Principles of PID with Shower Counter}
The shower Counter is used for identification of electrons and
separating them from hadrons. This is possible because for high energy
electrons the main mechanism of interaction with matter is a
production of electron-positron pairs but for hadrons the nuclear
interaction ( and at low energy ionization ) dominates.  The SF5 lead
glass has a nuclear interaction length of $\sim 22$ cm, which is about
10 times larger than the radiation length. Differences in the
longitudinal profiles of electro-magnetic and hadronic showers provide
a way to use of a shower counter for PID.  Shower counters complement
other means of PID such as time-of-flight (TOF) or threshold Cherenkov
counters, due to the independent physical processes responsible which
result in different detector limitations. Independent PID allows
multiple detectors ($i.e.$, a Cherenkov counter followed by a shower
counter) to obtain excellent rejection ratios that are the product of
the individual rejection ratios.

Shower counters measure the energy deposited by the incoming particle.
The detected light output is linearly proportional to the energy lost
by the incoming particle, however coefficients are different for
hadrons and electrons.  Electro-magnetic showers are stopped in the
counters, whereas hadronic showers, due to the longer hadronic mean
free path, are not.  Looking at the longitudinal distribution of the
energy deposited in the calorimeter differentiates between
electro-magnetic and hadronic showers and therefore identifies the
incident particle.  Typical pion rejection with a lead glass counter
is of the order of 100-1000:1 in the 1 to 10 GeV region.  The Hall A
electro-magnetic shower counter is meant to offer rejection ratios
better than 100:1. The figure \ref{2d_sh_psh} and figure \ref{fen}
demonstrate electron separation from pion at momentum of 1500 MeV/c.
%

\begin{figure}
\begin{center}
\includegraphics[width=15cm,height=16cm,clip]{shower_fig5.eps}
{\linespread{1.}
\caption[Typical correlation amplitudes in shower and pre-shower]{Shower vs pre-shower}
\label{2d_sh_psh}}
\end{center}
\end{figure}
%
\begin{figure}
\begin{center}
\includegraphics[width=15cm,height=16cm,clip]{shower_fig6.eps}
{\linespread{1.}
\caption[PID parameter fen]{PID by shower counter. Upper plot for electrons. 
Bottom plot for pions.}
\label{fen}}
\end{center}
\end{figure}
%

The limitation in using a shower counter comes from separating the
tails of the distributions, and is therefore dependent on energy
resolution.  At higher energy the relative resolution of a shower
counter improves, leading to better separation between distributions.
Conversely, other techniques perform worse at higher energy.

\subsection{Operation}
\subsubsection{Amplitude and calibration}

The average number of photons produced in the lead glass as result of
Cherenkov radiation per one GeV of energy deposited in
electro-magnetic cascade is about 4500. These photons are released in
the glass almost instantly as the shower develops. Photons can travel
long and different paths inside of lead glass block before hitting the
photo cathode. The current pulse from the PMT anode has a duration
about 15-20 ns. For initial calibration of the counter it is possible
to use cosmic rays. The amplitude of the signal from cosmic muons
corresponds to a shower with energy deposition of about 6 MeV per cm
passing through the lead glass. For the pre-shower counter the cosmic
signal equivalent to 60 MeV and clearly visible in the amplitude
spectra. For the total-shower a clean cosmic signal also can be
observed when event samples are filtered with a cut on the angles of
the particles.  Determination of the amplitude to energy conversion
coefficients done by using events from special calibration runs when
the energy of the electrons was measured in the spectrometer. Because
of the large dynamic range of signals from shower blocks the PMT gain
is relatively low ( $~\sim 1*10^6$ ) and a cosmic signal is not easy
to find with the oscilloscope. It has amplitude about 1-3 mV.

\subsubsection{Operating Procedures}

\paragraph{Power Supplies and Electronics Procedures} 
A LeCroy HV 1460 is used to supply HV power for the trigger
counters. It can be controlled from a VT100 terminal connected through
a terminal server or through the EPICS system running on the computer
HAC.  Current HV settings for trigger counters should be found from
print-outs of EPICS controls in the last experimental logbook. The
LeCroy HV power supply provides -1500 V nominal to the total absorber
and 1200 V to the pre-shower. The figure~\ref{hv_psh} and the
figure~\ref{hv_sh} presents actual high voltage settings.
%
\begin{figure}
\begin{center}
\includegraphics[width=15cm,height=16cm,clip]{shower_fig1.eps}
{\linespread{1.}
\caption[High voltage settings of pre-shower]{HV pre-shower}
\label{hv_psh}}
\end{center}
\end{figure}
%
\begin{figure}
\begin{center}
\includegraphics[width=15cm,height=16cm,clip]{shower_fig2.eps}
{\linespread{1.}
\caption[High voltage settings of shower]{HV shower}
\label{hv_sh}}
\end{center}
\end{figure}
%

The power supply is located in the detector hut on the upper level of
the Detector Frame. Connections from the power supply to the base are
made using standard SHV connectors mounted on red RG-59/U HV cable
which is rated to handle voltages upto 5 kV.  If at all possible, the
HV power supplies should be left on continuously.  This allows the
tube noise to quiet down after high voltage is applied and the
temperature of the base components stabilizes.
%
\paragraph{Signal Handling, Summing, Amplification and the Multiplexer}
The upstairs electronics racks hold two NIM bins containing the
summing modules of the multiplexer.  The output of these is sent to
the ADCs.  The individual channels are sent into an ADC where it can
later be analyzed by the software.  The sum of every six channels are
used to form combinations between shower and pre-shower blocks and
finally sent to the trigger signal.  The operations manual for the
multiplexer was written by H. Breuer, UMd and is available from him
upon request.  The detector signals are physically plugged into the
ADC channels and connected to the signal wires labeled as shown for
both the pre-shower counter figure~\ref{map_psh} and total absorber
counter figure~\ref{map_sh}.
\begin{figure}
\begin{center}
\includegraphics[width=15cm,height=16cm,clip]{shower_fig3.eps}
{\linespread{1.}
\caption[Wiring map for pre-shower ]{Pre-shower detector map}
\label{map_psh}}
\end{center}
\end{figure}
%
\begin{figure}
\begin{center}
\includegraphics[width=15cm,height=16cm,clip]{shower_fig4.eps}
{\linespread{1.}
\caption[Wiring map for shower ]{Shower detector map}
\label{map_sh}}
\end{center}
\end{figure}
%
The shower counters are very delicate devices which are easily
damaged.  Thus, care must be exercised whenever they are moved or
used.

\begin{itemize}
\item{Before turning on the high voltage for the shower counters (HAC5 
for the pre-shower and HAC6 for the total absorber), check the shower
counter log book located in the Hall A counting house for the latest
values of the high voltage.}

\item{Never disconnect or connect any of the high voltage cables to 
the bases with the high voltage power turned on.  Doing so will damage
the bases (the zener diodes will be destroyed).}

\item{Never service the lead glass with the high voltage on.  The
high voltage poses a safety hazard.}

\item{Never drop anything onto the bases and tubes; they are 
extremely fragile.  They should not be used as support or to hold any
weight.  Also, do not drop the lead glass blocks.  The blocks are made
of glass and will become damaged or destroyed.}

\item{When ramping the HV, keep an eye on the current drawn 
by the individual bases.  A light leak in the wrapping will produce a
large current ( $ > 1 \mu$A at low voltages (1 kV).  If the tubes
draw an excess amount of current turn them off and check for leaks
in the wrapping.}
\end{itemize}
%

\subsection{Safety Assessment}
The following potential hazards have been clearly identified.
\begin{description}
\item {\bf The High Voltage System} The LeCroy 1443 HV crate equipped
with LeCroy 1461N negative high voltage cards provides upto 3.3 kV of
low current power.  Red HV RG-59/U cable rated for voltages upto to 5
kV with standard SHV connectors are used to connect the power supply to
th photomultiplier tube voltage divider bases.  A given base on the
TA draws typically 500-600 $\mu A$ of current with the high voltage
on at between 1400 and 1500 Volts.  The PS bases typically draw 900
$\mu A$  with the high voltage on at between 1100 and 1200 Volts.
%
\item {\bf The Lead Glass Support Structure}
The lead glass shower counters are mounted on top of the space
frame for the detectors.  Access for servicing the shower counters
requires climbing on top of the support frame.  Only the responsible
personnel identified below should attempt to service the shower counter;
such work requires proper safety precautions and prior training/experience.
\item {\bf The Lead Glass blocks}
The lead glass shower blocks and tubes are approximately 70 -- 80 pounds
in weight apiece.  Lifting, replacing, or moving such blocks should
be done properly to avoid muscle problems and damage to the blocks.
\end{description}
%
\subsection{Authorized Personnel} 
The following individuals are authorized to work on shower counters. 
\begin{itemize} 
\item[~]Breuer, Herbert - 301-405-6108 
\item[~]Markowitz, Pete - x7237, 305-348-1710
\item[~]Segal, Jack - x7242 
\item[~]Voskania, Hakob - x5105
\item[~]Wojtsekhowski, Bogdan - x7191 
\end{itemize} 

\subsection{Software Algorithms}
The purpose of the shower cluster reconstruction in each of Preshower and 
Total Absorber parts of the shower detector is:
\begin{list}{$\bullet$}{}
\item Define all clusters of fired blocks, which belong to the showers, 
      registered in the detector;
\item Calculate parameters of showers in the detector: energy deposition of 
      showers, $X$ and $Y$ coordinates of the shower center;
\item Set parameters and identify the, so called, ``main'' cluster.
\end{list}

\noindent A cluster in the shower detector is determined as follows:
\begin{list}{---}{}
\item A cluster is a group of continuous blocks; 
\item A cluster can occupy a maximum 6 ($2 \times 3$) blocks in the case of the
      Pre-shower and 9 ($3 \times 3$) blocks in the case of the Total Absorber; 
\item The central block is defined as block that has maximum energy deposited.
\end{list}

The ``main'' cluster in Pre-shower/Total Absorber is the cluster with
the largest energy which is coincident with the ``golden track'', or
coincident with some Shower/Pre-shower cluster.  A coincidence of a
cluster with the ``golden track'' means that distance between the
shower cluster center and the crossing point of the ``golden track''
with the detector plane is less than a certain magnitude.  Such a
coincidence of Pre-shower and Total Absorber clusters means, that
distance between the clusters centers is less than a certain magnitude
assuming that both of these points are considered on the same
$Z$--plane.  

Shower clusters reconstruction in the Pre-shower and
Total Absorber is performed by the following steps:
\begin{enumerate}
\item Sort the fired blocks of the detector in decreasing order of their 
      energy deposition values;
\item Pick out the block with the maximum energy deposition and all fired blocks in
      it's geometric vicinity ($2 \times 3$ blocks for Pre-shower, $3 \times 3$ blocks
      for Shower), as belonging to one cluster;

\item Remove all of blocks, associated in the cluster, from following
      considerations;
\item Repeat steps 2 and 3, until all of the fired blocks are organized into 
      a cluster;
\item Calculate energy deposition, $X$ and $Y$ coordinates of each cluster and 
      sort clusters based on their energies in decreasing order;
\item Define coordinates of the crossing point of the ``golden track'' with the detector 
      plane in the detector local coordinate system;
\item Analyze geometrical position of the ``golden track'' point on the 
      detector plane and clusters centers in order to determine the ``main'' 
      cluster and to set it parameters and software identifier.
\end{enumerate}

Energy deposition $E$, $X$ and $Y$ coordinates of the shower center are 
calculated by the formulas: 

$$E=\sum_{i \in M}e_i\ ,\ \ \ \ 
X=\sum_{i \in M}e_i \cdot x_i/E\ ,\ \ \ \ Y=\sum_{i \in M}e_i \cdot y_i/E\ ,$$

where: $i$ --- number of detector block, included into the cluster; $M$ --- set
of blocks numbers, included into the cluster; $e_i$ --- energy deposition in 
the block $i$ of detector; $x_i$, $y_i$ --- $X$ and $Y$ coordinates of center 
of the block $i$ of detector.

The shower cluster reconstruction algorithm described above has been
implemented by the ESPACE analysis subroutine {\bf tot\_shower}.  A
complete description of the program as well as information about the
ESPACE routines and kumac files used to perform the analysis,
photographs of the detectors and more information can be found at the
URL:

http://www.jlab.org/\~ armen/sh\_web\_page/sh\_page\_init.html
\end{document}
