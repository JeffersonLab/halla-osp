\documentclass[12pt]{article} 
\usepackage{epsfig} 
\usepackage{graphicx} 
\begin{document}
\section{Detector Package of HRS}
\subsection{Overview}

The detector package of each spectrometer has trigger, tracking, and particle 
ID components. Hadron spectrometer in addition has unique proton polarimeter.
Tracking part comprises of two identical vertical drift chambers. The VDC
placed first on the way of the particles passed through the magnetic 
elements for minimization of the multiple scattering contribution to the angular and 
energy resolutions of the spectrometer. Trigger detectors include two planes of the
thin plastic scintillator counters, gas and aerogel Cherenkov counters and shower 
counter ( on EA ). On the hadron spectrometer the shower counter or large scintillator counter 
can be also used in the trigger. Particle ID provided by means of several techniques. 
For electron identification EA has the gas Cherenkov counter and two layers segmented 
lead glass shower counter. Because HA also can be used for experiment with electrons 
and it is equipped with short version of the gas Cherenkov counter and one layer segmented 
lead glass shower counter. Pion identification on both spectrometers rely on aerogel 
Cherenkov counters which presently have aerogel radiator with refraction index n = 1.025. 
Aerogel Cherenkov counters commissioning doesn't complete yet. For particle momentum  
below 800 MeV/c the energy loss in the scintillator and shower counters can be used for 
separation pions and protons. Large distance between planes of the trigger scintillator 
counters ( 2 - 3 m ) allows direct measurement of the particle speed with resolution ( sigma )
of $\sim 0.07$. Measurement of the time of flight on long pass from the target to the 
spectrometer focal plane ( $\sim 25$ m ) provides another powerful particle ID for 
coincidence experiments.
Focal plane polarimeter on HA operates in proton momentum range up to 3 GeV/c with 
the figure of merit $\sim 0.03$.\\

Detector packages installed inside of the Shielding Huts (SH). Access to the Shielding Hut 
realized by means of heavy swinging front doors. The main structure of SH made from three
inches steel plates. The side walls and bottom surfaces of the SH covers inside with one inch
lead slabs. Outside of the steel box the concrete is used for neutron protection. 
The front door has about 34 inches of concrete and three inches of lead. Side walls are 
covered with 17 inches of concrete. The roof of SH has only 10 inches of concrete above 
of three inches of steel. The side walls towards to the beam dump have an additional cover on lower 
half part with 15 inches of concrete.
SH reduces the counting rate of the single scintillator counter by factor of $\sim 10-20$.
Additional component of the shielding is so-called
``Line of Sight Shielding'', which made of $\sim 2-3$ m of concrete and installed on the distance 
$\sim 5$ m from the target. Most of high energy pions will be slow down and absorbed in the LSS.    
LSS is a main technique for reduction of the rate of high energy muons, which produced in pion decay.\\
  
The side views of the Detectors are shown on figure~\ref{Electron_Arm_Detector_side_view} and 
figure~\ref{Hadron_Arm_Detector_side_view}. VDC are mounted on the movable frame which can 
slides along Thompson rails to the hard stops. Geometry of the VDC on the frame and  position of the 
Thompson rails stable relatively to the spectrometer dipole magnet on level of 0.1 mm. Thomson rails are
surveyed relatively to the Hall A center. The rest of the detectors are mounted 
on the Detector Frame. Their positions are known with accuracy of 5-10 mm and
reproducible on level of 1-2 mm . Detector Frame can be moved out of the Shielding Hut 
for detector maintenance.   

\begin{figure}
\begin{center}
\includegraphics[width=8cm,height=16cm,clip]{new_fig1.eps}
{\linespread{1.}
\caption{Electron Arm Detector side view.}
\label{Electron_Arm_Detector_side_view}}
\end{center}+
\end{figure}
%
\begin{table}[hbtp]
\begin{center}
\caption{Locations of the detectors on Electron Arm.}
\medskip
\begin{tabular}{cccccccc}
detector&location&  location&  width &   width &    beam    & envelope &    \\
        & actual &   IDEAS  &   X    &     Y   &      X(+)  &  X($-$)&   Y   \\  \hline
       &        &           &        &         &            &        &       \\  \hline    
VDC1*   &      0 &          &   1942 &    271  &     843    & - 824  &  114  \\
VDC2*   &     572&          &   1942 &    271  &     932    & - 911  &  170  \\
S1      &    1311&     1321 &   1718 &    356  &     696    & -1022  &  250*  \\ 
AERO    &    1646&          &   199  &    414  &     709    & - 888  &  364  \\
GAS     &    2535&          &   2200 &    650  &     886    & -1110  &  558  \\ 
S2      &    3358&     3378 &   2197 &    540  &     897    & -1124  &  400* \\
preSHOW &    3502&     3546 &   2400 &    700  &     925    & -1158  &  602  \\ 
SHOW2   &    3780&     3912 &   2400 &    900  &     964    & -1207  &  644  \\  \hline

\end{tabular}
\end{center}
\end{table}

\begin{figure}
\begin{center}
\includegraphics[width=8cm,height=15cm,clip]{new_fig2.eps}
{\linespread{1.}
\caption{ Hadron Arm Detector side view.}
\label{Hadron_Arm_Detector_side_view}}
\end{center}
\end{figure}
%
\begin{table}[hbtp]
\begin{center}
\caption{Locations of the detectors on Hadron Arm.}
\begin{tabular}{cccccccc}
detector&location& location&  width &   width &    beam  & envelope&   \\
        & actual &   IDEAS &     X  &     Y   &      X(+)&  X($-$) &   Y       \\  \hline
       &        &          &        &         &          &         &           \\  \hline    
VDC1*  &        &         0&    1942&     271 &     843  &  - 824  &   114     \\    
VDC2*  &        &       500&    1942&     271 &     932  &  - 911  &   170     \\ 
S1     &        &      1287&    1760&     360 &     675  &  - 845  &   326     \\    
AERO   &        &      1617&    1872&     414 &     709  &  - 888  &   364     \\
SC1    &        &      1837&    1780&     480 &     738  &  - 924  &   396     \\    
GAS    &        &      2409&    2200&     650 &     857  &  -1073  &   526     \\    
SC2    &        &      2952&    2080&     640 &     865  &  -1083  &   536     \\   
S2     &        &      3141&    2220&     640 &     877  &  -1099  &   548     \\   
Analyzer&       &      3495&    2190&     680 &     916  &  -1147  &   592     \\   
SC3    &        &      3907&    2540&    1000 &    1099  &  -1343  &   914     \\  
SC4    &        &      4264&    3170&    1500 &    1382  &  -1645  &  1410     \\    
S3     &        &      4477&    3600&    1550 &    1437  &  -1704  &  1504     \\  \hline
\end{tabular}
\end{center}
\end{table}

The position in the table indicates the place of the central point of the 
detector. The origin of the coordinate system (0,0,0) located at cross of 
mid plane of the spectrometer and nominal focal plane ( middle of the Bottom VDC ).

\end{document}






























