%\author{G.J. Lolos}
%\address{Dept. of Physics, University of Regina, Regina, SK, S4S 0A2 Canada}

\ssubsection{ 5" PMT Bases for S3 Counters}

%\ssubsection{Overview}

   The general layout of the 5" bases is similar to that of the 2" bases
described above. It consists of a front tubular housing and a rear tubular
housing, both made out of aluminum. They join at a coupling nut, as shown in
the schematic diagram of figure~\ref{scint_pmt5}. The actual base where the PMT,
$\mu$-metal
shield, and the electronic dynode amplification chain are located, is
different. The corresponding middle section, incorporating the above
components, is made out of a moulded structure used in both the 5" bases and
the aerogel cerenkov counters. The PMT socket is moulded integral to the
section and it is also spring loaded. The collets are different, due to the
size of the light guide, but the method of assembly is very similar to that of
the 2" base. 

\snfig{./figs/scint_pmt5.eps}{The 5" PMT base used in S3  trigger
scintillators}{scint_pmt5}{5in}

\paragraph{Assembly Instructions}

   The collet assembly consists of several expanding rings, one set of solid
tapered and the other a spring collet. They are placed on top of each other,
alternating between the two kinds, with a spring collet on the top facing the
scintillator and the collet nut. As the collet nut is screwed in it presses
against the assembly and the spring collets slide inward against the tapered
solid ones, thus clamping against the light guide. Care should be taken, when
placing these two different kind of collets inside each other, so as not to
align the gaps in the plastic and create light leaks. The collets are not
continuous rings, otherwise they would not be compressible, and the gaps can
allow light all the way into the PMT if they are aligned. 

The collet nut is placed first around the light guide and the successive layers 
of the moulded collets are then placed around the light guide; one should make 
sure that enough free length of exposed light guide remains to enter the 
housing to slightly compress the PMT and base assembly, as in the case of the 
2" base. Once the slight compression of the spring loaded PMT-socket assembly 
is verified, the set of collet rings is moved to enter the housing and rest 
firmly against the snap/stop ring inside the front tubular housing. Then the 
collet ring is screwed in using the special tool until the light guide is 
firmly bedded in the housing and cannot be moved in or out. The remaining 
procedure is the same as that of the 2" base.

Due to the two diameter shape of the 5" PMT and its mu-metal shield, insertion  
of the PMT pins into the socket is a blind operation. It has to be done by feel 
and experience; the pins can only go in a specific PMT-socket geometry and it 
needs experience to learn when the correct alignment has been achieved. Let 
someone who has experience do it, bending of the socket pins at the base of the 
PMT result in destroyed PMTs, and they are expensive.

WARNING: The plastic insulator sleeve inside the front tubular aluminum 
housing should be in place BEFORE the mu-metal shield/PMT assembly is inserted. 
Failure to assure the proper location of the insulating plastic sleeve may 
result in electrical shock.
 
\paragraph{The Electronic Amplification Chain}


The dynode amplification chain is mounted on a PC board, as is the case of the 
2" base. A schematic diagram of the resistor chain is shown in 
figure~\ref{scint_base5}. As in 
the case of the 2" base, it incorporates an adjustable potentiometer (0-500 
Ohm) and the nominal critical matching value is 90 Ohm. It has, in addition, a 
51 Ohm resistor in series with the 14th dynode, to further improve the shape of 
the anode pulse. A safe -HV to start with is -1,800 V; it provides the best 
timing resolution with adequate pulse height for electrons during in beam 
testing.

\snfig{./figs/scint_base5.eps}{The 5" PMT base dynode resistor chain}
{scint_base5}{5in}


   Both 2" and 5" bases have been extensively tested under beam conditions. 
They have several safety related features but these cannot protect anyone who 
is bent on violating operating procedures and common sense. They allow the 
removal of the PMT/Base assembly, for repairs of the electronics or replacement 
of a PMT, without decoupling the housing and collets from the light guide. 
Thus, replacement of PMTs can be done in minutes without the need to remove the 
scintillator counters from their subframes.

   The safety features of the 5" base are the same as those in the 2" base. The
same precautions apply and it is important to assure that HV is OFF before any
handling of the base takes place. 

