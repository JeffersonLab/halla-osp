
\ssubsection{Overview}

Both spectrometers have front-end devices  for calibrating the optical
properties of the spectrometers. These are known as the collimator boxes.
These boxes are positioned between the scattering chamber and the 
first quadrupoles (Q1). Each box is carefully aligned and rigidly attached
to the  entrance flange of the Q1 of the respective spectrometer.  The boxes are
part of the vacuum system of the spectrometer.


Inside each box is a ladder is mounted which is guided by a linear bearing
and moved up and down by a ball screw. On this ladder 3 positions are 
available to insert collimators. Below this ladder
a special valve is mounted that can isolate the vacuum in the spectrometer
from the target system. This valve should be activated when it is moved
in front of the holes connecting the box with spectrometer and target chamber.
A schematic view of the collimator box is shown in figure%\ref{coll}.

\snfig{./figs/collimator.eps}{Schematic layout of the collimator box}{coll}{5in}


Vacuum requirement is $10^{-6}$ Torr. The material for the box is 
aluminum. It is possible to open one side of the box so that
collimators can be exchanged. The reproducibility of collimator positions after moving
the ladder and/or after replacing a collimator is
better than 0.1 mm in horizontal and vertical direction.The dimensions of the box are
roughly height=175 cm , width=35 cm and depth=15 cm. The tolerance in the dimension
of the 7-msr collimator hole is in each direction $\pm0.5$ mm. The tolerance in the position
of each of the sieve-slit holes is $\pm0.1$ mm in each direction.


\snfig{./figs/sieveslit.eps}{Sieve slit collimator for optics calibration}{sieve}{5in}

A typical sieve slit collimator is shown in figure~\ref{sieve}. It consists of 
a plate of roughly 14cm x 20 cm contains 49 holes
positioned in a regular 7x7 pattern. This slit is made out of 5 mm-thick tungsten.
The holes have a diameter of 2 mm except for the central one and one positioned
off-diagonal which have a diameter of 4 mm. The horizontal distance between the
holes is 12.5 mm while the vertical distance is 25.0 mm.


\ssubsection{Authorized  Personnel} 

\begin{itemize} 
\item[~]E. Folts - x7857 (mechanical and vacuum systems).
\item[~]J. Proffitt - x5006 (computer controls and electrical systems).
\end{itemize} 

\ssubsection{Safety Assessment}
The collimator boxes for part of the vacum system for each spectrometer. All hazzards
identified in section spectrometer vacuum section applies to the collimator box as well.

In addition, safe access to the top of
the collimator boxes is needed  during manual operation of the box as outlined below.
Due to the proximity of the collimator boxes to the scattering chamber, and Q1 quadrupoles
all necessary safety precautions with regards to vacuum windows, electrical power cables, 
cyrogenic transfer lines, and high magnetic field should be taken.

\ssubsection{Operating Procedure}
In order to change the  position of the collimator ladder, an access to the Hall is
required. The positioning of the collimator ladder is accomplished manually.
A hand-cranked acutator at the top of the collimator box is used to change the position.
The position is read out with a volt meter connected to the output of the encoder.

After accessing the Hall, the following operation procedure needs to be followed:
\begin{itemize}
\item[~] Select the electron or the hadron collimator box as necceray
\item[~] Gain safe access to the top of the collimator box via a temporary ladder.
\item[~] Hand-crank the  ladder actuator until the desired position is reached. The
following table summarizes the positioin calibrations for the two collimator boxes.
\end{itemize}
\begin{tabular}{|l | c| c|}
\hline
\multicolumn{2}{|c|}{Encoder reading in VDC} & \\
 & HRSE & HRSH \\
 \hline
 Slit & 6.15 & 6.17 \\ \hline
 Sieve & 3.64 & 3.66 \\ \hline
 Valve & 0.834 & 0.85 \\ 
 \hline
 \end{tabular}
\vfill\eject
