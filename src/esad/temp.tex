\section{Tritium Target}

The current run period involves having a target cell containing one kC tritium inside a seal cell
inside the Hall A scattering
chamber.    As tritium is a low energy beta emitter, it is not dangerous extenally as the beta particles 
cannot penetrate the skin; but inhaling or ingesting tritium or tritiated materials is hazardous. 
As the scattering chamber acts as a secondary containment of the tritium, an twenty foot area around
the chamber has been fenced off to not only keep people away from the cell and minimize the chances
that the scattering chamber window is accidently damaged.
Sensors have been placed in the Hall to detect the presence of tritium.   If tritium is detected,
green beacons and a siren will activate and all personnel should leave the area.   
For emergencies call the RadCon Cell phone at 876-1743.   For extensive details about the tritium
target and its safety systems, please 
see~\href{https://wiki.jlab.org/jlab_tritium_target_wiki/index.php/Main_Page}{Jefferson Lab Tritium Target Wiki}.

