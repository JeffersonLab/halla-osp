
%
% LaTeX Version 12 GeV ESAD 
%

\chapter{Introduction}

The ESAD document describes identified hazards of an experiment and the measures taken to eliminate, control, or mitigate them.
This document is part of the CEBAF experiment review process as defined in
\href{http://www.jlab.org/ehs/ehsmanual/manual/3120.html}{Chapter 3120 of the Jefferson Lab EHS\&Q manual},
and will start by describing general types of hazards that might be present in any of the  
JLab experimental halls.  This document then addresses the hazards associated 
with sub-systems of the base equipment of the experimental hall and their 
mitigation.  Responsible personnel for each item is also noted.  
In case of life threatening 
emergencies call 911 and then notify the guard house at 5822 so that the guards can help
the responders.  This document does not attempt to describe the function 
or operation of the various sub-systems. Such information can be found in
the experimental hall specific Operating Manuals.

%{\it{{\bf{TO DO LIST}}
%\begin{itemize}
%\item update the outline in Chapter 3120 to match our document final ESAD
%\item review as well as remove
%the es\&h coordinators as the physics division liason does those tasks
%\item use "responsible personnel' notation through-out
%\item we can add a reference list of names at the end; but for now I have them all with
%      the sections they go with
%\end{itemize}
%}}

\chapter{General Hazards}

\section{Radiation}
	
CEBAF's high intensity and high energy electron beam is a potentially lethal direct radiation source. 
It can also create radioactive materials that are hazardous even  after the beam has been turned off. 
There are many redundant measures aimed at preventing accidental exposure of personnel to the beam 
or exposure to beam-associated radiation sources that are in place at JLab. The training and mitigation 
procedures are handled through the JLab Radiation Control Department (RadCon). The radiation safety 
department at JLab can be contacted as follows: For routine support and surveys, or for emergencies 
after-hours, call the RadCon Cell phone at 876-1743. For escalation of effort, or for emergencies, 
the RadCon manager (Keith Welch) can be reached as follows: Office: 269-7212, Cell: 876-5342 or Home:  875-1707.

Radiation damage to materials and electronics is mainly determined by the neutron 
dose (photon dose typically causes parity errors and it is easier to shield against). 
Commercial-of-the-shelf (COTS) electronics is typically robust up to neutron 
doses of about 10^13 n/cm^2. If the experimental equipment dose as calculated 
in the RSAD is beyond this damage threshold, the experiment needs to add 
an appendix on "Evaluation of potential radiation damage" in the experiment 
specific ESAD. There, the radiation damage dose, potential impact to equipment 
located in areas above this damage threshold as well as mitigating measures taken should be described.

\section{Fire}

	The experimental halls contain numerous combustible materials and flammable gases. 
In addition, they contain potential ignition sources, such as electrical wiring and equipment. 
General fire hazards and procedures for dealing with these are covered by JLab emergency 
management procedures. The Jlab fire safety officer (Dave Kausch) can be contacted at 269-7674.

\section{Electrical Systems}

	Hazards associated with electrical systems are the most common risk in the experimental halls. 
Almost every sub-system requires AC and/or DC power. Due to the high current and/or high voltage 
requirements of many of these sub-systems they and their power supplies are potentially lethal 
electrical sources. In the case of superconducting magnets the stored energy is so large that 
an uncontrolled electrical discharge can be lethal for a period of time even after the actual 
power source has been turned off.  Anyone working on electrical power in the experimental Halls 
must comply with \href{http://www.jlab.org/ehs/ehsmanual/manual/6200.html}{Chapter 6200 of the Jefferson Lab EHS\&Q manual}
and must obtain approval of one of the responsible personnel. 
The JLab electrical safety point-of-contact (Todd Kujawa) can be reached at 269-7006.

\section{Mechanical Systems}

	There exist a variety of mechanical hazards in all experimental halls at JLab. 
Numerous electro-mechanical sub-systems are massive enough to produce potential fall 
and/or crush hazards.  In addition, heavy objects are routinely moved around within 
the experimental halls during reconfigurations for specific experiments. 

Use of ladders and scaffold must comply 
with \href{http://www.jlab.org/ehs/ehsmanual/manual/6132.html}{Chapter 6231 of the 
Jefferson Lab EHS\&Q manual}.
Use of cranes, hoists, lifts, etc. must comply with
\href{http://www.jlab.org/ehs/ehsmanual/manual/6141.html}{Chapter 6141 of the 
Jefferson Lab EHS\&Q manual}. 
Use of personal protective equipment 
to mitigate mechanical hazards, such as hard hats, safety harnesses, and safety 
shoes are mandatory when deemed necessary.
The JLab technical point-of-contact (Suresh Chandra) can be contacted at 269-7248.

\section{Strong Magnetic Fields}

	Powerful magnets exist in all JLab experimental halls. Metal objects being attracted 
by the magnet fringe field, and becoming airborne, possibly injuring body parts or striking 
fragile components resulting in a cascading hazard condition. Cardiac pacemakers or other 
electronic medical devices may no longer functioning properly in the presence of magnetic fields. 
Metallic medical implants (non-electronic) being adversely affected by magnetic fields. Lose of 
information from magnetic data storage driver such as tapes, disks, credit cards may also occur. 
Contact Jennifer Willams at 269-7882, in case of questions or concerns.

\section{Cryogenic Fluids and Oxygen Deficiency Hazard (ODH)}

	Cryogenic fluids and gasses are commonly used in the experimental halls at JLab. 
When released in an uncontrolled manner these can result in explosion, fire, cryogenic 
burns and the displacement of air resulting in an ODH condition. The hazard level and 
associated mitigation are dependent on the sub-subsystem and cryogenic fluid. However, 
they are mostly associated with cryogenic superconducting magnets and cryogenic target systems. 
Flammable cryogenic gases used in the experimental halls include hydrogen and deuterium which 
are colorless, odorless gases and hence not easily detected by human senses. Hydrogen air 
mixtures are flammable over a large range of relative concentrations: from 4\% to 75\% H2 by volume. 
Non-flammable cryogenic gasses typically used include He and nitrogen.  Contact Kelly Mahoney at
269-7024 or Mathew Wright at 269-7722 in case of questions or concerns.

\section{Vacuum and Pressure Vessels}

	Vacuum and/or pressure vessels are commonly used in the experimental halls. Many 
of these have thin Aluminum or kevlar/mylar windows that are close to the entrance 
and/or exit of the vessels or beam pipes. These windows burst if punctured accidentally 
or can fail if significant over pressure were to exist. Injury is possible if a failure 
were to occur near an individual. All work on vacuum windows in the experimental halls 
must occur under the supervision of appropriately trained Jlab personnel. Specifically, 
the scattering chamber and beam line exit windows must always be leak checked before service. 
Contact Will Oren 269-7344 for vacuum and pressure vessels issues.

\section{Hazardous Materials}

	Hazardous materials in the form of solids, liquids, and gases that may harm people 
or property exist in the JLab experimental halls. The most common of these materials include 
lead, beryllium compounds, and various toxic and corrosive chemicals. 
Material Safety Data Sheets (MSDS) for hazardous materials 
in use in the Hall is available from the Hall safety warden.  These are being replaced by the new standard
Safety Data Sheets (SDS) as they become avaliable in compliance with the new OSHA standards.    Handling of these materials 
must follow the guidelines of the EH\&S manual. Machining of Lead or Beryllia, that 
are highly toxic in powdered form, requires prior approval of the EH\&S staff. 
Lead Worker training is required in order to handle Lead safely in the Hall. 
In case of questions or concerns, the Jlab hazardous materials specialist (Jennifer Willams) can be contacted at 269-7882.

\section{Lasers}

	High power lasers are often used in the experimental areas for various purposes. Improperly 
used lasers are potentially dangerous. Exposure to laser beams at sufficient power levels 
may cause thermal and photochemical injury to the eye including retina burn and blindness. 
Skin exposure to laser beams may induce pigmentation, accelerated aging, or severe skin burn. 
Laser beams may also ignite combustible materials creating a fire hazard. At Jlab, lasers with power 
higher than 5 mW (Class IIIB) can only be operated in a controlled environment with proper eye protection 
and engineering controls designed and approved for the specific laser system. Each specific laser systems 
shall be operated under the supervision of a Laser System Supervisor (LSS) following the Laser 
Operating Safety Procedure (LOSP) for that system approved by the Laboratory Laser Safety Officer (LSO). 
The LSO (Dick Owen) can be reached at 269-6381.

%
% details for each of the Halls
%
\chapter{Hall Specific Equipment}

\section{Overview}

        The following Hall A subsystems are considered part of the experimental endstation base equipment.
Many of these subsystems impose similar hazards, such as those induced by magnets and magnet power supplies,
high voltage systems and cryogenic systems.  Note that a specific sub-system may have many different hazards associated with it.
For each major system, the hazards, mitigations, and responsible personnal are noted.

The material in this chapter is a subset of the material in the full Hall A operations manual and is only intended to familiarize
people with the hazards and responsible personnel for these systems.  It in no way should be taken as sufficent information to
use or operate this equipment.

%
% include your Hall sections here
%

\obsolete{
3 		Hall X Specific Safety Information  	
3.1 	Introduction
	The following subsystems are considered part of the experimental endstation base equipment. 
Many of these subsystems impose similar hazards, such as those induced by magnets and magnet power supplies, 
high voltage systems and cryogenic systems. Note that a specific sub-system may have many different hazards associated with it.

3.2		General Issues
	The EH\&S manual can be found on the web at:
 http://www.jlab.org/ehs/manual/EHSbook-1.html. The principal contacts for the JLab EH\&S group are: Bert Manzlak - x7556 (Physics EH\&S) and Charles Hightower - x7608 (Physics EH\&S).
3.3 		Electrical Systems
	Hazards associated with electrical systems are the most common risk among all the experimental endstations. Almost every subsystem requires AC and/or DC power. Due to the high current and/or high voltage requirements of many of these sub-systems the power supplies are potentially lethal.
3.3.1 		AC power
	The experimental endstations have megawatts of installed AC power with voltages available at 480 VAC, 240 VAC and 120 VAC. All of which are potentially lethal!
Hazard Mitigation 
	Aside from the resetting of approved low voltage (120 VAC) circuit breakers you should not attempt to solve any problems associated with AC power distribution without consulting responsible personnel. Anyone working on AC power in the experimental endstations must be familiar with the EH\&S manual and must contact one of the responsible personnel. 
3.3.2 		Remote Control Systems
	There are numerous computer based remote control systems to operate various sub-systems.
Hazard Mitigation	
	All non-routine maintenance shall be performed in strict accordance with the Jefferson Lab EH\&S Manual, and in particular, with the chapters on Lockout, Tagout, and Electrical Safety.
3.3.3	High Voltage \& Detector Systems
	Experimental detector systems typically require High Voltages, up to 3,000 VDC, drawing currents up to a few mA. Note: In the vicinity of these detectors are often cold cathode gauges which are used to measure the pressure in the spectrometers and/or beam line. These utilize high voltage! These High Voltages represent a potential hazard to personnel as well as a potential source of ignition.
Hazard Mitigation	
	Cable and SHV connectors are shielded and must meet existing EH\&S standards. Common guidelines for safe operation have been established and are outlined in the experimental endstation Operating Manual. The basic operating policy is to turn off the High Voltages before work occurs on or around the detector that does not absolutely require the HV to be on. DO NOT attach/remove HV cables when voltages are present on a channel. Turn off the main HV supplies or at least the HV to the specific channel when attaching or removing HV cables. Voltage should not be applied to cables that are not fully connected.
3.3.4 		Counting House Electronics
	Most of the electronic modules and devices require DC power. The DC voltages are typically provided by the AC powered supplies of the crates in which the modules are installed.
Hazard Mitigation
	Physics division recommends that the AC to the crates be turned prior to their removal or reconfiguration in the racks. Cables and SHV connectors are shielded and meet existing EH\&S standards. Common guidelines for safe operation have been established and are outlined in the experimental endstation Operating Manuals. The basic operating policy is to turn off the High Voltages (preferable the AC power to the crate itself) before configuration work occurs around the electronic crates/racks that does not require the HV to be on. DO NOT attach/remove HV cables when voltages are present on the channels. Turn off the main HV supplies or at least the HV to specific channels when attaching/removing HV cables.
3.4 		Electromagnets and Magnet Power Supplies
	The experimental endstations utilize a number of electro-magnets, both resistive and superconducting. The magnets in the experimental areas are typically energized via remote control. During major down times periods the magnets are powered down for personal safety reasons as well as to reduce electrical power consumption. During short interruptions of beam delivery, personnel entering the hall in the controlled access mode, many magnets are typically left energized. The primary reason is that the time constants of large size magnets can be long (of the order of hours), and frequent ramping or cycling will lead to inefficient operation. Also, every ramp of a large superconducting magnet involves some risk of permanent damage to the magnet coil. The principal hazards associated with the magnets are:
3.4.1		Electromagnetic Fields
	The electro-magnets produce large fields up to (Bcentral ? 10 kG). The magnets typically have return yokes and thus the external fields (near the magnets) are not nearly as large as the central fields but they may still be significant. The beam raster magnets do not have a return yoke and may have external AC magnetic fields of a few hundred Gauss. Personnel working in the proximity of energized magnets are exposed to the following magnetic hazards:
	? 	Danger of metal tools coming into contact with exposed leads, shorting out the
 			leads, depositing a large amount of power in the tool, vaporizing the metal, and
 			creating an arc.
	? 	Danger of metal objects being attracted by the magnet fringe field, and 
			becoming airborne, possibly pinching body parts or striking fragile components
 			(such as vacuum windows) resulting in a cascading hazard condition.
	? 	Danger of cardiac pacemakers or other electronic medical devices no longer 
			functioning properly in the presence of magnetic fields.
	? 	Danger of metallic medical implants (non-electronic) being adversely affected
 			by magnetic fields.
	? 	Lose of information from magnetic data storage driver such as tapes, disks, 
			credit cards.
Hazard Mitigation
Although Hazard mitigation is very configuration dependent certain common sense actions apply to all setups:
 1. Whenever possible de-energized magnets when not in use or when close proximity by personnel is require.
	2. Post and obey warning signs and flashing lights indicating a magnet is energized.
	3. Follow all safety procedures when working near electro-magnets.
	4. Contact appropriate JLab technical staff for all servicing/configuration changes.
3.4.2		Power Supplies 
	All the spectrometer magnets, and some of the beam line magnets, are high current devices. The power supplies that provide this current are potentially lethal. The most exposed and hence most dangerous places are inside the supplies themselves and at the magnet power leads.
Hazard Mitigation	
	The practice of keeping electro-magnets energized in the experimental areas during short accesses provides substantial benefits to the quality and effectiveness of the physics program. The resulting hazards have been mitigated by a combination of protective shields, personnel training, warning lights and signs, and administrative procedures. Two different modes of operation need to be distinguished: (1) routine operation involving work in the vicinity of the magnets, but not in close proximity to the electrical connections, and not involving any work that could result in purposely getting into contact with the coils or the leads, and (2) non-routine operation involving work on or near the exposed current conductors or connections (typically requiring removal of the shield) or any work that could result in contact, intentional or otherwise, with the coils or the leads.
	During routine operation the following measures shall be taken by the cognizant hall engineer (or his designee) to mitigate the hazards during routine operation:
? 	The current carrying conductors must be protected against accidental contact or
  	mechanical impact by appropriate measures (e.g. run cables in grounded metal
 	conduits or cable trays).
	? 	All exposed current leads and terminations shall be covered by non-conductive
			or grounded shields in such a manner as to make it impossible for personnel to
 			accidentally touch exposed leads with either their body or with a tool. Personnel
 			shall be instructed not to reach inside the shields. Warning signs shall to be
 			placed on the shields; the signs shall read:
    ------------ INSERT  CURRENT VERSION OF WARNIMG NOTICE  -------------
	? 	Whenever a magnet is energized, a flashing light on the magnet or on the magnet support structure must be activated to notify and warn personnel of the associated electrical and magnetic field hazards.
	? 	Administrative measures shall be implemented, as appropriate for the situation, to reduce the danger of metal objects being attracted by the magnet fringe field and becoming airborne. (Note that for most magnets strong magnetic fields are only encountered within non-accessible areas inside the magnet.) Areas where these measures are in effect shall be clearly marked.
	? 	To reduce the danger of magnetic fields to people using pacemakers or other medical implants, warning signs shall be prominently displayed at the entrance to each hall. The sign shall read:
    ------------ INSERT  CURRENT VERSION OF WARNIMG NOTICE  -------------
	During electrical work restrictions are established according to hazard class and mode of work. The mode of work is determined by the nature of the work:
	1. 	De?energized
	2. 	Energized with reduced safety and restricted manipulative operations
	3. 	Energized with manipulative operations
The hazard class is determined by the type work (electrical or electronic) and the combination of voltage and current.
	? 	Anyone working on magnet power supplies must comply with the Standard Operating Procedure (S.O.P.). They must be trained and qualified and obey the new arc flash and shock hazard procedure.
	? 	All maintenance shall be performed in strict accordance with the Jefferson Lab EH\&S Manual, and with Lock Tag and Try (LTT). The following references should be consulted before power supply maintenance or operations are attempted: (1) the operating procedure provided by the manufacturer, (2) any JLab magnet power supply maintenance procedures.
	? 	Removal of any protective shield or cover for an electrical conductor shall be performed using administrative lockout procedures. The lockout shall be performed by the cognizant hall engineer (or his designee). The administrative lock shall not be removed until the protective shield or cover has been fully re-installed.
	Power-on Maintenance: There will be no mode 3 (?hot work?) on any power supply in the experimental endstations. Mode 3 work is defined as manipulative operations that are conducted with equipment fully energized and with some or all normal protective barriers removed.
3.4.3		Low Conductivity Water System (LCW)
	The magnet coils (in the case of resistive magnets), the magnet power supplies and in some cases the current leads are cooled by the Low Conductivity Water system. This is a high-pressure system, P = 240 PSI, and an unconfined stream of water at this pressure could cause injury.
Hazard Mitigation	
	In case of problems with LCW contact one of the responsible personnel listed in Section 5.

3.4.4		Cryogenic Fluid 		
	The energy stored in superconducting magnets may be sufficient to cause an unrecoverable quench if all the energy stored is dumped into the magnets. When a magnet is energized a quench can happen if the superconducting magnet has its helium level dropped below a level that exposes the coil or their current leads have insufficient cooling. The release and subsequent expansion of cryogenic fluids presents the possibility of an oxygen deficiency hazard. Rapid expansion of a cryogenic fluid in a confined space presents an explosion hazard. Cryogenics in experimental endstations are present in the superconducting magnets, and the scattering chamber with its cryogenics. Contact with cryogenic fluids also presents the possibility of severe burns (frostbite) on extremities and if inhaled as a cold gas in the lungs.
Hazard Mitigation	
	With respect to superconducting magnet systems, in all cases quench protection circuit are incorporated. In addition, normally accessible areas of the experimental endstations are listed as an Oxygen Deficiency Hazard area of Class 0. No unescorted access is allowed without an up-to-date JLab ODH training. All volumes in the cryogenic systems which can be isolated by valves or any other means are equipped with pressure relief valves to prevent explosion hazards. All issues concerning the safe operation of cryogenic magnet systems and the experimental endstations cryotarget systems were reviewed by outside panels. The responsible personnel for access to an ODH-area are the principal physics division contacts for the JLab EH\&S group are: Bert Manzlak - x7556 (Physics EH\&S ) and Charles Hightower - x7608 (Physics EH\&S ).
3.5	Target Systems
	Scattering target systems are present in all four endstation complexes and their configuration can change as required by the specific experiment currently running. The basic types of target materials include cryogenic H2, D2, 4He, 3He, radioactive material and solid targets. 
3.5.1	Hydrogen and Deuterium Cryogenic Targets
	The experimental endstation cryogenic target systems depending upon the experiment specific configuration required and can consists of multiple instrumented targets, normally containing cryogenic H2 and D2. The hydrogen and deuterium targets present a number of potential hazards. The most notable of these are associated with the fire/explosion hazard of the flammable gas, the hazards connected with the vacuum vessel and those of handling cryogenic liquids (ODH and high pressure). In this document the hydrogen target will be referred to but the deuterium target is essentially identical and almost all comments apply to both targets.
	Hydrogen and deuterium are flammable, colorless, odorless gases and hence not easily detected by human senses. Hydrogen air mixtures are flammable over a large range of relative concentrations: from 4% to 75% H2 by volume. Detonation of explosions can occur with very low energy input, less than that required by mixtures of air and gasoline. At temperatures above -250? C hydrogen gas is lighter than (STP) air and hence will rise. At atmospheric pressure the ignition temperature is approximately 1000 degrees F but mixtures at pressures of 0.2 to 0.5 Atm can be ignited at temperatures as low as 650 degrees F. Hydrogen mixtures burn with a colorless flame. The total volume of liquid hydrogen in the target depends on which type of target cell is installed. The total explosive gas inventory associated with the targets is typically quite substantial. The basic idea behind safe handling of any flammable or explosive gas is to eliminate oxygen (required for burning) and to prevent exposure to any energy source that could cause ignition. In the experimental endstation environment the most likely source of oxygen is of course the atmosphere and the most likely ignition sources are from electrical equipment.
	Potential electrical installation sources exist within the experimental endstations. The experimental endstations contain a lot of electrical equipment and almost all of it could serve as an ignition source in the presence of an explosive oxygen and hydrogen mixture. JLab has made an effort to minimize the dangers from the equipment that is most likely to come into contact with hydrogen/deuterium gas. However, there are a number of electrically powered devices associated with the target gas handling system. The solenoid valves on the gas panels are approved for use in a hydrogen atmosphere, as are all the pressure transducers in the system. In addition to the electrical devices in the gas handling system there are a number of devices inside of or mounted on the scattering chamber. 
Hazard Mitigation
	There are flammable gas detectors installed to provide early detection of hydrogen/deuterium leaks. These detectors are sensitive (and calibrated) over the range from 0% to 50% Lower Explosive Limit (LEL) of hydrogen. The low level alarm is tripped at 20% LEL while 40% LEL activates the high level alarm. These detectors require periodic calibration. This calibration is checked by the JLab Target Group. The relays are connected to the Fast Shut Down System, FSD, which removes the beam from the hall.
	The target cells themselves represent the most likely failure point in the hydrogen system. The outer wall and downstream window of each cell is typically made of thin aluminum. The cell block components have been pressure tested hydrostatically at Jlab. The targets are normally temperature regulated using a software PID loop that takes as input the temperature as given by one of the Cernox resistors and outputs a control voltage to the power supply of the high power heaters.
	The most important aspect of hydrogen safety is to minimize the possibility of explosive mixtures of hydrogen and oxygen occurring. To this end the gas handling system has been made of stainless steel components (wherever possible) and as many junctions as possible have been welded. The pressure in the gas handling system is monitored in numerous places. The gas handling and controls system have been designed to prevent excessive pressure build up in the system in order to protect the target cells from rupture. The relief line of the target leads directly to the large recovery tanks. All target pressure reliefs are connected to a dedicated hydrogen relief line. 
	Solid hydrogen is more dense than the liquid phase so freezing does not endanger the mechanical integrity of a closed system. The chief hazard is that relief routes out of the system will become clogged with hydrogen ice making the behavior of the system during a warm up unpredictable. The freezing point of deuterium is higher than that of hydrogen and higher than the temperature of the gas used for cooling. 
	The target controls have been implemented with the EPICS control system and with hardware very similar to that employed by the accelerator. The pressure at various places in the system is monitored and alarm states are generated if a transducer returns a value that is outside user defined limits. 
	Whenever hydrogen (or deuterium) is condensed in the system, a responsible person must be on duty in the counting house. This individual is designated the target operator. He/she must be authorized to operate the target and the local target expert must keep a list of all the authorized individuals. In order to become eligible to act as a target operator a person must first be trained. 
3.5.2	Helium Cryogenic Targets
	3He and 4He gas targets are sometimes used instead of the more usual LH2 and LD2 target configuration. The helium targets are typically operated at 5.5K and 200 psia. Clearly, similar cryogenic concerns as with the use of LH2 and LD2 targets discussed in the previous section apply to cryogenic Helium targets. These targets are also contained in a scattering chamber, so thin vacuum window eye and ear protection is still required in the vicinity of the target region while the scattering chamber is evacuated. 
Hazard Mitigation
	All hazard mitigations discussed previously with respect to LH2 and LD2 targets apply with the exception of flammability. The high pressure associated with these targets also presents a hazard. In addition, new deliveries of 3He are checked for tritium contamination. The JLab Radiation Control Group has insured that the tritium concentration in the current 3900 STP liter 3He inventory is less than 15 ?Ci.
3.5.3	Polarized Targets
	Experiments that use a polarized target have a supplemental experiment specific ESAD
which includes information about the polarized target, hazards and hazard mitigation.
Hazard Mitigation
3.5.4	Radioactive Targets 
	Experiments that use a radioactive target (such as tritium) have a supplemental experiment specific ESAD which includes information about target specific risks and mitigation.
Hazard Mitigation
3.5.5	Solid Targets
	Some of the target materials may pose a serious safety concern. At this moment the only two special target materials we own are ceramic Beryllium-Oxide (BeO) and Beryllium (Be). In solid form, BeO is completely safe under normal conditions of use. The product can be safely handled with bare hands. However, in powder form all Beryllia is toxic when airborne. Over exposure to airborne Beryllium particulates may cause a serious lung disease called Chronic Berylliosis. Beryllium has also been listed as a potential cancer hazard. Furthermore exposure to Beryllium may aggravate medical conditions related to airway systems (such as asthma, chronic bronchitis, etc.). 
Hazard Mitigation
	Since beryllia are mainly dangerous in powdered form, do not machine, break, or scratch these products. Machining of the Beryllia can only be performed after consulting the EH\&S staff. It is good practice to wash your hands after handling the ceramic BeO. If handling the pure Beryllium target wear gloves and an air filter mask. These target materials are stored in the yellow target storage cabinets, either in the back room of the counting house or in the black safe downstairs in the experimental endstations experimental area. 
3.6	Gas Detectors
Hazard Mitigation
3.7	Laser Systems
Hazard Mitigation
3.8 	Spectrometer Carriage(s) (if applicable)
	The carriages are the support structures of the spectrometers used in Halls A and C. First and foremost (as it is a multi-level structure) it is important to keep in mind that people may be working above you. This means that the wearing of hard hats in the experimental endstations may be required depending on other activities going on in the endstation. Safety railings have been installed everywhere along the carriage perimeters. Be aware that some of these may be removed during the experimental data taking to enable spectrometer rotation and will need to be installed (or you need to wear a safety harness) before accessing these areas.
Hazard Mitigation
	Around the Spectrometer ?Pivot? or ?Platform? Area: The pivot area (in the case of a rotating spectrometer) or access platform (in the case of a fixed spectrometer system) is the walk-able area giving access to the scattering chamber. If the scattering chamber is evacuated, ear plugs will need to be worn when working closer than 3ft from the vacuum windows. This is also true if you need to work within 3ft. of the spectrometer vacuum windows if those spectrometers are under vacuum. In case a polarized target is used, special safety measures are taken to allow access to the pivot area (see the separate section on Polarized Targets).
	Fall Hazards: It cannot be overemphasized that one of the most significant hazards in the experimental endstations is a simple fall. Standard access routes such as stairs or ladders can also lead to serious injury if proper care is not taken. The risk can be multiplied if the individual is carrying a load of equipment such as oscilloscopes. Another fall hazard exists in the form of non-standard access routes. Generally speaking, these are to be avoided. However, use of a non-standard access route such as a well-secured ladder may occasionally be necessary. Certain areas on the pivot and the carriages will have the handrails removed during experiment operations. When access to these areas is required, use fall protection as mandated by the EHS manual. Another fall hazard exists in the form of non-standard access routes. Generally speaking, these are to be avoided. However, use of a non-standard access route such as a well-secured ladder may occasionally be necessary. Certain areas on the pivot and the carriages will have the handrails removed during experiment operations. When access to these areas is required, use fall protection as mandated by the EHS manual.
	Spectrometer Rotation (if applicable): The obvious problem with spectrometer rotation is that one rotates multi-ton object which will crush whatever is in its way. Rotation of the spectrometer is accomplished by using powerful electrical motors/gearing on the carriage itself. These motors may only be controlled by trained personnel when in manual mode.  Remote spectrometer operation within certified limits may be permitted when the hall is in beam permit, or controlled access with no onein the hall.  A spotter (in communication with a remote operator) is required when the spectrometer is remotely rotated when the hall is populated.
3.9 		Vacuum and Pressure Hazards
	The greatest safety concern for the vacuum and/or pressure vessels possibly in use in experimental endstations are the thin Aluminum or kevlar/mylar windows that close the entrance and/or exit of these vessels. All work on vacuum windows in experimental endstations must occur under the supervision of appropriately trained Jlab personnel. Some spectrometers (e.g. the HMS) have a shutter (for additional safety) that can go in front of the window in the detector hut.  The scattering chamber and beamline exits windows must always be leak checked before service, but obviously the possibility of vacuum loss cannot be eliminated. The most likely sources of vacuum failure are:
	1.  	Spectrometer Windows: The scattering chamber has two aluminum windows, one for each spectrometer.
	2.  	Beam Exit Window: There is a thin beam exit foil at the downstream end of the beamline at the entrance to the dump tunnel.
	3. 	Target Cell Failure: This is a multiple loop system. If a target cell fails then the remaining targets will have their insulating vacuum spoiled. In the unlikely event that a line which carries helium coolant were to rupture the large chamber relief valve is capable of handling the full coolant flow rate.
Hazard Mitigation	
	Installation of vacuum windows on beamlines and scattering chambers can only be done by the responsible and trained personnel. A beam interlock prevents beam delivery with the fast raster off. This interlock is properly checked at the end of a long (installation) down time. In addition, an administrative current limit is imposed on beam delivery conditions, as given in the accelerator operational restrictions: http://opweb.acc.jlab.org/internal/ops/ops webpage/restrictions/ops restrictions.php.
	Working Near Vacuum Window: Before entering the detector huts or pivot area, all personnel should check the spectrometer and/or scattering chamber vacuum gauges. If the spectrometers and/or scattering chamber are under vacuum:
	1. 	Before entering the area, put on hearing protection. It is recommended that nobody should be closer than 3 feet from the windows without ear protection and that only those personnel who need to approach the windows be in their immediate vicinity.
	2. 	If entering the area, check both spectrometer windows visually (from a distance greater than 3 feet if possible). If you observe any of the following problems, vacate the area and contact one of the responsible personnel: Visual defects, particularly wrinkles, discoloration, or uneven fiber stress. Date of removal on tag near spectrometer window indicates a date near or after current date.
	3. 	Never touch the vacuum windows, neither with your hands nor with tools.
	4. 	Never attempt to bypass the any shutter or valve interlock system.
	5. 	Use careful judgment if it is necessary to work near the vacuum windows. Do not place objects so that they may fall on the windows, etc.
	6.	 Do not work near the windows any longer than is absolutely necessary.
3.10 	Crane and Mechanical Equipment Operation
	Heavy objects are routinely moved around within the endstations during reconfigurations for specific experiments. This work can only be performed by approved and trained crain operators. No ?user? crane operation is permitted!
Hazard Mitigation
	Hard hat use is required in experimental endstations whenever mechanized equipment is in use in the hall (Cranes, JLG lifts, man lifts, fork lifts), or when the entrances to the hall are marked with hard hat required signs. Hard hats are not required when fully within the beamline tunnel, the Spectrometer detector bunkers or electronics bunker. The Hall work coordinator may also grant exceptions for specific work with sensitive equipment when there is no overhead work above the work area.
3.11		Hazardous and Toxic Materials
Note that the lead shielding blocks we use also form a potentially toxic material.
Hazard Mitigation
	Unwrapped or painted lead blocks may only be handled by certified lead workers who have undergone lead worker training. Gloves must be worn when handling uncovered blocks (this excludes blocks that are completely painted or wrapped in Heavy-Duty Aluminum Foil). Lead worker training is not required for the handling of lead bricks contained plastic bags. However, steel?toed shoes must always be worn when handling lead bricks of any type. Do not machine lead yourself, contact the EH\&S personnel or the Jefferson Lab workshop to ask for the procedure to machine lead.	
4		Radiation Safety and the Personnel Safety System
	CEBAF?s high intensity, high energy electron beam is a potentially lethal radiation source and hence many redundant measures aimed at preventing accidental exposure of personnel to the beam or exposure to beam-associated radiation sources are in place. When the hall is in an accessible state, all beamline or target chamber equipment requiring machining or disassembly, and all components which need to be removed from the Hall, must be surveyed by an ARM or a member of the Radiation Control Department (RadCon). 
Hazard Mitigation
	Only RadCon can approve unrestricted release of items from the hall (ARMs surveys may allow relocation, under the ARM?s direction, but not release). The radiation safety department at JLab can be contacted as follows: For routine support and surveys, or for emergencies after-hours, call the RadCon Cell phone at 876-1743. For escalation of effort, or for emergencies, the RadCon manager (Keith Welch) can be reached as follows: Office: 269-7212, Cell: 876-5342 or Home:  875-1707.
5		Fire Hazard
6		Experiment Readiness Clearance
6.1	Overview
	The Readiness checklist traditionally includes both safety items as well as those actions and checks which should be completed before the hall is closed in order to avoid unnecessary controlled accesses later on. In this document, personnel and/or equipment safety items are denoted in boldface. The experimental endstations Physics Liaison is responsible for ensuring that the checklist has been completed.
6.2	Assumptions 
	Example: ?A break in the schedule of at least ten days has occurred. All target options are to be available: LH2/LD2, empty, solid, halo, and no target. The LH2/LD2 target was warmed up at the start of the break in the schedule and now needs to be re-condensed.?
6.3	Signers 
	The only people authorized to initial this document are those whose initials appear next to each line item, or their designate. The following is an example checklist upon which experiment specific checklists will be based.
7 		Responsible Personnel




15
}

