\section[The Hall A Gas System]{The Hall A Gas System
\footnote{
  $CVS~revision~ $Id: gas.tex,v 1.4 2003/11/14 18:54:53 segal Exp $ $
}
\footnote{Authors: J.Segal \url{mailto:segal@jlab.org}}
}

\subsection{Overview}
The Hall A detector gas systems are located in the Hall A Gas Shed
alongside of the truck ramp for Hall A.  The gas cylinders in use
are along the outside of the Gas Shed in a fenced area.
There are racks next to
the Gas Shed for storage of full gas cylinders.  On the other side of the
truck ramp there are racks for storage of both full and empty cylinders.
Hall A currently uses ethane, argon, ethanol, carbon dioxide, methane, 
and nitrogen.
Details of these systems can be found in the Hall A Gas Systems (HAGS) manual.
A copy of the current manual is in Counting Room A and on the Hall A web page.

Four systems are supplied from two cylinders of Coleman grade CO2.
One system is for the gas Cerenkov counters in the HRS detector arrays.
One system is for
flushing the mirror aerogel Cerenkov counter in the HRS detector arrays.
One system is for
the gas Cerenkov counters in the (e,p) setup in the beamline.
One system is for the FPP straw tube wire chambers.
Argon and carbon dioxide for the FPP straw tube wire chambers
are mixed inside the Gas Shed.

Three systems are supplied from two cylinders of UHP grade argon.
One system is for the VDC wire chambers
of both arms.  Argon and ethane for the VDC wire chambers are
mixed inside the Gas Shed and bubbled
through ethyl alcohol.
One system is for the FPP straw tube wire chambers.
Argon and carbon dioxide for the FPP straw tube wire chambers
are mixed inside the Gas Shed.
One system is for flushing clean, inert gas through the RICH detector
wire chamber.

One system is supplied from two cylinders 
of Chemically Pure grade ethane.  This is for the VDC wire chambers
of both arms.  Argon and ethane for the VDC wire chambers
are mixed inside the Gas Shed and bubbled
through ethyl alcohol.

Two systems are supplied from two cylinders of UHP grade nitrogen.
One system is used to provide pressurized
gas for the automatic cylinder switch-overs in the systems.
One system is used to flush impurities from the RICH detector
freon resevoir.

One system is supplied from two cylinders of UHP grade methane.
The system is for the wire chamber of the RICH detector.

Maintenance of the gas systems is routinely performed by the Hall A
technical staff.  Shift personnel are not expected to be responsible
for maintaining the detector gas systems.  Unexpected maintenance
requirements should be handled by contacting

\begin{itemize} 
\item[~]Jack(John) Segal - pager and phone are both extension 7242
\item[~]Hall A Technician on call
\end{itemize} 

\subsection{Gas Alarms}

In Counting Room A there are two alarm panels associated with the gas
systems for the detectors.  They are located on the far left end of the
control console, mounted one above the other.  The upper panel is a
Gas Master flammable gas monitoring system.  The lower panel is a gas
systems status indicator.  The Gas Master system will go into alarm if
elevated levels of flammable gas are present in either of the Detector
Shielding Huts or the Gas Shed.
The gas systems status will
alarm if any of a number of faults are detected in the Hall A Wire-chamber
Gas System.  The LED for the specific fault will turn red to indicate which
fault caused the alarm.

Response to an alarm should be to contact either of the personnel listed above.
% ===========  CVS info
% $Header: /group/halla/analysis/cvs/tex/osp/src/hrs_det/gas.tex,v 1.4 2003/11/14 18:54:53 segal Exp $
% $Id: gas.tex,v 1.4 2003/11/14 18:54:53 segal Exp $
% $Author: segal $
% $Date: 2003/11/14 18:54:53 $
% $Name:  $
% $Locker:  $
% $Log: gas.tex,v $
% Revision 1.4  2003/11/14 18:54:53  segal
% Screw this file up
%
% Revision 1.3  2003/06/06 17:00:27  gen
% Revision printout changed
%
% Revision 1.2  2003/06/05 23:30:01  gen
% Revision ID is printed in TeX
%
% Revision 1.1.1.1  2003/06/05 17:28:30  gen
% Imported from /home/gen/tex/OSP
%
%  Revision parameters to appear on the output
