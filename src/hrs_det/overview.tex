
\section{Overview}

The detector package of each spectrometer has trigger, tracking, and particle 
ID components. In addition, the Hadron spectrometer has a unique
proton polarimeter. Particles 
which have passed through the magnetic 
elements first encounter the tracking detectors to  minimize 
the multiple scattering contribution to the angular and 
energy resolutions of the spectrometer. The tracking part consists of 
two identical vertical drift chambers. The trigger detectors include two planes of
thin plastic scintillator counters, gas and aerogel Cerenkov
counters and a shower 
counter. On the hadron spectrometer the shower counter or large scintillator counter 
can be used in the trigger. Particle ID is provided by several techniques. 
For electron identification the electron arm, EA, has the gas
Cerenkov counter and two layers of a segmented 
lead glass shower counter. Because the hadron arm, HA, also can
be used for experiments with electrons 
it is equipped with a short version of the gas Cerenkov counter
and one layer of a segmented 
lead glass shower counter. Pion identification in both
spectrometers relies on an aerogel 
Cerenkov counters which presently have aerogel radiator
with a refraction index, n, of 1.025. 
Aerogel Cerenkov counter commissioning is not yet completed. For particle momenta  
below 800 MeV/c, the dE/E in the scintillator and shower counters can be used for 
separation of pions and protons. The large distance between
planes of the trigger scintillator 
counters (2 -3 $m$) allows a direct measurement of the particle
speed with resolution ( sigma )
of 0.07. Measurement of the time of flight on the long path from the target to the 
spectrometer ( $~ 25 m$ ) provides another powerful particle ID
for coincidence experiments.
The focal plane polarimeter on the HA operates with proton
momenta up to 3 GeV/c with a figure of merit $~ 0.03$.\\

The detector packages are installed inside of the Shielding Huts (SH). Access to the Shielding Huts 
is via very heavy swinging front doors. The main structure of the
SH is made from 3 $in$ thick steel plates.
The side walls and bottom surfaces of the SH are covered inside
with 1 $in$ thick lead slabs. Outside of the steel box, concrete
is used for neutron protection. 
The front door has about 34 $in$ of concrete and 3 $in$ of lead. Side walls are 
covered with 17 $in$ of concrete. The roof of the SH has 10 $in$ of concrete above 
3 $in$ of steel. The lower half of the side walls facing the beam dump have an
additional cover of 15 $in$ of concrete. Additional ``Line of Sight 
Shielding'', LSS, is installed at a distance of $~5 m$ from the
target. This consists of 2 to 3 $m$ of concrete. 
High energy pions interact in this concrete before decaying. The
LSS reduces the rate of 
high energy muons, which are produced in pion decay. The overall
result from SH and LSS is a reduction 
factor of 10 to 20 in the counting rate of a single scintillator counter
( according to calculations ). 

The 2 VDCs provide accurate tracking information. They are
mounted on a movable frame which slides 
along Thompson rails to hard stops. 
The position of the VDC on the frame and location of the 
Thompson rails is surveyed relative to the Hall center. The
rest of the detectors are mounted 
on a detector frame which can be moved out of the Shielding Hut
for detector maintenance.   

\subsection{Geometry of the Spectrometer Detector Packages}

Tables ~\ref{ta:Edetg} and ~\ref{ta:Hdetg} give geometry
information for the electron arm and hadron arm detector packages. 
The values in the tables
indicate the position of the central point of the detector.
The origin of coordinate system (0,0,0) is located at the intersection of 
the mid plane of the spectrometer and the nominal focal
plane ( $\sim$ middle of the Bottom VDC ).

\begin{table}[hptb]
\begin{center}
\begin{tabular}{cccccccc}
detector&location&  location& width &   width &      BEAM  &        & ENVELOPE \\
        & actual &IDEAS model&   X  &     Y   &      X(+)&  X($-$) &   Y\\  \hline
       &        &          &        &         &          &         &           \\  \hline    
VDC1*   &      0 &          &   1942 &    271  &     843    & - 824  & +/-  57  \\
VDC2*   &     572&          &   1942 &    271  &     932    & - 911  & +/-  85  \\
S1      &    1311&     1321 &   1718 &    356  &     696    & -1022  & +/- 163  \\ 
AERO    &    1646&          &   199  &    414  &     709    & - 888  & +/- 182  \\
GAS     &    2535&          &   2200 &    650  &     886    & -1110  & +/- 279  \\ 
S2      &    3358&     3378 &   2197 &    540  &     897    & -1124  & +/- 285  \\
preSHOW &    3502&     3546 &   2400 &    700  &     925    & -1158  & +/- 301  \\ 
SHOW2   &    3780&     3912 &   2400 &    900  &     964    & -1207  & +/- 322  \\  \hline
\end{tabular}
\end{center}
\caption[Detectors: Electron ARM Detector Locations]{Locations of
the detectors on Electron Arm in mm.}
\label{ta:Edetg}
\end{table}

\begin{table}[hptb]
\begin{center}
\begin{tabular}{cccccccc}
detector&location&  location& width &   width &      BEAM  &        & ENVELOPE \\
        & actual &IDEAS model&   X  &     Y   &      X(+)&  X($-$) &   Y\\  \hline
       &        &          &        &         &          &         &           \\  \hline    
VDC1*  &        &         0&    1942&     271 &     843  &  - 824  &  +/-  57  \\    
VDC2*  &        &       500&    1942&     271 &     932  &  - 911  &  +/-  85  \\ 
S1     &        &      1287&    1760&     360 &     675  &  - 845  &  +/- 163  \\    
AERO   &        &      1617&    1872&     414 &     709  &  - 888  &  +/- 182  \\   
SC1    &        &      1837&    1780&     480 &     738  &  - 924  &  +/- 198  \\    
GAS    &        &      2409&    2200&     650 &     857  &  -1073  &  +/- 263  \\    
SC2    &        &      2952&    2080&     640 &     865  &  -1083  &  +/- 268  \\   
S2     &        &      3141&    2220&     640 &     877  &  -1099  &  +/- 274  \\   
Analyzer&       &      3495&    2190&     680 &     916  &  -1147  &  +/- 296  \\   
SC3    &        &      3907&    2540&    1000 &    1099  &  -1343  &  +/- 457  \\  
SC4    &        &      4264&    3170&    1500 &    1382  &  -1645  &  +/- 705  \\    
S3     &        &      4477&    3600&    1550 &    1437  &  -1704  &  +/- 752  \\  \hline
\end{tabular}
\end{center}
\caption[Detectors: Hadron Arm Detector Locations]{Locations of
 the detectors on Hadron Arm in mm.}
\label{ta:Hdetg}
\end{table}

%\vfill\eject



















% ===========  CVS info
% $Header: /group/halla/analysis/cvs/tex/osp/src/hrs_det/overview.tex,v 1.1 2003/06/05 17:28:30 gen Exp $
% $Id: overview.tex,v 1.1 2003/06/05 17:28:30 gen Exp $
% $Author: gen $
% $Date: 2003/06/05 17:28:30 $
% $Name:  $
% $Locker:  $
% $Log: overview.tex,v $
% Revision 1.1  2003/06/05 17:28:30  gen
% Initial revision
%
