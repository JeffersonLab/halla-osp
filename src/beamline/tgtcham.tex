\infolevone{
\chapter[Target Chamber]{Target Chamber
\label{sec:target_chamb}
\footnote{
  $CVS~revision~ $Id: tgtcham.tex,v 1.11 2005/04/04 22:27:25 gen Exp $ $
}
\footnote{Authors: ?? \email{??@jlab.org}}
}

The cryo-targets and the waterfall targets 
(see Sec.~\ref{sec:targets-overv}) 
are contained in a special target chamber which is a large 
evacuated  multistaged can. So far, three chambers have been designed:
\begin{list}{\arabic{enumi}.~}{\usecounter{enumi}\setlength{\itemsep}{-0.15cm}}
  \item a chamber used up to 2003;
  \item a chamber designed for use with septum magnets, starting in 2003;
  \item a chamber designed for use with the BigBite spectrometer.
%\footnote{
%        No yet manufactured by Dec,2003.}.
\end{list}

Here, chamber 1 is described. Chambers 2 and 3 are only different in 
size and slightly in shape. The safety considerations fully apply to chambers 2 and 3.
The chamber was designed to isolate the beam line vacuum from  each
HRS so that each HRS could rotate
around the target without vacuum coupling and without jeopardizing
certain desired kinematic and acceptance  specifications of 
both high resolution spectrometers
needed for approved experiments.  It  was also designed to simultaneously
 contain a liquid or gas target and an array of water cooled thin
 metallic foils, both remotely controlled and also be adaptable for
the waterfall target. The desired kinematic specifications that were
 considered included momentum and energy resolution in both arms,
 angular range of spectrometers, angular acceptance, and luminosity.
The chamber vacuum is isolated from the  HRS by using thin aluminum foils. 

The target chamber is designed so that
each spectrometer will have continuous coverage in the standard tune from
$\theta_{min}=$12.54$^\circ$ to $\theta_{max}=$165$^\circ$.
The aluminum window is 6~$in$ high and 0.016~$in$ thick made of 5052 H34 aluminum foil.
The foil forms regularly spaced vertical ridges when
placed under load. The window had an inter-ridge
spacing of 3 inches.
If the window is treated as a collection
of smaller rectangular windows which have the full vertical height
of 6 inches and the inter-ridge spacing as a width,
then stress formulas predict that the 0.016 $in$
material would reach ultimate stress at a pressure higher than 35 PSID
(for both over-pressure and under-pressure). 
There is a gate valve between the 
scattering chamber and the beam entrance (exit) 
pipe. Both 
valves will be closed automatically in the
event that the chamber vacuum begins to rise and an FSD will be caused
( this is done via a relay output of the scattering
chamber vacuum gauge). If either valve is closed an FSD will result.

The target chamber is supported by a 24 $in$ diameter pivot post
secured in concrete, rising about 93.6 $in$ above the Hall A cement floor.
The Hall A target chamber
consists of an aluminum middle ring, a stainless steel base ring,
each with a 41.0 $in$ inner diameter,
and a stainless steel cylindrical top hat with 40 $in$ inner diameter
to enclose the cryotarget and secure the cryogenic connections.

When the scattering chamber is under vacuum, there is a potential
danger of window rupture.
The loud noise from the rupture could hurt
one's ears if not protected. Therefore when the chamber is under vacuum,
protective covers are put on if possible. These must be taken off
for data taking. For restricted access, the protective cover is required
to be on when the chamber is under vacuum. Before switching from controlled
access to restricted access, the protective cover is required to be installed.
Anytime that the scattering chamber
is under vacuum, the pivot area is enclosed in a rope or tape barrier
and a warning sign is posted.
Hearing protection is required in the enclosed area.

\infolevone{
	The aluminum ring with an outer diameter of 45.0 $in$ and
wall thickness 2.0 $in$  is necessary for a sturdy support structure and
to permit machining of the outside surface to accommodate
the flanges for fixed and sliding seals mounted on
opposite sides of the ring that vacuum connect the chamber to each HRS.
The height of the aluminum ring shown is 36.0 $in$, which is
designed to accommodate the mounting flanges.
The stainless steel base ring 
is 11.50 $in$ in height with
one pump-out 6 $in$ diameter port  and with
seven 4 $in$ viewing and electrical feed-through ports.
The base ring will also contain support mechanisms for the solid
target ladder assembly, a rotisserie for collimating slits, radiators, and
magnetic
fingers for
removing the solid target vacuum-lock can. The total height of the top
ring, middle ring, and
base ring is 93.81 $in$. This length is partly determined by our desire to
include with the cryogenic extended target a solid target vertical ladder
secured in an inverted hat through a hole in the base of the chamber.

	The base ring includes an end plate through which the
inverted hat will be adapted to fit into the large vertical pipe serving
as the pivot post for the Hall A spectrometers.

	The stainless steel cylindrical top hat  has
40.0 $in$ inner diameter, and is 0.375 $in$ thick and
46.31 $in$ high , which is necessary to permit the
cryotarget to be withdrawn and to make space available to expose the solid
targets to the electron beam.

   The 200 $\mu$A electron beam, focused to a $\sim$\(0.1\, mm\times
0.1\) mm spot and rastered $\pm$5 mm horizontally or vertically on the
target, enters through a oval hole in the middle ring which
is 2.06 $in$ wide and exits through a 1.81 $in$ hole connected to the
exit pipe.
}

\infolevone{
\section{Target Chamber - Spectrometer Coupling}

   The aluminum middle ring will support a flange on each side for each high
resolution spectrometer. Four flanges will be available: Two flanges will
contain a 6 $in$ window opening which will be covered with a thin foil
(e.g., 10 mil aluminum) .
These two flanges will be used for experiments utilizing
extended  targets that do not require optimum momentum resolution.
The other two flanges will have two fixed ports (with a 8 $in$ $\times$ 6 $in$
opening)
which will be mainly used for calibration of the spectrometers . Fixed ports are
centered at 16.11 $^\circ$ and
45 $^\circ$ for one flange and at 16.11 $^\circ$ and 90 $^\circ$ for the second
flange.

   For a point beam on target a vertical opening in the walls of the chamber
of height 57.15 cm x 0.065 x 2 = 7.43 cm is required so that the scattered
beam is within the full acceptance of the spectrometer.
If the beam is rastered on target $\pm$0.5 cm in the vertical direction,
then the opening in the outer side of the chamber must be at least 8.5 cm for
full acceptance.

From consideration of the angular range of the spectrometers in the standard
tune, the scattered beam acceptance envelope, the effects of an
extended gas target on acceptance,
and the effects of a rastered beam $\pm$ 5 mm on acceptance,
the target chamber requires a window of at least 8.5 cm
high in the aluminum ring extending from 6.33 $^\circ$ (2.48 in) from the
beam exit point to 8.83 $^\circ$ (3.47 in) from the beam entrance point on one
side and a similar window on the other side of the beam.
For future considerations (e.g., using a third arm or sliding seal) the
width of the window on the middle ring was actually constructed
to be 17.78 cm (7 $in$).

\section{Stress Analysis of the Middle Ring}

Since the middle ring has an extensive cut across the midplane on both sides as
well as
entrance and exit holes and loaded with about 25,000 lbs, calculations of the
stresses
 and deformation of  the
midplane support area of the middle ring and deflection of the window opening
were made using the finite element analysis code ANSYS . The work was conducted
by a graduate student in the Department of Civil Engineering at the
University of
Virginia and a REU student.  A scaled down model of the middle ring was
constructed and then tested by applying forces to it using the Materials Testing
Service of the Department of Transportation at the University. ANSYS was first
checked by comparing calculations of the test model deflections to the actual
data. Agreement was  within $\pm$10\%. Results of ANSYS for the target
chamber showed that the maximum deflection of the opening of the window in the
middle ring varied from 0.007 $in$ to 0.015 $in$ depending on how the
middle ring
was loaded. This was decided to be a safe limit. In the final design, several
movable
7 $in$ long, 2 $in$ diameter aluminum support rods are placed in the
window for added support. In addition, flanges defining the ports and
coupling to
the spectrometers can be added, giving additional support to the middle ring.
Compressional stresses, calculated using ANSYS assuming the middle ring was
attached to the
top hat and loaded with 25,000 lbs, were less than 3000 psi 
almost everywhere.
However, stresses over small areas rose to levels 6000 psi near the entrance
and exit holes. These calculations indicated that we did not exceed the safety
limit of 15,000 psi for aluminum. A simple model calculation shown in Appendix
A  gives the result 1434 psi, which represents some average value over the
midplane
contact area.

\section{Vacuum Pumping System}

The vacuum in the target chamber is maintained by an Alcatel ( 880 l/s)
 turbomolecular vacuum pump. The pump is connected to a 6 $in$ port in the
stainless steel ring between 130
 $^\circ \le \theta_p \le 180 ^\circ$. The vacuum pump is
fastened to a horizontal pipe connected to the chamber. The vacuum pressure in
the chamber is about $10^{-5}$ mm. An additional Alcatel pump connected
to an 8 $in$ port should be added to obtain lower vacuum. Both
pumps may be isolated
from the target chamber using gate valves which are remotely operated
from the vacuum control rack and interlocked to the FSD system.


A 2 $in$ all metal gate valve is located between the entrance flange to the
chamber and the beam profile monitor.   
 An additional gate valve is located 2 m downstream of the
 target chamber to isolate the chamber from the exit beam pipe.
}
\begin{safetyen}{10}{15}
\section{Safety Assessment}
\end{safetyen}

The scattering chamber is typically a low maintenance item but it is a vacuum
system and hence problems may occur. The day to day operations of the cryogenic
targets are managed by the Hall A Staff while major maintenance operations are
handled by the Cryogenic Target Group (Physics Division). Occasionally the
cryogenic targets experience difficulties due to failures of the End Station
Refrigerator which supplies the coolant. In these cases the Cryogenics Group
of the Accelerator Division should be contacted.

\noindent{}The target chamber may pose several hazards:

\begin{list}{\arabic{enumi}.~}{\usecounter{enumi}\setlength{\itemsep}{-0.15cm}}
  \item {\bf Rupture of vacuum windows}. This hazard is mitigated by
        lexan guards on the vacuum windows, installed by the hall technicians
        either at the beginning of a ``restricted access'' period 
        %(see Sec.\ref{sec:Access}),
        or during ``control access'', in case an access to the target chamber area is needed.
        Installation and removal of the guards is included in the technician's checklists.
        When the chamber is under vacuum, it is mandatory to use ear protection in the chamber
        vicinity. The appropriate signs must be installed by the technicians. 

  \item {\bf Induced radioactivity}. The RADCON surveyor measures the level of induced
        radiation as a part of the general survey and may declare the target area 
        as ``High Radiation Area'', installing a rope protection around\cite{RWIcebaf}. 

\end{list}

Some other safety issues are discussed in the cryo-target chapter 
(see Sec.~\ref{sec:target-cryo-safety}) and also in the polarized target
chapter (see Sec.~\ref{sec:target-he3-general}).

\begin{safetyen}{10}{15}
\section[Authorized  Personnel]{Authorized  Personnel}
\end{safetyen}

\begin{namestab}{tab:targ_chamb:personnel}{Target chamber: authorized personnel}{%
      Target chamber: authorized personnel. ``W.B.'' stands for the white board 
      in the counting house.}
  \TechonCall{\em Contact}
  \EdFolts{}
  \DaveMeekins{Target group}
  \JianPingChen{}
\end{namestab}
}
