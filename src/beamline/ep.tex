\chapter[eP Beam Energy Measurement]{eP Beam Energy Measurement
\footnote{
  $CVS~revision~ $Id: ep.tex,v 1.6 2003/12/13 06:23:37 gen Exp $ $
}
\footnote{Authors: B.Reitz \email{reitz@jlab.org}}
}
\label{sec:ep}
\section {Purpose and Layout}
\label{sec:ep_purpose}

The Hall A eP system is a stand-alone device to measure the 
energy of the electron beam. It is located along the beamline
17~m upstream of the target. The beam energy $E$ is determined by measuring
the scattered electron angle $\Theta_e$ and the recoil proton angle
$\Theta_p$ in the $^1$H$(e,e'p)$ elastic reaction according to the kinematic
formula:
\begin{equation}
E = M_p \frac{\cos(\Theta_e) + \sin(\Theta_e)/\tan(\Theta_p) - 1}{1 - \cos(\Theta_p)} + O(m_e^2/E^2),
\end{equation}
in which $M_p$ denotes the mass of the proton and $m_e$ the mass of the electron.
The schematic diagram of the eP system is presented in Fig. \ref{fig:ep_layout}. 
Two identical arms, each consisting of an electron and a corresponding proton 
detector system, made up of a set of 2~x~8 silicon micro-strip detectors in the
reaction plane, are placed symmetrically with respect to the beam along the 
vertical plane. The target consists of a rotating CH$_2$ tape.
Simultaneous measurements of the beam energy with both arms result
in cancellation, to first order, of uncertainties in the knowledge of the position
and direction of the beam. 
 \begin{figure}[htb]
    \begin{center}
        \includegraphics*[angle=0,width=0.9\textwidth]{ep_layout}
    \end{center}
    \caption[eP: Layout]{
            Schematic layout of the eP energy measurement system,
            showing the arrangement of its components, the polyethylene (CH$_2$) 
            target, the Cherenkov detectors, the silicon micro-strip detectors (SSD) 
            for protons and electrons, and the scintillator detectors.
            }
    \label{fig:ep_layout} 
 \end{figure}  
%\clearpage

\infolevone{
\section{Description of Components}
\label{sec:ep_desc_comp}

\subsection{High Voltage}
\label{sec:ep_highvoltage}

The eP system is equipped with two gas Cherenkov detectors and 
altogether 18 scintillators. The high voltage for the photomultiplier
tubes of these detectors are provided by a LeCroy 1450 HV power supply,
located in the electronics racks along the beamline. The channel 
assignment and HV voltages (as of summer 2003) are given in
Table \ref{tab:ep_hv}.

\begin{table}[ht]
\begin{center}
\begin{tabular}{|l|r|l|} \hline
Channel & HV (Volts) & Detector  \\ \hline \hline
 1.2 & 2201 & S1 (bottom) \\  \hline
 1.3 & 2200 & S2 (bottom) \\  \hline
 1.4 & 1963 & S1 (top) \\  \hline
 1.5 & 1963 & S2 (top) \\  \hline
 1.8 & 1039 & S3 \\  \hline
 1.9 & 1027 & S3 \\  \hline
 2.0 & 2250 & Cherenkov  \\  \hline
 2.1 & 2250 & Cherenkov  \\  \hline
 3.0 & 1004 & S3 \\  \hline
 3.1 & 1113 & S3 \\  \hline
 3.2 & 1097 & S3 \\  \hline
 3.3 & 1144 & S3 \\  \hline
 3.4 & 1126 & S3 \\  \hline
 3.5 & 1119 & S3 \\  \hline
 3.6 & 1006 & S3 \\  \hline
 3.7 & 1112 & S3 \\  \hline
 3.8 & 1104 & S3 \\  \hline
 3.9 & 1071 & S3 \\  \hline
 3.10 & 1061 & S3 \\  \hline
 3.11 & 1051 & S3 \\  \hline
\end{tabular}
\end{center}
\caption[eP System: HV Summary]{HV connections and HV values. }
\label{tab:ep_hv}
\end{table}

\infolevtwo{
The standard way to control the high voltage is the use of the 
Hall A MEDM~\cite{MEDMwww} graphical user interface (EPICS~\cite{EPICSwww}), which is running 
on the \mycomp{hacsbc2} computer. This computer is located in the counting house,
but can also be accessed from other terminals. Usually at least one terminal 
in Hall A itself has a MEDM screen running, as well. If it is not running, log into \mycomp{hacsbc2}
as user \mycomp{hacuser}, and start the GUI with the command
\mycomp{hlamain}. A screen labeled ``Hall A Main Menu'' will appear (Fig. \ref{fig:medm-hlamain}).
Chose \mycomp{LeCroy HV}, and select \mycomp{Beamline} in the screen which will pop 
up (Fig. \ref{fig:ep_hvlecroy}). 


 \begin{figure}[bht]
    \begin{center}
        \includegraphics*[angle=0,width=6cm]{ep_lecroy}
    \end{center}
    \caption[eP: LeCroy HV Screen]{
	    Epics Menu for the LeCroy High Voltage supplies in Hall A. All slots related
            to the eP system can be accessed from the Beamline button.
            }
    \label{fig:ep_hvlecroy} 
 \end{figure}  
}

For a measurement, all HV channels defined in Table \ref{tab:ep_hv}
should be turned on. The demand voltages in these slots
(Slot 1, Slot 2 ``(e,p) \& ARC'' and Slot 3 ``Moller'') should have 
the correct preset values. 
To turn the HV on (or off), or to change the 
preset values,
press the button below the title of the slot. Another screen will pop-up,
where status and preset values can be adjusted. \infolevtwo{
(See Figs. \ref{fig:ep_hvbeamline}, \ref{fig:ep_hvslot1}, \ref{fig:ep_hvslot2}, and \ref{fig:ep_hvslot3})

\begin{figure}[bht]
    \begin{center}
        \includegraphics*[angle=0,width=0.9\textwidth]{ep_hvbeamline}
    \end{center}
    \caption[eP: Beamline HV Screen]{
	    Overview screen for the high voltage status of devices belonging to the 
            beamline instrumentation.
            }
    \label{fig:ep_hvbeamline} 
 \end{figure}  

\begin{figure}[bht]
    \begin{center}
        \includegraphics*[angle=0,width=0.9\textwidth]{ep_hvslot1}
    \end{center}
    \caption[eP: HV Screen for Slot 1]{
	    Control screen for all high voltage channels from Slot 1.
            }
    \label{fig:ep_hvslot1} 
 \end{figure}  

\begin{figure}[bht]
    \begin{center}
        \includegraphics*[angle=0,width=0.9\textwidth]{ep_hvslot2}
    \end{center}
    \caption[eP: HV Screen for Slot 2]{
	    Control screen for all high voltage channels from Slot 2.
            }
    \label{fig:ep_hvslot2} 
 \end{figure}  


 \begin{figure}[bht]
    \begin{center}
        \includegraphics*[angle=0,width=0.9\textwidth]{ep_hvslot3}
    \end{center}
    \caption[eP: HV Screen for Slot 3]{
	    Control screen for all high voltage channels from Slot 3.
            }
    \label{fig:ep_hvslot3} 
 \end{figure}
}

During a measurement, the alarm handler should be running, so that the 
operator will be informed, should one of the detectors trip. \infolevtwo{This can
also be done manually, by watching the beamline screen Fig. \ref{fig:ep_hvbeamline}.
All fields should be green and showing a voltage close to the values given
in Table \ref{tab:ep_hv}.}
If the EPICS screens are not working, there is an alternative way to 
control the HV, by connecting via telnet directly to the LeCroy 1450.
This can be done from nearly any Linux PC in the counting house with the 
command: \mycomp{$>$ telnet hatsv5 2011}.

%\clearpage

\subsection{MEDM Controls}
\label{sec:ep_medm}

\infolevtwo{
 \begin{figure}[bht]
    \begin{center}
        \includegraphics*[angle=0,width=0.3\textwidth]{ep_slow}
    \end{center}
    \caption[eP: Slow Controls Screen]{
	    EPICS main screen for the controls of the various devices in the eP system. 
            }
    \label{fig:ep_slow} 
 \end{figure} }
The target, the silicon micro-strip detectors, and the setting of the 
Cherenkov detector are controlled by an EPICS GUI \infolevtwo{(Fig. \ref{fig:ep_slow})}. 
It can be started from the ``Hall A Main Menu'' \infolevtwo{(Fig. \ref{fig:medm-hlamain})}
running on \mycomp{hacsbc2} by pressing the \mycomp{EP Energy Measure} button.
(see previous chapter, to learn how to start the ``Hall A Main Menu'' in case
it is not already running)
The controls are actually running on a VME computer \mycomp{hallasc6} 
(Bob calls this \mycomp{e-p~2}). It is located in the eP electronics 
racks along the beamline in Hall A \infolevfour{(Fig. \ref{fig:ep_pic_slow_ctrl})}. This computer
sometimes requires rebooting. \infolevtwo{ The computer is reached through 
the portserver \mycomp{hatsv5} at port 12. To reboot:\\
\\
\mycomp{$>$ telnet hatsv5 2012 \\
user: adaq\\
password: ******* \\
\\ }
if you do not see a prompt, press \mycomp{Ctrl C}.\\
\\
\mycomp{-$>$ reboot}\\
\\
wait for it to finish and then load EPICS:\\
\\
\mycomp{-$>$ $<$ epics \\
...\\
-$>$ Ctrl $]$ \\
telnet$>$ q \\
$>$\\ }

\infolevfour{
 \begin{figure}[bht]
    \begin{center}
        \includegraphics*[angle=0,width=0.75\textwidth]{ep_pic_slow_ctrl}
    \end{center}
    \caption[eP: Picture Slow Controls]{
	    VME crate containing modules for the slow controls of the eP system.
            }
    \label{fig:ep_pic_slow_ctrl} 
 \end{figure}  }
}

\infolevtwo{
\subsection{Silicon Micro-Strip Detectors}
\label{sec:ep_ssd}

There are three GUI's associated with the silicon micro-strip detectors. 
Two of them are important for everyday operations. They are labeled 
\mycomp{MicroStrip Polarization} 
and \mycomp{MX7RH Power Supply and Currents}. To operate the SSDs, pull up
the micro-strip polarization display and turn on all the bias voltages (see Fig. \ref{fig:ep_ssd_bias_control}). 
Make sure that the bias voltages are set to a reasonable value (30 Volts).
Pop up both current strip charts so that you can see when the currents 
have stabilized.
Pull up the MX7RH display and turn on all the supply's (see Fig. \ref{fig:ep_mx7_control}). 
Pop up the power supply strip charts. It takes at 
least 30 minutes for the strips to stabilize.

 \begin{figure}[bht]
    \begin{center}
        \includegraphics*[angle=0,width=0.9\textwidth]{ep_ssd_bias_control}
    \end{center}
    \caption[eP: SSD Bias Voltages Screen]{
            EPICS screen to control the bias voltages for the silicon micro-strip detectors.
            }
    \label{fig:ep_ssd_bias_control} 
 \end{figure}

 \begin{figure}[bht]
    \begin{center}
        \includegraphics*[angle=0,width=0.9\textwidth]{ep_mx7_control}
    \end{center}
    \caption[eP: MX7 Controls Screen]{
	    EPICS screen for the MX7 power supplies. 
            }
    \label{fig:ep_mx7_control} 
 \end{figure}  

%\clearpage
}

\subsection{Target}
\label{sec:ep_target}

The target of the eP system is made of a thin polyethylene (CH$_2$) tape, which 
is moving while it is in the electron beam. \infolevtwo{ To operate the target one has to
pull up the target GUI (Fig. \ref{fig:ep_target_control}). There are two controls, one to start the target moving
labeled \mycomp{Motor Control}
and another labeled \mycomp{Target Motion} to place the target in the beam. 
 \begin{figure}[bht]
    \begin{center}
        \includegraphics*[angle=0,width=0.6\textwidth]{ep_target_control}
    \end{center}
    \caption[eP: Target Control Screen]{
	    EPICS screen for the MX7 power supplies. 
            }
    \label{fig:ep_target_control} 
 \end{figure}  }
The CH$_2$ tape  must always be moving before 
it is placed in the beam. There are two monitors of the tape motion:
an output that shows the motor is powered and a diode-pin combination 
that triggers on a reflective strip. The diodes are often damaged.\\
\begin{safetyen}{10}{5}
Always make sure, that the target is moving while it is in the beam !!!\\
\end{safetyen}
The target movement and motion can also be controlled locally.
\infolevfour{The control box is located under the beamline next to the eP system
(see Fig. \ref{fig:ep_pic_trgtctrl}.)}\\
\begin{safetyen}{10}{5}
If you operate the target manually, make sure that the system
is set back to remote control afterwards.\\
\end{safetyen}
The CH$_2$-tape has only a limited life time. Therefore it
should be exchanged on a regular basis (twice per year, or 
before a long beam time). This work has to be done by the 
Hall A technical staff. 
\infolevfour{
 \begin{figure}[bht]
    \begin{center}
        \includegraphics*[angle=0,width=0.9\textwidth]{ep_pic_trgtctrl}
    \end{center}
    \caption[eP: Picture of Target Control Box]{
	    Control box for the eP target system.
            }
    \label{fig:ep_pic_trgtctrl} 
 \end{figure}  
}
%\clearpage

\subsection{Cherenkov}
\label{sec:ep_cer}

The detectors for the protons (the scintillators S1 and S2, and 
a silicon micro-strip detector) are installed at a fixed angle of
60$^o$. Therefore the scattering angle of the electron varies 
between 9$^o$ and 40$^o$ depending on the beam energy.
There are seven mirrors in each arm, covering the full angular range,
but only one photomultiplier tube per arm, which only looks at one 
mirror at a time. Depending on the beam energy the PMT has to be rotated 
to see the corresponding mirror.
\infolevtwo{ This movement is controlled by the Cherenkov GUI (see Fig. \ref{fig:ep_cer_control}). 
To change the setting, pull up the Cherenkov GUI and 
enter the desired energy in MeV into the widget. One arm at
a time will move. After the first PMT is in position you must re-enter an
energy that is 1 or 2 MeV different in order to move the second PMT.
This is a rather slow process, and can take several minutes.

 \begin{figure}[bht]
    \begin{center}
        \includegraphics*[angle=0,width=0.5\textwidth]{ep_cer_control}
    \end{center}
    \caption[eP: Cherenkov Controls Screen]{
	    EPICS control screen for the Cherenkov detector. User input is only
 	    possible for the beam energy. Be aware that only one detector at a time
            is moved.
            }
    \label{fig:ep_cer_control} 
 \end{figure}  }

The Cherenkov detector is filled with pure CO$_2$-gas. \infolevtwo{The schematic of the gas 
system is shown in Fig. \ref{fig:ep_cer_gas_layout}, \infolevfour{ a picture of the gas-controller
in Fig. \ref{fig:ep_cer_gas_ctrl}}.} The gas-controller is located in the same rack as 
the DAQ system. This rack is located in Hall A next to the beamline.
\infolevtwo{ When performing an eP measurement, the gas system
should be in \mycomp{Pressure}-mode. Therefore the left rotary switch should be at
\mycomp{PRESSION} and the right one at \mycomp{FERME}. The two digital displays
should both indicate a pressure of roughly 10.0~mbar, and the two flow-meters should
be at zero. However the flow regulator under the left flow meter needs to be open.
In this mode the system is pressurized, if the pressure falls below 10~mbar
the automated valve on the gas inlet side opens, until the pressure is restored.
On the other hand, if the pressure rises above 15~mbar, the automated valve in the exit pipe
opens, to release pressure.

If the gas Cherenkov detector needs to be opened, one should turn down the gas flow
on the regulator beneath the left flow meter and open the exit valve (right switch, \mycomp{OUVERT}). 
After the work on the detector is finished,
and the volume is closed again, the detector needs to be set in \mycomp{Flow Mode}.
The left rotary switch needs to be in the \mycomp{DEBIT} and the right one in the
\mycomp{OUVERT} position, the gas flow regulator needs to be opened. After the 
detector is purged for a sufficient time, one should switch back to the \mycomp{Pressure}-mode,
and verify that a pressure of 10~mbar is restored. The CO$_2$ is supplied by the Hall A 
gas system, which also supplies the Cherenkov detectors in the HRS with CO$_2$. The cylinders
and the main vallve (operated manually) are located in the gas-shack.

\begin{figure}[bht]
    \begin{center}
        \includegraphics*[angle=0,width=0.8\textwidth]{ep_cer_gas_layout}
    \end{center}
    \caption[eP: Layout of CO2 Gas System]{
	    Scheme of the gas system for the two carbon dioxide gas Cherenkov detectors.
            }
    \label{fig:ep_cer_gas_layout} 
\end{figure}  

\infolevfour{
\begin{figure}[bht]
    \begin{center}
        \includegraphics*[angle=0,width=0.8\textwidth]{ep_cer_gas_ctrl}
    \end{center}
    \caption[eP: Picture of CO2 Gas Controller]{
            Picture of the gas controller of the eP gas Cherenkov detectors.
            }
    \label{fig:ep_cer_gas_ctrl} 
\end{figure} }
}
%\clearpage

\subsection{Data Acquisition}
\label{sec:ep_daq}

The data acquisition (DAQ) is running on \mycomp{adaqep} in the
\mycomp{epmeas} user account. It is a standard CODA 2.2 system.
The DAQ system also downloads and initializes logic modules,
and thresholds of discriminators. Since these settings depend
on the beam energy, they have to be configured individually for 
each measurement.
\infolevfour{The DAQ hardware itself is located in two racks along the beamline 
in Hall A (see Figs. \ref{fig:ep_pic_daq1}, \ref{fig:ep_pic_daq1}, and \ref{fig:ep_pic_daq3} ). 

 \begin{figure}[bht]
    \begin{center}
        \includegraphics*[angle=0,width=0.75\textwidth]{ep_pic_daq1}
    \end{center}
    \caption[eP: DAQ VME Crate]{
	    VME crate for the eP data acquisition.
            }
    \label{fig:ep_pic_daq1} 
 \end{figure}  
 \begin{figure}[bht]
    \begin{center}
        \includegraphics*[angle=0,width=0.75\textwidth]{ep_pic_daq2}
    \end{center}
    \caption[eP: DAQ NIM Bin]{
	    NIM bin for the eP data acquisition.
            }
    \label{fig:ep_pic_daq2} 
 \end{figure}  
 \begin{figure}[bht]
    \begin{center}
        \includegraphics*[angle=0,width=0.75\textwidth]{ep_pic_daq3}
    \end{center}
    \caption[eP: DAQ CAMAC Crate]{
	    CAMAC crate for the eP data acquisition. 
            }
    \label{fig:ep_pic_daq3} 
 \end{figure}  
%\clearpage 
}

\infolevtwo{
\subsubsection{Trigger-configuration \\ }

Before data taking can start, 
a trigger file appropriate for the nominal beam energy must be created. This
file (\mycomp{settings.conf}) insures that the trigger MLU is programmed 
correctly. You have to be logged into \mycomp{adaqep} as user
\mycomp{epmeas}. There you have to change to the correct directory
(use \mycomp{goconf}) and run a short program (\mycomp{trigger}) to 
generate the trigger file. An example is shown in Fig. \ref{fig:ep_trgcnf}.
Make sure that you give the beam energy in MeV.
The file is read in by CODA during the \mycomp{PRESTART}.

 \begin{figure}[bht]
    \begin{center}
        \includegraphics*[angle=0,width=0.65\textwidth]{ep_trgconfig}
    \end{center}
    \caption[eP: Trigger Configuration]{
	    Example for the generation of a trigger configuration file.
            }
    \label{fig:ep_trgcnf} 
 \end{figure}  

\subsubsection{Rebooting Acquisition-VME \\ }

The DAQ system utilizes a VME computer as its Readout Controller (ROC). This
computer is designated \mycomp{hallasc15} and can be
accessed from the portserver \textbf{hatsv5} at port~2. To reboot it, use the following 
procedure:\\
\\
\mycomp{epmeas@adaqep.jlab.org$>$ telnet hatsv5 2002\\
user: adaq \\
password: ******** \\
}
\\
if you do not see a prompt, press: \mycomp{Ctrl C}\\
\\
\mycomp{-$>$ reboot\\
-$>$ Ctrl $]$ \\
telnet$>$ q \\
epmeas@adaqep.jlab.org$>$\\} 
\\
If the reboot fails, or if CODA afterwards still does not work, 
check that the ROC is configured for CODA 2.2.
Therefore one has to interrupt the reboot by pressing the \mycomp{any}-key.
Press \mycomp{p} to show the present setting, it should look the following
way:\\
\\
\mycomp{boot device          : ei \\
processor number     : 0 \\
host name            : adaqs3-ep.jlab.org \\
file name            : /home/epmeas/vxworks/vx162lc-8MB \\
inet on ethernet (e) : 129.57.188.14:ffffff00 \\
inet on backplane (b): \\
host inet (h)        : 129.57.164.45 \\
gateway inet (g)     : 129.57.188.1 \\
user (u)             : epmeas \\
ftp password (pw) (blank = use rsh): \\
flags (f)            : 0x20 \\
target name (tn)     : hallasc15 \\
startup script (s)   : /home/epmeas/vxworks/epmeas\_22.boot \\
other (o)            : \\
}
\\
Press \mycomp{c} to change these settings and 
reboot the ROC by pressing \mycomp{@} afterwards.

\subsubsection{Running CODA \\ }

To run CODA, you have to be logged into \mycomp{adaqep} as user
\mycomp{epmeas}. From the prompt CODA can be started with the
command \mycomp{runcontrol}. Withing CODA you have to click 
on \mycomp{Configure} and choose configuration \mycomp{epm1},
then click on \mycomp{Download}, and finally on \mycomp{Prestart}.
At this point the information in the settings.conf file,
 that controls the acquisition
(thresholds, discriminator widths, and trigger MLU logic) is downloaded to the
hardware and spooled to the diagnostics window. This provides an opportunity
to check this information.

The actual data taking starts after pressing \mycomp{Go}. The rate 
is usually rather low, below one per second. However if after 
a few minutes the number of events is not increasing, one has to 
verify if:
\begin{itemize}
\item the trigger is programmed correctly,
\item all components of the DAQ are running,
\item the Cherenkov is at the correct position,
\item the target is in the beam and moving.
\end{itemize}
After collecting enough data, the \mycomp{End} button should be used
to end data-taking, and to ensure that all data is written into the 
datafile.

\subsection{Data Analysis}
\label{sec:ep_analysis}

The data analysis is currently done in two steps, using
two different programs. Both run on \mycomp{adaqep} in the
\mycomp{epmeas} account.

In the first step, the CODA raw file is converted into an
ASCII file.
For this part of the analysis one has to change to the \mycomp{epcoda}
directory, which can be done by typing \mycomp{goep}, and start
the program \mycomp{eplong}:\\
\\
\mycomp{epmeas@adaqep.jlab.org$>$ goep \\
epmeas@adaqep.jlab.org$>$ eplong \\
~How many events (-1= lots) ? \\
-1 \\
~What file name ? \\
epmeas02\_\#\#\#.dat \\
~What output filename ? \\
\#\#\# \\
~opening/adaqep/data1/epraw/epmeas02\_\#\#\#.dat    \\
\\
~Have opened  epmeas02\_\#\#\#.dat \\
\\
~bank length is wrong \\
~bank length is wrong \\
~Finished;  events read =   234 \\
epmeas@adaqep.jlab.org$>$  \\
}
\\
In this example \#\#\# is the three-digit CODA run number. \mycomp{eplong} can be started,
while CODA is still taking data for that run.

The second step of the analysis utilizes a stand-alone analysis code,
which asks for nominal beam energy, beam position, beam intensity
and duration and uses the output of \mycomp{eplong}. One has to
change into the \mycomp{ep} directory and start the code:\\
\\
\mycomp{epmeas@adaqep.jlab.org$>$ cd \\
epmeas@adaqep.jlab.org$>$ cd ep \\
epmeas@adaqep.jlab.org$>$ ep \\
}
\\
Make sure, that the nominal beam energy is given in \textbf{GeV}.
The program prints the result for the energy, together with
the path and name for log-files and ntuple files.
It is recommended to repeat the analysis with a slightly changed
nominal energy value or with slightly changed cuts, to verify that
the automatic fitting procedure does really find the eP events,
and does not trigger on noise. One also has to be aware, that one
needs elastic events in both arms to get a reliable results.
Furthermore, for beam energies between 2.7~GeV and 3.4~GeV,
where micro-strip detector E$_3$ is used, the obtained
values are systematically shifted as compared to the results from the ARC energy measurements, 
probably due to a misalignment of this detector.
}}

\infolevtwo{
\section{Operating Procedure}
\label{sec:ep_ops_proc}

In preparation of an eP measurement, the mirrors of the Cherenkov 
should be driven to the appropriate position (see Sec. \ref{sec:ep_cer}), 
and the silicon micro-strip detectors should be turned on (see Sec. \ref{sec:ep_ssd}).
These two measures should be started several hours before the actual 
eP measurement is scheduled.

Shortly before the measurement, the high voltages for the scintillator photomultiplier tubes
and for the Cherenkov photomultiplier tubes need to be turned on (see Sec. 
\ref{sec:ep_highvoltage}). Finally the DAQ should be 
prepared (see Sec. \ref{sec:ep_daq}).

For the eP measurement, the following requirements need to 
be communicated to MCC:
\begin{itemize}
\item 3-4 $\mu$A CW beam 
\item Raster OFF
\item OTR target 1C12 OUT
\item Physics target empty ( or be able to stand unrastered, uncentered beam )
\item Centered on BPM 1H01 absolute
\item Fast Feedback must be ON
\end{itemize} 
To check the beam position (recommended!), you can use the 
\mycomp{Monticello} screen from MCC, which is usually also available
on one monitor in the Hall A counting house. On the 
\mycomp{Monticello} main menu
select \mycomp{BPM}, and there click on
\mycomp{BPM Spikes and Position Summary}.
This will pop up a new screen, go to the top row of this screen 
(\mycomp{``Injector, BSY, Hall A, B and C Transport''}) 
and select \mycomp{Pos Sum}.  From here select \mycomp{Hall A Transport}.
A screen will show up, which summarizes beam positions at various 
locations. For the eP system the numbers in \mycomp{BPM 1H01 absolute}
are the only ones relevant.

When MCC has established those conditions, the high voltages and 
the micro-strip detectors should be checked one more time.
Next the eP target tape motion should be turned on 
(\mycomp{Motor Control}) and then the 
target can be moved into the beam (\mycomp{Target Motion}, 
see Sec.\ref{sec:ep_target}.)

Now the actual data-taking can start, by pressing \mycomp{Prestart/Go}
in the CODA runcontrol screen. The rate should be a few 
tenth of a Hz. If the BPM position changes, the fast feedback system fails, or a 
lot of beamtrips accrue, consider stopping the run and starting a new 
one.

One should analyze the data, while CODA is still running. 
With a hundred events one can already check the quality of the 
data, and estimate how much more statistics are needed.
Typically one needs 40-50 minutes of stable beam or a few 
hundred events.

After data taking is finished, and it is verified, that there
is a sufficient number of events to extract a number for the beam energy, 
the following steps should be taken:
\begin{itemize}
\item eP target: should be moved out of the beam
\item eP target: motor should be turned off (after it is moved out)
\item MCC can restore the beam needed for the experiment: 
\begin{itemize}
\item restore beam position at target
\item restore raster
\item insert OTR 1C12, if needed for the experiment
\item restore beam current
\end{itemize}
\item Shift workers can go back to physics target
\item high voltages for eP scintillators and eP Cherenkov should be turned off
\item MX7 power supplies and micro-strip bias voltages should be turned off
\item CODA windows should be closed
\item remaining windows from the \mycomp{epmeas} account should be closed
\end{itemize}
Before posting the result of the eP measurement, one should make sure,
that the full statistics of the run is analyzed, that the result is 
independent of the chosen cuts, and that there are events on both 
arms of the eP system.

\section{Maintenance}
\label{sec:ep_maintenance}

The CH$_2$ tape of the eP target 
should be exchanged on a regular basis (twice per year, or 
before a long beam time). This work involves opening the 
eP scattering chamber and therefore breaking the vacuum in
this section of the beamline. This work has to be coordinated
by the Hall A work coordinator, and can only be done by the 
Hall A technical staff personnel. 
}

\begin{safetyen}{0}{0}
\section {Safety Assessment}
\label{sec:ep_safety}

\subsection{High Voltage}

The LeCroy 1450~HV~crate equipped with LeCroy~1461N
high voltage cards provides up to 3~kV of low current power.
RG-59/U~HV~cables, certified for up to 5~kV, with standard SHV 
connectors are used to connect the power supply to the photomultipliers.
The PMTs for S1,S2 and for the Cherenkov detector are usually operated 
at 1900~-~2300~V and draw up to 1.5~mA currents. The PMTs for the 
S3 scintillators are operated at 1000~-~1150~V, drawing 0.9~mA current.
The high voltage MUST be turned off during all work on the detector.

\subsection{Silicon Micro-Strip Detectors}

The SSD are prone to radiation damage, regardless if they are 
turned on or off. Ion chambers next to the eP measure radiation levels
in this part of the beamline and interrupt beam delivery via the
fast shutdown system (FSD), in case the levels are not acceptable. Therefore these
ion chambers should never be masked. 

\subsection{Target}

The target is controlled by the experimenters, not by MCC. Therefore it is
the responsibility of the eP operator to ensure that it is properly
operated.
To avoid damage to the eP target, the following instructions have to be 
followed:
\begin{itemize}
\item The target should only be in the beam during an eP measurement
\item Before inserting the target into the beam, the tape motion has to
be turned on. The target should not be in the beam when the tape is not moving. 
\item The target should not be in the beam 
if the beam current is greater than 5 $\mu$A. 
\item After finishing the eP measurement, the target should be moved out
of the beam, and then the tape motion stopped. 
\item The tape should not run, and the target should not be in the beam
without an eP operator being present.
\end{itemize}

\subsection{Cherenkov}

If for work on the Cherenkov detector the detector needs to be opened, 
the CO$_2$ gas flow needs to be stopped. After the work is finished the 
detector needs to be purged and later the operating mode needs to be restored.
\infolevone{(see Sec. \ref{sec:ep_cer})}

\end{safetyen}

% include the personnel list
\begin{safetyen}{0}{0}
\section[Authorized  Personnel]{Authorized  Personnel
\footnote{
   $CVS~revision~ $Id: ep-personnel.tex,v 1.5 2008/05/09 23:08:11 doug Exp $ $
 }
\footnote{Authors: D.~W.~Higinbotham \email{doug@jlab.org}}
}
\end{safetyen}
%
% Authorized Personnel 
%
The list
of the presently authorized personnel is given in Table~\ref{tab:ep:personnel}.
Individuals must notify and receive permission from
the Hall A work coordinator (see Table~\ref{tab:ep:personnel}) before working any beamline part 
of the system.  
\begin{namestab}{tab:ep:personnel}{eP System: authorized personnel}{%
   eP System: authorized personnel}
 \DouglasHiginbotham{\it Contact}
 \EdFolts{Work Coordinator}
 \JackSegal{Gas System}
 \ToddEwing{CH2 Targets}
 \ScotSpiegel{CH2 Targets}
\end{namestab}



% ===========  CVS info
% $Header: /group/halla/analysis/cvs/tex/osp/src/beamline/ep.tex,v 1.6 2003/12/13 06:23:37 gen Exp $
% $Id: ep.tex,v 1.6 2003/12/13 06:23:37 gen Exp $
% $Author: gen $
% $Date: 2003/12/13 06:23:37 $
% $Name:  $
% $Locker:  $
% $Log: ep.tex,v $
% Revision 1.6  2003/12/13 06:23:37  gen
% Septum added. Name tables. Polishing
%
% Revision 1.5  2003/12/08 15:34:23  reitz
% change paragraph to subsubsection
%
% Revision 1.4  2003/12/05 05:48:30  gen
% Polishing
%
% Revision 1.3  2003/11/18 22:34:16  reitz
% personnel added
%
% Revision 1.2  2003/11/17 06:50:16  gen
% CVS records added
%
