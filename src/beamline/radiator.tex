\section[Bremsstrahlung Radiator]{Bremsstrahlung Radiator
\footnote{
  $CVS~revision~ $Id: radiator.tex,v 1.3 2003/06/06 15:19:03 gen Exp $ $
}
\footnote{Authors: A.Saha \url{mailto:saha@jlab.org}}
}

\subsection{Overview}

The Bremsstrahlung radiator is the last element in the Hall A beam line 
before the scattering chamber, and is about 72.6 cm from the center 
of the physics targets.
Its design is based on the Hall C radiator 
system built by David Meekins, and documented in the Hall C operations
manual.

The central component of the system is a U-shaped, oxygen-free copper target
ladder, with six positions for differing thicknesses of oxygen-free Cu foils.
The ladder is designed so that it never intersects the beam.
The 3.175-cm wide gap in the ladder is spanned only by 
the target foils, which are 6.35 cm wide, 3.175 cm high,
and 3.332 cm apart (center to center).
A stepper motor moves the target ladder with foils up and down,
into and out of the beam.
Hard stops prevent motion of the ladder beyond the limit switches.
Water cooling of the radiator ladder cools the foils, preventing
damage from overheating by the beam.

The interaction of the beam with the foils produces
background radiation in the Hall.
At 3 GeV, ion chamber trip levels do not need to be
adjusted, and increases in detector background rates are minimal;
further tests are planned for 0.8 GeV.
No local shielding is installed, as calculations indicate
that this will not significantly affect dose at the site boundary.
Any installation and/or subsequent modifications must be coordinated
with RadCon.

\subsection{Safety Issues}

The only safety issue concerning the Bremsstrahlung radiator is that of
induced radioactivity in the Cu targets and in the water 
used for cooling the targets.
The water cooling system is a closed loop,
using a portable welding-torch water cooler, located
under the beam line just upstream of the target. 
The cooler is kept in a tray which is intended to provide secondary
containment in case of a leak.
The cooling system must not be breached or drained without concurrence from 
the RCG. Accidental breach or spill constitutes a radiation contamination hazard.  
A spill control kit, capable of containing a system leak or spill, is
staged by the door to the hall. In the event of a spill notify
the RCG.

\subsection{Operations}

Although the radiator foils are water cooled, a high current electron beam
may melt the foils.
Beam currents with the radiator will be limited to 30 micro-amperes.
Including a safety factor, the raster radii given in Table 1
will limit the temperature rise to 100 $^{\circ}$C.

\begin{table}
\begin{center}
\caption[Bremsstrahlung Radiator: Raster Radius]{Raster radius as a function of beam current.}
\begin{tabular}{cc}
\hline
\hline
Current ($\mu$A) & Minimum raster radius (mm) \\
\hline \\
10 & (not needed) \\
15 & 0.2 \\
20 & 0.7 \\
25 & 1.3 \\
30 & 2.1 \\
\hline \hline \\
\end{tabular}
\end{center}
\end{table}

The only operational control consists of moving the ladder in and out.
Radiator position is determined by the ratio of the readback voltage from 
a linear encoder to the voltage applied to it.
Table 2 gives the radiator position as a function of this ratio.
The foil thicknesses are set so that the thickness, in percent
of a radiation length, equals the foil number, except that
no foils are mounted in position 1.

\begin{table}
\begin{center}
\caption[Bremsstrahlung Radiator: Encoder Calibration]{Encoder voltage calibration. See text.}
\begin{tabular}{ccc}
\hline
\hline
Position & Voltage ratio & $V_{\rm encoder}$ for $V_{\rm supply}$ = 5 V \\
\hline \\
out limit & 0.030 & 0.15 \\
 foil 1 & 0.102 & 0.511 \\
 foil 2 & 0.269 & 1.346 \\
 foil 3 & 0.436 & 2.179 \\
 foil 4 & 0.603 & 3.013 \\
 foil 5 & 0.769 & 3.847 \\
 foil 6 & 0.936 & 4.681 \\
 in limit & 0.966 & 4.831 \\
\hline \hline \\
\end{tabular}
\end{center}
\end{table}

Software controls of the ladder position are under development;
radiator position is changed by calling
MCC and requesting that the radiator be set to some foil position,
or to the out limit.
The position may be changed with beam on.
A manual-control backup system also exists.

Both software and manual backup systems
control an Oregon Micro Systems MH10DX step motor driver,
which drives a Slo-Syn M063-LS09 stepper motor.
The MH10 driver, power supplies, and other control circuitry, 
are in a custom-built box located in the hall in rack 1H75B10.
The linear encoder voltage ADC is in slot 5 of the CAMAC crate in 
rack 1H75B02; radiator inputs use channels 15 and 16, and
are connected through a patch panel to block 30 in rack 1H75B08.

When the radiator is not being used, the system should be set to the
out-limit position, so that it is clear of the beam.
Power to the control box in the hall may be turned off with a front-panel 
switch if the radiator will not be used for a long time - and should be 
turned off if work is to be done on the radiator.
This deactivates the limit switches and the linear encoder, 
but does not affect positioning.
Additional hard stops should be installed as a safety measure.
The Hall A technical staff checklist, done as part of preparations for
closing the Hall for beam, includes checking the radiator
position, the status of the control box, and the installation of hard stops.

\subsection{Special Instructions}

Care must be taken in case any removal or disassembly of the radiator
system is needed.
Disconnecting the stepper motor from the motor driver
while power is on can damage the motor, motor driver, and VME44 board.

The Cu targets will certainly be activated in the course of an experiment.
Therefore, only remove the Cu target, the target ladder,
and/or the whole radiator system in the presence of a Radcon officer.

Ron Gilman should be informed in case of any problems with the radiator.
Except for normal operations of the radiator, any work on the
system hardware requires that RadCon has concurred in the work and
either Ron Gilman or David Meekins is present.


% ===========  CVS info
% $Header: /group/halla/analysis/cvs/tex/osp/src/beamline/radiator.tex,v 1.3 2003/06/06 15:19:03 gen Exp $
% $Id: radiator.tex,v 1.3 2003/06/06 15:19:03 gen Exp $
% $Author: gen $
% $Date: 2003/06/06 15:19:03 $
% $Name:  $
% $Locker:  $
% $Log: radiator.tex,v $
% Revision 1.3  2003/06/06 15:19:03  gen
% Revision printout changed
%
% Revision 1.2  2003/06/05 23:29:59  gen
% Revision ID is printed in TeX
%
% Revision 1.1.1.1  2003/06/05 17:28:32  gen
% Imported from /home/gen/tex/OSP
%
