\section[Arc Energy Measurement]{Arc Energy Measurement
\footnote{
  $CVS~revision~ $Id: arc.tex,v 1.4 2003/11/17 21:11:03 saha Exp $ $
}
\footnote{Authors: A.Saha \url{mailto:saha@jlab.org}}
}

%%%%%%%%%%%%%%%%%%%%%%%%%%%%%% LyX specific LaTeX commands.
\providecommand{\LyX}{L\kern-.1667em\lower.25em\hbox{Y}\kern-.125emX\@}

%%%%%%%%%%%%%%%%%%%%%%%%%%%%%% Textclass specific LaTeX commands.
\newenvironment{lyxlist}[1]
  {\begin{list}{}
    {\settowidth{\labelwidth}{#1}
     \setlength{\leftmargin}{\labelwidth}
     \addtolength{\leftmargin}{\labelsep}
     \renewcommand{\makelabel}[1]{##1 \hfill}}}
  {\end{list}}

\newenvironment{lyxcode}
  {\begin{list}{}{
    \setlength{\rightmargin}{\leftmargin}
    \raggedright
    \setlength{\itemsep}{0pt}
    \setlength{\parsep}{0pt}
    \ttfamily}%
   \item[]}
  {\end{list}}

The ARC energy measurement is under EPICS control through a MEDM display. Two
independent control systems are used: the beam bend angle measurement through
the arc (\char`\"{}scanners\char`\"{}) and the field integral of
the arc (\char`\"{}integral\char`\"{}). To measure the energy: 

\begin{itemize}
\item perform several angle measurements 
\item perform an integral measurement 
\item analyze the integral measurement and note the value of the arc field 
integral 
\item analyze the angle measurements, average the results (proposed by the 
software),
then ask for the energy calculation, enter the above arc field integral and
you will get the beam energy computed from the average angle. 
\end{itemize}

\vspace{0.3cm}

\subsection{Summary of ARC operations: }

Six scanners of the same type, called \char`\"{}ARC scanner\char`\"{} and labelled
from scanner \#1 to \#6, are installed on the Hall-A beamline. Scanners \#1
to \#4 are used for the ARC energy measurement and they are located on the Hall-A
arc: \#1 [1HA1C07A] and \#2 [1HA1C07B] just upstream of the arc, in the BSY, and 
\#3 
[1HA1C18A] and \#4 [1HA1C18B] in the Hall-A
tunnel, just upstream the Compton polarimeter. Scanners \#5 [1HA1H03A] and \#6 
[1HA1H03B] 
are located
between the Moller and the target to control the beam geometry on the target
and their use will not be discussed here. 

Procedure for running a harp scan is described in:

\href{http://hallaweb.jlab.org/equipment/beam/harp_halla/harp.html}{scanrun}.


Each scanner has a motor/ball-screw/shaft-encoder/vacuum-penetrator system moving
accurately a set of 3 tungsten wires through the beam. Each time a wire crosses
the beam a PMT located a few meters downstream records a signal due to the 
electromagnetic
shower induced by the beam in the wire. Both forward and backward passes are
recorded. The motion is a horizontal translation and, for a forward pass: 

-the translation is from beam left to beam right, 

-the two first wire crossing the beam are at 45deg from the vertical, 

-the third wire, which is the only important for the ARC energy measurement,
is vertical. 

Recording, during the scan, the scanner position and the PMT output voltage
allows us to determine the beam position at each scanner location. Then, using
calibration data not detailed here, we deduce the net beam bend angle through
the arc. This result measured in dispersive arc tuning, along with the field
integral of the arc dipoles, provides an accurate determination of the beam
energy. 

\vspace{0.3cm}

\subsection{Summary of field integral }

The purpose is to measure absolutely the straight field integral of a 
\char`\"{}BA\char`\"{}
3m long dipole, called the \char`\"{}9th dipole\char`\"{} and located in the
\char`\"{}Dipole Shed\char`\"{}. It is of the same type as the 8 arc dipoles
and is powered in series with them. 

The ARC integral setup is basically made of a 3m long plate (the 
\char`\"{}probe\char`\"{})
which is able to move inside the 9th dipole gap along the beam axis and carrying 
two
field measurement devices: a pair of pick-up coils connected in series and a
set of NMR probes. The coils are on both ends of the probe and the NMRs close
to the center. 

-at the \char`\"{}upstream\char`\"{} probe position, the 
\char`\"{}downstream\char`\"{}
coil is close to the dipole center, the \char`\"{}upstream\char`\"{} is outside
the dipole and the NMRs at one end of the dipole: 

Door$<-$-- ....................$<-$-------DIPOLE-----$--->$ 

.............$<-$-------PROBE------$--->$ 

-at the \char`\"{}central\char`\"{} probe position, each coil is at one end
of the 3m long dipole and the NMRs close to the dipole center: 

Door$<-$-- ...................$<-$-------DIPOLE-----$--->$ 

..................................$<-$-------PROBE------$--->$ 

-at the \char`\"{}downstream\char`\"{} probe position, the 
\char`\"{}upstream\char`\"{}
coil is close to the dipole center, the \char`\"{}downstream\char`\"{} is outside
the dipole and the NMRs at one end of the dipole: 

Door$<-$-- ...................$<-$-------DIPOLE-----$--->$ 

....................................................$<-$-------PROBE------$--->$ 

We call upstream the position where the probe is the closest to the shed access
door. Among the 3 above positions, the only one where the NMR can lock on the 
dipole
field is the central one as in the extreme position of the probe, the field 
homogeneity
is not sufficient. The probe position is controlled by a linear encoder. The
Z axis refers to the \char`\"{}beam\char`\"{} direction, increasing from upstream
to downstream. We use three kinds of \char`\"{}Z\char`\"{}: 

-Zm to locate a point inside the magnet. The dipole center is at Zm=0 and the
yoke ends at +-1500.mm 

-Zp to locate a point inside the probe. The probe center is at Zp=0. Each of
the 4 NMR probes has a Zp given in the file \char`\"{}magnet.dir\char`\"{}.
At a temperature of 21C, the coils are at Zp=+-1519.815mm (from magnet.dir) 

-Zd to refer to a displacement of the probe w.r.t. the dipole. Zd=0 refers to
the upstream (home) position of the probe. The integral measurement is performed
from Zd=0.000mm (1st PDI trigger) to Zd=3199.000mm (last PDI trigger), for forward
pass. Zd is given by the display (at the top of the rack) or by the master screen
(\char`\"{}OUT\char`\"{}). 

The relationship between Zm, Zp and Zd is: 

Zd-Zm+Zp=C 

where C is a constant given in magnet.dir (C=1604.000 nomin.). Example of use:
to have the probe center at the dipole center, one must set Zd=1604.000mm (set
Zm=0 and Zp=0 in the above formula, and solve for Zd) 

The integral measurement sequence is the following: 

-from the current position (a priori arbitrary) move the probe upstream, up
to a limit (optic) switch. 

-move downstream by a few mm to cross the encoder index (encoder initialization) 

-move to the central position to measure the central field by NMR, the system
checks if the NMR locks and if the reading is stable, it will be the 
\char`\"{}before\char`\"{}
field 

-move back to upstream position 

-move to downstream position while integrating the flux through the coil system,
this measurement will be called the \char`\"{}forward\char`\"{} integral (duration
\( \sim  \) 7s) 

-move back to upstream position while integrating the flux through the coil
system, this measurement will be called the \char`\"{}backward\char`\"{} integral
(duration \( \sim  \)7s) 

-move to the central position to measure the central field by NMR, the system
checks if the NMR locks and if the reading is stable, it will be the 
\char`\"{}after\char`\"{}
field. 

In addition to the central field, 4 probe temperatures, a local excitation current
measurement, the setting of the dipoles P.S, the readback of the dipoles P.S
and the probe position at NMR measurement time are recorded 
\char`\"{}before\char`\"{}
and \char`\"{}after\char`\"{}. 

To perform an integral field measurement: 

1-check if the system works (see \char`\"{}details on integral system 
check\char`\"{}
below) 

2-run the above integral sequence (see \char`\"{}details on integral run\char`\"{}
below) 

3-fix the error(s) if any (see \char`\"{}details on integral errors\char`\"{}
below) 

4-save the data in a file (see \char`\"{}details on integral data save\char`\"{}
below) 

5-analyze the data (see Arun Saha). 


\subsection{Details on integral run }

To run the integral measurement sequence, call the 
\char`\"{}arc\_integral.adl\char`\"{}
medm screen, then: 

-push \char`\"{}start\char`\"{} to start the full sequence 

-look at the results displayed: 

-after the \char`\"{}before\char`\"{} NMR measurement: the 
\char`\"{}before\char`\"{}
data set 

-after the \char`\"{}forward\char`\"{} integral pass: the forward velocity profile
and the forward voltage-after-gain profile 

-after the \char`\"{}backward\char`\"{} integral pass: the backward velocity
profile and the backward voltage-after-gain profile 

-after the \char`\"{}after\char`\"{} NMR measurement: the 
\char`\"{}after\char`\"{}
data set 

-if \char`\"{}BAD NMR\char`\"{} or \char`\"{}PDI saturation\char`\"{} flags
are set, or if something is obviously wrong in the data or plots, call expert. 

-data are ready to be saved (see \char`\"{}Details on integral data save\char`\"{}
below) 


\subsection{Details on temperatures }

The AC system of the shed is made of two cooling units, a heating unit and a
controller connected to two temperature sensors : one located in the shed and
one located in the BSY. This system is programmed in such a way that the 
temperature
of the shed follows the BSY temperature within +-2C. The BSY temperature can
be anywhere in the {[}18C,35C{]} range, regardless of the season. The BSY 
temperature
and the shed temperature are given (in F) by a display panel located close to
the workstation, on the wall. The AC system can be set in manual control by
turning from \char`\"{}auto\char`\"{} to \char`\"{}manual\char`\"{} a set of
switches controlling the cooling units and the heater unit. These switch boxes
are located on the shed wall. If the shed temperature is above 34.4C (94F),
call Arun Saha (the electronics can be damaged) and cool down the shed in manual
AC mode. The 4 temperature sensors of the probe are labelled Tx+z+, Tx+z-, Tx-z+,
Tx-z- depending on their position w.r.t. the following (x,z) frame: 

\begin{lyxcode}
~~~~~~~~~~~~~~~~~~~~~~~~~~~~~~~~~~~~~~x~

~~~~~~~~~~~~~~~~~~~~~~~~~~~~~~~~~~~~~~\^{}~

~~~~~~~~~~~~~~~~~~~~~~~~~~~~~~~~~~~~~~|~

~~~~~~~~~~~~~~~~~~~~~~~~~~~~~~~~~~~~~~|~

Door$< -${}-{}-~~~~~~~~~~~$< 
-${}-{}-{}-{}-{}-{}-{}-{}-{}-{}-{}-{}-D~I~P|O~L~E~-{}-{}-{}-{}-{}-{}-{}-{}-{}-{}$-
>$~

~~~~~~~~~~~~~~~~~~~~~~~~~~~~~~~~~~~~~~|~

~~~~~~~~~~~~~~~~~~~$< 
-${}-{}-{}-{}-Tx+z-~-{}-{}-{}-{}-{}-{}-|-{}-{}-{}-{}-{}-Tx+z+-{}-{}-{}-{}-{}$->$~

~~~~~~~~~~~~~~~~~~~|~~~~~~~~~~~~~~~~~~|~~~~~~~~~~~~~~~~~|~

~~~~~~~~~~~~~~~~~~~|~~~~~~~~~~~~~~~~~~|-{}-{}-{}-{}-{}-{}-{}-{}-{}$->$z~~~~~|~

~~~~~~~~~~~~~~~~~~~|~~~~~~~~~~~~~~~~~~~~~~~~~~~~~~~~~~~~|~

~~~~~~~~~~~~~~~~~~~|~~~~~~~~~~~~~~P~R~O~B~E~~~~~~~~~~~~~|~

~~~~~~~~~~~~~~~~~~~$< 
-${}-{}-{}-{}-Tx-z-~-{}-{}-{}-{}-{}-{}-{}-{}-{}-{}-{}-{}-{}-Tx-z+-{}-{}-{}-{}-{}$-
>$
\end{lyxcode}

Both \char`\"{}x+\char`\"{} sensors are on the probe edge which is inside the
dipole gap and both \char`\"{}x-\char`\"{} sensors on the opposite edge which
is outside the dipole gap. Both \char`\"{}z-\char`\"{} sensors are at 1/4 of
the long dimension of the probe and both z+ at 3/4 of this length. The average
of the 4 temperatures is used by the analysis program to correct the coil distance
from the thermal expansion of the probe, so it is important to make sure that
the 4 sensors are working well. The user can just make sure that the temperatures
displayed in \char`\"{}arc-master.adl\char`\"{} or recorded in 
\char`\"{}arc-integral.adl\char`\"{}
are realistic. In \char`\"{}arc-integral.adl\char`\"{} they are given in the
order: Tx+z-, Tx+z+, Tx-z-, Tx-z+ Tx-z- and Tx-z+ should be close to the shed
temperature. Tx+z- and Tx+z+ depend on the probe position, as the gap (iron
yoke) is warmer than the shed and the dipole coil (at both ends of the dipole)
is warmer than the iron yoke. For a probe in a central position for more than
about one hour, the Tx+z- and Tx+z+ sensors should give the yoke temperature,
i.e the shed temperature plus 0. to 5.C, depending on the current, LCW temperature
and the magnet/shed temperature history. The 4 temperatures are also displayed
inside the shed, on the electronics rack. These values are digitized by separate
ADCs, so they may differ from the remote values by \( \sim  \)0.1C. 

\subsection{Shed access and safety }

For safety reasons, the access to the shed is limited to authorized
persons which are listed in the ESAD and listed below. To be added to the list, 
ask the 
Hall-A leader. 
The standard
operation mode of the integral measurement setup is the remote mode, through
the network, from the counting house. In case of problem needing an access in
the shed, unauthorized users must contact Arun Saha.

\subsection{List of Authorized Personnel for Shed Access}


Pascal Vernin

Jacques Marroncle

Christian Veyssiere

Francois Gougnaud

Arunava Saha

Mike Tiefenback

Yves Roblin

Rick Gonzales

Bill Merz

Mark Augustine

Hari Areti

Pete Francis

Douglas Higinbotham

Scott Higgins

David Seidman

Ron Lauze

Tony Day


% ===========  CVS info
% $Header: /group/halla/analysis/cvs/tex/osp/src/beamline/arc.tex,v 1.4 2003/11/17 21:11:03 saha Exp $
% $Id: arc.tex,v 1.4 2003/11/17 21:11:03 saha Exp $
% $Author: saha $
% $Date: 2003/11/17 21:11:03 $
% $Name:  $
% $Locker:  $
% $Log: arc.tex,v $
% Revision 1.4  2003/11/17 21:11:03  saha
% update
%
% Revision 1.3  2003/06/06 15:19:02  gen
% Revision printout changed
%
% Revision 1.2  2003/06/05 23:29:59  gen
% Revision ID is printed in TeX
%
% Revision 1.1.1.1  2003/06/05 17:28:32  gen
% Imported from /home/gen/tex/OSP
%
%  Revision parameters to appear on the output
