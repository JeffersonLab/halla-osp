
%%%%%%%%%%%%%%%%%%%%%%%%%%%%%% LyX specific LaTeX commands.
\providecommand{\LyX}{L\kern-.1667em\lower.25em\hbox{Y}\kern-.125emX\@}

%%%%%%%%%%%%%%%%%%%%%%%%%%%%%% Textclass specific LaTeX commands.
\newenvironment{lyxlist}[1]
  {\begin{list}{}
    {\settowidth{\labelwidth}{#1}
     \setlength{\leftmargin}{\labelwidth}
     \addtolength{\leftmargin}{\labelsep}
     \renewcommand{\makelabel}[1]{##1 \hfill}}}
  {\end{list}}

\newenvironment{lyxcode}
  {\begin{list}{}{
    \setlength{\rightmargin}{\leftmargin}
    \raggedright
    \setlength{\itemsep}{0pt}
    \setlength{\parsep}{0pt}
    \ttfamily}%
   \item[]}
  {\end{list}}

The ARC energy measurement is under EPICS control through a MEDM display. Two
independent control systems are used: the beam bend angle measurement through
the arc (\char`\"{}scanners\char`\"{}) and the field integral of
the arc (\char`\"{}integral\char`\"{}). To measure the energy: 

\begin{itemize}
\item perform several angle measurements 
\item perform an integral measurement 
\item analyze the integral measurement and note the value of the arc field integral 
\item analyze the angle measurements, average the results (proposed by the software),
then ask for the energy calculation, enter the above arc field integral and
you will get the beam energy computed from the average angle. 
\end{itemize}

\subsection{Computers and softwares used by scanners and integral: }

\vspace{0.3cm}
{\centering \begin{tabular}{|c|c|c|}
\hline 
&
SCANNERS&
INTEGRAL\\
\hline 
\hline 
Computer:&
hac&
pascal1(IP=129.57.188.20)\\
\hline 
type:&
hp&
sun\\
\hline 
login under:&
gougnaud&
gougnaud\\
\hline 
password:&
see Arun&
see Arun\\
\hline 
Disk mounted on CUE?&
yes (use cp for transfer)&
no (use ftp or rcp)\\
\hline 
.adl files directory:&
\( \sim  \)/arc2/adl/&
\( \sim  \)/EPICS/adl/\\
\hline 
standard .adl file:&
ARC6.adl&
arc\_integral.adl\\
& & (arc\_master.adl)\\
\hline 
expert .adl file:&
expert.adl&
arc\_nmr.adl \\
& & (arc\_pdi.adl...)\\
\hline 
Directive files dir.:&
\( \sim  \)/arc2/ARC/&
\( \sim  \)/EPICS/ioc.mg/\\
\hline 
standard directive file:&
data1234 (data56)&
magnet.dir\\
\hline 
output file:&
\( \sim  \)/arc2/ARC/scan\_nnn.data&
\( \sim  \)/EPICS/integral/\\
& & integral\_nnn.data\\
\hline 
analysis code:&
\( \sim  \)/arc2/ARC/pascal/arc-scan.sun&
\( \sim  \)/pascal/arc-integral\\
\hline 
result file:&
\( \sim  \)/arc2/ARC/scan\_nnn.data.log&
\\
\hline 
ioc name:&
arcioc (IP=129.57.188.24)&
arcioc2  (IP=129.57.188.21)\\
\hline 
ioc login name:&
None&
target\\
\hline 
ioc password:&
None&
password\\
\hline 
ioc \char`\"{}iam\char`\"{}:&
gougnaud&
gougnaud\\
\hline 
ioc type and RAM size:&
MVME162, 8Mo&
MVME162, 8Mo\\
\hline 
ioc boot file:&
\( \sim  \)/arc2/ioc/up&
\( \sim  \)/EPICS/ioc.mg/up\\
\hline 
\end{tabular}\par}
\vspace{0.3cm}


\subsection{Summary of ARC operations: }

Six scanners of the same type, called \char`\"{}ARC scanner\char`\"{} and labelled
from scanner \#1 to \#6, are installed on the Hall-A beamline. Scanners \#1
to \#4 are used for the ARC energy measurement and they are located on the Hall-A
arc: \#1 and \#2 just upstream of the arc, in the BSY, and \#3 and \#4 in the Hall-A
tunnel, just upstream the Compton polarimeter. Scanners \#5 and \#6 are located
between the Moller and the target to control the beam geometry on the target
and their use will not be discussed here. 

Each scanner has a motor/ball-screw/shaft-encoder/vacuum-penetrator system moving
accurately a set of 3 tungsten wires through the beam. Each time a wire crosses
the beam a PMT located a few meters downstream records a signal due to the electromagnetic
shower induced by the beam in the wire. Both forward and backward passes are
recorded. The motion is a horizontal translation and, for a forward pass: 

-the translation is from beam left to beam right, 

-the two first wire crossing the beam are at 45deg from the vertical, 

-the third wire, which is the only important for the ARC energy measurement,
is vertical. 

Recording, during the scan, the scanner position and the PMT output voltage
allows us to determine the beam position at each scanner location. Then, using
calibration data not detailed here, we deduce the net beam bend angle through
the arc. This result measured in dispersive arc tuning, along with the field
integral of the arc dipoles, provides an accurate determination of the beam
energy. 

The list of the operations for an ARC energy measurement is given here and will
be detailed after: 

1/check the pulser (scanners) 

2/check the HV (scanners) 

3/check EPICS (scanners) 

4/check space on disk(scanners) 

5/check the integral (see \char`\"{}Details on integral system check\char`\"{}
below) 

6/ask for chicane off, put target in a safe position, ask \( \sim\)5$\mu$A (achromatic) 

7/perform a scan test in achromatic at \( \sim  \)5$\mu$A
$-->$ solve the trips $-->$ final
adjustment of gains of scanners \#1 and \#2 $-->$ save the last (non saturated)
scan 

8/if everything is OK, ask for dispersive beam \( \sim  \)5$\mu$A 

9/check dispersive beam: 

-CW \( \sim  \)5$\mu$A 

-Fast Feedback ON 

-Energy Feedback ON $\leftarrow$ important 

-Kresting ON 

-arc's quads and steerers OFF 

-beam stable at entrance and exit (Bscope) (see the test plane of the dispersive
mode in MCC) 

10/perform a scan in dispersive at \( \sim  \)5uA (see details below) 

-solve the trips 

-finish the 4 gains adjustment 

-perform and save 3 good scans 

11/perform a field integral (see details below) 

12/restore beam for the experiment, unmask the diagnostics, restore achromatic
mode and chicane, quit MEDM and turn both HV channels OFF. 

13/analyze scan data (dispersive only) and integral data. 


\subsection{Preliminary details about MEDM }

To fill or update an input field of MEDM, put the mouse cursor in the field,
do the edit or change, and then PRESS CARRIAGE RETURN WHILE THE CURSOR IS STILL
INSIDE THE FIELD!!! 


\subsection{Details on pulser check }

A pulser is used to trigger the scanner acquisition. The ARC scanner's pulse generator
(middle counting room) must be adjusted on 1000Hz for CW beam (on \char`\"{}external
trigger\char`\"{} in pulsed mode, for any scanner). Check the output with a
scope. 

Settings for CW mode:

-frequency: 1KHz 

-delay: not important in CW, should be always 0. 

-time width: 10$\mu$s 

-level: 2.5V on 50 $\Omega$ load, 5V on 1 M$\Omega$, polarity: + (TTL). 


\subsection{Details on HV check }

The energy measurement uses scanners \#1 to \#4 with PMT read out. One PMT (PMT0)
is in the BSY reading scanners \#1 and \#2. One other is in the Compton region
(PMT1), reading scanners \#3 and \#4. They both are energized by the \char`\"{}beam
line HV module\char`\"{}, card\#1 channel\#0 (PMT0) and channel\#1 (PMT1), under
EPICS control of the general Hall A control. Both voltages must be -1200V, make
sure that \char`\"{}Meas. V\char`\"{} indicates this voltage. If needed, push
on \char`\"{}enable/disable\char`\"{} or edit the \char`\"{}SET V\char`\"{}
field. 


\subsection{Details on EPICS check (scanners) }

-login as \char`\"{}gougnaud\char`\"{} on hac, you need to know the password,
ask Arun Saha. If you connect by \char`\"{}cd \( \sim  \)gougnaud\char`\"{},
you will not be able to edit the command file nor save the data. hac$>$ cd arc2/adl
hac$>$ medm\& 

-on the medm access window, click on \char`\"{}file\char`\"{}, and then \char`\"{}open\char`\"{} 

-in the open file window, select \char`\"{}ARC6.adl\char`\"{}; the ARC window
must appear then 

-on the medm access window, click on \char`\"{}execute\char`\"{} 

-if then the ARC window remains partly blank, or if you have some doubt about
the identity of the program running in the ioc, go to the \char`\"{}reboot ioc\char`\"{}
section below. 


\subsection{Details on ioc reboot (scanners) }

Reboot arcioc by pushing the \char`\"{}Beamline Arc Measurement VME Reset\char`\"{}
green button (middle counting room): the arcioc is loaded from the \char`\"{}up\char`\"{}
file. If the ARC window (in execute mode) was not blank, it becomes blank, then
wait for 1 minute and it returns to grey: the scanner is then operational. If the
ioc is not dead, you can also reboot it by software through the network: 

hac$>$rlogin arcioc 

$->$ reboot (the rlogin connection is then broken by the reboot) 

10/98:The hard reboot (green button labelled \char`\"{}ARC\char`\"{}) is temporarily
connected also to another VME crate (e,p energy measurement), use it only if
e,p is not running. 


\subsection{Details on disk space check (scanners) }

Run the \char`\"{}quota -v\char`\"{} UNIX command under \( \sim  \)gougnaud/arc2/ARC: 

Filesystem usage quota limit timeleft files quota limit timeleft 

/u/home 119308 512000 512000 924 -1 -1 

Do it several times, sometimes the answer is crazy! In the above example you
have about 400Mo free, a scan file of scanners \#1 to \#4 (energy measurement)
occupies about 1Mo when unzipped and 300Ko when zipped, so you have room for
\( \sim  \)400 unzipped files or \( \sim  \)1200 zipped files. 


\subsection{Details on running a scan }

-On the MEDM screen load \char`\"{}data1234\char`\"{} as Command File. Content
of \char`\"{}data1234\char`\"{}: 

+1 P 5. 

+2 P 5. 

+3 P 5. 

+4 P 5. 

-1 P 5. 

-2 P 5. 

-3 P 5. 

-4 P 5. 

means: use scanners \#1 to \#4, do first the 4 forward pass and then the 4 backward
ones (to give the wires time to cool down) at 5 turns/s = 12.500 mm/s. Warning:
lower velocity may cause the wire to melt. An error message will be edited if
the velocity is outside {]}0., 6.1 t/s{[}. P is for PMT readout (S for Secondary
emission for scanners \#5 and \#6). The file is in \( \sim  \)gougnaud/arc2/ARC.
In pulsed mode (60Hz) use a velocity of 0.3 turn/s. You need to edit a special
file setting this velocity. On the MEDM screen update CW/60Hz, beam current
and DISPersive/ACHROmatic (for the record). 

-Push \char`\"{}START\char`\"{} when ready, you can follow the operations by
looking at: 

-running command: the line of the \char`\"{}Command file\char`\"{} currently
executed 

-the cursor position of the scanners 

-voltage versus position plot. The location of the 6 scanners on the MEDM window
is the following: 

\vspace{0.3cm}
{\centering \begin{tabular}{|c|c|}
\hline 
\#1&
\#2\\
\hline 
\hline 
\#3&
\#4\\
\hline 
\#5&
\#6\\
\hline 
\end{tabular}\par}
\vspace{0.3cm}

The forward plot is green, the backward one is red. In case of saturation, redo
the scan with lower gain (see details below). The only peak used for energy
measurement is the rightmost one (H profile produced by the V wire): focus on
this peak to adjust the gain, the 2 other peaks may saturate. 

-the names on the top of the plots are the CEBAF device names of the scanners. 

WARNING: operating at too high a beam current or too small a velocity may cause
the wire to melt, with no opportunity to repare it before the next shutdown.
The total travel time is about 15 s per scanner. During the travel, have a look
at the beam current. If the beam trips, inquire why, correct if it was caused
by the scan (set masks), and redo the scan (see details on trips below). 

-Save the data (see details below). 

-Print the MEDM window (see details below). 

-you can plot the profile (see details below). 

After the scan, check the peak (rightmost) quality: it must be compact (Gauss
curve) with a good signal/noise ratio ($>$5) If scan \#1 and \#2 are not compact:
report to MCC and make sure that the fast feedback is ON. If scan \#3 and \#4
are not compact: report also to MCC and make sure that the energy feedback and
the kresting are ON. 

If the scan seems OK: start a field integral measurement and
 report to \\
 \char`\"{}saha@jlab.org\char`\"{} \\
with as much details as possible (at least the file names) for data analysis
and beam energy determination. Record the mail in the e-logbook. Don't forget
to save the files. 


\subsection{Details on MEDM window print }

To print the MEDM window: put the mouse cursor somewhere in the background part
of the ARC window (outside the plots), push the mouse right button, select print,
release the button. It should go on \char`\"{}cha1hp\char`\"{} if you work from
the counting house. 


\subsection{Details on profile plot }

To plot the profile: use your favorite curve plotter (hvplot, xmgr...) The scan
file is a plain ASCII file with a line per acquisition (trigger) and two float
per line: position in encoder unit (4096.0 per 2.500 mm or turn) and voltage
after gain in V. First the forward profile with increasing \char`\"{}x\char`\"{},
then the backward one, with decreasing \char`\"{}x\char`\"{} (x=coordinate transverse
to the beam and horizontal. Towards beam right for scanners \#1 to \#4, towards
beam left for \#5 and \#6). Some lines are header lines. In this case the 1st
character is a \char`\"{}!\char`\"{} seen as a comment line by hvplot (needs
to be changed for other plotters). The rightmost peak (high x values) was produced
by the vertical wire (it is a pure horizontal profile), the 2 other peaks by
the +45deg. and -45deg. wires. 


\subsection{Details on gain adjustment (scanners) }

Adjust the gain of the signal amplifier to get {]}-10V, +10V{[} at the input
of the ADC, range= 001, 002, 004,...128, 256. 

Gain adjustment: in the ARC scanner MEDM window, click (right
button for Arun Saha,
left one in general!) on the gain command button located close to the plot of
the profile of the scanner you want to adjust. Then select a gain and release
the mouse. 

After a scan: 

-Select higher gain if the rightmost peak is $<$ 3V (in absolute value) 

-Select lower gain if the scanner saturates. To know if the scanner saturates,
do not believe the MEDM plot (on which several adjacent channels are averaged),
trust the red diode associated to each scanner (close to the plot). If you want
to know what these diodes look like, click on \char`\"{}edit\char`\"{} on the
MEDM access window: you will see the diodes. Then return to \char`\"{}execute\char`\"{}. 

Start with the gains (dispersive case): 

\vspace{0.3cm}
{\centering \begin{tabular}{|c|c|}
\hline 
scanner&
gain\\
\hline 
\hline 
\#1&
4\\
\hline 
\#2&
8\\
\hline 
\#3&
32\\
\hline 
\#4&
32\\
\hline 
\end{tabular}\par}
\vspace{0.3cm}

Note that in dispersive mode the beam is dispersed at scanners \#3 and \#4 locations,
so these scanners need a higher gain than \#1 and \#2. 

To zoom on a peak, change the MEDM parameters of the window (\char`\"{}edit\char`\"{}): 

-first determine the range (in mm) on which you want to zoom 

-on the MEDM access window, push on \char`\"{}edit\char`\"{} 

-on the ARC window, click anywhere inside the plot you want to change 

-on the \char`\"{}Resource Palette\char`\"{}, go down to \char`\"{}Axis Data\char`\"{},
select it 

-on the \char`\"{}Cartesian Plot Axis Data\char`\"{}, in the \char`\"{}X Axis\char`\"{}
part (X means horizontal), change \char`\"{}Axis Range\char`\"{} from \char`\"{}auto-scale\char`\"{}
to \char`\"{}user-specified\char`\"{}, then edit \char`\"{}Minimum Value\char`\"{}
and \char`\"{}Maximum Value\char`\"{} to specify the horizontal range you want
to look at, the unit is mm. If you used the \char`\"{}edit\char`\"{} facility
of MEDM, it will suggest that you save the changes when killing the window. Do
not save the changes! 

-on the MEDM access window, push on \char`\"{}execute\char`\"{} 

Short-cut: push on the right button while the cursor is inside the plot, you
will access directly to the MEDM axis data menu Other MEDM tricks (may depend
on the version of MEDM you are running): 

-Control+middle button+vertical drag will scale the plot up or down 

-Shift+middle button+drag will translate the plot. 


\subsection{Details on file save (scanners) }

To save the data in a disk file: push on \char`\"{}Save\char`\"{}. The software
builds automatically a name of the type: \char`\"{}scan\_nnn.data\char`\"{},
where nnn is the scan run number, incremented automatically. The current run
number is stored on disk, so unless a disk crashes, you can reboot the VME or
delete a data file without resetting this number. Read the run number on the
MEDM screen and record the file name in the e-logbook. The file will be stored
in \( \sim  \)gougnaud/arc2/ARC. As the file is big (2{*}7000 channel histogram
per scanner), purge or compress your own data files. \( \sim  \)gougnaud belongs
both to hac and the CUE, thanks to Javier. So you do not need \char`\"{}ftp\char`\"{}
to copy it inside the CUE; the cp command works as far as you have the good
file access. Write in the e-logbook the final path of the file in the CUE. \\


Example of scan file: scan\_159.data \\


!version:2 

!THU MAY 27 12:55:25 1999 

!CW ,CW\_60Hz, beam time structure 

! 5.000,Microamp, beam current in microamp 

!ACHRO,DISPersif\_ACHROmatique, arc optics tuning 

data1234,Command\_Filename 

! 

!+1 P 1 5. ,Current\_Command\_Line 

!IHA1C07A a,p,TJ\_Device\_Name 

!-1. ~p,Displacment\_X, dimensionless, X=beam left 

!0. ~p,Displacment\_Y, dimensionless, Y=top vertical (Z=beam) 

!1 ~,Rod\_Serial\_Number 

!235360 ~p,Fiducial\_To\_Beam\_Micron (at survey scanner position) 

!1 ~,Cartridge\_Serial\_Number 

!81270 ~a?,p,Cartridge\_Range\_Micron 

!20 ~,Wire\_Diameter\_Micron 

!W ~,Wire\_Material 

!3 ~p,Number\_Of\_Wire 

!20 ~p,Survey\_Wire\_Diameter\_Micron (Wire used for above data) 

!284375,257840,234730 p,Fidicial\_To\_Wire\_Micron 

!.707,0.,-.5 p,Differential\_Fiducial\_To\_Wire (per micron of wire diam) 

!44.97,-45.04,-0.06 p,Angle\_Degree (around Z axis, =0 when perp. to displ.) 

!0.,0.00315,0. p,Differential\_Angle\_Degree\_Per\_Micron (of wire diam) 

!4096 ~a,p,Encoder\_Home 

!118405 ~p,Encoder\_Survey (Survey of 11 Aug 98) 

!1 ~a?,p,Encoder\_Sign (+1 if increase at forward) 

!4096 ~a?,p,Encoder\_unit\_per\_Turn 

!10 ~a?,p,Motor\_Micro\_Step\_Per\_Step 

!200 ~a?,p,Motor\_Step\_Per\_Turn 

!2500 ~a?,p,Screw\_Pitch\_Micron 

!1 ~a? ,Motor\_Polarity (used for forward displacment) 

!begin forward 1 Time Stamp (s)=0 

4093.0 -0.002 

4093.0 0.007 

4093.0 0.007 

........ 

135163.0 -0.002 

135163.0 0.002 

!end forward 1 Time Stamp (s)=1305473 

!-1 P 1 5. ,Current\_Command\_Line 

!begin backward 1 Time Stamp (s)=1305498 

135163.0 0.007 

135163.0 0.007 

........ 

4092.0 0.002 

4092.0 0.002 

!end backward 1 Time Stamp (s)=1305505 

!+2 P 2 5. ,Current\_Command\_Line 

!IHA1C07B a?,p,TJ\_Device\_Name 

!-1. ~p,Displacment\_X, dimensionless, X=beam left 

!0. ~p,Displacment\_Y, dimensionless, Y=top vertical (Z=beam) 

!4 ~,Rod\_Serial\_Number !235120 ~p,Fiducial\_To\_Beam\_Micron (at survey scanner
position) 

!4 ~,Cartridge\_Serial\_Number 

!81270 ~a?,p,Cartridge\_Range\_Micron 

!20 ~,Wire\_Diameter\_Micron 

!W ~,Wire\_Material 

!3 ~p,Number\_Of\_Wire 

!20 ~p,Survey\_Wire\_Diameter\_Micron (Wire used for above data) 

!284310,257775,234665 p,Fidicial\_To\_Wire\_Micron 

!.707,0.,-.5 p,Differential\_Fiducial\_To\_Wire (per micron of wire diam) 

!45.08,-44.96,0. p,Angle\_Degree (around Z axis, =0 when perp. to displ.) 

!0.,0.00315,0. p,Differential\_Angle\_Degree\_Per\_Micron (of wire diam) 

!4096 ~a,p,Encoder\_Home 

!118407 ~p,Encoder\_Survey (Survey of 11 Aug 98) 

!1 ~a?,p,Encoder\_Sign (+1 if increase at forward) 

!4096 ~a?,p,Encoder\_unit\_per\_Turn 

!10 ~a?,p,Motor\_Micro\_Step\_Per\_Step !200 ~a?,p,Motor\_Step\_Per\_Turn 

!2500 ~a?,p,Screw\_Pitch\_Micron 

!1 ~a? ,Motor\_Polarity (used for forward displacment) 

!begin forward 2 Time Stamp (s)=0 

4092.0 -0.007 

4092.0 0.017 

......

...... 

4090.0 0.032 

4090.0 0.129 

!end backward 4 Time Stamp (s)=1305529


\subsection{Details on trip handling }

Trips are frequent at 5$\mu$A, caused by scanners \#3 and \#4 and triggered by the
Compton ion chamber or BLM, even with chicane OFF. They are invasive for the
other halls. If a trip occurs, write in the e-logbook which scanner, forward/backward
(for this, you have to monitor the beam current during the scan), beam current,
arc tuning mode, which diagnostic tripped (ask MCC), get a strip chart of
this diagnostic (MCC). Contact the beamline coordinator ( Arun Saha) or the
hall leader (Kees de Jager) to allow MCC to mask the diagnostic. But in any case start
with unmasked diagnostics to have all safety systems responding. If masking
is impossible or not sufficient, reduce the beam current down to 1$\mu$A but the
quality of the energy measurement will then be affected by the poor signal/noise
ratio. 

If the trip occurred from scanner \#3 or \#4 during the preliminary test in
achromatic mode, leave this step and ask for dispersive mode. In this mode,
due to the dispersion of the beam, the luminosity will be reduced and hopefully
the trip will not occur. Nevertheless, the test in achromatic mode is useful
to check the system (including the trip protection!) and to anticipate a gain
change for scanners \#1 and \#2. 


\subsection{Detail on encoder change (scanner) }

The main screw shaft is coupled to the stepper motor by a bellow coupler (bellow
\#1) and the motor to the encoder by a second bellow coupler (bellow \#2). If
for any reason (encoder change, motor change, one of the bellows slept) the
angular relationship between the screw and the encoder is lost, then: 

-these two objects must be linked again in good angular agreement (see detail) 

-the theodolite survey of the bench must be redone (see detail) 


\subsection{Detail on encoder re-mount (scanner) }

The procedure is based on 2 rules: 

-the \char`\"{}home\char`\"{} position is defined by an encoder reading=4096
enc. units 

-this \char`\"{}home\char`\"{} position must be at about 1/2 turn (\( \sim  \)2048=1.25mm)
from the Low Limit Switch (LLS) transition. So this transition must happen
at encoder reading \( \sim  \)2048. 

-if the Low Limit Switch was also changed, then the rule is that the external
fiducial must be at 230mm from the beam nominal axis when the encoder reading
is 118407 (\char`\"{}survey\char`\"{} position). This rule needs the help of
a survey. 

The procedure is the following: 

-by software (see \char`\"{}scanner expert task detail\char`\"{}) or by hand
(by acting on the bellow \#1), set the scanner in the convenient position (LLS
or survey): 

-decouple the bellow \#2 

-connect the encoder to the control box readout 

-by hand turn the encoder shaft up to have a good reading 

-strongly fasten the coupler \#2 Note: the encoder is absolute over 64 turns
with 4096 units per turn. So, starting from 0 and turning it clockwise, one
should read: 

-4096 after one turn 

-8192 after two turns 

-... 

-up to 262143 just (one puls) before 64 turns 

-0 again one puls after. 

-4096 after one turn .... 


\subsection{De
tails on scanner expert task }

The path of the \char`\"{}expert task\char`\"{} MEDM file is
 given in the first subsection. 
 Use this task to set the encoder in any position between the limit switches,
to dismount the wire cartridge, to set the encoder,... 

-Call this task in MEDM. 

-You get a small window with a blue button. 

-Left click on it: it gives you a choice between the 4 ARC scanners. 

-Select the scanner you are interested in, then release the left mouse button. 

-An \char`\"{}Expert-task\char`\"{} bigger window appears. 

-In this new window, left click on the blue button \char`\"{}click to initialize...\char`\"{}. 

-Then the designated scanner name is displayed on the top of the window. 

-Edit the motion you want in the \char`\"{}First goal\char`\"{} input field.
This is an ABSOLUTE position (i.e. not an increment), from an origin which can
be arbitrary, in unit of motor micro-step. +2000 microstep= +1 turn= +4096 encoder
units, so one has 2.048 encoder units per microstep. 

-push on \char`\"{}Pos$-->$0\char`\"{} to set to zero the \char`\"{}Motor current
position\char`\"{} register. The current scanner position becomes the new origin
of the motor positions. -once the motor position is entered, push on \char`\"{}1
WAY\char`\"{} to execute the motion 

-If you want to move the scanner fron its current position by 10mm towards the
beam, and if the current position is 100000, then enter 100000.+(10./2.500)x2000.=108000.,
and push on \char`\"{}1 WAY\char`\"{}. 

-If you want to go to the LLS position, enter a large negative value like -1000000,
and push on \char`\"{}1WAY\char`\"{} Note: Limit switches will not stop the
motor at a very reproducible position.


\subsection{Detail on theodolite survey of a bench }

The goal is to measure the H angle between the segment joining the fiducials
of the 2 scanners of the bench in their \char`\"{}survey\char`\"{} position,
and the normal to the autocollimation mirror associated with this bench. The
procedure is the same for both benches, except that: -the upstream one needs
a BSY access, the downstream one just a Hall A access, providing the light tubes
were left in straight position. 

-the upstream angle is \( \sim  \)0, only affected by a change in the scanners;
the downstream one is \( \sim  \)6 deg and it can be affected by a change in
the scanners or by a motion of the tunnel. 

-the upstream autocollimation is at short distance, the downstream one is at
long distance, through the light tubes, and the line of sight is bent by a pair
of mirrors, both at forward and at backward pass. 

The procedure is: 

-using the \char`\"{}expert task\char`\"{} (see detail), set both scanners in
their \char`\"{}survey\char`\"{} position (encoder=118407+-2) 

-install the theodolite on its support and level it carefully. 

-mount the battery (2 autocollimation batteries and chargers are stored in the
shed, make sure the batteries are charged) 

-push the \char`\"{}ON/Enter\char`\"{} button, the bottom display prompts \char`\"{}V
zero 1\char`\"{} 

-release the \char`\"{}V\char`\"{} break (small knob of the top pair of knobs) 

-initialize the V (Vertical angle) encoder by turning slowly and continuously
the optics around the horiz. axis in a direction, until you get a beep and \char`\"{}V
zero 2\char`\"{} displayed. Then turn in the opposite direction, until you get
a beep and \char`\"{}A zero\char`\"{} displayed in the top display. 

-release the \char`\"{}A\char`\"{} break (small knob of the bottom pair of knobs) 

-initialize the A (Horizontal angle) encoder by turning slowly and continuously
the whole theodolite around its vertical axis, until you get a beep and \char`\"{}Theo\char`\"{}
displayed in the top display. Push \char`\"{}ON/Enter\char`\"{} button to accept
the proposed \char`\"{}theodolite\char`\"{} menu. 

-from this point, \char`\"{}V\char`\"{} and \char`\"{}A\char`\"{} angles will
be permanently displayed in their own displays, in units of decimal degrees from
0.0000 to 359.9999 If the decimal part is not displayed, correct the levelling. 

Note: the origin of the A angles is arbitrary (this is not the case 
for the V angles), so you may decide to set it to zero, for example at the first
autocollimation measurement: this way, the measured angles will be more intuitive.
To do this: 

-from the \char`\"{}THEOD\char`\"{} menu push on \char`\"{}+\char`\"{} (means
go to the next menu called \char`\"{}SET A\char`\"{}) 

-\char`\"{}SET A\char`\"{} menu is proposed: push \char`\"{}enter\char`\"{}
to accept, you have a \char`\"{}WAIT\char`\"{} displayed, after which the angles
are displayed again, but with A=0.0000. 

-push \char`\"{}-\char`\"{} to proceed: the sub-menu \char`\"{}MEAS P\char`\"{}
is proposed, you must accept (\char`\"{}enter\char`\"{}), you have then the
angles displayed, push \char`\"{}-\char`\"{} to come back to the \char`\"{}THEOD\char`\"{}
menu, then push \char`\"{}enter\char`\"{} to accept it, you should still have
A=0.0000: you have a new origin of the A angles.

-measure the autocollimation angle: 

-open the light tube 

-plug the autocollimation light cable 

-turn ON the autocollimation power switch on the battery 

-adjust the autocollimation power potentiometer to half range 

-remove the theodolite handle, to let the cable go through. 

-adjust the occular (small knob on the optics) to your vision, until you have
the cross hairs well focused 

-adjust the main focusing (large knob on the optics) to infinity (turn counter
clockwise up to the limit and come back by \( \sim  \)1/4 turn) 

-adjust the V angle to 90.00?? deg.: first move the optics, second
fasten the V break and use the V knob. 

-adjust the A angle to point in the direction of the light tube: first move
the theodolite, second fasten the A break and use the A knob. 

-using the A knob, search for a light signal 

-focus on it (main focusing): the autocollimation pattern is a yellow/green cross 

-adjust A and V angles to get the cross hairs on the autocollimation cross,
fine adjust both focusings, then check if the pattern you have is the good one
(there are 2 wrong autocollimation patterns): when you change by a small amount
the A/V angle, then the A/V coincidence between both crosses must be destroyed.
If it is not the case, you are focusing on one of the two wrong patterns. Then
change the main focusing towards infinity (counter clockwise) to get another
patern. 

-fine adjust the autocollimation power potentiometer to get the best vertical
line of the autocollimation cross -fine adjust A and V angles, record their
values in the Shed logbook. 

-change both angles by 180.000 deg., WITHOUT CHANGING THE MAIN FOCUSING, and
redo the previous step. 

-close the light tube 

-turn OFF the autocollimation power switch on the battery, Note: the above measurement
is easy from upstream, more tricky from downstream (long distance autocollimation).
New operators should start with the upstream measurement.

-measure the scanner \#2/\#3 (the one which is on the autocollimation side) 

-remove the scanner cover -unplug the connexion of the encoder to the VME -plug
the control box on the fiducial lamp and on the encoder 

-record the encoder reading in the logbook. It should be \( \sim  \)118407+-2 

-adjust the main focusing on the fiducial wire 

-adjust the V angle to the middle of the aperture across which the wire appears
-fine adjust the A angle, record A and V values in the Shed logbook. 

-change both angles by 180.000 deg., WITHOUT CHANGING THE MAIN FOCUSING, and
redo the previous step. 

-record the encoder reading in the logbook. It should give the previous reading
+-1 encoder unit 

-unplug the control box 

-plug the connexion of the encoder to the VME 

-cover the scanner, Note: in case of scanner \#2, a foil closing the apertures
in the mirror support must de removed prior to the survey, and reinstalled after.

-measure the scanner \#1/\#4 as for \#2/\#3

-re-measure the autocollimation angle.

-turn OFF the theodolite (push simultaneously the + and - buttons) 

-release both breaks 

-mount the handle 

-dismount the theodolite 

-dismount the battery and the autocollimation cable 

-store the theodolite in its box 

-store the theodolite box and the control box in the Shed 

-charge the battery(ies)

Note: in case you have to change the battery during the measurement, redo the
complete measurement with the new battery. A battery in good shape can provide
\( \sim  \)2h of work with autocollimation ON. The survey of
a bench will take between 1h and 1/2h. If possible, discharge the battery before
charging it.

-analyze the angle: 

-in the directory \( \sim  \)gougnaud/arc2/ARC/angles, edit a plain ASCII file
named upstream\_ddmmyy.dat or downstream\_ddmmyy.dat, depending on the bench,
where ddmmyy stands for the date of the survey. Use upstream\_270798.dat or
downstream\_270798.dat as a model. Units are encoder unit and decimal degrees.
If several measurements of the same quantity were done, enter the average. If
the full measurement of a bench was done twice, edit both sets of data as in
the model. 

-run \char`\"{}angle\char`\"{} on a Sun machine and follow the instructions
of the program, accept the proposed mirror calibration angle and ask for an
average of all your measurements. If only one bench was measured (encoder change,
BSY closed,...), use for the other bench the most recent file available in \( \sim  \)gougnaud/arc2/ARC/angles. 

-analyze individual measurements and compare them to the above average. The
reproducibility of the reference angle measures several times on the same day
should be $\leq$ 10-3 deg (17 urad \( \sim  \)3 10-5 on the energy). 

-compare the new average to the previous one: if nothing changed in the scanners,
and if only the downstream angle changed, this can be interpreted as a motion
of the arc (the code gives the detail of the 3 angles: upstream, downstream
and reference). 

-the files used today (01/07/99) are upstream\_250699.dat (1 meas.) and downstream\_250699.dat
(2 meas.). They result in a reference angle of 34.32322 deg. (averaged) 

-write the new reference angle in the arc-scan source code (include file \char`\"{}arc-incl-plot.f\char`\"{},
parameter \char`\"{}Reference\_Angle\_Degree\char`\"{}) 

-compile and test the arc-scan code. 

-report in the e-logbook about the changes and results (file names).\\


Example of upstream file: upstream\_250699.dat (1 meas.)\\


UPSTREAM BENCH MEASUREMENT 

{*}{*}{*}{*}{*}{*}{*}{*}{*}{*}{*}{*}{*}{*}{*}{*}{*}{*}{*}{*}{*}{*}{*}{*}{*}{*} 

Used unit: decimal degrees 

Enter data in this order: 

\begin{itemize}
\item - Autocollimation 
\item - Fiducial upstream of upstream bench \#1 
\item - Fiducial downstream of upstream bench \#2 
\end{itemize}
- Horizontal Distance (point A, up. fiducial)= 1996.00 mm \#1 

- Horizontal Distance (point A, down. fiducial)= 1975.00 mm \#2 \\


{*}{*}{*}{*}{*}{*}{*}{*}{*}{*}{*}{*}{*}{*}{*}{*}{*}{*}{*}{*}{*}{*} 

25/06/99 Andrew+Tim Mesur. Nb 1 

First measurement on 25/06 

Encoder reading for upstream scanner of upstream bench: 118406 \#1 

Encoder reading for downstream scanner of upstream bench: 118406 \#2 

\vspace{0.3cm}
{\centering \begin{tabular}{|c|c|c|c|c|}
\hline 
H1&
H2&
V1&
V2&
\\
\hline 
\hline 
0.000100&
179.995500&
90.002100&
270.003500&
Autocoll.\\
\hline 
359.977700&
179.973900&
250.509900&
109.498800&
\#1\\
\hline 
179.970000&
359.967100&
250.382500&
109.619700&
\#2\\
\hline 
\end{tabular}\par}
\vspace{0.3cm}


Example of downstream file: downstream\_210699.dat (2 meas.) \\

DOWNSTREAM BENCH MEASUREMENT \\

{*}{*}{*}{*}{*}{*}{*}{*}{*}{*}{*}{*}{*}{*}{*}{*}{*}{*}{*}{*}{*}{*}{*}{*}{*}{*}{*}{*} 

Used unit: decimal degrees 

Enter data in this order: 

\begin{itemize}
\item Autocollimation 
\item Fiducial upstream of down. bench \#3 
\item Fiducial downstream of down. bench \#4 
\end{itemize}

- Horizontal Distance (point A, up. fiducial)= 1870.00 mm \#3 \\
- Horizontal Distance (point A, down. fiducial)= 1840.00 mm \#4 \\


{*}{*}{*}{*}{*}{*}{*}{*}{*}{*}{*}{*}{*}{*}{*}{*}{*}{*}{*}{*}{*} 

21/06/99 Pascal Mesur. Nb 1 

First measurement on 21/06 

Encoder reading for upstream scanner of downstream bench: 118405 \#3 

Encoder reading for downstream scanner of down. bench: 118408 \#4

\vspace{0.3cm}
{\centering \begin{tabular}{|c|c|c|c|c|}
\hline 
H1 &
H2&
V1&
V2&
\\
\hline 
\hline 
0.000100&
179.998050&
90.001400&
270.000000&
Autocoll\\
\hline 
6.100100&
186.10110&
110.667500&
249.336900&
\#3\\
\hline 
6.176100&
186.175700&
249.129900&
110.876300&
\#4 \\
\hline 
\end{tabular}\par}
\vspace{0.3cm}

{*}{*}{*}{*}{*}{*}{*}{*}{*}{*}{*}{*}{*}{*}{*}{*}{*}{*}{*}{*}{*} 

21/06/99 Pascal Mesur. Nb 2 

Second measurement on 21/06 

Encoder reading for upstream scanner of downstream bench: 118405 \#3 

Encoder reading for downstream scanner of down. bench: 118408 \#4

\vspace{0.3cm}
{\centering \begin{tabular}{|c|c|c|c|c|}
\hline 
H1&
H2&
V1&
V2&
\\
\hline 
\hline 
0.000100&
179.998100&
90.003800&
270.003000&
Autocoll.\\
\hline 
6.100800&
186.100400&
110.667600 &
249.338500&
\#3\\
\hline 
6.176000&
186.176100&
249.130700&
110.875600 &
\#4 \\
\hline 
\end{tabular}\par}
\vspace{0.3cm}


\subsection{More on the theodolite menu }

It is a 2-level menu, the entry point after power ON and indexes initialization
is at the 2nd level (Theodolite ?), the +/- buttons are used to navigate among
several choices, the \char`\"{}enter\char`\"{} button to validate the currently
proposed choice. In the above list, \char`\"{}+\char`\"{} will move to the next
option (down), \char`\"{}-\char`\"{} to the previous one (up), and \char`\"{}enter\char`\"{}
will answer \char`\"{}yes\char`\"{} to the proposition. In some cases (SET A)
the choice is just yes/no. Then \char`\"{}enter\char`\"{} means \char`\"{}yes\char`\"{}
and \char`\"{}+\char`\"{} or \char`\"{}-\char`\"{} means \char`\"{}no\char`\"{}:

1st level menu: 

\begin{itemize}
\item $--->$M (Measure)? 
\item A (Adjust)? 
\item U (Unit)?
\end{itemize}
2nd level menu M (Measure): 

\begin{itemize}
\item $--->$ M-0 Return to 1st level? 
\item M-2 THEODOLITE? $--->$ theodolite 
\item M-5 Zero angle? $--->$ A=0,Set angle? $--->$ Set A to value 
\item M-7 Hold angle? $--->$ Hold angle
\end{itemize}
2nd level menu A (Adjust): 

\begin{itemize}
\item --$->$ A-0 Return to 1st level? 
\item A-1 Adjust V index?$--->$Adjust V index 
\item A-2 Adjust A coll.?-$->$Adjust A coll. 
\item A-4 Illumin. ON/OFF?-$->$Change Illumin. mode? 
\item A-5 Compensator ON/OFF?-$->$Change Compens. mode? 
\item A-6 Levelling with compensation ?-$->$Levelling
\end{itemize}
2nd level menu U (Unit): 

\begin{itemize}
\item $--->$ U-0 Return to 1st level? 
\item U-1 Reverse A reading?$--->$Reverse A reading 
\item U-2 Change V display?-$->$Select this V display? 
\item U-3 Change angle unit?-$->$Sel. this angle unit? 
\item U-7 Sound ON/OFF?-$->$Change Sound mode? 
\item U-8 Select configuration?-$->$Select this configur?
\end{itemize}

\subsection{Summary of field integral }

The purpose is to measure absolutely the straight field integral of a \char`\"{}BA\char`\"{}
3m long dipole, called the \char`\"{}9th dipole\char`\"{} and located in the
\char`\"{}Dipole Shed\char`\"{}. It is of the same type as the 8 arc dipoles
and is powered in series with them. 

The ARC integral setup is basically made of a 3m long plate (the \char`\"{}probe\char`\"{})
which is able to move inside the 9th dipole gap along the beam axis and carrying two
field measurement devices: a pair of pick-up coils connected in series and a
set of NMR probes. The coils are on both ends of the probe and the NMRs close
to the center. 

-at the \char`\"{}upstream\char`\"{} probe position, the \char`\"{}downstream\char`\"{}
coil is close to the dipole center, the \char`\"{}upstream\char`\"{} is outside
the dipole and the NMRs at one end of the dipole: 

Door$<-$-- ....................$<-$-------DIPOLE-----$--->$ 

.............$<-$-------PROBE------$--->$ 

-at the \char`\"{}central\char`\"{} probe position, each coil is at one end
of the 3m long dipole and the NMRs close to the dipole center: 

Door$<-$-- ...................$<-$-------DIPOLE-----$--->$ 

..................................$<-$-------PROBE------$--->$ 

-at the \char`\"{}downstream\char`\"{} probe position, the \char`\"{}upstream\char`\"{}
coil is close to the dipole center, the \char`\"{}downstream\char`\"{} is outside
the dipole and the NMRs at one end of the dipole: 

Door$<-$-- ...................$<-$-------DIPOLE-----$--->$ 

....................................................$<-$-------PROBE------$--->$ 

We call upstream the position where the probe is the closest to the shed access
door. Among the 3 above positions, the only one where the NMR can lock on the dipole
field is the central one as in the extreme position of the probe, the field homogeneity
is not sufficient. The probe position is controlled by a linear encoder. The
Z axis refers to the \char`\"{}beam\char`\"{} direction, increasing from upstream
to downstream. We use three kinds of \char`\"{}Z\char`\"{}: 

-Zm to locate a point inside the magnet. The dipole center is at Zm=0 and the
yoke ends at +-1500.mm 

-Zp to locate a point inside the probe. The probe center is at Zp=0. Each of
the 4 NMR probes has a Zp given in the file \char`\"{}magnet.dir\char`\"{}.
At a temperature of 21C, the coils are at Zp=+-1519.815mm (from magnet.dir) 

-Zd to refer to a displacement of the probe w.r.t. the dipole. Zd=0 refers to
the upstream (home) position of the probe. The integral measurement is performed
from Zd=0.000mm (1st PDI trigger) to Zd=3199.000mm (last PDI trigger), for forward
pass. Zd is given by the display (at the top of the rack) or by the master screen
(\char`\"{}OUT\char`\"{}). 

The relationship between Zm, Zp and Zd is: 

Zd-Zm+Zp=C 

where C is a constant given in magnet.dir (C=1604.000 nomin.). Example of use:
to have the probe center at the dipole center, one must set Zd=1604.000mm (set
Zm=0 and Zp=0 in the above formula, and solve for Zd) 

The integral measurement sequence is the following: 

-from the current position (a priori arbitrary) move the probe upstream, up
to a limit (optic) switch. 

-move downstream by a few mm to cross the encoder index (encoder initialization) 

-move to the central position to measure the central field by NMR, the system
checks if the NMR locks and if the reading is stable, it will be the \char`\"{}before\char`\"{}
field 

-move back to upstream position 

-move to downstream position while integrating the flux through the coil system,
this measurement will be called the \char`\"{}forward\char`\"{} integral (duration
\( \sim  \) 7s) 

-move back to upstream position while integrating the flux through the coil
system, this measurement will be called the \char`\"{}backward\char`\"{} integral
(duration \( \sim  \)7s) 

-move to the central position to measure the central field by NMR, the system
checks if the NMR locks and if the reading is stable, it will be the \char`\"{}after\char`\"{}
field. 

In addition to the central field, 4 probe temperatures, a local excitation current
measurement, the setting of the dipoles P.S, the readback of the dipoles P.S
and the probe position at NMR measurement time are recorded \char`\"{}before\char`\"{}
and \char`\"{}after\char`\"{}. 

To perform an integral field measurement: 

1-check if the system works (see \char`\"{}details on integral system check\char`\"{}
below) 

2-run the above integral sequence (see \char`\"{}details on integral run\char`\"{}
below) 

3-fix the error(s) if any (see \char`\"{}details on integral errors\char`\"{}
below) 

4-save the data in a file (see \char`\"{}details on integral data save\char`\"{}
below) 

5-analyze the data (see Arun Saha). 


\subsection{Shed access and safety }

For safety reasons, the access to the shed is limited to authorized
persons which are listed in the ESAD. To be added to the list, ask the Hall-A leader. The standard
operation mode of the integral measurement setup is the remote mode, through
the network, from the counting house. In case of problem needing an access in
the shed, unauthorized users must contact Arun Saha.


\subsection{Details on integral system check }

First - from any workstation or X-terminal connected by telnet to pascal1 workstation,
run MEDM with arc\_master.adl data file (see Arun for the password of gougnaud
on pascal1 and the first subsection for the path of arc\_master.adl). If pascal1 is not
responding, go in the shed and check UPS (see \char`\"{}Details on AC power
(integral)\char`\"{} below). If the UPS is not beeping, check the room temperature.
If the temperature is above 35C, the thermal protection of the workstation is
probably activated, see \char`\"{}detail on temperatures\char`\"{} below. If
the temperature is below 35C, ask the computer center to restart pascal1. 

Second - check if the 4 temperatures and the 3 currents are stable, in agreement between
them and if the beam energy computed from the current is realistic compared
to what you know from MCC. In not, call Arun. If the IOC is dead (all displays
on \char`\"{}arc-master.adl\char`\"{} are blank), reboot the ioc (see \char`\"{}integral
ioc reboot\char`\"{} below) Note: the current/field ratio is about 1000A/1.98T
and the energy/field ratio is 12.03 GeV/T. The current used by MCC for the accelerator
tuning is the set current. The readback can be different from the set by 0.1A.
The local current can be different from the set by 5A. For the temperatures
see \char`\"{}detail on temperatures\char`\"{} below. 

Third - check if the NMR is locked. If not see \char`\"{}details on NMR lock\char`\"{}
below, 

Forth - check if the NMR reading is stable within 10-5 relative (P.S. stability).
If not, inquire about a recent or running setting change by the MCC. If necessary
call Arun Saha. 


\subsection{Details on NMR lock }

-check if the dipole current is above 22A. The integral setup can't work for
lower currents (22A \( \sim  \) .043T \( \sim  \)517MeV), due to the NMR probe
limitation. 

-check if the probe is at a central position, corresponding to Zd\( \sim  \)1604mm.
It should be at this position if no special motion was ordered since the previous
integral measurement. If not, enter 1604 RETURN in the \char`\"{}set Zd position\char`\"{}
input field, wait for the end of the motion (look at cursor and position readback
labelled \char`\"{}out\char`\"{}) and wait for the NMR lock for up to 1 minute.
Note: due to the software, the position readback may be updated \( \sim  \)10
seconds after the real probe motion. Zd\( \sim  \)0mm is for the upstream position,
\( \sim  \)1604mm for the central position and \( \sim  \)3208mm for the downstream
position. If the probe does not obey, call Arun. If Arun is not available, reboot
the ioc (see \char`\"{}integral ioc reboot\char`\"{} below) 

-check if the NMR probe selected is the good one. The system has 4 probes to
cover the field/energy range {[}0.043T/0.517GeV,1.05T/12.63GeV{]}. The software
selects automatically the probe from an estimated field and the following range
table (from file \char`\"{}magnet.dir\char`\"{}): 

\vspace{0.3cm}
{\centering \begin{tabular}{|c|c|c|}
\hline 
mini&
maxi&
probe\\
\hline 
\hline 
T&
T&
\\
\hline 
0.043&
0.1254&
\# 1\\
\hline 
0.1250&
0.250&
\# 2\\
\hline 
0.177&
0.470&
\# 3\\
\hline 
0.338&
1.06&
\# 4\\
\hline 
\end{tabular}\par}
\vspace{0.3cm}

The estimated field is computed from the \char`\"{}set\char`\"{} current and
the coefficients given above. Note that there is an overlap between the range
of each probe. In the standard mode (\char`\"{}auto selection\char`\"{}), two
selection algorithms are used: 

-initial selection: select the probe in which the estimated field will be the
most \char`\"{}centred\char`\"{} 

-routine check: change the probe only if the estimated field is outside the
selected probe range. 

Initial selection is executed at boot time, at each integral start and when
you leave the \char`\"{}manual selection\char`\"{} optional mode. 

The routine check is performed periodically. 

In some cases in the past this algorithm did not select the best probe. If it
occurs again, force the probe selection by opening arc\_nmr.adl medm screen,then
switch to manual selection mode, click on the probe you want to select and wait
for NMR lock. Don't forget to return to the \char`\"{}auto selection\char`\"{}
mode before leaving the integral measurement. Report about the problem in the
e-logbook. 

-expert users can also use nonstandard search modes and DAC values as proposed
by the arc\_nmr.adl screen (\char`\"{}Dac Value\char`\"{}, \char`\"{}Dac+-5\%\char`\"{}
and \char`\"{}All Probes Range\char`\"{}). Note that \char`\"{}Dac Value\char`\"{}
needs a scope. They are asked to restore the standard mode (\char`\"{}SeLected
Probe Range\char`\"{}) at the end. The DAC to field relationship used by the
software is linearly interpolated in the table (from file \char`\"{}magnet.dir\char`\"{}): 

\vspace{0.3cm}
{\centering \begin{tabular}{|c|c|c|c|c|c|}
\hline 
dac=0&
dac=1000&
dac=2000&
dac=3000&
dac=4095&
Probe\\
\hline 
\hline 
T&
T&
T&
T&
T&
\\
\hline 
.040362&
.05388&
.07443&
.104184&
.132404&
\# 1\\
\hline 
.080713&
.107771&
.148845&
.208340&
.264790&
\# 2\\
\hline 
.161430&
.215539&
.297680&
.416680&
.529570&
\# 3\\
\hline 
.322870&
.431112&
.595471&
.833455&
1.059210&
\# 4\\
\hline 
\end{tabular}\par}
\vspace{0.3cm}

Note: as the 9\( ^{th} \) dipole has a uniform field, there is no gradient
coil around the probes and the lock is in general easier to obtain than with
the HRS dipoles. Nevertheless some difficulties may occur at low field due to
the hysteresis gradient. The NMR system is the same as for the HRS (Metrolab)
but the EPICS implementation is different. The field polarity is \char`\"{}+\char`\"{}.
Some lock problems may apear in the {[}60A,70A{]} range. In this case select
the \char`\"{}Dac+-5\%\char`\"{} option, compute the approximate field as explained
above, estimate the corresponding DAC value from the table above (depending
on field value and probe \#), set this estimated value in the arc\_nmr.adl screen,
correct this setting by using the \char`\"{}+\char`\"{}, \char`\"{}++\char`\"{},
\char`\"{}-\char`\"{}, and \char`\"{}--\char`\"{} buttons up to have the \char`\"{}OUT
IN T\char`\"{} value close to the field you computed, and then wait for the
lock. It should work, you have 100s to get the lock at the begining and at the
end of the integral sequence! The best is to get the lock by this way before
starting the integral sequence. 


\subsection{Details on integral run }

To run the integral measurement sequence, call the \char`\"{}arc\_integral.adl\char`\"{}
medm screen on pascal1 (see \char`\"{}Details on integral system check\char`\"{}
above), then: 

-push \char`\"{}start\char`\"{} to start the full sequence 

-look at the results displayed: 

-after the \char`\"{}before\char`\"{} NMR measurement: the \char`\"{}before\char`\"{}
data set 

-after the \char`\"{}forward\char`\"{} integral pass: the forward velocity profile
and the forward voltage-after-gain profile 

-after the \char`\"{}backward\char`\"{} integral pass: the backward velocity
profile and the backward voltage-after-gain profile 

-after the \char`\"{}after\char`\"{} NMR measurement: the \char`\"{}after\char`\"{}
data set 

-if \char`\"{}BAD NMR\char`\"{} or \char`\"{}PDI saturation\char`\"{} flags
are set, or if something is obviously wrong in the data or plots, fix the problem
(see \char`\"{}Details on integral error\char`\"{} below) and start a new integral
run. 

-data are ready to be saved (see \char`\"{}Details on integral data save\char`\"{}
below) 


\subsection{Details on integral error }

-\char`\"{}BAD NMR\char`\"{} flag is set if at least one of the 4 NMR status
(locked and stable, before and after) is wrong 

-the full NMR field measurement is made of 3 successive elementary measurements
separated by a 5s delay. 

-the \char`\"{}lock\char`\"{} status is set to wrong if at least one of the
3 NMR measurements did not lock before a 100s timeout. The measurement sequence
is aborted as soon as a measurement does not lock, so the maximum time one has
to wait for a full NMR measurement is about 100s (in case of a zero current
measurement for example) 

-the \char`\"{}stable\char`\"{} status is set to wrong if the 2nd or the 3rd
field values differs from the 1st by more than 5 10-4 relative. In any case,
the field value returned is the last one. 

-\char`\"{}In position\char`\"{} is set to false if the probe position was outside
1604.0+-0.1mm at NMR measurement time. 

-\char`\"{}PDI saturation\char`\"{}: the PDI (Mertolab's Precision Digital Integrator)
consists in a programmable amplifier and a VFC integrator working in the {[}-5V,
+5V{]} input range. The PDI gain is the result of an automatic selection in
the {[}1,2,5,10,...1000{]} range, similar to the NMR probe selection. In case
of a wrong gain selection (maximum voltage after gain, as plotted in the MEDM
screen, outside the {[}1.5V,5V{[} range), call the \char`\"{}arc\_pdi.adl\char`\"{}
MEDM screen to switch the gain selection mode from \char`\"{}auto\char`\"{}
to \char`\"{}manual\char`\"{} and then select manuallay the gain. Return to
\char`\"{}auto\char`\"{} mode before leaving the integral measurement. 

-\char`\"{}before\char`\"{} and \char`\"{}after\char`\"{} data must be very
close: $<$ 0.1 C difference for the temperatures of a given probe, <0.1A for the
currents and <10-5 for the field. 

-plots: the blue curves are from forward pass, yellow ones for backward pass. 

-the top plot gives the probe velocity (in m/s) as function of its position
(in mm). The forward velocity should be +0.6m/s everywhere except at the center
where it is reduced to +0.06m/s. The backward velocity plot should be the mirror
image of the forward ons (-0.6 and -0.06m/s) 

-the bottom plot gives the coil voltage after gain (in V) as function of the
probe position (in mm). It should be a null voltage everywhere except a set
of oscillation at the center. The backward voltage plot should be the mirror
image of the forward one (V-$->$-V). See \char`\"{}PDI saturation\char`\"{} above. 

-to zoom a MEDM plot, see \char`\"{}Details on gain adjustment (scanners)\char`\"{}
above. 


\subsection{Details on integral data save }

The procedure to save the datafile from \char`\"{}arc-integral.adl\char`\"{}
is the sane as for the scanners (see \char`\"{}Details on file save (scanners)\char`\"{}
above), with an independent integral run number. The file name will be of the
type \char`\"{}integral-nnn.data\char`\"{}, stored on pascal1 in \( \sim  \)gougnaud/EPICS/integral/.
As pascal1 disk is not mounted on the CUE, use ftp or rcp to move the file to
the CUE. 

The size of a file is 207Ko. 

The data file is made of: 

-a header containing basically the \char`\"{}before\char`\"{} and \char`\"{}after\char`\"{}
data 

-the forward integral data 

-the backward integral data. 

Forward end backward integral data are made of 3200 lines of 3 data each: 

-a line per trigger, i.e. per mm of probe motion over the 3.2m of the total
motion. The first trigger line is missing. 

-the first data of the line is the probe position in mm, which should be a round
value from 1.000mm to 3199.000mm at forward pass and from 3199.000mm to 1.000mm
at backward pass, 

-the 2nd data are the flux increment measured during the current step (i.e. between
the previous trigger and the current one) in unit of 10-8Vs, corrected from
the gain. 

-the 3rd data are the time of the trigger since the start, in microsecond units. 

These field increments,like the scan profiles, can be plotted by standard plotters,
see \char`\"{}Details on profile plot\char`\"{} above. The curve plotted online
in \char`\"{}arc-integral.adl\char`\"{} is the time derivative of the above
flux data, multiplied by the gain to get the input VFC voltage after gain.\\


Example of an integral file:\\


!ARC magnetic measurement integral data file 

!version:1 

!date: THU MAY 27 15:28:09 1999 THU MAY 27 15:29:21 1999 

!local current (A) 142.25 142.25 

!remote current set (A) 140.11 140.11 

!remote current readout (A) 140.15 140.14 

!NMR field (T) 0.2752619 0.2752621 !NMR locked? T T 

!NMR stable? T T 

!NMR in position? T T 

!Zd plate position (mm) 1604.003 1604.003 

!temperature x+,z+(deg.C) 33.2 33.1 

!temperature x+,z-(deg.C) 32.8 32.7 

!temperature x-,z+(deg.C) 31.1 31.1 

!temperature x-,z-(deg.C) 31.1 31.1 

!PDI gain: 10 

!NMR probe: 3 

!forward data: 

1.000 -100 5138 

2.000 200 9695 

3.000 -100 13830 

4.000 150 17611 

5.000 0 21096 

6.000 200 24340 

....... 

3198.000 -100 7051626 

3199.000 150 7058284 

!end forward, start backward 

3199.000 -50 5160 

3198.000 -50 9730 

..... 

2.000 -100 7052428 

1.000 -50 7059066 

!end backward


\subsection{Integral ioc reboot }

One of the VME boards (PMAC's motor board) has its own internal boot process
triggered at \char`\"{}power on\char`\"{} time. So the usual boot procedure
(red push button on the ioc or reboot command through the network) is not sufficient
to make sure that the integral VME is correctly initialized. Thus, boot the
VME by switching the AC power of the VME crate OFF and ON. The AC power switch
is located in the lower part of the crate front panel. 


\subsection{Details on temperatures }

The AC system of the shed is made of two cooling units, a heating unit and a
controller connected to two temperature sensors : one located in the shed and
one located in the BSY. This system is programmed in such a way that the temperature
of the shed follows the BSY temperature within +-2C. The BSY temperature can
be anywhere in the {[}18C,35C{]} range, regardless of the season. The BSY temperature
and the shed temperature are given (in F) by a display panel located close to
the workstation, on the wall. The AC system can be set in manual control by
turning from \char`\"{}auto\char`\"{} to \char`\"{}manual\char`\"{} a set of
switches controlling the cooling units and the heater unit. These switch boxes
are located on the shed wall. If the shed temperature is above 34.4C (94F),
call Arun Saha (the electronics can be damaged) and cool down the shed in manual
AC mode. The 4 temperature sensors of the probe are labelled Tx+z+, Tx+z-, Tx-z+,
Tx-z- depending on their position w.r.t. the following (x,z) frame: 

\begin{lyxcode}
~~~~~~~~~~~~~~~~~~~~~~~~~~~~~~~~~~~~~~x~

~~~~~~~~~~~~~~~~~~~~~~~~~~~~~~~~~~~~~~\^{}~

~~~~~~~~~~~~~~~~~~~~~~~~~~~~~~~~~~~~~~|~

~~~~~~~~~~~~~~~~~~~~~~~~~~~~~~~~~~~~~~|~

Door$< -${}-{}-~~~~~~~~~~~$< -${}-{}-{}-{}-{}-{}-{}-{}-{}-{}-{}-{}-D~I~P|O~L~E~-{}-{}-{}-{}-{}-{}-{}-{}-{}-{}$->$~

~~~~~~~~~~~~~~~~~~~~~~~~~~~~~~~~~~~~~~|~

~~~~~~~~~~~~~~~~~~~$< -${}-{}-{}-{}-Tx+z-~-{}-{}-{}-{}-{}-{}-|-{}-{}-{}-{}-{}-Tx+z+-{}-{}-{}-{}-{}$->$~

~~~~~~~~~~~~~~~~~~~|~~~~~~~~~~~~~~~~~~|~~~~~~~~~~~~~~~~~|~

~~~~~~~~~~~~~~~~~~~|~~~~~~~~~~~~~~~~~~|-{}-{}-{}-{}-{}-{}-{}-{}-{}$->$z~~~~~|~

~~~~~~~~~~~~~~~~~~~|~~~~~~~~~~~~~~~~~~~~~~~~~~~~~~~~~~~~|~

~~~~~~~~~~~~~~~~~~~|~~~~~~~~~~~~~~P~R~O~B~E~~~~~~~~~~~~~|~

~~~~~~~~~~~~~~~~~~~$< -${}-{}-{}-{}-Tx-z-~-{}-{}-{}-{}-{}-{}-{}-{}-{}-{}-{}-{}-{}-Tx-z+-{}-{}-{}-{}-{}$->$
\end{lyxcode}

Both \char`\"{}x+\char`\"{} sensors are on the probe edge which is inside the
dipole gap and both \char`\"{}x-\char`\"{} sensors on the opposite edge which
is outside the dipole gap. Both \char`\"{}z-\char`\"{} sensors are at 1/4 of
the long dimension of the probe and both z+ at 3/4 of this length. The average
of the 4 temperatures is used by the analysis program to correct the coil distance
from the thermal expansion of the probe, so it is important to make sure that
the 4 sensors are working well. The user can just make sure that the temperatures
displayed in \char`\"{}arc-master.adl\char`\"{} or recorded in \char`\"{}arc-integral.adl\char`\"{}
are realistic. In \char`\"{}arc-integral.adl\char`\"{} they are given in the
order: Tx+z-, Tx+z+, Tx-z-, Tx-z+ Tx-z- and Tx-z+ should be close to the shed
temperature. Tx+z- and Tx+z+ depend on the probe position, as the gap (iron
yoke) is warmer than the shed and the dipole coil (at both ends of the dipole)
is warmer than the iron yoke. For a probe in a central position for more than
about one hour, the Tx+z- and Tx+z+ sensors should give the yoke temperature,
i.e the shed temperature plus 0. to 5.C, depending on the current, LCW temperature
and the magnet/shed temperature history. The 4 temperatures are also displayed
inside the shed, on the electronics rack. These values are digitized by separate
ADCs, so they may differ from the remote values by \( \sim  \)0.1C. 


\subsection{Details on AC power (integral) }

All the shed electronics and computer (except the motor power, see details about
the Power Switch below) are powered through a UPS (Uninterruptible Power Supply)
whose role is to protect the setup from short (\( \sim  \)1/4 h) power outages
and to convert the US AC power (60Hz, 200V between two phases) in a well stabilized
AC power of European type (220V, 50Hz). When working in the shed, be careful
about the electrical hazard as the setup can remain powered a long time after
interrupting the UPS input (see \char`\"{}Shed access and safety\char`\"{} above). 

The UPS unit is located under a table, in front of the electronics rack. It
has a display, a small keyboard and a beeper. The beeper activated means that
some important message about the UPS is displayed. In this case, use the keys
to scan the UPS memory, read the messages, record them in the e-logbook, stop
the beeper and inform Arun. Refer to the UPS manual (available in the shed)
for details of operation. 


\subsection{Details on mechanics (integral) }

The probe is capable of a fast motion (\( \sim  \)1m/s) and it is hazardous to
access or to introduce something inside the device, under the altuglass covers,
or to dismount a cover. It is also hazardous to manipulate the motor shaft (see
\char`\"{}Shed access and safety\char`\"{} above). To do it safely, the user
must first turn off the motor power box. The power box is located under the
motor. It has two external switches: one controlling the logics and one controlling
the power: 

%\newpage

\begin{lyxcode}
~~~~~~~~~~~~~~~~~~~~~Logics~switch~

~~~~~~~~~~~~~~~~~~~~~~~~~~~~|~

(TOP~VIEW)~~~~~~~~~~~~~~~~~~|~

~~~~~~~~~~~~~~~~~~~~~~~~~~~~V~~

.~~~~~~~~~~~~~~~~~~~~~~~~-{}-{}-{}-{}-{}-{}-~~~~-{}-{}-{}-{}-{}-{}-{}-{}-{}-{}-{}-{}-{}-{}-{}-{}-{}-{}-{}-{}-{}-{}-{}-{}-{}-{}-~

.~~~~~~~~~~~~~~~~~~~~~~~|~~~~~~~|~~|~~~$< -${}-{}-{}-{}-{}-{}-DIPOLE-{}-{}-{}-{}-{}$->$~~~|~

.~~~~~~~~~~~Power~box-{}$->$|~~~~~~~|~~|~~~~~~~~~~~~~~~~~~~~~~~~~~~|~

DOOR$< -${}-{}-~~~~~~~~~~~~~~~~|~~~~MOTOR-|~~~~~~~~~~~~~~~~~~~~~~~~~~~|~

.~~~~~~~~~~~~~~~~~~~~~~~|~~~~~~~|$< -${}-{}-Power~switch~~~~~~~~~~~~~~|~

.~~~~~~~~~~~~~~~~~~~~~~~|~~~~~~~|~~|~~~~~~~~~~~~~~~~~~~~~~~~~~~|~

.~~~~~~~~~~~~~~~~~~~~~~~~-{}-{}-{}-{}-{}-{}-~~~~-{}-{}-{}-{}-{}-{}-{}-{}-{}-{}-{}-{}-{}-{}-{}-{}-{}-{}-{}-{}-{}-{}-{}-{}-{}-{}-~
\end{lyxcode}


It is important NOT to switch off the logics (some data in memory will be lost),
but to switch off the motor power (see the exact Power switch location above)
prior to any intervention on the mechanics. The detailed procedure is the following: 

-turn OFF the power switch 

-wait for \( \sim  \)1min to empty its buffer-capacitor 

-you can work safely on the mechanics 

-make sure that the path of the probe is free (tools...) and the
 covers are in position 

-turn the power switch back ON 

-reboot the VME (see \char`\"{}Integral ioc reboot\char`\"{} above) 

The above procedure may be necessary if, for some reason, the probe went outside
its allowed range limited by a pair of electric limit switches. The probe range
is limited by the following set of devices: 

-upstream energy damper for Zd in {[}-40.,-14.mm{]} 

-upstream electric limit switch for Zd$<-$7.5mm 

-upstream optic limit switch for Zd$<-$3.8mm 

(operation range, 3.2m long) 

-downstream optic limit switch for Zd>3200.8mm 

-downstream electric limit switch for Zd>3212.3mm 

-downstream energy damper for Zd in {[}3214.,3238.mm{]} 

The current status of the 4 limit switches is given in the \char`\"{}arc-master.adl\char`\"{}
screen. Optical limit switches are used for the encoder initialization procedure
(see \char`\"{}Summary of field integral\char`\"{} above). Electric limit switches
should never be activated. But if this happens, then the motor control: 

-stops the motor 

-raises an error flag (the letted \char`\"{}d\char`\"{} for defect is displayed
inside the power box, the usual display being a vertical segment) 

-waits for a manual repositioning of the probe inside its allowed range and
for a reboot of the VME. 

The manual repositioning of the probe inside its allowed range can be done safely
by manipulating the motor shaft while the motor power is off, according to the
above procedure. 


\subsection{Details on detailed field mapping }

The main use of the integral setup is to measure the integral of the 9th dipole
along a straight line, located on the axis of the pole tips. By moving (manually)
the dipole in the transverse direction and then performing the integral sequence,
one will measure the integral along different axes, and hence
extract the gradient of the dipole integral (used to determine the dipole positionning tolerances).
By inserting a jumper (short) on one of the pick-up coils and reducing the gain
of the PDI, the same sequence as for the integral will provide, after a specific
analysis, the field profile of the dipole as seen by the other coil. This profile
will cover one half of the magnet, including one of the two fringing fields.
Reduce the PDI gain by one step ( example: if the automatic gain is 10, select
manually 5) By inserting the jumper on the other coil, one can get the other
half profile. These two single coil measurements will provide together the complete
profile of the dipole including both fringing fields. For the central part,
where the field is uniform, more accurate data can be taken using the NMR probes.
Here, there is no change in the hardware, but one must use a specific software:
the \char`\"{}mapping\char`\"{} MEDM window. By combining, for a set of transverse
positions: 

-standard integral measurement 

-upstream profile (upstream coil alone) 

-downstream profile (downstream coil alone) 

-NMR map of the plateau, 

one gets a 2D map of the dipole field in its mid-plane with an optimum accuracy
and redundancy and a minimum time spent by the operator. This 2D map is used
to get the straight-to-curved correction to the integral and a set of tolerances.

Call \char`\"{}medmMew\char`\"{} (and NOT \char`\"{}medm\char`\"{}), then open
\char`\"{}arc\_map.adl\char`\"{}, as this display is written in a new version
of MEDM. Due to a bug in this Solaris/MEDM combination, you must enter the name
of the display in the open menu, you can't just click on the display file name.
You then get the \char`\"{}ARC-MAPPING\char`\"{} screen. You must define the
position inside the magnet (Zm) where you want to measure the field, inside the
{[}-1500mm,+1500mm{]} range. The software will compute the probe position Zd
depending on the NMR probe selected (automatic or manual selection). You have
to define 3 regions, where the steps may differ. Edit the fields: 

-STARTING Zm 

-region 1 : NUMBER OF STEPS and STEP SIZE, then the intermediate position
Zm is computed automatically by the software as a result of the above input 

-region 2 : NUMBER OF STEPS and STEP SIZE, then the intermediate position
Zm is computed as above 

-region 3 : NUMBER OF STEPS and STEP SIZE, then the final position
Zm is computed as above. 

Then push \char`\"{}Start\char`\"{}. It takes about 20s per
point, unless there is a lock
problem, where it can take up to 100s (timeout). If the NMR does not lock
even far from the ends, push \char`\"{}Stop Map here\char`\"{}, use the DAC+-5\%
lock mode (see \char`\"{}detail on NMR lock\char`\"{}), and start a new map. 

Note: after \char`\"{}Stop Map here\char`\"{} the incomplete map can still be
saved. Use \char`\"{}Stop\char`\"{} in case of emergency, and then reboot the
VME (see detail). 

The measured fields are displayed on the screen as a function of Zm. The sequence
is: 

-a measurement at Zm=0, called \char`\"{}Bref1\char`\"{} 

-the list of measurement you entered 

-a measurement at Zm=0, called \char`\"{}Bref2\char`\"{}. 

When done, save the data by pushing \char`\"{}Save\char`\"{}. The file is numbered
automatically, the name is of the type mapping\_nnn.data, where \char`\"{}nnn\char`\"{}
stands for the map number. The directory is given by the MEDM screen. An example
of file (\char`\"{}mapping\_7.data\char`\"{}) is given below (EPICS was not
responding for the P.S. readback and set, the real current was 90A):

!ARC magnetic measurement mapping data file 

!version:1 

!date: SAT JUN 12 04:49:40 1999 SAT JUN 12 05:41:32 1999 

-1460.00 50 20.00 

-460.00 46 20.00 

460.00 50 20.00 

!temperature x+,z+(deg.C) 29.8 30.2 

!temperature x+,z-(deg.C) 31.7 32.0 

!temperature x-,z+(deg.C) 28.4 28.2 

!temperature x-,z-(deg.C) 28.6 28.2 

!tal nbr pts: 146 

!initial \& final NMR probe: 2 2 

Zm (mm), B (T), NMR lckd, NMR flag,local crt(Vadc),crt set(A),crt rdt(A) 

0.00 0.1770036 1 1 92.19 0.04 0.00 

-1460.00 0.1770068 1 1 92.19 0.04 0.00 

-1440.00 0.1770157 1 1 92.19 0.04 0.00 

....... 

1439.99 0.1770756 1 1 92.19 0.04 0.00 

1460.00 0.1770665 1 1 92.19 0.04 0.00 

0.00 0.1770044 1 1 92.19 0.04 0.00 

!end

Note: \char`\"{}1\char`\"{} is the normal status for NMR lock and NMR (stability)
flag. The lines \#4,\#5 and \#6 of the header give the data entred by the operator.
The result is a map from -1460 to +1460mm with an uniform 20mm step size. This
input is the standard one for the needs of ARC.

% ===========  CVS info
% $Header: /group/halla/analysis/cvs/tex/osp/src/beamline/arc.tex,v 1.1 2003/06/05 17:28:32 gen Exp $
% $Id: arc.tex,v 1.1 2003/06/05 17:28:32 gen Exp $
% $Author: gen $
% $Date: 2003/06/05 17:28:32 $
% $Name:  $
% $Locker:  $
% $Log: arc.tex,v $
% Revision 1.1  2003/06/05 17:28:32  gen
% Initial revision
%
