
\subsection{Electron detector}
\label{sec:compton_edet}

The electron detector is made of 4 planes of silicon strips composed
of 192 strips each of width 250 (200 + 50) microns and 500 microns
thick. The planes are staggered by 200~microns to allow for better
resolution. The detector is placed on a vertical rod allowing vertical motion.  The first strip of the first plane can reach as close as 5 mm away from the
beam. Illustrated in Fig.\ref{fig:compton_edet} is a view of the actual electron detector.
 \begin{figure}[htp]
    \begin{center}
        \includegraphics*[angle=0,scale=0.6]{Edet-new-EPICS}
    \end{center}
    \caption[compton:electron detector]{
            View of the Compton electron detector
            }
    \label{fig:compton_edet}
 \end{figure}

Distance between the CIP and the first strip is 5750 mm. We recall that
between the CIP and the end of the Dipole 3 is 2150 mm.
For a beam of 3.362 GeV the Compton
edge is at 3.170 GeV. This corresponds to a deviation of 17 mm. Thus at
this energy, only one half of the Compton spectrum is covered and it extends
to the 13th strip of the first plane.
The trigger logic looks for a coincidence
between a given number of plane in a "road" of 2 strips. For each trigger
it outputs a signal check by the Polarimeter DAQ.

\subsection{Fast acquisition system}
\label{sec:compton_daq}
The goal of this system is to acquire
for each electron helicity state the energie of the scattered photons at a
rate up to 100 kHz. The energy of each Compton event can be
reconstructed from the signals of the photon calorimeter
with front-end electronics and ADCs. Each helicity state, given
by the accelerator, is also numbered.
Further information is given for each
event (type of event, status of the polarimeter at event's time) and for
each polarization period (duration, dead time, counting rate,...).
A specific tool, the so-called spy\_acq, has been developped in Tcl/Tk
to manage all acquisition system parameters. Finally, a web-based logbook
is available on this site at \url{http://hallaweb.jlab.org/compton/Logbook/index.php}.

} %infolev

\infolevtwo{
\section {Operating Procedure }
\label{sec:compton_oper}

The main operations computer for the compton polarimeter is compton.jlab.org located
in the central isle of the Hall A counting house. A dedicated console labelled as Compton
is in the backroom. This machine, running RedHat Linux 7.3, runs the compton data acquisition,
analysis, and the EPICS~\cite{EPICSwww} slow control system. To begin compton polarimeter activity log on to:


\noindent machine:  \mycomp{compton.jlab.org}\\
username: \mycomp{compton}\\
password: \mycomp{*******}(contact Sirish Nanda (7176))\\

All necessary environment variables are automatically defined on logon. Follow the steps below
paying careful attention to ensuring that you have checked the result of each step:


\subsection{DAQ Setup}
\begin{itemize}
\item Go to the CODA desktop and open a new terminal window. Type

\$ coda start

The general syntax is coda start|stop|restart. If CODA is in a bad state, do coda restart.
All relevant CODA processes are started and you whould get the runcontrol panel 
as shown  in Fig.\ref{fig:compton_rc_connect}. 
\begin{figure}[htp]
    \begin{center}
        \includegraphics*[angle=0,width=10cm]{compton_rc_connect}
    \end{center}
    \caption[compton:runcontrol connect]{Compton DAQ setup: {\it connect} }
    \label{fig:compton_rc_connect}
\end{figure}

\item Click on the "CONNECT" button. You will get the  window shown in
Fig.\ref{fig:compton_rc_configure}. \\
\begin{figure}[htp]
    \begin{center}
        \includegraphics*[angle=0,width=10cm]{compton_rc_configure}
    \end{center}
    \caption[compton:runcontrol configure]{Compton DAQ setup: {\it configure} }
    \label{fig:compton_rc_configure}
\end{figure}

\item Click on the {\bf Configure} button and you will get the sub-panel shown in 
Fig.\ref{fig:compton_rc_fastacq}.
\begin{figure}[htp]
    \begin{center}
        \includegraphics*[angle=0,width=6cm]{compton_rc_fastacq}
    \end{center}
    \caption[compton:runcontrol run type]{Compton DAQ setup: {\it run type} }
    \label{fig:compton_rc_fastacq}
\end{figure}
Click on the "Run type" button and choose of of the following configurations:

\item {\bf Fastacq}
This is the standard run type to take data in a nominal situation.
\item {\bf Beamtune} This run type could be use during beam tuning phases.
It does not record good data, but by connecting to ROC1 (telnet cptaq1),
you could monitor the trigger counting rate more quickly than with spy\_acq.

\item {\bf Scanacq} This run type is used during the electrons detector harp scan procedure.
It stores in particular the electrons detector ruler values at high counting
rates.
\item {\bf Linscan}
This is the procedure to determine the non-linearity of the ADC.
This run type is used to scan the response of the photon detector electronic
with test pulses of different amplitude. The acquisition system set the
electronics to send 50ns long test pulses to the integration and ADC devices,
and increase progressively the amplitude of these pulses.
\item {\bf Others}
The other run types are essentially used for debug purposes. If you are
not an expert of the acquisition do not use them.

\item Confirm via the "OK" button
You should see in the window below the following message "transition
configure succeeded"
\item Click on the "Download" Button to get the  messages shown in 
Fig.\ref{fig:compton_rc_configure}. \\
\begin{figure}[htp]
    \begin{center}
        \includegraphics*[angle=0,width=10cm]{compton_rc_download}
    \end{center}
    \caption[compton:runcontrol download]{Compton DAQ setup: {\it download} }
    \label{fig:compton_rc_download}
\end{figure}
\item Start the run clicking on the "Start Run" button\\
You should see the run control display as in 
Fig.\ref{fig:compton_rc_configure}. \\
\begin{figure}[htp]
    \begin{center}
        \includegraphics*[angle=0,width=10cm]{compton_rc_start}
    \end{center}
    \caption[compton:runcontrol start run]{Compton DAQ setup: {\it start} }
    \label{fig:compton_rc_start}
\end{figure}
Check that the following happens:\\
{\it transition Go succeeded}\\
the counting rates distribution\\
the number of events in this run is updating\\
the run status {\it active}\\
the run number updated\\

\item To end a run, click on End Run (Fig.\ref{fig:compton_rc_end}) button to stop the acquisition.
\begin{figure}[htp]
    \begin{center}
        \includegraphics*[angle=0,width=10cm]{compton_rc_end}
    \end{center}
    \caption[compton:runcontrol end run]{Compton DAQ setup: {\it end} }
    \label{fig:compton_rc_end}
\end{figure}

\newpage
\item {\bf Start the acquisition control panel}\\

In a fresh terminal window, excecute the command to start 
the spy acqusition control panels\\

\mycomp{\$ spy\_acq}\\

Don't be surprised. 7 windows will be open but regroup in only one
        within few seconds. From time to time, it may happen that one window
        does not go inside \mycomp{spy\_acq} window. Click on the corresponding widget.

\item Check the High Voltages applied on the Photon detector PMT's.
Go to the "Acq" panel where \mycomp{spy\_acq} is currently runnning,
Go in the Logbook panel and select photon detector. You will get the 
the  panel shown in Fig.\ref{fig:compton_spy_ht} with the high voltages values:
\begin{figure}[htp]
    \begin{center}
        \includegraphics*[angle=0,scale=0.7]{compton_spy_ht}
    \end{center}
    \caption[compton:spy\_acq high voltage]{spy\_acq high voltage status panel }
    \label{fig:compton_spy_ht}
\end{figure}

\item If the HV are off, switch them on.\\
The cards of the COMPTON Polarimeter PMT HV are located in crate \#2
telnet hatsv5 2011, then 1450, vt100 and usual display.

High Voltage channel for the Compton polarimeter are in cards \# 12, 13, 14 and 15.
Typical HV for Beam Diag and Cristal PMTs is 1500 V. The voltage of the monitoring
LED (channel 15.10) should stay in the range 110-130 V.

\par NB: Only one user can connect on hatsv5 at the same time!! If you can't connect
check if others are logged in. See also procedure posted in rack \#CH01B05.
\end{itemize}

\newpage
\subsection{Cavity Setup}
Choose the EPICS desktop and in a fresh terminal window and start the MEDM~\cite{MEDMwww} EPICS panel 
 by executing  the command:\\

\mycomp{\$ epicshall}\\

This will open the main EPICS menu for the Compton as shown in Fig.\ref{fig:compton_epicsall}.
 \begin{figure}[htp]
    \begin{center}
        \includegraphics*[angle=0,width=0.8\textwidth]{compton_epicsall}
    \end{center}
    \caption[compton:epics main control]{Compton polarimeter main EPICS control panel }
    \label{fig:compton_epicsall}
 \end{figure}


\begin{itemize}
\item {\bf Switch on the laser}\\

On the EPICS control panel, pull the "OPTICS" menu down. Click on "Mini Optic" 
(see Fig.\ref{fig:compton_optic_mini}).
\begin{figure}[htp]
    \begin{center}
        \includegraphics*[angle=0,scale=0.8]{compton_optic_mini}
     \end{center}
    \caption[compton:epics mini control]{Compton polarimeter mini optics control panel }
    \label{fig:compton_optic_mini}
 \end{figure}


To turn the Laser On .... Click on the Laser On button.\\
        Check LASER STATUS and INCIDENT POWER.\\
        A Laser spot may blink on the CCD control TV screen\\ 
	(second from left among the 4 screens)\\
        and you should see a bright spot on the miror control TV screen labelled "laser." 
	(see Fig.\ref{fig:compton_laser_photo}).
 \begin{figure}[htp]
    \begin{center}
        \includegraphics*[angle=0,width=6cm]{compton_laseroff_photo}
        \includegraphics*[angle=0,width=6cm]{compton_laseron_photo}
    \end{center}
    \caption[compton:laser spot]{TV viewer images of laser spot }
    \label{fig:compton_laser_photo}
 \end{figure}

        If Laser doesn't turns ON most problable problem is an interlock
        fault.
        You need an access in Hall A and check the different parts of the
        laser interlock around the optic table:\\
        Two crash buttons on the left wall, inside and outside the green
        tent.\\
        The optic table is protected by a metallic cover. Four magnetic switches connect the top and the walls
        of this cover. Some of them might be opened. {\bf CAUTION!!}: you are standing close to the optic table.
        Do not try to open the cover. Just move the top carefully sideways to try to close the switches.
	
\item {\bf Lock the cavity}\\
	
 To lock the cavity click on the Servo On button shown in Fig.\ref{fig:compton_servo_on}.
 \begin{figure}[htp]
    \begin{center}
        \includegraphics*[angle=0,scale=0.8]{compton_servo_on}
    \end{center}
    \caption[compton:servo control]{Laser servo control}
    \label{fig:compton_servo_on}
 \end{figure}
You should see the cavity locking
on the CCD control TV as in  Fig.\ref{fig:compton_accroche}, and you should have more than 1000 W stored
in the optical cavity.
\begin{figure}[htp]
    \begin{center}
        \includegraphics*[angle=0,width=5cm]{compton_accroche}
    \end{center}
    \caption[compton:vacity lock]{Compton cavity lock indicator}
    \label{fig:compton_accroche}
\end{figure}

If it isn't the case, turn on the Slow Ramp and then Click on the Slow On button shown in 
Fig.\ref{fig:compton_slow_on}.
\begin{figure}[htp]
    \begin{center}
        \includegraphics*[angle=0,scale=0.8]{compton_slow_on}
    \end{center}
    \caption[compton:laser ramp control]{Compton Laser slow ramp control}
    \label{fig:compton_slow_on}
\end{figure}

    If successfull you can turn OFF the Slow Ramp button.\\
    Photons are now ready to meet electrons and give some Compton photons children.\\

    If the cavity still doesn't lock after few minutes with SERVO and Slow Ramp ON:\\
Check the Yokogawa generator in the Compton rack (CH01B00).
Frequency should be 928 kHz, Amplitude 80 mVpp and phase -4 deg.
Pull down the OPTICS menu in the main epics window. Click on "Optic table" and then on "Servo".
The laser servo control panel appears as shown in Fig.\ref{fig:compton_servo}
Gain should be close to 167. A too high traking level in the feedback can prevent the
cavity from locking. Bring the "tracking Level" cursor down to low values (0.20 - 0.40)
and try to lock again with Servo and Slow Ramp on.
\begin{figure}[htp]
    \begin{center}
        \includegraphics*[angle=0,scale=0.8]{compton_servo}
    \end{center}
    \caption[compton:servo settings]{Compton Laser servo control}
    \label{fig:compton_servo}
\end{figure}

\item {\bf Unlock the Cavity}

On the EPICS control panel, pull the "OPTICS" menu down. Click on "Mini Optic" 
\begin{figure}[htp]
    \begin{center}    
        \includegraphics*[angle=0,scale=0.8]{compton_laseron}
    \end{center}
    \caption[compton:laser control ]{Compton Laser control setup to unlock the cavity}
    \label{fig:compton_laseron}
\end{figure}


To unlock the cavity, click on the Servo off button as shown in Fig.\ref{fig:compton_laseron}.
\end{itemize}
    
\newpage
\subsection{Electron Detector Setup}
\begin{itemize}
\item {\bf Turn on the electron detector}\\
        The detector system needs to be powered. In the tunnel,
        there is a crate attached to the wall above the electron detector, it also needs to be turned ON.
        When it is ON a red LED is lit (at the right end of the crate).

\begin{figure}[htp]
    \begin{center}
        \includegraphics*[angle=0,scale=0.6]{compton_map1_elec}
    \end{center}
    \caption[compton:electron detector circuit breaker]{Location of the electron detector circuit breaker}
    \label{fig:compton_map1_elec}
\end{figure}

\item {\bf Slow control of the electron detector }\\
To perform operations on the electron detector. Open the Compton Polarimeter... screen from the main Polarimeter EPICS screen
and then choose "Electron Detector". On this screen, ashown in Fig.\ref{fig:compton_epicsElectron},
 active buttons appear in blue
and readback values appear on a yellow background. 
To use the electron detector a high voltage
(120 V) must be applied to polarize the silicon
microstrips and a low voltage must be provided to the preamplifier
circuit board and some threshold must be set for each plane for
the detection of the signals. To do this execute the following operations :

Turn the low voltage ON on both side\\
Turn the high voltage ON. Put the final DAC value (up to 255 for max HV ) and click the ramp.\\
Set thresholds to 17. The return value should read 17.\\
Turn calibration OFF.\\

The electron detector can be put in data
taking position remotely. When the detector is inserted {\bf the chicane must be ON},
when it is being moved {\bf the beam must be OFF too}. If it is not the case the detector will eventually be destroyed.

Click on either "garage" or "beam" depending on where you want to put the detector.\\
To make sure the detector is where you want watch the detector move on the TV screen (there is one
        in tha Hall A counting house and one in the back room). The switches readback must oscillate a
        little bit if the system is running properly.\\
	
\begin{figure}[htp]
    \begin{center}
        \includegraphics*[angle=0,scale=0.8]{compton_epicsElectron}
    \end{center}
    \caption[compton:Electron detector control]{Compton electron detector control panel}
    \label{fig:compton_epicsElectron}
\end{figure}

The electron detector must be on the garage position.

        Check the status of the electron detector on\\
          the video screen shown in Fig.\ref{fig:compton_electron_out}. The arrow must be exactly in front\\
          of  OUT (outside) nominal position.\\
\begin{figure}[htp]
    \begin{center}
        \includegraphics*[angle=0,width=8cm]{compton_electron_out}
    \end{center}
    \caption[compton:Electron detector viewer]{Compton electron detector TV viewer}
    \label{fig:compton_electron_out}
\end{figure}
	  
\item {\bf Switch on the the Compton chicane}\\

        This procedure is only performed by MCC operators.\\

        First of all,
        the Hall A Run Coordinator must request that
        MCC tune the beam through the Compton chicane.
        MCC operators have to apply the section 2 of
        the procedure MCC-PR-04-001. If necessary
       (after a long shutdown for exemple), let's remind to the operator
       to open valves located on the Compton line.
       The complete procedure is available on the MCC web site \\
\href{http://opsntsrv.acc.jlab.org/ops_docs/online_document_files/MCC_online_files/HallA_beam_delivery_proc.pdf}%
{ at this URL}
       
\item {\bf Lock once again the cavity}\\
\end{itemize}

\subsection{Vertican Scan}
Perform a vertical scanning of the electron beam inside the magnetic chicane 
in order to maximize the counting rates in the Photon detector.

        In the case the crossing of the electron and Laser beams
        has been lost, or is not optimal, a "Vertical scan" has to be performed.
        By stepping the magnetic field of the
        chicane dipoles, the beam is moved vertically. Step size should be
        small with respect to the laser spot size (\~100 micro m). Here are some
        step sizes corresponding to a {\bf 25 or 100$\mu$m  } vertical displacements versus
        typical beam energies, MCC operator are used to Gauss.cm unit:
\begin{table}[ht]
\begin{center}
\begin{tabular}{|l|l|l|} \hline
step 25$\mu$m & step 100$\mu$m & Energy \\ \hline\hline
10 G.cm & 40 G.cm & 0.8 GeV \\ \hline
20 G.cm &  80 G.cm &  1.6 GeV \\ \hline
30 G.cm &  120 G.cm & 2.4  GeV \\ \hline
40 G.cm &  160 G.cm & 3.2  GeV \\ \hline
50 G.cm &  200 G.cm & 4.0  GeV \\ \hline
60 G.cm &  240 G.cm &  4.8 GeV \\ \hline
70 G.cm &  280 G.cm &  5.6 GeV \\ \hline 
\end{tabular}
\end{center}
\caption[Compton:vertical scan]{Chicane vertical scan step values for various energies.
}
\label{tab:compton_vscan}
\end{table}
	
Although the procedure is non-invasive for Hall A, let the shift leader know
when you start and finish the scan.

The scan is done in contact with MCC (7047) by checking the online evolution
        of the counting rate using "spy\_acq".
        The optimal Y-position is at the upper part of the bell appearing
        on the middle of the spy\_acq picture (see Fig.\ref{fig:compton_spy_bell}) 
	("{\bf Counting Rates versus DAQ Y
        at vertex}"). As a first pass, one can use bigger step size to locate
        the maximum and then go back to small steps to fine tune the position.

When this procedure is over, come back to the right Y-position and
        ask to the machine operator to lock the Y-position of the beam. Then an
        automatic magnetic feed-back will run to keep the electron beam Y-position
        within 50 micro m of this optimal position. Click on {\bf Set X} and
        {\bf Set Y} buttons in the frame "Beam drift alarm on Epics pos at vertex".
        Click on {\bf Alarm ON} to set it green. It will bip when the 50 micons limit
        is reached. This is an important task to avoid sensitivity to beam position
        false asymmetry.
\begin{figure}[htp]
    \begin{center}
        \includegraphics*[angle=0,height=0.8\textheight]{compton_spy_bell}
    \end{center}
       \caption[compton:vertical scan]{Vertical scan trace in spy\_acq panel}
        \label{fig:compton_spy_bell}
\end{figure}

\begin{itemize}
\item {\bf Ask to the MCC operators to switch the beam off.}


\item {\bf Insert the electron detector in the beam line.}

First of all, the electron beam must
be off (see Hall A run coordinator and call MCC operator)
If it is not the case the detector will eventually be destroyed.
To perform operations on the electron detector. Open the Compton... screen from 
the main Hall A EPICS screen
and then choose "Electron Detector" to get the panel shown in 
Fig.\ref{fig:compton_epicsElectron}.\\

Click the "COMPTON" button.\\
To make sure the detector is where you want watch the detector move on the TV screen (there is one
in tha Hall A counting house and one in the back room). The switches readback must oscillate a
little bit if the system is running properly.

Finally, Ask to the MCC operators to switch the beam on.
\end{itemize}

\subsection{Taking data}
This is a list of check points to run Compton. It assumes the polarimeter 
has already been started up as described in the previous sections and  
that a run has just ended and you want to take a 
new one.

Bring up the  following three screens to control the data taking:

\begin{itemize}
\item {\bf EPICS screen}: slow control for the optic table + cavity, the 
photon and electron detectors, beam parameters.
\item {\bf Acq screen}: runcontrol and spy\_acq display.
\item {\bf Netscape screen}: Electronic logbook of the Compton polarimeter, 
check histograms.
 \end{itemize}

Follow the following procedure:
\begin{itemize}
\item Go to the {\bf EPICS screen}, check the cavity is loocked with \~1200 Watts 
or more.
\item Go to the {\bf Acq screen} and click on the {\bf Trigger window} in spy\_acq. 
   
\begin{itemize}
\item Check Random, Mouly and central cristal are activated.
\item Check {\bf Raw data rates}. Assuming a trigger rate of 1kHz/muA, 
   prescaler factors should keep the read data rates at the few kHz level.
\item Check the trigger condition in {\bf General Daq setup} (Photon only, 
   e only, coinc, ...). 
\end{itemize}
\item Check the state of the acquisition in the {\bf Acquisition system window} 
of spy\_acq. After an "End run succeded" each module should be in "downloaded" 
state. 
{\it If not, follow error messages displayed in the bottom window. Most of 
the problems are fixed by clicking on Reset or Reboot + Download buttons. Display needs 
some delay to refresh after these  actions. Don't click like crazy on every 
enabled button. If everything is stuck try "restart this window" in the 
spy\_acq menu to refresh the display.}
\item Click on {\bf Start Run} in the {\bf Runcontrol window}. Check that 
each module of the acquisition goes from downloaded to paused and then active 
state.
\item Click on the {\bf Online Counting Rate window} in spy\_acq. Check the rate 
in the central cristal (red curve) is close to the optimal value from the last 
vertical scan (it should be around 1kHz/muA).
{\it If counting rates are low and Beam Drift Alarm keeps ringing, the 
crossing of the electron and Laser beams is not optimal. Stop the run and 
perform a 
vertical scan.}
\item Start the photon polarization reversal by clicking on {\bf Procedure ON} 
in the {\bf Left-Right procedure} frame. Periods of cavity ON and OFF will 
alternate, starting with OFF (bkg measurement).
\item Take a {\bf \~1 hour run}.
\item Before ending a run, fill up the {\bf LogBook window} in spy\_acq (name, 
run type, title). Ensure the Logbook and Checklist buttons are not inhibited
if you want the run to be analysed and stored in the electronic logbook.
\item Click on {\bf End Run} in the {\bf runcontrol window}. Look for the 
"End of run succeeded" message in the bottom frame.
{\it If End of run failed, go to Acquisition system in spy\_acq and follow 
error messages.}
\item A {\bf yellow window} should pop up for few second after the end of run 
indicating that the run is saved and the online analysis (checklist) is 
runnig.
\item Go to {\bf Netscape screen} and reload the {\bf Compton logbook web 
page}. Last run should appear on the first line. 
\item By clicking on {\bf more} you access to detailed informations about 
the running conditions as well as to control histograms generated by the 
{\bf checklist} script. This script may take few minutes to run.
{\bf It is important task to check the control histograms after each end of 
run. Quality off the data depends on it.} See section "Control Histograms".
\item Go to first point, {\bf start a new run}. 
\end{itemize}

Two kinds of alarm can turn ON during data taking:

\begin{itemize}
\item {\bf Y Position:} Go to {\bf On Line Counting Rate window} in spy\_acq and check 
the "Beam drift alarm" frame. If the "Average Y" differs to "Y settings" by more than 
50 microns, Alarm is ringing and stop bell button is red. Click on stop bell and 
wait few seconds to see if the position feedback brings Y back to its nominal 
value. If it doesn't, call MCC (7047) and ask them to check if the position 
feedback on Y in the Compton chicane is still running.
If necessary stop the run, perform a vertical scan and re-lock the vertical 
position at the new optimal value.
\item {\bf DAQ system:} If something goes wrong in the DAQ system during data 
taking you should see an effect on the "photon read" counting rate. Go to 
{\bf Acquisition system} window in spy\_acq, click on stop bell button if alarm is 
ringing. End Run in runcontrol window. Follow error messages displayed in spy\_acq 
to fix the problem.
\end{itemize}

ADC spectrum of the photon detector should show the 
pedestal peak
(pink), the diode signal (green), Compton + background spectrum (blue). 
Gain has to be adjusted so that the Compton edge is between 1/2 and 2/3 
of the ADC range.An automatic fit procedure substracts the background and calibrate 
the threshold value using the Compton edge. If the fit doesn't succeed 
it won't affect the quality of the data but prevents further online 
analysis. Call a Compton expert to fix it.Typical experimental Compton asymmetries 
are of the order a 1\%. 
Check the electron beam current asymmetry stays below few 100 ppm.{\bf Vertical position} 
of the electron beam is the {\bf most important parameter}. 
It drives our luminosity (electron and Laser 
beam crossing) as well as our sensitivity to beam position differences 
correlated with the helicity.
Check we spend most of the running time at the {\bf optimal Y position}, 
which is at the {\bf summit of the bottom left curve (counting rate \% Ybpm)}.
If the most probable position is off by more than 50 micron, perform a 
vertical scan.


Any comment about the running conditions, shift summary, ... are welcome to 
help the offline analysis. You can insert them in the Compton electronic 
logbook by filling up the {\bf LogEntry window} in \mycomp{spy\_acq}. Click on 
{\bf Submit} to dowload your comments in the logbook.

} %infolev

\infolevtwo{
\subsection{Turning off the compton polarimeter}
\begin{itemize} 
\item  Stop the magnetic chicane\\

        This procedure is only performed by MCC operators.
    
        First of all, 
        the Hall A Run Coordinator must request that
        MCC tune the beam through the Compton chicane.
        For a foreseen shutdown or maintenance days, you do not need this step.
    
        MCC operators have to apply the section 3 of 
        the procedure MCC-PR-04-001. Let's remind to the operator
       to close valves located on the Compton line. It is very important
       to keep the best vacuum in the Compton line and avoid
       dust deposit on the high reflectivity mirrors of the cavity
       The complete procedure is available on our web site at :
       
       \url{http://hallaweb.jlab.org/compton/Documentation/Procedures/compton_off.frm.ps}

\item Set the electron detector on the GARAGE position\\


First of all, the electron beam must
be off (see Hall A run coordinator and call MCC operator) 
If it is not the case the detector will eventually be destroyed.
%Slow control of the electron detector. 
%To perform operations on the electron detector. 
Open the MISC... screen from the main Polarimeter EPICS screen
and then choose "Electron Detector". On this screen, active buttons appear in blue 
and readback values appear on a yellow background. This screen is also reachable, 
by loading the file \mycomp{ComptonElectron.adl} located 
under the home directory of hacuser on the hac computer.
    
Click on "garage" button.\\

To make sure the detector is where you want watch the detector move on the TV screen (there is one 
in tha Hall A counting house and one in the back room). The switches readback must oscillate a 
little bit if the system is running properly.

\item  Switch off the PMT High Voltage of the photon detector\\
\par The cards of the COMPTON Polarimeter PMT HV are located in crate \#2
telnet hatsv5 2011, login:adaq, paswd:{\it ask people on shift}then 1450, vt100 and usual display.
\par High Voltage channel for the Compton polarimeter are in cards \# 12, 13, 14 and 15.
Switch off all the channels.
\par NB: Only one user can connect on hatsv5 at the same time!! If you can't connect 
check if others are logged in. See also procedure posted in rack \#CH01B05.

\item Unlock the cavity\\
On the EPICS control panel, pull the "OPTICS" menu down. \\
Click on "Mini Optic".\\
To unlock the cavity, click on the Servo off button.\\
\item Switch off the laser\\
On the EPICS control panel, pull the "OPTICS" menu down.\\
 Click on "Mini Optic".\\
 To turn the Laser Off .... Click on the Laser Off button.
 Check LASER STATUS and INCIDENT POWER.\\
 The Laser spots would switch off on the CCD control TV screen
\end {itemize}
} %infolev

\begin{safetyen}{0}{0}
\section {Safety Assessment}
\label{sec:compton_safety}
\end{safetyen}

\begin{safetyen}{10}{15}
\subsection{Magnets}

Particular care must be taken in working in the vicinity of the
magnetic chicane dipoles of the compton polarimeter as they can have large currents
running in them. Only members of the Compton polarimeter group are
authorized to work in their immediate vicinity, and only when they are
not energized. As with all elements of the
polarimeter which can
affect the beamline, the magnets are controlled by MCC. All four dipoles are
powered in series from a common power supply. The power supply for
the dipoles is located in the Beam Switch yard Building (Building 98).
The maximum current for the dipole is 600A. There is a
red light which indicate the status of the dipoles. The warning red
light is activated via a magnetic field sensitive switch placed
on the coils of one of the dipole. Lock and tag training is required of all personnel working
in the vicinity of the Compton magnets. \\

\subsection{Laser}
The primary hazzard in the optical table of the compton polarimeter is the Class IIIB, 240~mw 
CW infra-red laser.
It is housed in the tunnel in a laser safety enclosure interlocked with the laser power.
Welding curtains are provided on all sides to isolate the laser enclosure from other pathways.
A flashing yellow beacon installed in the tunnel indicates laser on status. Three crash  buttons
are provided in the tunnel for emergency shutdown of the laser.

All functions of the laser are remotely controlled and personnel access to the laser "hut" 
is not necessary during routine operation of the compton polarimeter. However, in case of repair  
or mainetance work, access to the laser enclosure may be necessary. The safe operating procedure for this laser is described in 
Jeffeson Lab Laser Standard Operating Procedure (LSOP) 101-2-99-1-4. A copy of the LSOP is 
available in the tunnel wall next to the laser hut. Only personnel aouthorized in the LSOP are
permitted to access the laser hut.
 

\subsection{High Voltage}

There are 25 photomultiplier tubes within the compton photon detector module.
Each tube is connected to a high voltage power supply located in the beamline instrumentation area
with SHV cables. The maximum voltage is 3000 Volts.
The high voltage supply must be turned off prior to accessing  any of the photon detector elements
for servicing purposes. Only members of the Compton group
are authorized to access the  detector.\\

\end{safetyen}

\subsection{Authorized Personnel}

The list
of the presently authorized personnel is given in Table~\ref{tab:compton:personnel}.
Other individuals must notify and receive permission from
the contact person (see Table~\ref{tab:compton:personnel}) before adding their names
to the above list.
\obsolete{
\begin{table}[ht]
\begin{center}
\begin{tabular}{|ll|l|l|l|l|c|} \hline
  \multicolumn{2}{|c|}{Name} & Dept. & \multicolumn{2}{c|}{Telephone} &
  \multicolumn{1}{c|}{e-mail} & Comment \\
  \cline{4-5}
    first & last & & JLab & Pager &  & \\
\hline
 {\em Sirish } & {\em Nanda} & JLab   & 7176  & 7176  & nanda@jlab.org     & Contact     \\
   Jack   &  Segal           & Jlab   & 5849  & 5849  & segal@jlab.org     & Technical \\
  Joseph  &  Zhang           & Jlab   & 5849  & 5849  & shukui@jlab.org    & Optics \\
  Martial &  Authier         &  CEA   & 4324* &       & mauthier@Cea.Fr    & Engineering \\
 Nathalie &  Colombel        &  CEA   & 8350* &       & ncolombel@Cea.Fr   & Mechanical \\
 Pascale  &  Deck            &  CEA   & 2426* &       & pdeck@Cea.Fr       & Electronics \\
 Alain    &  Delbart         &  CEA   & 3454* &       & adelbart@Cea.Fr    & Optics \\
 David    &  Lhuillier       &  CEA   & 9497* &       & dlhuillier@Cea.Fr  & Analysis  \\
  Yves    &  Lussignol       &  CEA   & 2828* &       & lussi@Cea.Fr       & EPICS \\
  Damien  &  Neyret          &  CEA   & 7552* &       & dneyret@Cea.Fr     & DAQ \\
  G�rard  &  Tarte           &  CEA   & 8464* &       & gtarte@Cea.Fr      & Electronics\\
Christian &  Veyssi�re       &  CEA   & 9704* &       & cveyssiere@Cea.Fr  & Electronics \\
\hline
\end{tabular}  
\end{center}
\caption[compton Polarimeter: authorized personnel]{
   Compton Polarimeter: authorized personnel. The primary contact person's
   name is marked with a slanted font. *For CEA extensions, dial 9-011-33-1-6908-XXXX.   
}
\label{tab:compton:personnel}
\end{table}
}

\begin{namestab}{tab:compton:personnel}{Compton Polarimeter: authorized personnel}{%
          Compton Polarimeter: authorized personnel}
 \SirishNanda{\it Contact}
 \JackSegal{Technical}
 \JosephZhang{Optics}
 \MartialAuthier{Engineering}
 \NathalieColombel{Mechanical}
 \PascaleDeck{Electronics}
 \AlainDelbart{Optics}
 \DavidLhuillier{Analysis}
 \YvesLussignol{EPICS}
 \DamienNeyret{DAQ}
 \GerardTarte{Electronics}
 \ChristianVeyssiere{Electronics}
\end{namestab}

