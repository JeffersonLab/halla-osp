\section{M{\o}ller Polarimeter}
\label{sec:moller}

\subsection {Purpose and Layout}
\label{sec:moller_purpose}
%\cbstart


The Hall A beam line is equipped with a M{\o}ller 
polarimeter
whose purpose is 
to measure the polarization of the electron beam delivered to the hall. 

 \begin{figure}[bht]
    \begin{center}
        \includegraphics*[angle=0,width=\textwidth]{moller_layout_v}
    \end{center}
    \caption{Layout of M{\o}ller polarimeter. The origin of the 
            coordinate frame is at the center of the polarimeter
             target, which is 17.5~m upstream of the Hall A target.
            }
    \label{fig:moller_layout} 
 \end{figure}  

The M{\o}ller polarimeter consists of (see Fig.\ref{fig:moller_layout}):
\begin{itemize}
\item  a magnetized ferromagnetic foil used as a polarized electron target,
       placed 17.5 m upstream of the central 
       pivot point of the Hall A High Resolution Spectrometers;

\item a spectrometer consisting of three quadrupole magnets and a dipole magnet,
      used to deflect the electrons scattered in a certain kinematic
      range towards the M{\o}ller detector;

\item a detector and its associated shielding house;
\item a stand alone data acquisition system;
\item off-line analysis software which helps to extract the beam polarization
      from the data immediately after the data are taken. 
\end{itemize}

The beam polarization is measured by measuring the difference
in the counting rates for two beam helicity samples.  
%This document discusses the safe operation of the M{\o}ller polarimeter
%in the Hall A beam line.

There are also external resources of 
information\htmladdnormallinkfoot{}{(Home page: \url{http://www.jlab.org/~moller/})}. 

\infolevthree{
\subsection {Principles of Operation}
\label{sec:moller_principles}

The cross-section of the M{\o}ller scattering 
$ \vec{e^-} + \vec{e^-} \rightarrow e^- + e^-$
depends on the beam and target polarizations ${\cal P}^{beam}$ and 
${\cal P}^{target}$ as:
\begin{equation}
         \sigma \propto ( 1 +  
            \sum_{i=X,Y,Z} (A_{ii}\cdot{}{\cal P}^{targ}_{i}\cdot{}{\cal P}^{beam}_{i})) , 
\end{equation}
where $i = X,Y,Z$ defines the projections of the polarizations. 
The analyzing power $A$ depends on the scattering angle in the CM frame 
$\theta{}_{CM}$. 
Assuming that the beam direction is along the Z-axis and that the scattering
happens in the ZX plane:
\begin{equation}
         A_{ZZ} = 
            -\frac{\sin^2\theta{}_{CM}\cdot(7+\cos^2\theta{}_{CM})}
                                   {(3+\cos^2\theta{}_{CM})^2},
         A_{XX} = -\frac{\sin^4\theta{}_{CM}}
                       {(3 + \cos^2\theta{}_{CM})^2},
         A_{YY} = -A_{XX}
\end{equation}

The analyzing power does not depend on 
the beam energy.
At $\theta{}_{CM}=90^o$ the analyzing power has its maximum $A_{ZZ}^{max}=7/9$. 
A transverse polarization also leads to an asymmetry, though the analyzing power is 
lower:  $A_{XX}^{max}=A_{ZZ}^{max}/7$. The main purpose of the polarimeter
is to measure the longitudinal component of the beam polarization.
 
The M{\o}ller polarimeter of Hall A detects pairs of scattered electrons in a 
range of $ 75^o < \theta{}_{CM} < 105^o$. The average analyzing power 
is about $<A_{ZZ}>=0.76$.

The target consists of a thin magnetically saturated ferromagnetic foil.
In such a material about 2 electrons per atom can be 
polarized. An average electron polarization of about 8\% can be obtained.
In Hall A M{\o}ller polarimeter the foil is magnetized along its plane and
can be tilted at angles $20-160^o$ to the beam. The effective target 
polarization is 
${\cal P}^{target}={\cal P}^{foil}\cdot{}\cos\theta{}^{target}$.

The secondary electron pairs pass through a magnetic spectrometer
which selects particles in a certain kinematic region. Two electrons
are detected with a two-arm detector and the 
coincidence counting rate of the two arms is measured.

The beam longitudinal polarization is measured as:
\begin{equation}
           {\cal P}^{beam}_{Z} = \frac{N_{+}-N_{-}}{N_{+}+N_{-}}\cdot{}
      \frac{1}{{\cal P}^{foil}\cdot{}\cos\theta{}^{target}\cdot{}<A_{ZZ}>},
\end{equation}
where $N_{+}$ and $N_{-}$ are the measured counting rates with two opposite
mutual orientation of the beam and target polarizations, while 
$<A_{ZZ}>$ is obtained using Monte-Carlo calculation of the M{\o}ller 
spectrometer acceptance, ${\cal P}^{foil}$ is derived from special
magnetization measurements of the foil samples and $\theta{}^{target}$
is measured using a scale, engraved on the target holder and seen
with a TV camera, and also using the counting rates measured at
different target angles.

The target is rotated in the
horizontal plane. The beam polarization may have a horizontal 
transverse component, which would interact with the horizontal transverse 
component of the target polarization. The way to cancel the influence
of the transverse component is to take an average of the asymmetries
measured at 2 complimentary target angles, say 25 and $155^o$. 
}

\infolevone{

\subsection{Description of Components}
\label{sec:moller_compon}

\subsubsection{MEDM Control}
\label{sec:moller_compon_medm}

 Several components of the polarimeter, namely the target position and
 the current in the magnets can be checked
 using the regular MEDM program of Machine Control Center (MCC). The appropriate
 window%
\infolevtwo{
   ~\ref{fig:moller_medm_1}
   \begin{figure}% [htb]
      \begin{center}
          \includegraphics*[angle=90,scale=0.45]{moller_medm_window}
      \end{center}
      \caption{The M{\o}ller target rotary dial.
            }
      \label{fig:moller_medm_1} 
   \end{figure}  
}
can be called from the Hall A MEDM menues. Only the MCC can change
the values in this window.

\subsubsection{Polarized Electron Target }
\label{sec:moller_compon_targ}

The M{\o}ller Polarized Electron Target is
placed on the beamline 17.5 m upstream of the main Hall A
physics target.
\infolevfour{
   The photograph on Fig.~\ref{fig:moller_targ_1} shows the
   target chamber, the beam pipe, Helmholtz coils and other
   elements.
   \begin{figure}% [hbt]
      \begin{center}
          \includegraphics*[angle=0,width=\textwidth]{moller_targ_phot1}
      \end{center}
      \caption{The M{\o}ller target.
            }
      \label{fig:moller_targ_1} 
   \end{figure}  
}

Two target slots exist, called ``bottom'' and ``top''.
For the M{\o}ller target at the moment
a supermendur foil 12~$\mu$m thick is used, positioned in the ``bottom'' 
slot. The target block can be moved vertically from the center,
which contains the hole for the beam, to either ``bottom'' or ``top'' 
position.  
The foil can be tilted to the beam at an angle required,
in a range from $20^o$ to $160^o$.

\infolevtwo{
Both vertical movement of the target and its rotation is controlled by the 
Machine Control Center (MCC)
operators. The vertical movement is controlled by 3 buttons
on the M{\o}ller MEDM display, named ``center'', ``bottom'' and ``top''.
Rotation is controlled by typing an angle, measured in certain units,
in a ``rotary'' window and pressing RETURN. The size of the unit is defined
by the end switches which stop the target rotation at low and high
end points. The distance between these 2 switches is about $140^o$,
and this distance is divided into 80 units. Therefore one unit is about
$1.75^o$. In order to turn the target to $90^o$ one should set a value
of about 38 units. The default rotary position is at 0 units and
the target cannot be moved vertically being at a different position.
}

\begin{safetyen}{20}{20}
Before the target is moved vertically the target
movement should be ``masked'' by the MCC operators, since the vertical
movement may bring thick construction elements into the beam area
and therefore is connected to the Fast Shutdown (FSD). The target rotation
is safer and can be performed without ``masking'' the target movement
and is normally performed without turning off the beam.
A potentially dangerous target rotation to angles lower than $15^o$
is prevented mechanically. 
\end{safetyen}

\infolevtwo{
he target motion is supervised using 2 TV cameras, displayed in
Hall A counting house.
One camera is looking from the side. A glass window in the target 
vacuum box allows one to see the beam area in the place where 
a target can be moved in. The central position (empty) of the target
block is seen as an empty round hole. A target moved in is clearly
visible. The second camera looks from the top at the target
holder. A scale 
}
\infolevfour{
   (Fig.~\ref{fig:moller_targ_2})
   \begin{figure}[hbt]
      \begin{center}
          \includegraphics*[angle=0,scale=0.5]{moller_targ_phot2}
      \end{center}
      \caption{The M{\o}ller target rotary dial.
            }
      \label{fig:moller_targ_2} 
   \end{figure}  
}
\infolevfour{
engraved on the holder shows its angle in degrees. 
This scale gives correct relative angles of the target.
The absolute angle of the target to the beam is measured
using the event rates, measured at a given target angle and
at about $90^o$. At the moment the scale has a shift, that:
$\theta^{target}~\approx~\theta^{scale}-2.5^o$.
}

The target is magnetically saturated using 2 external
Helmholtz coils, providing a field of about 240~Gs along the beam
axis at the target center. The coils are turned on by the M{\o}ller 
CODA task, during the data taking. Its polarity is reversed for each new
run of data taking (one run typically takes 2-3~min).

\begin{safetyen}{20}{20}
The beam of a few $\mu$A may heat up the target locally by 20-40K,
which may change slightly the target polarization. Therefore
the beam current is limited to about 2$\mu$A\footnote{ 
A system for cooling the target with liquid nitrogen has been built. 
However, it has not been used and there are no plans to use it.}
because of heating, while the dead time problems actually
limit the current to about 0.5$\mu$A.
\end{safetyen}

\infolevtwo{
The temperature of the target holder is measured using cernox resistors.

The target magnetization has been measured before the installation\footnote{
The magnetization can be also measured in situ, by changing the field in
the Helmholtz coils and measuring the voltage at pick-up coils, wound
around the target foils, although this method is less accurate
than the lab method and has not been used.}.
}
}

%\cbend
\infolevone{

\subsubsection {Spectrometer Description }
\label{sec:moller_compon_spectr}

The M{\o}ller polarimeter spectrometer consists of three quadrupole magnets and one
 dipole magnet (see Fig.\ref{fig:moller_layout}
 \infolevtwo{
   and also Fig.\ref{fig:moller_medm_1}
 }
 ):
\begin{list}{$\bullet$}{\setlength{\itemsep}{-0.15cm}}
   \item quadrupole {\bf MQM1H02} (on the yoke labeled {\bf $PATSY$});
   \item quadrupole {\bf MQO1H03} (on the yoke labeled {\bf $TESSA$}); 
   \item quadrupole {\bf MQO1H03A} (on the yoke labeled {\bf $FELICIA$})
   \item dipole {\bf MMA1H01} (on the yoke marked as {\bf $University \, of \,
    Kentucky$}).
\end{list} 
\noindent
\infolevfour{
   The photograph on Fig.~\ref{fig:moller_spectr_phot1} shows the side view
   of the spectrometer.
   \begin{figure}% [hbt]
      \begin{center}
          \includegraphics*[angle=0,width=\textwidth]{moller_spectr_phot1}
      \end{center}
      \caption{The M{\o}ller spectrometer. The target is located at the right
            side of the photograph, the blue dipole magnet is close to the center.
            }
      \label{fig:moller_spectr_phot1} 
   \end{figure}  
}

These magnetic elements are controlled by MCC operators.

The spectrometer accepts electrons scattered close to the horizontal 
plane (see Fig.\ref{fig:moller_layout}). The acceptance in the asymuthal
angle is limited by a collimator in front of the dipole magnet, while
the detector vertical size and the magnetic field in the dipole magnet limit 
the acceptance in the scattering angle $\theta_{CM}$.

\infolevtwo{
 The electrons have to pass through
the beam pipe in the 
region of the quads, through the collimator in front of 
the dipole magnet, with a slit of 0.3-4~cm high, 
through two vertical
slits in the dipole, about 2~cm wide, positioned at $\pm$4~cm 
from the beam. These slits are ended with vacuum tight windows
at the end of the dipole. The dipole deflects the scattered 
electrons down, towards the detector. 
The detector, consisting
of 2 arms - 2 vertical columns -  
is positioned such that electrons, scattered at $\theta_{CM}=90^o$ pass
close to its center. 
This acceptance is about $76<\theta_{CM}<104^o$. At beam energies below 1~GeV
the vertical slits in the dipole limit the acceptance to about
$83<\theta_{CM}<97^o$.
}

For a given beam energy there is an optimal setting of the currents in 
these 4 magnets%
\infolevtwo{ (see section~\ref{sec:moller_oper_magset})}.
At energies higher than 2.5~GeV it is possible
to optimize the beam line for both regular running and for 
M{\o}ller measurements. Typically, the dipole magnet should be turned on
only for the M{\o}ller measurements. 

\subsubsection {Detector}
\label{sec:moller_compon_det}

The M{\o}ller polarimeter detector is located in the shielding box downstream 
of the dipole and consists of two identical modules placed symmetrically about a vertical
plane containing the beam axis, thus enabling   
coincidence measurements.
Each part of the detector includes:
\begin{itemize}
  \item An aperture detector consisting of a 28$\times$4cm$\times$1cm 
        scintillator  with Plexiglas light guide and a Hamamatsu R1307 (3 inch)
        photomultiplier tube.
  \item A ``spaghetti'' lead - scintillating fiber calorimeter\footnote{
         before summer 2002 a lead glass calorimeter consisting of 4 
         8$\times$8$\times$30 cm$^3$
         was used. It lost a big fraction of the amplitude due
         to the radiation damage and deterioration of the optical contact.},
        consisting of 2 blocks 9$\times$15$\times$30~cm$^3$, each separated into 2
        channels equipped with Photonis XP2282B (2 inch) photomultiplier tubes. Thus,
        The vertical aperture is segmented into 4 calorimeter channels.
\end{itemize}
The HV crate is located in the Hall A rack 15 and is connected
to a portserver {\em hatsv5}, port 11.
HV for the lead glass detectors is tuned in order to align the M{\o}ller peak
position at a ADC channel 300 for each module, which means that the gain
of the the bottom modules is about 50\% higher than the gain of the top
modules.

\obsolete{
The values of HV are presented in table \ref{tab:moller_hv}.

\begin{table}[ht]
\begin{center}
\begin{tabular}{|r|r|c||r|r|r|r|r|r|} \hline
    & & & \multicolumn{6}{c|}{Voltage (V) for given beam energies (MeV)} 
  \\ \cline{4-9}
      \multicolumn{1}{|r|}{slot}
    & \multicolumn{1}{r|}{\#}
    & \multicolumn{1}{c||}{PMT}
    & 845$^{~}$ & 1645$^{~}$ & 2445$^{~}$ & 3245$^{~}$ & 3334$^{~}$ & 4252$^{~}$  
  \\ \hline
 4  &  0 & LG (L) 1  & 1974 & 1896 & 1851 & 1819 & 1817 & 1790  \\  \hline
 4  &  1 & LG (L) 2  & 1914 & 1835 & 1791 & 1759 & 1757 & 1730  \\  \hline
 4  &  2 & LG (L) 3  & 1909 & 1842 & 1803 & 1775 & 1773 & 1750  \\  \hline
 4  &  3 & LG (L) 4  & 1870 & 1806 & 1770 & 1744 & 1742 & 1720  \\  \hline
 4  &  4 & LG (R) 5  & 1903 & 1814 & 1762 & 1727 & 1724 & 1694  \\  \hline
 4  &  5 & LG (R) 6  & 1897 & 1824 & 1781 & 1751 & 1749 & 1724  \\  \hline
 4  &  6 & LG (R) 7  & 1934 & 1843 & 1790 & 1754 & 1751 & 1720  \\  \hline
 4  &  7 & LG (R) 8  & 1935 & 1865 & 1824 & 1797 & 1794 & 1770  \\  \hline
 4  &  8 & Ap scin L & 1800 & 1800 & 1800 & 1800 & 1800 & 1800  \\  \hline
 4  &  9 & Ap scin R & 1900 & 1900 & 1900 & 1900 & 1900 & 1900  \\  \hline
\end{tabular}
\end{center}
\caption[M{\o}ller Polarimeter: HV Summary]{HV connections and
  HV values. 

}
\label{tab:moller_hv}
\end{table}

In order to obtain the HV values for an arbitrary energy:
\begin{list}{--}{\setlength{\itemsep}{0.cm}}
\item login to {\it adaqs3} as {\it moller}; 
\item $adaqs3>~cd~paw/analysis$, start PAW, select Workstation type 3; 
\item $paw>~exec~sett\_{}magp~e0=3.2$, the required values of $GL$ or $BdL$ are printed. 
\end{list} 
} 


\subsubsection {Electronics}
\label{sec:moller_compon_ele}

The electronics, used for M{\o}ller polarimetry, is located
in several crates in the Hall:
\begin{list}{}{\setlength{\itemsep}{-0.15cm}}
  \item[1.] VME, board computer \mycomp{halladaq14} - for Helmholtz coils control;
  \item[2.] VME, board computer \mycomp{hallavme5} - for DAQ;
  \item[3.] CAMAC - for the trigger and data handling;
  \item[4.] NIM - for the trigger and data handling;
  \item[5.] LeCroy 1450 - HV crate, slot 4.
\end{list}
\noindent
\infolevfour{
   The photograph on Fig.~\ref{fig:moller_electr_phot1} shows the crates 2-4.
   \begin{figure}% [hbt]
      \begin{center}
          \includegraphics*[angle=0,height=20cm]{moller_electronics_phot1}
      \end{center}
      \caption[The M{\o}ller electronics crates.]{The M{\o}ller electronics,
            located in the Hall, at the right side of the beam line. 
            The top crate is the VME DAQ crate, the middle one is the CAMAC
            crate and the bottom one is the NIM crate. The first VME
            crate, used to control the Helmholtz coils, is above these three.
            }
      \label{fig:moller_electr_phot1} 
   \end{figure}  
}
\noindent
One can connect to the CPU boards and the HV crate via a portserver:
\begin{list}{}{\setlength{\itemsep}{-0.15cm}}
  \item[1.] \mycomp{halladaq14} - \mycomp{hatsv5} port \mycomp{3};
  \item[2.] \mycomp{hallavme5}~ - \mycomp{hatsv5} port \mycomp{4};
  \item[5.] LeCroy 1450 - \mycomp{hatsv5} port \mycomp{11}.
\end{list}

\noindent
\infolevfour{
    The M{\o}ller target positions are connected to the Fast ShutDown (FSD)
    system of the accelerator.
   The photograph on Fig.~\ref{fig:moller_fsd_phot1} shows the crate which controls
   the signals.
   \begin{figure}[hbt]
      \begin{center}
          \includegraphics*[angle=0,width=\textwidth]{moller_fsd_phot1}
      \end{center}
      \caption[The M{\o}ller FSD crate.]{The M{\o}ller target
            motion may cause the Fast ShutDown (FSD) of the accelerator.
            The LEDs in the top right corner show the appropriate
            signals from the target. The top LED lit indicates
            no FSD signal.
            }
      \label{fig:moller_fsd_phot1} 
   \end{figure}  
}

\subsubsection {DAQ}
\label{sec:moller_compon_daq}

The DAQ%
\htmladdnormallinkfoot{}{(More details in: \url{http://www.jlab.org/~moller/guide1_linux.html})} 
 is based on CODA\cite{CODAwww} and runs at \mycomp{adaql2},
connecting to \mycomp{hallavme5}. The database server for CODA
is located on \mycomp{adaqs2}. 

\subsubsection {Slow Control}
\label{sec:moller_compon_slow}

The Helmholtz coils are controlled via a script starting automatically
at the beginning of each CODA run. The polarity of the current in the coils
is reversed at every new run.

The HV, the electronics settings and the collimator position
are controlled from a Java program, equipped with a GUI.

 Start the slow control task:
 \begin{list}{--}{\setlength{\itemsep}{-0.15cm}}
   \item Login to \mycomp{adaql1} as \mycomp{moller};
   \item \mycomp{adaql1$>$~cd~Java/msetting/}
   \item \mycomp{adaql1$>$~./mpc} - start the slow control task.
 \end{list}
 It may take about a minute to start all the components and read out
 the proper data from the electronic crates.
\infolevtwo{
  The slow control console is presented on Fig.~\ref{fig:moller_slowc_1}.
   \begin{figure}[htb]
      \begin{center}
          \includegraphics*[angle=0,width=\textwidth]{moller_slowcntrl_window}
      \end{center}
      \caption{The slow control console (Java).
            }
      \label{fig:moller_slowc_1} 
   \end{figure}  
}
 The components are: 
 \begin{list}{--}{\setlength{\itemsep}{-0.15cm}}
   \item \mycomp{EPICS Monitor}: these EPISC variables are stored for every DAQ run
   \item \mycomp{Detector Settings} is used to set up the thresholds, delays etc.
   \item \mycomp{High Voltage Control} for the photomultiplier tubes
   \item \mycomp{Motor Control} to move the collimator
   \item \mycomp{Target Monitor} information on the target position, magnets etc.
 \end{list}
\infolevtwo{

  High voltage can be changed or turne on/off using the HV console (Fig.~\ref{fig:moller_slowc_hv}),
  where the first 8 channels belong to the calorimeter and the other 2 channels
  belong to the aperture counters.
   \begin{figure}[htb]
      \begin{center}
          \includegraphics*[angle=0,width=\textwidth]{moller_slowcntrl_window_hv}
      \end{center}
      \caption{HV control console.
            }
      \label{fig:moller_slowc_hv} 
   \end{figure}  
  The settings of the CAMAC electronics used to make the trigger
  and control DAQ are controlled using the \mycomp{Detector Setting} window (Fig.~\ref{fig:moller_slowc_de}):
   \begin{figure}[htb]
      \begin{center}
          \includegraphics*[angle=0,scale=0.55]{moller_slowcntrl_window_de}
      \end{center}
      \caption{Detector setting console.
            }
      \label{fig:moller_slowc_de} 
   \end{figure}  
   \begin{list}{--}{\setlength{\itemsep}{-0.15cm}}
     \item \mycomp{Delay line} - the delays for the calorimeter and counter signals;
     \item \mycomp{LedDiscriminator} - discriminator thresholds for the calorimeter and the counters
     \item \mycomp{PLU Module} - settings of the logical unit
   \end{list}
  The collimator width can be changed using
  \mycomp{Motor Control} window (Fig.~\ref{fig:moller_slowc_co}),
   \begin{figure}[htb]
      \begin{center}
          \includegraphics*[angle=0,scale=0.55]{moller_slowcntrl_window_co}
      \end{center}
      \caption{Control console for the collimator (and also the slide, which is not relevant
               here).
            }
      \label{fig:moller_slowc_co} 
   \end{figure}  
}

}

\infolevtwo{
\subsection {Operating Procedure }
\label{sec:moller_oper}

The procedure includes general steps as follows: 
\begin{list}{$\bullet$}{\setlength{\itemsep}{-0.15cm}}
  \item ``Non-invasive'' preparations - start the appropriate
          computer processes, turn on the HV and learn the
          magnet settings needed;
  \item ``Invasive'' preparation: beam tuning with the regular
          magnet settings, loading the M{\o}ller settings,
          beam tuning, if neccessary, installing the M{\o}ller target;
  \item Detector check/tuning;
  \item Measurements;
  \item Restoring the regular settings.
\end{list}
The ``non-invasive'' preparations can be done without disturbing the running 
program in the Hall. It is reasonable to perform these preparations
before starting the ``invasive'' part.

In more details, the ``invasive'' procedure looks as follows:
 \begin{list}{--}{\setlength{\itemsep}{-0.15cm}}
   \item Remove the main target;
   \item Tune the beam position with any convenient beam current;
   \item Load the M{\o}ller settings in the magnets; 
   \item Check the beam position;
   \item Tune the beam to $\sim{}0.3~\mu$A for M{\o}ller measurements; 
   \item Pull in the M{\o}ller target;
   \item Make measurements at the forward target angle ($\sim{}23^\circ$);
   \item Make 2 short runs at the normal target angle ($\sim{}90^\circ$);
   \item Make measurements at the backward target angle ($\sim{}163^\circ$);
 \end{list}

\subsubsection {Initialization}
\label{sec:moller_oper_initial}

In order to control the operations several sessions
of \mycomp{moller} account must be opened at computers \mycomp{adaql1,...}.
The data analysis and some initial calculations are done using a PAW\cite{PAWwww}
session on  \mycomp{adaql1}:
 \begin{list}{--}{\setlength{\itemsep}{-0.15cm}}
   \item Login to \mycomp{adaql1} as \mycomp{moller};
   \item \mycomp{adaql1$>$~cd~paw/analysis}, start PAW (type \mycomp{paw}), select Workstation type 3.  
 \end{list}
\noindent
CODA runs on \mycomp{adaql2}: 
 \begin{list}{--}{\setlength{\itemsep}{-0.15cm}}
   \item Login to \mycomp{adaql2} as \mycomp{moller}, make two sessions;
   \item \mycomp{adaql2$>$ kcoda} - clean up the old coda;
   \item \mycomp{adaql2$>$ et\_start moller \&} - start ET if it is not running;
   \item Reset \mycomp{hallavme5} and \mycomp{halladaq14} by pressing two top left green
         reset buttons in the middle room of the counting house;
   \item \mycomp{adaql2$>$ runcontrol} - start CODA;
   \item Click \mycomp{Connect} and select the configuration \mycomp{beam\_pol};
   \item Click \mycomp{Download} to download the program into the VME board.
 \end{list}

\noindent
Slow control:
 \begin{list}{--}{\setlength{\itemsep}{-0.15cm}}
   \item Login to \mycomp{adaql1} as \mycomp{moller};
   \item Start the slow control (see section~\ref{sec:moller_compon_slow});
   \item Load the regular settings and the appropriate HV.
 \end{list}

\subsubsection {Initial Beam Tune}
\label{sec:moller_oper_initbeam}

 Typically, the M{\o}ller measurements are taken during the regular Hall A
 running, when the beam has been tuned for this running. However,
 the M{\o}ller measurements require a different magnetic setting.
 At least the dipole magnet has to be turned on. This magnet
 slightly deflects the beam downward. The deflection at the main target
 could be 2-8~mm, depending on the beam energy. It is, therefore, 
 useful to tune the beam position before the dipole is turned on. It can be done
 before the magnets are set to th M{\o}ller mode.
 The requirements are: 
 \begin{list}{--}{\setlength{\itemsep}{-0.15cm}}
   \item On BPM IPM1H01 (in front of the M{\o}ller target) $|X|<0.2$~mm, $|Y|<0.2$~mm.
   \item On BPM IPM1H04A/B $|X|<2$~mm, $|Y|<2$~mm.
 \end{list}
 The request should be given to MCC. 

\subsubsection {The Magnet Settings}
\label{sec:moller_oper_magset}

 In order to find the proper settings for the given beam energy, say 3.25~GeV, 
 type on the PAW session: \\ 
 \hspace*{0.5cm} \mycomp{PAW$>$~exec~sett\_magp~e0=3.25 nq=3} for 3-quad configuration \\
 \hspace*{0.5cm} \mycomp{PAW$>$~exec~sett\_magp~e0=3.25 nq=2} for 2-quad configuration \\
 The printed values for $GL$ and $BdL$ should be checked with the current values,
 displayed on the MEDM window~\ref{sec:moller_compon_medm}. 
 The MCC should be asked to load the M{\o}ller settings in the magnets -
 they have a tool to load the proper settings for the M{\o}ller
 magnets and a few other magnets on the beam line. 
 The set values should be compared with the calculated values%
 \footnote{The reasonable acuracy in the  magnets settings is about 1-2\%}.
 \begin{safetyen}{5}{2}
   The beam must be turned off when the magnets are tuned.
 \end{safetyen}


\subsubsection {Final Beam Tune}
\label{sec:moller_oper_finalbeam}

 The beam parameters for M{\o}ller measurements
 are: 
   \begin{list}{--}{\setlength{\itemsep}{-0.15cm}}        
     \item Remove the main target;
     \item \begin{safetyen}{5}{0} the beam current $\sim{}0.3~\mu$A and $<2~\mu$A;
            \end{safetyen}
     \item the beam current should be reduced mainly by closing the ``slit''
           in the injector (not by the laser attenuator), in order to
           reduce the effect of current leak-through from the other halls.
   \end{list}


\subsubsection {Target Motion}
\label{sec:moller_oper_target}
 MCC should be asked to put in one of the targets (typically, the \mycomp{bottom}
\infolevtwo{
 (see Fig.~\ref{fig:moller_medm_1})
}
 target is used). The requirements for the target motion are:
 \begin{safetyen}{10}{10}
    \begin{list}{--}{\setlength{\itemsep}{-0.15cm}}
       \item Vertical motion: beam OFF, target motion MASKED.
       \item Rotation: no constraint, the beam can be ON. 
    \end{list}
 \end{safetyen}

 Ask the MCC to resume the same beam and check the Ion Chamber
 {\bf SLD1H03} reading. It should not exceed 1000. At beam energies
 above 1.6 GeV it should not exceed 300.

\subsubsection{Detector Tuning and Checking }
\label{sec:dettune}

The goal is to check that the detector is working, that the counting rates
are normal and that the M{\o}ller peaks are located at about ADC channel 300
for all the calorimeter blocks.

\begin{list}{}{\setlength{\itemsep}{0.5cm}}
  \item[A.] Data taking with CODA
        \begin{list}{}{\setlength{\itemsep}{0.cm}}
             \item[1.] Take a RUN for about 20k events. Let us assume the run 
                       number is 9911.
        \end{list}
  \item[B.] Data analysis with PAW
        \begin{list}{--}{\setlength{\itemsep}{0.cm}}
             \item[1.]  \mycomp{PAW$>$~exec~run~run=9911}: build an NTUPLE and 
                       attach it to the PAW session;
             \item[2.]  \mycomp{PAW$>$~exec~lg\_spectra~icut=60~run=9911}: look at the 
                       ADC distributions. The peaks should be at about ADC channel 300
                       for all 8 modules. If the peaks are off - try to adjust
                       the HV (do not go beyond 1990V).
        \end{list}
  \item[C.] Check of the background
        \begin{list}{}{\setlength{\itemsep}{0.cm}}
             \item[1.] Raise the thresholds to 240~mV of the channels 1 and 2
                       of the discriminator, using the slow control window 
                       (see section~\ref{sec:moller_compon_slow});
             \item[2.] Take a run of about 20k events, say run=9915;
             \item[3.]  \mycomp{PAW$>$~exec~lg\_spectra~icut=60~cut=11~run=9915}: look at the 
                       ADC distributions. The peaks should be at about ADC channel 300
                       for all 8 modules. The histograms 9 and 10 present the sums
                       of the left and right arms.
                       The histogram 11 (sum of both arms) should contain a clean peak at
                       about channel 600;
             \item[4.] \mycomp{PAW$>$~exec~asyms~angl=23.0~run=9915}: polarization analysis should
                       provide a reasonable number. Check the scaler rates per second.
                       The counting rates in each arm should not exceed 600kHz. If they
                       are higher ask the MCC to reduce the beam current.
        \end{list}
\end{list}

\subsubsection {Polarization Measurement }
\label{sec:polmeas}

\begin{list}{$\bullet$}{\setlength{\itemsep}{0.cm}}
  \item[1.] Make a note in the logbook of the target angle on the scale, 
            seen with the TV camera.  
  \item[2.] Take 4 runs of data with the given angle, each run of about 20-30k
            events (30k at $E_{beam}<2$~GeV).
  \item[3.] Ask the MCC to turn the target to 38 units 
            and make a note of the angle on the scale.
  \item[4.] Take 2 runs of data with the given angle, each run of about 15k events.
  \item[5.] Ask the MCC to turn the target to 80 units
            and make a note of the angle on the scale.
  \item[6.] Take 6 runs of data with the given angle, each run of about 20-30k
            events (30k at $E_{beam}<2$~GeV).
  \item[7.] Analyze the data
        \begin{list}{--}{\setlength{\itemsep}{0.cm}}
             \item[1.] \mycomp{PAW$>$~exec~run~run=????} and
             \item[2.] \mycomp{PAW$>$~exec~asyms~run=????~angl=tang}, for each RUN,
                       \mycomp{tang} is the target angle observed on the scale.
             \item[3.] \mycomp{PAW$>$~call~prunpri.f(9000,20000)}, print a table
                       with the results for a given range of runs.
%             \item[4.] \mycomp{PAW$>$~shell~ cp ~runs.tab~ \tilde{}{\hbox{}}moller/public\_html/tab/run.tab.9916-9950}, 
%                       copy the table to a database.
        \end{list}
\end{list}

}
\begin{safetyen}{0}{0}
\subsection {Safety Assessment}
\label{sec:moller_safety}
\end{safetyen}


\begin{safetyen}{50}{50}
\subsubsection{Magnets}

Particular care must be taken in working in the vicinity of the
magnetic elements of the polarimeter as they can have large currents
running in them. Only members of the M{\o}ller polarimeter group are
authorized to work in their immediate vicinity, and only when they are
not energized. The quadrupole magnets and the leads for the dipole magnet 
are protected with Plexiglas shields. As with all elements of the
polarimeter which can
affect the beamline, the magnets are controlled by MCC. There are four
red lights which indicate the status of the magnets. The dipole has two
lights which are activated via a magnetic field sensitive switch placed
on the coils of the dipole. One light is placed on the floor on beam left,
and the other is placed on the raised walkway on beam right. The quadrupoles
have similarly placed lights (one on the floor on beam left and one
on the walkway), and are lit up when any one of the M{\o}ller quads is 
energized. The status of the quadrupole power supplies is on the 
checklist for
closing up Hall A. Lock and tag training is required of all personnel working
in the vicinity of the M{\o}ller magnets. \\ 

The power supply for
the dipole is located in the Beam Switch yard Building (Building 98). 
The maximum current for the dipole is 450A.
The quadrupole power supplies are located in Hall A electronics rack 13,
2 supplies connected in parallel per one quadrupole. The maximum
current per one power supply is 60A at about 20V. 
\\

\noindent
{\bf Vacuum System}\\

One must be careful in working near the downstream side of the dipole
magnet, as there are two 2 by 16 cm, 4 mil thick titanium windows.
Only members of the M{\o}ller polarimeter group should work in this
area.\\

\noindent
{\bf High Voltage}\\

There are 38 photomultiplier tubes within the detector
shielding hut, with a maximum voltage of 3000 V. The detector is serviced
by sliding it back on movable rails. The high voltage must be turned
off during any detector movement. Only members of the M{\o}ller group
should move the detector.\\

\noindent
{\bf Target}\\

To avoid damage to the M{\o}ller target, the target should not be in the beam 
if the beam current is greater than
5 $\mu$A. Only MCC can move the target, but the experimenters
are responsible for ensuring that it is properly positioned.\\
\end{safetyen}


\begin{safetyen}{0}{0}
\subsection{List of Authorized  Personnel}
\end{safetyen}
%All authorized persons must sign next to the correct listed
%name. This signature indicates that they have read and understand this
%OSP.
The list of the presently authorized personnel is given in Tab.~\ref{tab:moller:personnel}.
Other individuals must notify and receive permission from
the contact person (see Tab.~\ref{tab:moller:personnel}) before adding their names 
to the above list.

% include the personnel list
%\begin{table}[ht]
\begin{center}
\begin{tabular}{|ll|l|l|l|l|r|} \hline
  \multicolumn{2}{|c|}{Name} & Dept. & \multicolumn{2}{c|}{Telephone} & 
  \multicolumn{1}{c|}{e-mail} & Comment \\ 
  \cline{4-5}
   &  &   & JLab & Pager &  & \\ 
\hline
% include the personnel list
%\begin{table}[ht]
\begin{center}
\begin{tabular}{|ll|l|l|l|l|r|} \hline
  \multicolumn{2}{|c|}{Name} & Dept. & \multicolumn{2}{c|}{Telephone} & 
  \multicolumn{1}{c|}{e-mail} & Comment \\ 
  \cline{4-5}
   &  &   & JLab & Pager &  & \\ 
\hline
% include the personnel list
%\begin{table}[ht]
\begin{center}
\begin{tabular}{|ll|l|l|l|l|r|} \hline
  \multicolumn{2}{|c|}{Name} & Dept. & \multicolumn{2}{c|}{Telephone} & 
  \multicolumn{1}{c|}{e-mail} & Comment \\ 
  \cline{4-5}
   &  &   & JLab & Pager &  & \\ 
\hline
% include the personnel list
%\input{\dircur/moller_personnel}
 {\em Eugene} & {\em Chudakov}  & JLab    & 6959 & 6959 & gen@jlab.org      & Contact     \\ 
 Alexander    & Glamazdin       & Kharkov & 6378 &      & glamazdi@jlab.org &  \\ 
 Viktor       & Gorbenko        & Kharkov & 6378 &      & gorbenko@jlab.org &  \\ 
 Roman        & Pomatsalyuk     & Kharkov & 6378 &      & romanip@jlab.org  &  \\ 
\hline
\end{tabular}
\end{center}
\caption{Moller Polarimeter: authorized personnel. The primary contact person's
 name is marked with with a slanted font. 
}
\label{tab:moller:personnel}
\end{table}
% ===========  CVS info
% $Header: /group/halla/analysis/cvs/tex/osp/src/beamline/Attic/moller_personnel.tex,v 1.1 2003/06/05 17:28:32 gen Exp $
% $Id: moller_personnel.tex,v 1.1 2003/06/05 17:28:32 gen Exp $
% $Author: gen $
% $Date: 2003/06/05 17:28:32 $
% $Name:  $
% $Locker:  $
% $Log: moller_personnel.tex,v $
% Revision 1.1  2003/06/05 17:28:32  gen
% Initial revision
%

 {\em Eugene} & {\em Chudakov}  & JLab    & 6959 & 6959 & gen@jlab.org      & Contact     \\ 
 Alexander    & Glamazdin       & Kharkov & 6378 &      & glamazdi@jlab.org &  \\ 
 Viktor       & Gorbenko        & Kharkov & 6378 &      & gorbenko@jlab.org &  \\ 
 Roman        & Pomatsalyuk     & Kharkov & 6378 &      & romanip@jlab.org  &  \\ 
\hline
\end{tabular}
\end{center}
\caption{Moller Polarimeter: authorized personnel. The primary contact person's
 name is marked with with a slanted font. 
}
\label{tab:moller:personnel}
\end{table}
% ===========  CVS info
% $Header: /group/halla/analysis/cvs/tex/osp/src/beamline/Attic/moller_personnel.tex,v 1.1 2003/06/05 17:28:32 gen Exp $
% $Id: moller_personnel.tex,v 1.1 2003/06/05 17:28:32 gen Exp $
% $Author: gen $
% $Date: 2003/06/05 17:28:32 $
% $Name:  $
% $Locker:  $
% $Log: moller_personnel.tex,v $
% Revision 1.1  2003/06/05 17:28:32  gen
% Initial revision
%

 {\em Eugene} & {\em Chudakov}  & JLab    & 6959 & 6959 & gen@jlab.org      & Contact     \\ 
 Alexander    & Glamazdin       & Kharkov & 6378 &      & glamazdi@jlab.org &  \\ 
 Viktor       & Gorbenko        & Kharkov & 6378 &      & gorbenko@jlab.org &  \\ 
 Roman        & Pomatsalyuk     & Kharkov & 6378 &      & romanip@jlab.org  &  \\ 
\hline
\end{tabular}
\end{center}
\caption{Moller Polarimeter: authorized personnel. The primary contact person's
 name is marked with with a slanted font. 
}
\label{tab:moller:personnel}
\end{table}
% ===========  CVS info
% $Header: /group/halla/analysis/cvs/tex/osp/src/beamline/Attic/moller_personnel.tex,v 1.1 2003/06/05 17:28:32 gen Exp $
% $Id: moller_personnel.tex,v 1.1 2003/06/05 17:28:32 gen Exp $
% $Author: gen $
% $Date: 2003/06/05 17:28:32 $
% $Name:  $
% $Locker:  $
% $Log: moller_personnel.tex,v $
% Revision 1.1  2003/06/05 17:28:32  gen
% Initial revision
%

\begin{table}[ht]
\begin{center}
\begin{tabular}{|ll|l|l|l|l|r|} \hline
  \multicolumn{2}{|c|}{Name} & Dept. & \multicolumn{2}{c|}{Telephone} & 
  \multicolumn{1}{c|}{e-mail} & Comment \\ 
  \cline{4-5}
   &  &   & JLab & Pager &  & \\ 
\hline
 {\em Eugene} & {\em Chudakov}  & JLab    & 6959 & 6959 & gen@jlab.org      & Contact     \\ 
 Alexander    & Glamazdin       & Kharkov & 6378 &      & glamazdi@jlab.org &  \\ 
 Viktor       & Gorbenko        & Kharkov & 6378 &      & gorbenko@jlab.org &  \\ 
 Roman        & Pomatsalyuk     & Kharkov & 6378 &      & romanip@jlab.org  &  \\ 
\hline
\end{tabular}
\end{center}
\caption{Moller Polarimeter: authorized personnel. The primary contact person's
 name is marked with with a slanted font. 
}
\label{tab:moller:personnel}
\end{table}
 
%Daniel Dale (606-257-2504, Jlab office x5375, pager 3727, e-mail: dale@pa.uky.edu)\\
%Ashot Gasparian (606-257-5565, e-mail: gasparian@jlab.org)\\
%Marie Keese (office x7635, pager x7635, e-mail: keesee@jlab.org)\\
%Roman Pomasalyk (office x6378, e-mail: romanip@jlab.org )\\
%Victor Gorbenko (office x6378, e-mail: gorbenko@jlab.org)\\
%Alexander Glamazdin (office x6378, pager 6370, e-mail: glamazdin@jlab.org))\\
%Eugene Chudakov (Office x6959, pager 5152, e-mail: gen@jlab.org )\\


%\begin{center}
%{ \bf Other individuals must notify and receive permission from
% Eugene Chudakov before adding their names to the above list }
%\end{center}

{\small
\begin{verbatim}CVS $Id: moller.tex,v 1.2 2003/06/05 23:04:19 gen Exp $\end{verbatim}
}



% ===========  CVS info
% $Header: /group/halla/analysis/cvs/tex/osp/src/beamline/moller.tex,v 1.2 2003/06/05 23:04:19 gen Exp $
% $Id: moller.tex,v 1.2 2003/06/05 23:04:19 gen Exp $
% $Author: gen $
% $Date: 2003/06/05 23:04:19 $
% $Name:  $
% $Locker:  $
% $Log: moller.tex,v $
% Revision 1.2  2003/06/05 23:04:19  gen
% Revision date added
%
% Revision 1.1.1.1  2003/06/05 17:28:32  gen
% Imported from /home/gen/tex/OSP
%
