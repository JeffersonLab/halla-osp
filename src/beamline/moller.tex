\chapter[M{\o}ller Polarimeter]{M{\o}ller Polarimeter
\label{sec:moller}
\footnote{
  $CVS~revision~ $Id: moller.tex,v 1.13 2010/06/23 14:32:05 gomez Exp $ $
}
\footnote{Authors: E.Chudakov (\email{gen@jlab.org}), O. Glamazdin (\email{glamazdi@jlab.org}), J. Gomez\hfill\break (\email{gomez@jlab.org})}
}
\section {Purpose and Layout}
\label{sec:moller_purpose}


The polarization of the electron beam coming into Hall A can be masured 
with a M{\o}ller 
polarimeter\htmladdnormallinkfoot{}{(Home page: \url{http://www.jlab.org/~moller/})}. 
The system consists of (see Fig.\ref{fig:moller_layout}),
\vspace{-\parskip}
\begin{itemize}
\item  A magnetized iron foil placed in the beam path. The foil acts as a polarized electron target and it can be selected
from a set of four different foils.
A pair of superconducting Helmholtz coils
($\sim$ 4 T peak field) magnetizes the in-beam foil. The foils are located 17.5 m upstream of the nominal pivot of the 
Hall A High Resolution Spectrometers.
\item A magnetic spectrometer system consisting of three quadrupole magnets and a dipole magnet.
The spectrometer focuses electrons scattered in a certain kinematic range
onto the M{\o}ller detector package. 
\item The detector package and its associated shielding house.
\item An stand-alone data acquisition system.
\item An off-line analysis software package to extract the beam polarization. Roughly, the beam polarization is calculated
by taking the difference in the counting rates of two different beam helicity samples.
\end{itemize}

 \begin{figure}[bht]
    \begin{center}
        \includegraphics*[angle=0,width=0.8\textwidth]{moller_layout_v}
    \end{center}
    \caption[M{\o}ller: layout]{
            Layout of M{\o}ller polarimeter. The origin of the 
            coordinate frame is at the center of the polarimeter
             target, which is 17.5~m upstream of the Hall A target.
            }
    \label{fig:moller_layout} 
 \end{figure}  


%There are also external resources of 
%information\htmladdnormallinkfoot{}{(Home page: \url{http://www.jlab.org/~moller/})}. 

\infolevthree{
\section {Principle of Operation}
\label{sec:moller_principles}

The cross-section of the M{\o}ller scattering 
$ \vec{e^-} + \vec{e^-} \rightarrow e^- + e^-$
depends on the beam and target polarizations ${\cal P}^{beam}$ and 
${\cal P}^{target}$ as:
\begin{equation}
         \sigma \propto ( 1 +  
            \sum_{i=X,Y,Z} (A_{ii}\cdot{}{\cal P}^{targ}_{i}\cdot{}{\cal P}^{beam}_{i})) , 
\end{equation}
where $i = X,Y,Z$ defines the projections of the polarizations. 
The analyzing power $A$ depends on the scattering angle in the CM frame 
$\theta{}_{CM}$. 
Assuming that the beam direction is along the Z-axis and that the scattering
happens in the ZX plane:
\begin{equation}
         A_{ZZ} = 
            -\frac{\sin^2\theta{}_{CM}\cdot(7+\cos^2\theta{}_{CM})}
                                   {(3+\cos^2\theta{}_{CM})^2},
         A_{XX} = -\frac{\sin^4\theta{}_{CM}}
                       {(3 + \cos^2\theta{}_{CM})^2},
         A_{YY} = -A_{XX}
\end{equation}

The analyzing power does not depend on 
the beam energy.
At $\theta{}_{CM}=90^o$ the analyzing power has its maximum $A_{ZZ}^{max}=7/9$. 
A transverse polarization also leads to an asymmetry, though the analyzing power is 
lower:  $A_{XX}^{max}=A_{ZZ}^{max}/7$. The main purpose of the polarimeter
is to measure the longitudinal component of the beam polarization.
 
The M{\o}ller polarimeter of Hall A detects pairs of scattered electrons in a 
range of $ 75^o < \theta{}_{CM} < 105^o$. The average analyzing power 
is about $<A_{ZZ}>=0.76$.

   \begin{figure}[htb]
      \begin{center}
          \includegraphics*[angle=0,width=\textwidth]{Moller_edm1}
      \end{center}
      \caption[M{\o}ller: MCC control screen ]{The M{\o}ller polarimeter control screen used by MCC.
            }
      \label{fig:Moller_edm1} 
   \end{figure}  


The target consists of a thin magnetically saturated ferromagnetic foil.
In such a material about 2 electrons per atom can be 
polarized.
The maximal electron polarization for fully saturated pure iron is 8.52\%.
In Hall A M{\o}ller polarimeter the foil is magnetized by a 3 T field parallel to the beam
axis and perpendicular to the foil plane.

The secondary electron pairs pass through a magnetic spectrometer
which selects particles in a certain kinematic region. Two electrons
are detected with a two-arm detector and the 
coincidence counting rate of the two arms is measured.

The beam longitudinal polarization is measured as:
\begin{equation}
           {\cal P}^{beam}_{Z} = \frac{N_{+}-N_{-}}{N_{+}+N_{-}}\cdot{}
      \frac{1}{{\cal P}^{foil}\cdot{}<A_{ZZ}>},
\end{equation}
where $N_{+}$ and $N_{-}$ are the measured counting rates with two opposite
mutual orientation of the beam and target polarizations, while 
$<A_{ZZ}>$ is obtained using Monte-Carlo calculation of the M{\o}ller 
spectrometer acceptance, ${\cal P}^{foil}$ is derived from special
magnetization measurements in bulk material.


} %infolevthree

\infolevone{


\section{Description of Components}
\label{sec:moller_compon}


\subsection{Control of the M{\o}ller Polarimeter}
\label{sec:moller_compon_medm}

Control of the M{\o}ller polarimeter is devided into two separate sections,
\vspace{-\parskip}
\begin{itemize}
\item The operators in the Machine Control Center (MCC) have sole control over target motion
and currents settings of the quadrupoles \& dipole that make up the magnetic spectrometer
of the polarimeter. Acccess to the Graphical User Interface (GUI)
screens used by MCC to control the polarimeter is done through the menu (``MyMenu'')
provided by the ``NewTools'' application in any of the Hall A
control computers (e.g. hacsbc2). Start by selecting ``EDM (OPS)''.
It opens a new edm window, labeled ``Accelerator Main Menu''.
Press the left-button under the ``Hall A'' entry. It brings up a drop-down menu.
Select ``Hall A Moller Polarimeter Control''. Fig.~\ref{fig:Moller_edm1} shows the final
screen.
\item The user has control over the superconducting magnet current used to magnetically saturate
the in-beam target. See section \ref{SHC_sssection}.
\end{itemize}






\subsection{Polarized Electron Target }
\label{sec:moller_compon_targ}

The M{\o}ller Polarized Electron Target is
located 17.5 m upstream of the main Hall A
physics target (see Fig. \ref{fig:moller_targ_1}). It consists of,
\vspace{-\parskip}
\begin{itemize}
  \item Two superconducting Helmholtz coils coaxial to the beam axis.
  \item A horizontally sliding rail with four (4) iron foils positioned 
        $90^\circ$ to the beam direction (see Fig. \ref{fig:moller_targ_rail}).
        The foils are described in Table \ref{tab:moller-foils}.
\end{itemize}


\subsubsection{Superconducting Hemholtz Coils}\label{SHC_sssection}

A pair of superconducting Hemholtz coils are used to magnetically saturate the in-beam iron foil.
A 3 T field is normally used. The magnet is capable of reaching 4 T.

   \begin{figure}%[htb]
      \begin{center}
          \includegraphics*[angle=0,width=0.9\textwidth]{Moller_edm2}
      \end{center}
      \caption[M{\o}ller: Cryo GUI]{M{\o}ller Cryo GUI}
      \label{fig:Moller_edm2} 
   \end{figure}  

Proper levels of liquid nitrogen and helium are required for the magnet to become superconducting
and remain so while performing a polarimetry measurement. A GUI is provided to monitor
the various temperatures and cryogenic levels of the magnet. It can be accessed
through the menu (``MyMenu'')
provided by the ``NewTools'' application in any of the Hall A
control computers (e.g. hacsbc2). Start by selecting ``EDM (Cryo Main)''.
It opens a window labeled ``CHL Main Menu''.
Pressing the button labeled ``ESR'" brings up the "ESR Menu" screen.
To reach the "Hall A Menu" screen, press the button with that label.
Pressing the "Hall A - DB/Matrix" button will bring a drop-down menu. Select "Hall A Moller"
to bring the screen shown in Fig.~\ref{fig:Moller_edm2}.

The levels of liquid nitrogen and helium are shown as CLL75M110 and CLL75M105 respectively in the cryogenic GUI
of Fig.~\ref{fig:Moller_edm2}. Filling of the magnet helium reservoir is done manually from a dewar
located in the hall. After filling, the helium reservoir lasts several days.
 \begin{safetyen}{0}{0}
 Only personnel designated by the Hall A Work Coordinator is allowed to refill the magnet.
 Notify the Hall A Work coordinator in advance of the need to use the M{\o}ller polarimeter 
 so that the magnet can be refilled at reasonable hours.
 \end{safetyen} 

   \begin{figure}[htb]
      \begin{center}
          \includegraphics*[angle=0,width=0.3\textwidth]{Moller_edm3}
      \end{center}
      \caption[M{\o}ller: control of Hemholtz coils.]{GUI to control the M{\o}ller target superconducting
      Hemholtz colils.}
      \label{fig:Moller_edm3} 
   \end{figure}
   
\infolevfour{
   \begin{figure}%[t]
      \begin{center}
         \includegraphics*[angle=0,width=0.8\textwidth]{ntarget_area}
      \end{center}
      \caption[M{\o}ller:target]{The M{\o}ller target area. The beam propagates from right to left.
      The object portruding towards the reader, is the target holder \& motion mechanism which slides
      the various iron foils in and out of the beam. The blue cylinder contains the
      superconducting magnet helium and nitrogen reservoirs.
            }
      \label{fig:moller_targ_1} 
   \end{figure}  
} % info 4

Power to the superconducting Hemholtz coils is controlled trough
the GUI shown in Fig.~\ref{fig:Moller_edm3}.
The magnet is equipped with a persistent mode switch.
The switch is a superconducting wire placed across the magnet leads
(inside the magnet can) which needs to be made normal conducting before the magnet current itself
can be changed. Press the button labeled ``SDBY'' to make the persistent mode switch normal conducting.
The status field ``HEATER'' should display ``ON''.
Wait for about 30 seconds for the persistent switch to become completely normal
conducting. Pressing afterwards the ``GO SET'' button will ramp the magnet
to the field value set in ``Set Field (Tesla)''. The magnet should not be ramped faster than
4 Amps/minute to 3 T. The ramp up process can be stopped at any time by hitting the ``HOLD''
button.

Use the button ``REM'' to establish communication with the power supply after a power cycle.
The supply is located in the Hall A corridor leading to Hall A. It is an IPS120-10 from OXFORD.
It is a polarity reversible device so, the sign of the field value chosen by the ``Set Field (Tesla)''
is important. To reverse polarity the magnet needs to be ramped to zero first. Use the provided external
gaussmeter to determine that the field is ramping to the desired value and orientation.



\subsubsection{M{\o}ller Target Foils}
Operation of the target slider (see Fig.~\ref{fig:moller_targ_rail}) is accomplished through the edm screen shown in
Fig.~\ref{fig:Moller_edm1}.
Pressing the blue button next to the label "Target Motion Control" in
Fig. ~\ref{fig:Moller_edm1} opens an expert screen (see Fig.~\ref{fig:Moller_edm4}).
\infolevfour{
    \begin{figure}[hbt]
      \begin{center}
          \includegraphics*[angle=0,width=0.8\textwidth]{Moller_edm4}
      \end{center}
      \caption[M{\o}ller: expert target motion control]{M{\o}ller target motion control - expert screen.
             }
      \label{fig:Moller_edm4} 
   \end{figure}  

  \begin{figure}[hbt]
      \begin{center}
          \includegraphics*[angle=0,width=0.8\textwidth]{ntarget_holder}
      \end{center}
      \caption[M{\o}ller:target foils]{The M{\o}ller target holder. It carries
            four ferromagnetic foils, stretched at $90^\circ$ to the beam. The beam comes from the
            bottom of the picture.
            }
      \label{fig:moller_targ_rail} 
   \end{figure}  

\begin{table}[htb]
\begin{center}
\begin{tabular}{|l|c|r|r|r|r|} \hline
          & \multicolumn{5}{c|}{Position} \\ \cline{2-6}
          & 0               &  1  &  2  &  3   &  4   \\
          &  (Retracted Limit) &     &     &      &      \\
%          &  Limit  &     &     &      &      \\
\hline 
  Material            & none & Fe & Fe & Fe & Fe \\
  Thickness ($\mu$m)   &  empty    & 1   &  1   & 4   & 12.5   \\
  Polarization (\%)      &           & 8.52 & 8.52 & 8.52 & 8.52 \\
  Comment             &  Beam  trough position &        &        &        &    \\
%                                &  position  &        &        &        &     \\
\hline
\end{tabular}
\end{center}
\caption[Moller target foils]{Target foil parameters. An ``Extended Limit'' stops the motion
of the target slider a short distance past foil\# 4 and the FSD interlock will not allow
for beam delivery. In the ``Retracted Limit'' (Position 0), the slider edge is several
inches away from the beam.}
\label{tab:moller-foils}
\end{table}
} % info 4




\begin{safetyen}{0}{0}
The target slider can be moved horizontally, placing different foils
into the beam. The CW beam current applied should not exceed $5~\mu$A. For "users mode" the 
beam current aplied should not exceed $15~\mu$A. Before the target is moved the target
movement has to be ``masked'' by the MCC operators, since the target
movement may bring thick construction elements into the beam area
and therefore is connected to the Fast Shutdown (FSD). It is forbidden to move
the target  while the beam is on. Only the MCC operator is allowed to move the target.

The target motion is supervised using TV camera, displayed in
Hall A counting house. The camera shows the horizontal position of the 
target slider. At the ``Retracted'' (default) position the target holder 
with foil 2 is seen in the camera. The camera shows a target number when the target 
is on the beam. A foil surface condition of targets 2, 3 and 4 can be seen in 
the camera.

The target is fully magnetically saturated using 2 superconducting 
Helmholtz coils (see section \ref{SHC_sssection}), providing a field of about 3~T
along the beam axis at the target center. 

The CW beam of $3~\mu$A may heat up the target locally by 20-40 K,
which may change slightly the target polarization (of about 0.3\%). Therefore
the beam current is limited to about $5~\mu$A

\end{safetyen}


\noindent
    The M{\o}ller target positions are connected to the Fast ShutDown (FSD)
    system of the accelerator.
%\infolevfour{
%   The photograph on Fig.~\ref{fig:moller_fsd_phot1} shows the crate which controls
%   the signals.
%   \begin{figure}[hbt]
%      \begin{center}
%          \includegraphics*[angle=0,width=0.8\textwidth]{Moller_FSD}
%      \end{center}
%      \caption[M{\o}ller: FSD crate.]{The M{\o}ller target
%            motion may cause the Fast ShutDown (FSD) of the accelerator.
%            The LEDs in the top right corner show the appropriate
%            signals from the target. The top LED lit indicates
%            no FSD signal.
%            }
%      \label{fig:moller_fsd_phot1} 
%   \end{figure}  
%}
} % infolevone
%\cbend


\infolevone{

\subsection {Spectrometer Description }
\label{sec:moller_compon_spectr}

The M{\o}ller polarimeter spectrometer consists of three quadrupole magnets and one
 dipole magnet (see Fig.\ref{fig:moller_layout}
 \infolevtwo{
   and also Fig.\ref{fig:Moller_edm1}
 }
 ):
\begin{list}{$\bullet$}{\setlength{\itemsep}{-0.15cm}}
   \item quadrupole {\bf MQM1H02} (on the yoke labeled {\bf $PATSY$});
   \item quadrupole {\bf MQO1H03} (on the yoke labeled {\bf $TESSA$}); 
   \item quadrupole {\bf MQO1H03A} (on the yoke labeled {\bf $FELICIA$})
   \item dipole {\bf MMA1H01} (on the yoke marked as {\bf $University \, of \,
    Kentucky$}).
\end{list} 
\noindent
\infolevfour{
   The photograph on Fig.~\ref{fig:moller_spectr_phot1} shows the side view
   of the spectrometer.
   \begin{figure}% [hbt]
      \begin{center}
          \includegraphics*[angle=0,width=0.8\textwidth]{moller_spectr_phot1}
      \end{center}
      \caption[M{\o}ller:spectrometer]{The M{\o}ller spectrometer. The target is located at the right
            side of the photograph, the blue dipole magnet is close to the center.
            }
      \label{fig:moller_spectr_phot1} 
   \end{figure}  
}

These magnetic elements are controlled by MCC operators.

The spectrometer accepts electrons scattered close to the horizontal 
plane (see Fig.\ref{fig:moller_layout}). The acceptance in the azimuthal
angle is limited by a collimator in front of the dipole magnet, while
the detector vertical size and the magnetic field in the dipole magnet limit 
the acceptance in the scattering angle $\theta_{CM}$.

\infolevtwo{
 The electrons have to pass through the beam pipe in the 
region of the quads, through the collimator in front of 
the dipole magnet see Fig.~\ref{fig:moller_collimator}), with a slit of 
0.3-4~cm high, through two vertical
slits in the dipole, about 2~cm wide, positioned at $\pm$4~cm 
from the beam. These slits are terminated with vacuum tight windows
at the end of the dipole. The dipole deflects the scattered 
electrons down, towards the detector. The detector, consisting
of 2 arms - 2 vertical columns -  is positioned such that electrons, 
scattered at $\theta_{CM}=90^o$ pass close to its center. 
This acceptance is about $76<\theta_{CM}<104^o$. 

   \begin{figure}% [hbt]
      \begin{center}
          \includegraphics*[angle=0,width=0.8\textwidth]{collimator}
      \end{center}
      \caption[M{\o}ller:collimator]{The M{\o}ller dipole collimator is located in front 
              of the dipole. The M{\o}ller electrons pass through  right and left 
             slits of the collimator. The electron beam passes through the collimator 
             in the  center.  
            }
      \label{fig:moller_collimator} 
   \end{figure}  
}

For a given beam energy there is an optimal setting of the currents in 
these 4 magnets%
\infolevtwo{ (see section~\ref{sec:moller_oper_magset})}.
The dipole magnet should be turned on
only for the M{\o}ller measurements. 

\subsection {Detector}
\label{sec:moller_compon_det}

The M{\o}ller polarimeter detector is located in the shielding box downstream 
of the dipole and consists of two identical modules placed symmetrically about a vertical
plane containing the beam axis, thus enabling   
coincidence measurements.
Each part of the detector includes:
\begin{itemize}
  \item An aperture detector consists of four  
        scintillators with light guides and  Hamamatsu R4124 (13~mm diameter)
        photomultiplier tubes connected to each segment. Size of the aperture 
        assembled detector is 31$\times$4cm$\times$3.6cm.
  \item A ``spaghetti'' lead - scintillating fiber calorimeter
        consisting of 2 blocks 9$\times$15$\times$30~cm$^3$, each separated into 2
        channels equipped with Photonis XP2282B (2 inch) photomultiplier tubes. Thus,
        the vertical aperture is segmented into 4 calorimeter channels.
\end{itemize}
The HV crate is located in the Hall A rack 15 and is connected
to a portserver {\em hatsv5}, port 11.
HV for the lead glass detectors is tuned in order to align the M{\o}ller peak
position at a ADC channel 300 for each module, which means that the gain
of the the bottom modules is about 50\% higher than the gain of the top
modules.

\obsolete{
The values of HV are presented in table \ref{tab:moller_hv}.

\begin{table}[ht]
\begin{center}
\begin{tabular}{|r|r|c||r|r|r|r|r|r|} \hline
    & & & \multicolumn{6}{c|}{Voltage (V) for given beam energies (MeV)} 
  \\ \cline{4-9}
      \multicolumn{1}{|r|}{slot}
    & \multicolumn{1}{r|}{\#}
    & \multicolumn{1}{c||}{PMT}
    & 845$^{~}$ & 1645$^{~}$ & 2445$^{~}$ & 3245$^{~}$ & 3334$^{~}$ & 4252$^{~}$  
  \\ \hline
 4  &  0 & LG (L) 1  & 1974 & 1896 & 1851 & 1819 & 1817 & 1790  \\  \hline
 4  &  1 & LG (L) 2  & 1914 & 1835 & 1791 & 1759 & 1757 & 1730  \\  \hline
 4  &  2 & LG (L) 3  & 1909 & 1842 & 1803 & 1775 & 1773 & 1750  \\  \hline
 4  &  3 & LG (L) 4  & 1870 & 1806 & 1770 & 1744 & 1742 & 1720  \\  \hline
 4  &  4 & LG (R) 5  & 1903 & 1814 & 1762 & 1727 & 1724 & 1694  \\  \hline
 4  &  5 & LG (R) 6  & 1897 & 1824 & 1781 & 1751 & 1749 & 1724  \\  \hline
 4  &  6 & LG (R) 7  & 1934 & 1843 & 1790 & 1754 & 1751 & 1720  \\  \hline
 4  &  7 & LG (R) 8  & 1935 & 1865 & 1824 & 1797 & 1794 & 1770  \\  \hline
 4  &  8 & Ap scin L & 1800 & 1800 & 1800 & 1800 & 1800 & 1800  \\  \hline
 4  &  9 & Ap scin R & 1900 & 1900 & 1900 & 1900 & 1900 & 1900  \\  \hline
\end{tabular}
\end{center}
\caption[M{\o}ller Polarimeter: HV Summary]{HV connections and
  HV values. 

}
\label{tab:moller_hv}
\end{table}

In order to obtain the HV values for an arbitrary energy:
\begin{list}{--}{\setlength{\itemsep}{0.cm}}
\item login to {\it adaqs3} as {\it moller}; 
\item $adaqs3>~cd~paw/analysis$, start PAW, select Workstation type 3; 
\item $paw>~exec~sett\_{}magp~e0=3.2$, the required values of $GL$ or $BdL$ are printed. 
\end{list} 
} 


\subsection {Electronics}
\label{sec:moller_compon_ele}

The electronics, used for M{\o}ller polarimetry, is located
in several crates in the Hall:
\begin{list}{}{\setlength{\itemsep}{-0.15cm}}
  \item[1.] VME, board computer \textcolor{blue}{hallavme5} - for DAQ;
  \item[2.] CAMAC - for the trigger and data handling;
  \item[3.] NIM - for the trigger and data handling;
  \item[4.] LeCroy 1450 - HV crate, slot 5.
\end{list}
\noindent
\infolevfour{
   The photograph on Fig.~\ref{fig:moller_electr_phot1} shows the crates 2-4.
   \begin{figure}% [hbt]
      \begin{center}
          \includegraphics*[angle=0,height=0.8\textheight]{moller_electronics_phot1}
      \end{center}
      \caption[M{\o}ller: electronics crates.]{The M{\o}ller electronics,
            located in the Hall, at the right side of the beam line. 
            The top crate is the VME DAQ crate, the middle one is the CAMAC
            crate and the bottom one is the NIM crate.
            }
      \label{fig:moller_electr_phot1} 
   \end{figure}  
}
\noindent
One can connect to the CPU boards and the HV crate via a portserver:
\begin{list}{}{\setlength{\itemsep}{-0.15cm}}
  \item[1.] \textcolor{blue}{hallavme5}~ - \textcolor{blue}{hatsv5} port \textcolor{blue}{4};
  \item[4.] LeCroy 1450 - \textcolor{blue}{hatsv5} port \textcolor{blue}{3}.
\end{list}

\subsection {DAQ}
\label{sec:moller_compon_daq}

The DAQ%
\htmladdnormallinkfoot{}{(More details in: \url{http://www.jlab.org/~moller/guide1.2_linux.html})} 
 is based on CODA\cite{CODAwww} and runs at \textcolor{blue}{adaql2},
connecting to \textcolor{blue}{hallavme5}. The ``msql'' database server for CODA
is located on \textcolor{blue}{adaql1}. 

\subsection {Slow Control}
\label{sec:moller_compon_slow}

The HV, the electronics settings and the collimator position
are controlled from a Java program, equipped with a GUI%
\footnote{More details in \url{http://www.jlab.org/~moller/slow_mpc.html}}.

 Start the slow control task:
 \begin{list}{--}{\setlength{\itemsep}{-0.15cm}}
   \item Login to \textcolor{blue}{adaql1} as \textcolor{blue}{moller};
   \item \textcolor{blue}{adaql1$>$~cd~Java/msetting/}
   \item \textcolor{blue}{adaql1$>$~./mpc} - start the slow control task.
 \end{list}
 It may take about a minute to start all the components and read out
 the proper data from the electronic crates.
\infolevtwo{
  The slow control console is presented on Fig.~\ref{fig:moller_slowc_1}.
   \begin{figure}[htb]
      \begin{center}
          \includegraphics*[angle=0,width=0.8\textwidth]{moller_slowcntrl_window}
      \end{center}
      \caption[M{\o}ller:slow control window]{The slow control console (Java).
            }
      \label{fig:moller_slowc_1} 
   \end{figure}  
}
 The components are: 
 \begin{list}{--}{\setlength{\itemsep}{-0.15cm}}
   \item \textcolor{blue}{EPICS~\cite{EPICSwww} Monitor}: these EPICS variables are stored for every DAQ run
   \item \textcolor{blue}{Detector Settings} is used to set up the thresholds, delays etc.
   \item \textcolor{blue}{High Voltage Control} for the photomultiplier tubes
   \item \textcolor{blue}{Motor Control} to move the collimator
   \item \textcolor{blue}{Target Monitor} information on the target position, magnets etc.
 \end{list}
\infolevtwo{

  High voltage can be changed or turned on/off using the HV console (Fig.~\ref{fig:moller_slowc_hv}),
  where the first 8 channels belong to the calorimeter and the other 2 channels
  belong to the aperture counters.
   \begin{figure}[htb]
      \begin{center}
          \includegraphics*[angle=0,width=0.8\textwidth]{moller_slowcntrl_window_hv}
      \end{center}
      \caption[M{\o}ller:HV control]{HV control console.
            }
      \label{fig:moller_slowc_hv} 
   \end{figure}  
  The settings of the CAMAC electronics used to make the trigger
  and control DAQ are controlled using the \textcolor{blue}{Detector Setting} window (Fig.~\ref{fig:moller_slowc_de}):
   \begin{figure}[htb]
      \begin{center}
          \includegraphics*[angle=0,scale=0.55]{moller_slowcntrl_window_de}
      \end{center}
      \caption[M{\o}ller: electronics control]{Detector setting console.
            }
      \label{fig:moller_slowc_de} 
   \end{figure}  
   \begin{list}{--}{\setlength{\itemsep}{-0.15cm}}
     \item \textcolor{blue}{Delay line} - the delays for the calorimeter and counter signals;
     \item \textcolor{blue}{LedDiscriminator} - discriminator thresholds for the calorimeter and the counters
     \item \textcolor{blue}{PLU Module} - settings of the logical unit
   \end{list}
  The collimator width can be changed using
  \textcolor{blue}{Motor Control} window (Fig.~\ref{fig:moller_slowc_co}).
   \begin{figure}[htb]
      \begin{center}
          \includegraphics*[angle=0,scale=0.55]{moller_slowcntrl_window_co}
      \end{center}
      \caption[M{\o}ller: collimator control]{Control console for the collimator 
               (and also the slide, which is not relevant here).
            }
      \label{fig:moller_slowc_co} 
   \end{figure}  
}

}

\infolevtwo{
\section {Operating Procedure }
\label{sec:moller_oper}

The procedure includes general steps as follows: 
\begin{list}{$\bullet$}{\setlength{\itemsep}{-0.15cm}}
  \item ``Non-invasive'' preparations - start the appropriate
          computer processes, turn on the HV and learn the
          magnet settings needed;
  \item ``Invasive'' preparation: beam tuning with the regular
          magnet settings, loading the M{\o}ller settings,
          beam tuning, if neccessary, installing the M{\o}ller target;
  \item Detector check/tuning;
  \item Measurements;
  \item Restoring the regular settings.
\end{list}
The ``non-invasive'' preparations can be done without disturbing the running 
program in the Hall. It is reasonable to perform these preparations
before starting the ``invasive'' part.

In more details, the ``invasive'' procedure looks as follows:
 \begin{list}{--}{\setlength{\itemsep}{-0.15cm}}
   \item Remove the main target;
   \item Load the M{\o}ller settings in the magnets, keep the dipole off; 
   \item Tune the beam position with any convenient beam current;
   \item Turn on the M{\o}ller dipole; 
   \item Check the beam position;
   \item Tune the beam to $\sim{}0.5~\mu$A for M{\o}ller measurements; 
   \item Pull in the M{\o}ller target, using the TV camera to make sure the foil is at the
         window center;
   \item Make at least two CODA runs in order to use both coil polarities;
%   \item Make measurements at the forward target angle ($\sim{}23^\circ$);
%   \item Make 2 short runs at the normal target angle ($\sim{}90^\circ$);
%   \item Make measurements at the backward target angle ($\sim{}163^\circ$);
 \end{list}

\subsection {Initialization}
\label{sec:moller_oper_initial}

In order to control the operations several sessions
of \textcolor{blue}{moller} account must be opened at computers \textcolor{blue}{adaql1,...}.
The data analysis and some initial calculations are done using a PAW\cite{PAWwww}
session on  \textcolor{blue}{adaql1}:
 \begin{list}{--}{\setlength{\itemsep}{-0.15cm}}
   \item Login to \textcolor{blue}{adaql1} as \textcolor{blue}{moller};
   \item \textcolor{blue}{adaql1$>$~cd~paw/analysis}, start PAW (type \textcolor{blue}{paw}),
   select Workstation type 3.  
 \end{list}
\noindent
Check that the portserver connections are available:
 \begin{list}{--}{\setlength{\itemsep}{-0.15cm}}
   \item Try \textcolor{blue}{telnet hatsv5 2003} and \textcolor{blue}{telnet hatsv5 2004};
   \item If a connection i refused - clean it up, by connecting
         \textcolor{blue}{telnet hatsv5} as root and typing \textcolor{blue}{kill 3} or
         \textcolor{blue}{kill 4},
         see instructions in \textcolor{blue}{\~adaq/doc/portserver.doc}.    
 \end{list}
\noindent
 \begin{list}{--}{\setlength{\itemsep}{-0.15cm}}
   \item Login to \textcolor{blue}{adaql1} as \textcolor{blue}{moller};
   \item Start the slow control (see section~\ref{sec:moller_compon_slow});
   \item Load the regular settings and the appropriate HV.
 \end{list}
Slow control:
 \begin{list}{--}{\setlength{\itemsep}{-0.15cm}}
   \item Login to \textcolor{blue}{adaql1} as \textcolor{blue}{moller};
   \item Start the slow control (see section~\ref{sec:moller_compon_slow});
   \item Load the regular settings and the appropriate HV.
 \end{list}
\noindent
CODA runs on \textcolor{blue}{adaql1}: 
 \begin{list}{--}{\setlength{\itemsep}{-0.15cm}}
   \item Login to \textcolor{blue}{adaql1} as \textcolor{blue}{moller}, make two sessions;
   \item \textcolor{blue}{adaql2$>$ kcoda} - clean up the old coda;
   \item Reset the VME board \textcolor{blue}{hallavme5} by: 
   \textcolor{blue}{telnet hatsv5 2004}, \textcolor{blue}{-$>$ reboot};  
   \item \textcolor{blue}{adaql2$>$ start\_coda} - start CODA;
   \item Click \textcolor{blue}{Connect} and select the configuration \textcolor{blue}{beam\_pol};
   \item Click \textcolor{blue}{Download} to download the program into the VME board.
 \end{list}

\subsection {Initial Beam Tune}
\label{sec:moller_oper_initbeam}

 Typically, the M{\o}ller measurements are taken during the regular Hall A
 running, when the beam has been tuned for this running. However,
 the M{\o}ller measurements require a different magnetic setting.
 At least the dipole magnet has to be turned on. This magnet
 slightly deflects the beam downward. The deflection at the main target
 could be 2-8~mm, depending on the beam energy. It is, therefore, 
 useful to tune the beam position before the dipole is turned on. It can be done
 before the magnets are set to the M{\o}ller mode.
 The requirements are: 
 \begin{list}{--}{\setlength{\itemsep}{-0.15cm}}
   \item On BPM IPM1H01 (in front of the M{\o}ller target) $|X|<0.2$~mm, $|Y|<0.2$~mm.
   \item On BPM IPM1H04A/B $|X|<2$~mm, $|Y|<2$~mm.
 \end{list}
 The request should be given to MCC. 

\subsection {The Magnet Settings}
\label{sec:moller_oper_magset}

 In order to find the proper settings for the given beam energy, say 3.25~GeV, 
 type on the PAW session: \\ 
 \hspace*{0.5cm} \textcolor{blue}{PAW$>$~exec~sett\_magp~e0=3.25 nq=3} for 3-quad configuration \\
 \hspace*{0.5cm} \textcolor{blue}{PAW$>$~exec~sett\_magp~e0=3.25 nq=2} for 2-quad configuration \\
 The printed values for $GL$ and $BdL$ should be checked with the current values,
 displayed on the MEDM window~\ref{sec:moller_compon_medm}. 
 The MCC should be asked to load the M{\o}ller settings in the magnets -
 they have a tool to load the proper settings for the M{\o}ller
 magnets and a few other magnets on the beam line. 
 The set values should be compared with the calculated values%
 \footnote{A reasonable accuracy in the  magnets settings is about 1-2\%}.
 \begin{safetyen}{5}{2}
   The beam must be turned off when the magnets are being changed.
 \end{safetyen}


\subsection {Final Beam Tune}
\label{sec:moller_oper_finalbeam}

 The beam parameters for M{\o}ller measurements
 are: 
   \begin{list}{--}{\setlength{\itemsep}{-0.15cm}}        
     \item \begin{safetyen}{5}{0} the beam current $\sim{}0.5~\mu$A and $<5~\mu$A;
            \end{safetyen}
     \item the beam current should be reduced mainly by closing the ``slit''
           in the injector (not by the laser attenuator), in order to
           reduce the effect of current leak-through from the other halls.
   \end{list}


\subsection {Target Motion}
\label{sec:moller_oper_target}
 The procedure is as follows:
% MCC should be asked to: %put in one of the targets (typically, the \mycomp{bottom}
   \begin{list}{--}{\setlength{\itemsep}{-0.15cm}}        
     \item Ask the MCC to mask the main target (cryotarget or whatever) motion
           and remove the main target, then ask the MCC to unmask the motion;
     \item Ask the MCC to mask the M{\o}ller target motion;
     \item Move to target to the position needed (say, 4) using
           the EDM screen \infolevtwo{(see Fig.~\ref{fig:Moller_edm1})}.
           Check that the target is close to the center of the window
           in the TV camera screen. 
   \end{list}

%\infolevtwo{
% (see Fig.~\ref{fig:moller_medm_1})
%}
% target is used). 
%The requirements for the target motion are:
% \begin{safetyen}{10}{10}
%    \begin{list}{--}{\setlength{\itemsep}{-0.15cm}}
%       \item Vertical motion: beam OFF, target motion MASKED.
%       \item Rotation: no constraint, the beam can be ON. 
%    \end{list}
% \end{safetyen}

 %Ask the MCC to resume the same beam and check the Ion Chamber
 %{\bf SLD1H03} reading. It should not exceed 1000. At beam energies
 %above 1.6 GeV it should not exceed 300.

\subsection{Detector Tuning and Checking }
\label{sec:dettune}

The goal is to check that the detector is working, that the counting rates
are normal and that the M{\o}ller peaks are located at about ADC channel 300
for all the calorimeter blocks.

\begin{list}{}{\setlength{\itemsep}{0.5cm}}
  \item[A.] Data taking with CODA
        \begin{list}{}{\setlength{\itemsep}{0.cm}}
             \item[1.] Take a RUN for about 20k events. Let us assume the run 
                       number is 9911.
        \end{list}
  \item[B.] Data analysis with PAW
        \begin{list}{--}{\setlength{\itemsep}{0.cm}}
             \item[1.]  \textcolor{blue}{PAW$>$~exec~run~run=9911}: build an NTUPLE and 
                       attach it to the PAW session;
             \item[2.]  \textcolor{blue}{PAW$>$~exec~lg\_spectra~icut=60~run=9911}: look at the 
                       ADC distributions. The peaks should be at about ADC channel 300
                       for all 8 modules. If the peaks are off - try to adjust
                       the HV (do not go beyond 1990V).
        \end{list}
  \item[C.] Check of the background
        \begin{list}{}{\setlength{\itemsep}{0.cm}}
             \item[1.] Raise the thresholds to 240~mV of the channels 1 and 2
                       of the discriminator, using the slow control window 
                       (see section~\ref{sec:moller_compon_slow});
             \item[2.] Take a run of about 20k events, say run=9915;
             \item[3.]  \textcolor{blue}{PAW$>$~exec~lg\_spectra~icut=60~cut=11~run=9915}: look at the 
                       ADC distributions. The peaks should be at about ADC channel 300
                       for all 8 modules. The histograms 9 and 10 present the sums
                       of the left and right arms.
                       The histogram 11 (sum of both arms) should contain a clean peak at
                       about channel 600;
             \item[4.] \textcolor{blue}{PAW$>$~exec~asyms~angl=23.0~run=9915}: polarization analysis should
                       provide a reasonable number. Check the scaler rates per second.
                       The counting rates in each arm should not exceed 600kHz. If they
                       are higher ask the MCC to reduce the beam current.
        \end{list}
\end{list}

\subsection {Polarization Measurement }
\label{sec:polmeas}

\begin{list}{$\bullet$}{\setlength{\itemsep}{0.cm}}
%  \item[1.] Make a note in the logbook of the target angle on the scale, 
%            seen with the TV camera.  
  \item[1.] Take an even (say, 2) number  of runs of data with the given angle, each run of about 20-30k
            events (30k at $E_{beam}<2$~GeV).
%  \item[3.] Ask the MCC to turn the target to 38 units 
%            and make a note of the angle on the scale.
%  \item[4.] Take 2 runs of data with the given angle, each run of about 15k events.
%  \item[5.] Ask the MCC to turn the target to 80 units
%            and make a note of the angle on the scale.
%  \item[6.] Take 6 runs of data with the given angle, each run of about 20-30k
%            events (30k at $E_{beam}<2$~GeV).
  \item[2.] Analyze the data
        \begin{list}{--}{\setlength{\itemsep}{0.cm}}
             \item[1.] \textcolor{blue}{PAW$>$~exec~run~run=????} and
             \item[2.] \textcolor{blue}{PAW$>$~exec~asymt~run=????~angl=tang}, for each RUN,
                       \textcolor{blue}{tang} is the target angle observed on the scale.
             \item[3.] \textcolor{blue}{PAW$>$~call~prunpri.f(9000,20000)}, print a table
                       with the results for a given range of runs.
%             \item[4.] \mycomp{PAW$>$~shell~ cp ~runs.tab~ \tilde{}{\hbox{}}moller/public\_html/tab/run.tab.9916-9950}, 
%                       copy the table to a database.
        \end{list}
\end{list}

} %levtwo

\begin{safetyen}{0}{0}
\section {Safety Assessment}
\label{sec:moller_safety}
\end{safetyen}


\begin{safetyen}{0}{0}
\subsection{Magnets}

Particular care must be taken in working in the vicinity of the
magnetic elements of the polarimeter as they can have large currents
running in them. The quadrupole magnets and the leads for the dipole magnet 
are protected with Plexiglas shields. The leads for the superconducting Hemholtz
coils are encased in a plastic hood. Removing the shields or hood
can only be done by Hall A technical staff, with the power supplies
turned off and using the ``Lock out / Tag out'' procedure. All the personnel involved
must have ``Lock out / Tag out'' training.
Only members of the M{\o}ller polarimeter group and Hall A technical staff are
authorized to work in the immediate vicinity of the magnets with the shields
removed.

As with all elements of the
polarimeter which can
affect the beamline, the magnets are controlled by MCC. There are four
red lights which indicate the status of the magnets. The dipole has two
lights which are activated via a magnetic field sensitive switch placed
on the coils of the dipole. One light is placed on the floor on beam left,
and the other is placed on the raised walkway on beam right. The quadrupoles
have similarly placed lights (one on the floor on beam left and one
on the walkway), and are lit up when any one of the M{\o}ller quads is 
energized. The status of the quadrupole power supplies is on the 
checklist for
closing up Hall A.

The superconducting Hemholtz coils are remotely controlled by the users.
A red beacon indicates that the magnet could be energized at any moment.

The power supply (62~V, 500~A rating) for
the dipole is located in the Beam Switch yard Building (Building 98). 
The maximum current for the dipole is 450A. The
quadrupole power supplies (40~V, 330~A rating) are located in Hall A electronics rack 13.
The power supply (10~V, 120~A) for the superconducting Hemholtz coils is located in the Hall A corridor.

\subsection{Magnetic field}
The superconducting Hemholtz coils are capable of producing a 4 T maximum field (normally a 3 T field
is used). It has no return yoke for the field lines but it has a relatively small warm bore of 76 mm.
Calculations and measurements by OXFORD indicate that for a 4 T bore field,
the free-air 5 Gauss boundary is located at 1.8 meters
from the magnet axis.
The possible presence of high magnetic fields will be indicated by
standard Jefferson Lab signs and by a flashing beacon. The area surrounding 
the magnet will be roped-off to indicate the 5 Gauss boundary
whenever it is possible that the magnet be 
energized.
Personnel with ferromagnetic or metal implants and those 
wearing electronic medical devices are not allowed inside the roped-off area. 
Personnel working inside the roped-off area 
should be aware of the possible presence of a magnetic fringe field.

The presence of magnetic material
represents an additional hazard when the magnet is operating.
Loose magnetic material could be pulled into the magnet by the magnetic field it generates.
All such materials must 
be removed and placed outside of the ropes.

A superconducting magnet could quench bringing a rapid collapse of the magnetic field.
The rapid change of the magnetic field will induce Eddy currents in nearby metal objects
experiencing an attractive force during a magnet quench. So, all loose conductive materials
must be also removed and placed outside the ropes.

\subsection{Cryogenics}
The supeconducting Hemholtz coils make use of liquid nitrogen and helium to operate.
 Only personnel designated by the Hall A Work Coordinator is allowed to cool, warmup
 or refill the magnet. These personnel must wear the proper Personal Protective Equipment (PPE) to handle cryogens and follow
 an approved work procedure.

\subsection{Vacuum System}

One must be careful in working near the downstream side of the dipole
magnet, as there are two 2 by 16 cm, 4 mil thick titanium windows.
The windows are partially protected by a lead collimator downstream
of the dipole. 
Only members of the M{\o}ller polarimeter group should work in this
area. 
If work is to be done on the collimators, the appropriate ear
and eye protection should be used.

\subsection{High Voltage}

There are 16 photomultiplier tubes within the detector
shielding hut, with a maximum voltage of 3000 V. The detector is serviced
by sliding it back on movable rails. The high voltage must be turned
off during any detector movement. Only members of the M{\o}ller group
should move the detector.\\

\subsection{Target}

To avoid damage to the M{\o}ller target, the target should not be in the beam 
if the beam current is greater than
5 $\mu$A. 
%Only MCC can move the target, but 
The experimenters
are responsible for ensuring that the M{\o}ller target is removed from the beam
for regular running and that its position is unmasked. \\

\end{safetyen}

\vfill\eject

% include the personnel list
%\begin{safetyen}{0}{0}
\section[Authorized  Personnel]{Authorized  Personnel
\label{sec:moller-pers}
\footnote{
   $CVS~revision~ $Id: moller-personnel.tex,v 1.6 2010/06/23 14:32:05 gomez Exp $ $
 }
\footnote{Authors: E. Chudakov (\email{gen@jlab.org}), O.Glamazdin (\email{glamazdi@jlab.org}), J. Gomez\hfill\break (\email{gomez@jlab.org})}
}
%\end{safetyen}
%All authorized persons must sign next to the correct listed
%name. This signature indicates that they have read and understand this
%OSP.
The list
of the presently authorized personnel is given in Table \ref{tab:moller:personnel}.
Other individuals must notify and receive permission from
the contact person (see Table \ref{tab:moller:personnel}) before adding their names 
to the above list.
%\obsolete{
\begin{table}[ht]
\begin{center}
\begin{tabular}{|ll|l|l|l|l|r|} \hline
  \multicolumn{2}{|c|}{Name} & Dept. & \multicolumn{2}{c|}{Telephone} & 
  \multicolumn{1}{c|}{e-mail} & Comment \\ 
  \cline{4-5}
   &  &   & JLab & Pager &  & \\ 
\hline
 Oleksandr    & Glamazdin       & Kharkov & 5441 & 5441 & glamazdi@jlab.org &  \\ 
  Javier & Gomez  & JLab    & 7498 & 7498 & gomez@jlab.org      & Primary contact     \\ 
Viktor       & Gorbenko        & Kharkov & 5441 &   -  & gorbenko@jlab.org &  \\ 
 Roman        & Pomatsalyuk     & Kharkov & 5395 & 0001 & romanip@jlab.org  &  \\ 
Jixie & Zhang & JLab & 5352 & 5352 & jixie@jlab.org \\
\hline
\end{tabular}
\end{center}
\caption[Moller Polarimeter: authorized personnel]{
   Moller Polarimeter: authorized personnel.
}
\label{tab:moller:personnel}
\end{table}
%}

\obsolete{
\begin{namestab}{tab:moller:personnel}{Moller Polarimeter: authorized personnel}{%
   Moller Polarimeter: authorized personnel}
  \EugeneChudakov{{\em Contact}}
  \OleksandrGlamazdin{}
  \RomanPomatsalyuk{}
  \ViktorGorbenko{}
\end{namestab}
}                  

% ===========  CVS info
% $Header: /group/halla/analysis/cvs/tex/osp/src/beamline/moller-personnel.tex,v 1.6 2010/06/23 14:32:05 gomez Exp $
% $Id: moller-personnel.tex,v 1.6 2010/06/23 14:32:05 gomez Exp $
% $Author: gomez $
% $Date: 2010/06/23 14:32:05 $
% $Name:  $
% $Locker:  $
% $Log: moller-personnel.tex,v $
% Revision 1.6  2010/06/23 14:32:05  gomez
% Moller Polarimeter update
%
% Revision 1.6  2010/02/10 15:35:01  glamazdi
% Personnel update
%
% Revision 1.5  2008/04/28 15:35:01  gen
% Personnel update
%
% Revision 1.4  2003/12/13 06:23:37  gen
% Septum added. Name tables. Polishing
%
% Revision 1.3  2003/12/05 05:48:30  gen
% Polishing
%
% Revision 1.2  2003/06/06 21:12:51  gen
% Revision printout changed
%
% Revision 1.1  2003/06/06 15:19:03  gen
% Revision printout changed
%
% Revision 1.2  2003/06/05 23:29:59  gen
% Revision ID is printed in TeX
%
% Revision 1.1.1.1  2003/06/05 17:28:32  gen
% Imported from /home/gen/tex/OSP


% ===========  CVS info
% $Header: /group/halla/analysis/cvs/tex/osp/src/beamline/moller.tex,v 1.13 2010/06/23 14:32:05 gomez Exp $
% $Id: moller.tex,v 1.13 2010/06/23 14:32:05 gomez Exp $
% $Author: gomez $
% $Date: 2010/06/23 14:32:05 $
% $Name:  $
% $Locker:  $
% $Log: moller.tex,v $
% Revision 1.13  2010/06/23 14:32:05  gomez
% Moller Polarimeter update
%
% Revision 1.12  2008/04/28 15:34:33  gen
% New target
%
% Revision 1.11  2008/04/01 16:10:54  gen
% Minor changes
%
% Revision 1.10  2005/04/04 22:27:24  gen
% Chnges after the review
%
% Revision 1.9  2004/12/15 00:13:00  gen
% minor update of the labels
%
% Revision 1.8  2003/12/13 06:23:37  gen
% Septum added. Name tables. Polishing
%
% Revision 1.7  2003/12/05 05:48:30  gen
% Polishing
%
% Revision 1.6  2003/06/06 21:41:39  gen
% Captions improvements
%
% Revision 1.5  2003/06/06 21:12:51  gen
% Revision printout changed
%
% Revision 1.1  2003/06/06 15:19:03  gen
% Revision printout changed
%
% Revision 1.2  2003/06/05 23:29:59  gen
% Revision ID is printed in TeX
%
% Revision 1.1.1.1  2003/06/05 17:28:32  gen
% Imported from /home/gen/tex/OSP
%
%  Revision parameters to appear on the output
 
%Daniel Dale (606-257-2504, Jlab office x5375, pager 3727, e-mail: dale@pa.uky.edu)\\
%Ashot Gasparian (606-257-5565, e-mail: gasparian@jlab.org)\\
%Marie Keese (office x7635, pager x7635, e-mail: keesee@jlab.org)\\
%Roman Pomasalyk (office x6378, e-mail: romanip@jlab.org )\\
%Victor Gorbenko (office x6378, e-mail: gorbenko@jlab.org)\\
%Alexander Glamazdin (office x6378, pager 6370, e-mail: glamazdin@jlab.org))\\
%Eugene Chudakov (Office x6959, pager 5152, e-mail: gen@jlab.org )\\


%\begin{center}
%{ \bf Other individuals must notify and receive permission from
% Eugene Chudakov before adding their names to the above list }
%\end{center}


% ===========  CVS info
% $Header: /group/halla/analysis/cvs/tex/osp/src/beamline/moller.tex,v 1.13 2010/06/23 14:32:05 gomez Exp $
% $Id: moller.tex,v 1.13 2010/06/23 14:32:05 gomez Exp $
% $Author: gomez $
% $Date: 2010/06/23 14:32:05 $
% $Name:  $
% $Locker:  $
% $Log: moller.tex,v $
% Revision 1.13  2010/06/23 14:32:05  gomez
% Moller Polarimeter update
%
% Revision 1.12  2008/04/28 15:34:33  gen
% New target
%
% Revision 1.11  2008/04/01 16:10:54  gen
% Minor changes
%
% Revision 1.10  2005/04/04 22:27:24  gen
% Chnges after the review
%
% Revision 1.9  2004/12/15 00:13:00  gen
% minor update of the labels
%
% Revision 1.8  2003/12/13 06:23:37  gen
% Septum added. Name tables. Polishing
%
% Revision 1.7  2003/12/05 05:48:30  gen
% Polishing
%
% Revision 1.6  2003/06/06 21:41:39  gen
% Captions improvements
%
% Revision 1.5  2003/06/06 21:12:51  gen
% Revision printout changed
%
% Revision 1.4  2003/06/06 15:19:03  gen
% Revision printout changed
%
% Revision 1.3  2003/06/05 23:29:59  gen
% Revision ID is printed in TeX
%
% Revision 1.2  2003/06/05 23:04:19  gen
% Revision date added
%
% Revision 1.1.1.1  2003/06/05 17:28:32  gen
% Imported from /home/gen/tex/OSP
%
