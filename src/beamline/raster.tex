\section[Fast Raster]{Fast Raster
\footnote{
  $CVS~revision~ $Id: raster.tex,v 1.7 2003/12/19 14:16:35 rom Exp $ $
}
\footnote{Authors: R.~Michaels \email{rom@jlab.org}}
}

The beam is rastered on target with an amplitude of
several millimeters to prevent overheating.  
The raster is a pair of horizontal (X) and vertical (Y)
air-core dipoles located 23 m upstream of the
target. Prior to 2003, the raster has been 
used in two different modes, sinusoidal and 
amplitude modulated.  After 2003 the raster
design was upgraded (see below).
For the old sinusoidal pattern both the X and Y magnet 
pairs were driven with pure sine waves
with $90^{\circ}$ relative phase, 
and with frequencies which do 
not produce a closed Lissajous pattern.
For the old amplitude modulated mode  
both the X and Y magnets were driven at 18 kHz with
a $90^{\circ}$ phase between X and Y producing a circular
pattern whose radius was amplitude-modulated at 1 kHz. 

The upgraded fast raster design (ca. 2003)
achieves a uniform rectangular
density distribution of beam on the target 
by moving the beam with a time-varying dipole
magnetic field whose waveform is triangular
with very little dwell time at the peaks.  
The electronics design is an ``H-bridge''
in which switches are opened and closed 
at 25 kHz, to switch between two directions 
of current (100 A peak-to-peak) 
through the raster coils.

One can view the 
status of the raster in the
EPICS overview screen called ``General Accelerator
Parameters'' where the set-point for the radius amplitude
and the readback of the peak-current in the raster are displayed.

Control of the raster is done by first asking the MCC
operators to set up the raster for a particular size
typically 2 mm square.
The control software assumes a field-free region between
the raster and the target, so it is only approximately
correct because there are four quadrupoles in this region.
It is important to check the raster spot size and
make adjustments if necessary.  The main adjustment is made
by asking MCC to change the size.
Relatively small independent adjustments to the 
gains on the X and the Y raster
coils are available in the middle room of the hall A
counting room using the ``PGA Controller'' knobs.
Near these knobs is also located an oscilloscope X-Y trace
of the current in the raster.  A fast shutdown (FSD) shuts
the beam down within 0.1 msec if the raster fails, thus
affording some protection of the target.

{\it NOTE:  If you are unsure of the status of the raster,
measure the spot size with very low current ($\le 2 \mu$A) or with
the target out of the beam.}  A potentially fatal error 
is to check the beam spot size with high current on target; by
the time you check it, the target might already be destroyed.
The rastered beam spot on target can be checked with
plots in the ROOT analyzer or by 
using the stand alone code called \mycomp{spot}.
For more details on usage, type \mycomp{spot -h} (help)
on the ADAQ computers.


% ===========  CVS info
% $Header: /group/halla/analysis/cvs/tex/osp/src/beamline/raster.tex,v 1.7 2003/12/19 14:16:35 rom Exp $
% $Id: raster.tex,v 1.7 2003/12/19 14:16:35 rom Exp $
% $Author: rom $
% $Date: 2003/12/19 14:16:35 $
% $Name:  $
% $Locker:  $
% $Log: raster.tex,v $
% Revision 1.7  2003/12/19 14:16:35  rom
% updated for 2003 uniform triangle pattern
%
% Revision 1.6  2003/12/17 03:59:47  gen
% authorized personnel tables unfied
%
% Revision 1.5  2003/12/13 06:23:37  gen
% Septum added. Name tables. Polishing
%
% Revision 1.4  2003/12/05 05:48:30  gen
% Polishing
%
% Revision 1.3  2003/06/06 15:19:03  gen
% Revision printout changed
%
% Revision 1.2  2003/06/05 23:29:59  gen
% Revision ID is printed in TeX
%
% Revision 1.1.1.1  2003/06/05 17:28:32  gen
% Imported from /home/gen/tex/OSP
%
