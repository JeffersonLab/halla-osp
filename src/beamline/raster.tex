\section[Fast Raster]{Fast Raster
\footnote{
  $CVS~revision~ $Id: raster.tex,v 1.8 2008/03/11 21:39:44 rom Exp $ $
}
\footnote{Authors: R.~Michaels \email{rom@jlab.org}}
}

The beam is rastered on target with an amplitude of
several millimeters at 25 kHz to prevent overheating.  
The raster is a pair of horizontal (X) and vertical (Y)
air-core dipoles located 23 m upstream of the
target.

Since 2003 we've used the triangle-wave 
raster pattern designed by Chen Yan.  
This achieves a very uniform rectangular
density distribution of beam on the target 
by moving the beam with a time-varying dipole
magnetic field whose waveform is triangular
with very little dwell time at the peaks.  
The electronics design is an ``H-bridge''
in which switches are opened and closed 
at 25 kHz, to switch between two directions 
of current (100 A peak-to-peak) 
through the raster coils.

One can view the 
status of the raster in the
EPICS overview screen called ``General Accelerator
Parameters'' where the set-point for the radius amplitude
and the readback of the peak-current in the raster are displayed.

Control of the raster is done by first asking the MCC
operators to set up the raster for a particular size
typically 2 mm square.
The control software assumes a field-free region between
the raster and the target, so it is only approximately
correct because there are several quadrupoles in this region.
It is important to check the raster spot size and
make adjustments if necessary.  The adjustment is made
by asking MCC to change the size and noting the 
linear relationship between what their software says
the size is and the actual size.
Relatively small independent adjustments to the 
gains on the X and the Y raster
coils are available in the middle room of the hall A
counting room using the ``PGA Controller'' knobs;
however, it is not recommended to touch these.
Near these knobs is also located an oscilloscope X-Y trace
of the current in the raster.  A fast shutdown (FSD) shuts
the beam down within 0.1 msec if the raster fails, thus
affording some protection of the target.

{\it NOTE:  If you are unsure of the status of the raster,
measure the spot size with very low current ($\le 2 \mu$A) or with
the target out of the beam.}  It would be a mistake
to check the beam spot size with high current on target; by
the time you check it, the target may already be destroyed.
The rastered beam spot on target can be checked with
plots in the ROOT analyzer or by 
using the stand alone code called \mycomp{spot},
also called \mycomp{raster}.
For more details on usage, type \mycomp{spot -h} (help)
on the ADAQ computers.

Regarding the BPM measurements, it should be noted that 
the stripline BPMs displayed by \mycomp{spot} have a high-frequency 
cutoff of approximately 30 kHz.  Since the raster frequency is 25 kHz
the plot of the amplitude distribution shows spikes at the 
limits of the orbit, instead of a flat distribution.  The scale
factor between what is seen in \mycomp{spot} and the real width of the beam
is $\sim 1.5$, i.e. the beam is 1.5 times bigger than the naive
reading of the \mycomp{spot} distribution.


% ===========  CVS info
% $Header: /group/halla/analysis/cvs/tex/osp/src/beamline/raster.tex,v 1.8 2008/03/11 21:39:44 rom Exp $
% $Id: raster.tex,v 1.8 2008/03/11 21:39:44 rom Exp $
% $Author: rom $
% $Date: 2008/03/11 21:39:44 $
% $Name:  $
% $Locker:  $
% $Log: raster.tex,v $
% Revision 1.8  2008/03/11 21:39:44  rom
% update for 2008
%
% Revision 1.7  2003/12/19 14:16:35  rom
% updated for 2003 uniform triangle pattern
%
% Revision 1.6  2003/12/17 03:59:47  gen
% authorized personnel tables unfied
%
% Revision 1.5  2003/12/13 06:23:37  gen
% Septum added. Name tables. Polishing
%
% Revision 1.4  2003/12/05 05:48:30  gen
% Polishing
%
% Revision 1.3  2003/06/06 15:19:03  gen
% Revision printout changed
%
% Revision 1.2  2003/06/05 23:29:59  gen
% Revision ID is printed in TeX
%
% Revision 1.1.1.1  2003/06/05 17:28:32  gen
% Imported from /home/gen/tex/OSP
%
