% Thanks to John LeRose for Rotation text. 07NOV2013
%
\infolevone{\chapter{Spectrometer Rotation}}
\infoleveqnull{\section{Spectrometer Rotation}}
%
Since each HRS weighs in excess of 1,000 tons it is very important that all safety
precautions are carefully adhered to. The good news is they move very slowly (a few degrees/min
maximum), BUT 1,000 tons moving even very slowly is hard to stop. 

\begin{safetyen}{0}{0}
\subsection{Hazards}

Hazards include:
\begin{itemize}
\item{Knocking items over.}
\item{The wheels crushing things (including fingers and toes) on the floor in the path of the 
spectrometer}
\item{Damaging the beamline or other equipment on the floor if one goes to too small 
or too large an angle, or if it just gets pushed around inadvertently.}
\item{Tearing out of cables etc. physically attached to the superstructure}
\end{itemize}

\subsection{Mitigations}

Hazard mitigations:
\begin{itemize}
\item{Guards on either side of the wheels prevent items from getting under them.}
\item{Large pins in the floor to stop the spectrometer rotated beyond the needed angular range.}
\item{Blinking lights on the spectrometers indicating they are in motion or that motion
is possible (controls engaged etc.)}
\item{During a running experiment the run coordinator and work coordinator should know in advance 
of any moves.  Moves at any other time must be cleared with the Hall work coordinator 
before implementation.}
\item{Careful inspection of the intended path to make sure it is clear. This is part of
the pre-run checklist performed by the technical staff prior to closing the Hall and
a remote camera allows shift worker to inspect the area.}
\item{Any motion that takes a spectrometer inside 14 degrees or outside \emph{X} degrees
(\emph{X} being specified in the pre-run checklist and noted on the whiteboard during a run) 
must be supervised by a trained Hall A technician.}
\end{itemize}
\end{safetyen}

\infolevone{
Remote Procedure for a shift worker:
\begin{itemize}
\item{Make sure the move is part of the approved runplan (if in doubt, check with the 
run coordinator).}
\item{Check that the pre-run checklist has been completed and note and comply with any 
possible limitations to spectrometer motion (if there is a conflict inform the Run
Coordinator and do not initiate any move until the conflict is cleared).}
\item{Visually inspect the Hall using the closed circuit TV cameras to verify that there
are no obstructions.}
\item{If people are in the Hall wait until they leave (during a Controlled Access MCC keeps
track of people in the Hall). (Maybe we could soften this to "Inform EVERYONE in the Hall of
the move".)}
\item{Activate the spectrometer motion controls (see the Wiki and below) and 
move to the desired angle.}
\item{Deactivate the controls (brakes on, power off, etc.)}
\item{Update the spectrometer position information on the Hall A Controls screen}
\item{Make a halog entry indicating you've moved the spectrometer including from what angle 
to what new angle.}
\end{itemize}

Procedure for a non-run associated move in the Hall:
\begin{itemize}
\item{Inform the work coordinator of the planned move}
\item{Perform a careful visual inspection to verify that the path is clear}
\item{Check to make sure there are no temporary connections to the spectrometer (wires etc.)
that could be damaged during the move.}
\item{Inform everyone in the Hall of the move and check with them re 3.}
\item{Activate the spectrometer motion controls (see the Wiki and below) verify 
that the warning lights are on and move to the desired angle.}
\item{Deactivate the controls (brakes on, power off, etc.).}
\end{itemize}

The full procedure for moving the spectrometer follows and can also be found on the Hall A wiki.

On hacsbc2, click the red "tool box" icon on the linux taskbar, as above. Choose 
bogies\_SetSpec so that you can determine the angle and vernier setting for the spectrometer.
Enter the spectrometer (L or R), and the angle, and you will get two options for the floor 
mark and the vernier. Generally choose the vernier closer to zero. Center the cameras on the 
desire vernier using the Move+/Move- buttons on the Hall A General Tools screen. The TV monitors 
for these cameras are on the middle shelf, in rack CH01A05.

Choose bogies\_Left (or bogies\_Right) in the tool box to bring up the bogies control screen. 
Click PSM enable and wait a few seconds for PSM OK to read YES. 
Click DM enable and wait a few seconds for DM OK to read YES.
Make sure the velocity is set to 0 and the direction is CW or CCW as desired. Click on Brake Release 
and wait for Brakes OK to read YES.

Click on ClampRelease, set the velocity to 700. Once you see the spectrometer start to move in the 
floor angle camera - you cannot see the spectrometer move in the Hall overview camera, as it only 
moves a few degrees per minute at maximum speed. For the left arm, to move to a larger angle, the 
direction should be CCW, while for the right arm CW moves the spectrometer to larger angle. The 
direction of the spectrometer is reversed by using a negative rpm. Watch the spectrometer motion 
on the cameras. When you are getting close to the desired angle, slow down to about 300 rpm. 
To stop, click on the Clamp Release button and the Brake button. Disable DM and PSM, and disconnect 
to close the GUI. Read off the floor angle mark and vernier, and input the values into the appropriate 
fields in the Alignment section of the Hall A General Tools GUI. 
}

\begin{safetyen}{0}{0}
\subsection{Responsible Personnel}

Following the experimental run plan, as posted in the counting house by the run coordinator,
shift workers are allowed to rotate the HRS following guidelines of the standard equipment manual.
In the event of a problem getting the spectrometers to rotate the run coordinator should notified.
If the run coordinator is unable to solve the problem, and with the run coordinators concurrence, 
qualified personnel should be notified to repair the problem (see Table \ref{tab:hrs:personnel_rotate}).  
On weekends and after hours please only use the tech-on-call number.
It should be noted that for experiments that are using thick targets at high current, it is
not uncommon that the produced radiation will cause the motion system to require a hard reset.

\begin{namestab}{tab:hrs:personnel_rotate}{HRS: authorized personnel}{%
      List of HRS responsible personnel where ``W.B.'' stands for the white board 
      in the counting house.}
   \TechonCall{\em Contact}
   \JackSegal{}
   \HeidiFansler{}
   \JessieButler{}
   \AndrewLumanog{}
\end{namestab}

\end{safetyen}

