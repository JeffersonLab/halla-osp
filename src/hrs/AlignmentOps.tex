\section[Spectrometer Alignment]{Spectrometer Alignment
\footnote{
  $CVS~revision~ $Id: AlignmentOps.tex,v 1.5 2003/11/14 22:54:26 gomez Exp $ $ 
}
\footnote{Authors: J.Gomez \url{mailto:gomez@jlab.org}}
}

At present, the systems implemented to determine the alignment of each spectrometer
(roll, vertical angle/pointing and horizontal angle/pointing) without the help of the
Accelerator Division Survey group are limited to roll, vertical angle and horizontal angle.
All alignment information is displayed in the ``ALIGNMENT'' mosaic of the Tools MEDM screen
(``Hall A Menu $-->$ ``Tools'').

A bi-axial inclinometer is used to determine the roll and vertical angle (also known as pitch)
of each spectrometer. These inclinometers are attached to the back of the dipoles at the power
supply platform level. The raw inclinometer measurements, in Volts,
are displayed as ``Tilt X'' and ``Tilt Y''. The inclinometer temperature is also given
(`` Tilt T''), in degree Celsius. From these values, the ``ROLL'' and ``PITCH'' values are
calculated
Agreement between the inclinometer readings and survey measurements
are better than $\pm$ 0.1 mrad over all presently available history.

The horizontal spectrometer angle is determined from floor marks set in
place by the survey group. Floor marks have been placed every 0.5 $^\circ$ covering the useful range of
both spectrometers.
There are two concentric rings of floor marks in the hall. We will concentrate in the
inner ring which covers the angular range of both spectrometers. The outer-ring covers only 
small angular sections but these floor marks are made on metal plates which allow 
to read them with higher resolution.
The inner-ring floor marks are located at a distance of $\sim$10 $m$ from the target center.
A ruler attached to each spectrometer dipole runs over the floor marks and it acts as a vernier to interpolate
between marks. The location of a given floor mark on the ruler can be viewed from the Hall A Counting
House through a TV camera (labeled ``Front Camera'') .
The camera is able to move along the length of the ruler so that any
parallax effect can be eliminated. The camera motion is controlled from the ``Tools'' screen
through two push buttons (``FRONT CAMERA'' - ``MOVE +'' and ``MOVE --'').
Two fields in the ``ALIGNMENT'' mosaic
(``Flr Mrk'' and ``Vernier'') allow to input
the values read from the TV monitor. The effective spectrometer angle is then calculated and displayed
as ``Angle''. The application ``HRS Floor Marks'' calculates the floor mark and vernier value
to which the spectrometer should be set
to obtain a given angle. Spectrometer horizontal angle surveys and floor mark determinations
agree to $\pm$ 0.2 mrad.

\subsection{Personnel Responsible}
J. Gomez (pager: 849-7498). 

% ===========  CVS info
% $Header: /group/halla/analysis/cvs/tex/osp/src/hrs/AlignmentOps.tex,v 1.5 2003/11/14 22:54:26 gomez Exp $
% $Id: AlignmentOps.tex,v 1.5 2003/11/14 22:54:26 gomez Exp $
% $Author: gomez $
% $Date: 2003/11/14 22:54:26 $
% $Name:  $
% $Locker:  $
% $Log: AlignmentOps.tex,v $
% Revision 1.5  2003/11/14 22:54:26  gomez
% *** empty log message ***
%
% Revision 1.4  2003/06/06 16:15:32  gen
% Revision printout changed
%
% Revision 1.3  2003/06/06 16:13:37  gen
% Revision printout changed
%
% Revision 1.2  2003/06/05 23:30:00  gen
% Revision ID is printed in TeX
%
% Revision 1.1.1.1  2003/06/05 17:28:31  gen
% Imported from /home/gen/tex/OSP
%
%  Revision parameters to appear on the output
