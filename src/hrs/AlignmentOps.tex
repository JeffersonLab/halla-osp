\section[Spectrometer Alignment]{Spectrometer Alignment
\footnote{
  $CVS~revision~ $Id: AlignmentOps.tex,v 1.3 2003/06/06 16:13:37 gen Exp $ $ 
}
\footnote{Authors: J.Gomez \url{mailto:gomez@jlab.org}}
}

At present, the systems implemented to determine the alignment of each spectrometer
(roll, vertical angle/pointing and horizontal angle/pointing) without the help of the
Accelerator Division Survey group are limited to roll, vertical angle and horizontal angle.

A bi-axial inclinometer is used to determine the roll and vertical angle (also known as pitch)
of each spectrometer. These inclinometers are attached to the back of the dipoles at the power
supply platform level. The inclinometer measurements are displayed in the main Hall A controls
screen (alignment mosaic). Agreement between the inclinometer readings and survey measurements
are better than $\pm$ 0.1 mrad over all presently available history.

The horizontal spectrometer angle is determined from floor marks set in
place by the survey group. Floor marks have been placed every 0.5 $^\circ$ covering the useful range of
both spectrometers. The marks are located at a distance of $\sim$10 $m$ from the target center.
A ruler attached to each spectrometer dipole runs over the floor marks and it acts as a vernier to interpolate
between marks. The location of a given floor mark on the ruler can be viewed from the Hall A Counting
House through a TV camera. The camera is able to move along the length of the ruler so that any
parallax effect can be eliminated. The camera motion is controlled from the main Hall A controls screen
(alignment mosaic) through two push buttons. Two fields on the
same mosaic (Flr Mrk/Vernier) allow one to input
the values read from the TV monitor. The effective spectrometer angle is then calculated and displayed
on the same mosaic. A terminal based program (setspec) is available on the controls computer ``hac'' which, for
a given angle, returns the floor mark value and its location on the ruler to which the spectrometer should be set
to obtain the desired angle. Spectrometer horizontal angle surveys and floor mark determinations
agree to $\pm$ 0.2 mrad.

\subsection{Personnel Responsible}
J. Gomez (pager: 849-7498). 

% ===========  CVS info
% $Header: /group/halla/analysis/cvs/tex/osp/src/hrs/AlignmentOps.tex,v 1.3 2003/06/06 16:13:37 gen Exp $
% $Id: AlignmentOps.tex,v 1.3 2003/06/06 16:13:37 gen Exp $
% $Author: gen $
% $Date: 2003/06/06 16:13:37 $
% $Name:  $
% $Locker:  $
% $Log: AlignmentOps.tex,v $
% Revision 1.3  2003/06/06 16:13:37  gen
% Revision printout changed
%
% Revision 1.2  2003/06/05 23:30:00  gen
% Revision ID is printed in TeX
%
% Revision 1.1.1.1  2003/06/05 17:28:31  gen
% Imported from /home/gen/tex/OSP
%
%  Revision parameters to appear on the output
{\small
\begin{verbatim}CVS $Id: AlignmentOps.tex,v 1.3 2003/06/06 16:13:37 gen Exp $\end{verbatim}
}
