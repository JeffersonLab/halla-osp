%\section[Overview]{ Overview
\chapter[Overview]{ Overview
\label{sec:targets-overv}
\footnote{
  $CVS~revision~ $Id: overview.tex,v 1.11 2008/04/18 20:05:18 jpchen Exp $ $ }
\footnote{Authors: J. P. Chen \email{jpchen@jlab.org}}
}

Three types of mutually exclusive target systems have been used in Hall A:

\begin{list}{\arabic{enumi}.~}{\usecounter{enumi}\setlength{\itemsep}{-0.15cm}}
  \item a system of cryo-targets and solid targets;
  \item a waterfall target;
  \item a target of polarized gaseous $^3$He.
\end{list}

%The physics program in Hall A utilizes a number of different
%target systems of varying complexity. 
The set of
cryogenic targets currently operates with liquid hydrogen, liquid
deuterium and gasous helium 3 or helium 4
as target materials.
A variety of solid targets are also provided; BeO, Carbon and
Aluminum are typical but other self supporting materials are available if need arises.
The combination of cryogenic targets and a few solid targets is
the standard configuration. \\

A waterfall target was used during the commissioning of
the hall spectrometers and for hypernuclear experiments. This system also 
requires a special installation. \\

In addition,
there is a large program based on polarized $^3$He. This
is a special installation and hence is not available at the same
time as the cryogenic target system.\\

Each of these systems is discussed in following chapters.
