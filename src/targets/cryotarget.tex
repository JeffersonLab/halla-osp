% Cryogenic Targets
\infolevone{
\chapter[Cryogenic Target System]{Cryogenic Target System}
\footnote{Authors: J. P. Chen \email{jpchen@jlab.org}
}}
\infoleveqnull{\section{Cryogenic Target System}}

\infolevone{
\section{Procedure for Normal Running of the Hall A Cryogenic Targets}

This procedure provides guidelines for the everyday running of the
Hall A cryogenic targets.

\subsection{Introduction }
}

The Hall A cryotarget system allows for multiple configurations depending
on the requirements of the experiment(s). In the standard configuration,
the system has three separate target loops. One of these loops contains
low pressure $^{4}$He gas with pressure up to 32 psia. The other
two loops are usually used for liquid hygrodgen and deuterium targets.
Each loop can have one or two target cells which is again dependent
experiment requirements.   Below the loops, solid targets, such as carbon foils
can be added.   

A short version of the cryotarget target control system user manual
is available at \url{http://hallaweb.jlab.org/equipment/targets/cryotargets/Hallatgt.html}.
An updated User's Guide to the Hall A Cryotarget is available at \url{https://polweb/guides/atarg/ATARGMAN.html}.
Other useful information for cryotarget operators is also available
at the above web sites.

\infolevone{
During the normal operation, the hydrogen and/or deuterium targets
shall have already been liquefied and are in a stable state of about
2 to 3 degrees sub-cooled liquid (19K for hydrogen and 22K for deuterium).
The normal operating conditions of the targets are given in Table~\ref{tab:target-cryo-param}.
Also listed in Table~\ref{tab:target-cryo-param} are the freezing
and boiling temperatures. These parameters should be reasonably stable
(temperature to $\pm0.1$ K, pressure to $\pm1$ psi) provided that
the End Station Refrigerator (ESR) is stable. The temperature is controlled
by a software PID loop with a high power heater (up to 1500 Watts).
The PID loops read the output of one of the temperature sensors and
adjust the power in the high power heater appropriately. The control
loop functions extremely well and the temperature fluctuations with
steady beam are typically measured in hundredths of degrees. The PID
control loop also monitors the electron beam current to keep the target
temperature stable by compensating for this heat load during unstable
beam situations.

\begin{table}
{hp} 

\begin{centering}
\begin{tabular}{|c|c|c|c|c|}
\hline 
Target  & Temperature (\ensuremath{^{\circ}K}
) & Pressure (PSIA) & Freezing T (\ensuremath{^{\circ}K}
) & Boiling T (\ensuremath{^{\circ}K}
)\tabularnewline
\hline 
H\ensuremath{_{2}}
 & 19 & 25 & 13.86 & 22.24\tabularnewline
\hline 
D\ensuremath{_{2}}
 & 22 & 22 & 18.73 & 25.13\tabularnewline
\hline 
\end{tabular}
\par\end{centering}

\caption[Cryo-target: operation conditions]{Normal operation conditions of the cryo-target cells}


\label{tab:target-cryo-param} 
\end{table}
}


\infolevone{ 


\section*{Graphical User Interface}

The principal interface with the target is through the Graphical User
Interface (GUI), of the control system. Every target operator shall
be familiar with the use of the target GUI.


\subsection{Alarm Handler}
}

\begin{safetyen}{0}{0} 
It is \emph{mandatory} to have an alarm
handler, ALH, running at all times when the target has been cooled-down.
Further, it is \emph{mandatory} that the alarm handler be visible
in all work spaces on the target control computer. \end{safetyen}
Even though the target safety is ultimately insured by mechanical
measures, the alarm handler can save you lots of time, grief and potentially
prevent problems with data. The ALH will alarm if any of its parameters
goes out of normal range. Servicing the alarm is the responsibility
of the target operator. At high beam current, the ALH will usually
alarm when the beam goes from on to off or from off to on, since the
temperature change is out of normal range. The ALH can also repeatedly
alarm if there are noisy analog channels. \begin{safetyen}{0}{0}
If the AH alarms repeatedly or the cause of the alarm is not clear,
the target operator should contact the on-call target expert. 
\end{safetyen}

\infolevone{
\subsection{Target Motion and Fast Raster}

The target motion is interlocked with the machine Fast Shut Down (FSD)
system. Therefore, it is \emph{mandatory} that operators call MCC
so that they can remove beam from the Hall and mask our FSD node \emph{before}
using \emph{any} target motion mechanism. In the case of Gmp experiment,
a separate pointing target is planned to be inside the scattering
chamber. This target will be manually positioned by a target group
expert. The cryotarget motion control will be dissabled, while the
pointing target is in position, by power lockout.
}

\begin{safetyen}{0}{0} 
When full power beam with tiny beam spot
hit the cryotarget, there is a danger that the beam can melt the target
cell. The fast raster is used to prevent this from happening. Every
time when moving the cryotarget into beam position, the target operator
\emph{must check to make sure} that the faster raster is on and has
a reasonable size for beam current above 5 $\mu$A. 
\end{safetyen}

\infolevone{


\subsection{Cryogenic Consumption}

The ESR is not a bottomless reservoir of helium coolant. Every effort
should be made to keep our consumption within reasonable bounds. This
means that heater overheads should be tens and not hundreds of Watts
and that loops which will be dormant for extended periods should be
powered down as much as possible. If you feel that the cryogenic consumption
is too high (or have received complaints from another ESR user) and
are uncertain about the appropriate action contact the on-call target
expert.


\subsection{Checklist}

The Hall A target experts and the JLab target group tracks the state
of the target. To facilitate this task the target checklist \emph{must
be logged in the Elog and the charts and the main target page screen
captured at least once per shift}.

} 

\infolevone{
\subsection{Target Operators}

One individual on each shift is responsible for target operations.
This individual is the dedicated target operator. To become a certified
target operator, one must be trained by one of the target experts
and to sit one shift with an already certified target operator. The
training usually takes place in the Hall A counting house and consists
of a guided walk through of the control system and procedures for
handling off normal events.

The target operator must read this document, the Safety Assessment
Document for the Hall A Cryogenic Targets, and the short version of
the GUI manual. The target operator shall be familiar with the GUI
system and be able to handle the normal target loop operation, the
cryostat operation and the target motion. He/she shall also be able
to deal with the GUI crash, the IOC crash and the usual alarms. 

After the target operator's training, if he/she feels comfortable
with the normal operation of the cryotargets, he/she should sign his/her
name on the target operator authorization list, which is  maintained in the
counting house by the target controls, indicating that he/she
has read this procedure and has been trained. The target expert who
trained him/her should inform the Hall A staff who is responsible
for the cryotarget system (J. P. Chen). 

\begin{safetyen}{0}{0} 

The following table contains the names of the currently recognized
target experts (who have worked on the Hall A cryotarget system and
have extensive knowledge of the system) and their pager numbers

\begin{namestab}{tab:cryotarg:personnel}{Cryo Target: experts}{Cryo
target: experts and authorized personnel, with their phone numbers}
\namestabheader{Hall A Physicists} \JianPingChen{cell:218-0722}
\JohnLeRose{cell:565-5060} \namestabheader{JLab Cryo-Target Group}
\DaveMeekins{cell:968-9076} \ChristopherKeith{cell:746-9277}
\end{namestab}

A cryotarget expert will be on call all the time when a cryotarget
is in cooled state. An on-call cryotarget-expert list will be posted
in the Hall A Counting House.
\end{safetyen}
}

\begin{safetyen}{0}{0} 
\infolevone{\section{Safety Assessment}}
\label{sec:target-cryo-safety}

\subsection{Hazards}

The cryogenic hydrogen and deuterium targets present a number of potential
hazards, such as the fire/explosion hazard of the flammable gas as
well as the hazards connected with the vacuum vessel and the of handling
cryogenic liquids (ODH and high pressure). 

\subsection{Mitigations}

\infoleveqnull{More detailed
information about these mitigations can be found in the full Hall A operations manual~\cite{HallAosp}.}

\paragraph{Flammable Gas}
\label{sec:targ-flammablegas}

\infolevone{
Hydrogen and deuterium are colorless, odorless gases and hence not
easily detected by human senses. Hydrogen air mixtures are flammable
over a large range of relative concentrations from 4 $\%$ to 75 $\%$
H$_{2}$ by volume. Detonation can occur with very low energy input,
less than 10\% of that required by mixtures of air and gasoline. At
temperatures above 23K hydrogen gas is lighter than (STP) air and
hence will rise. At atmospheric pressure, the ignition temperature
is approximately 811K but air H$_{2}$ mixtures at pressures of 0.2
to 0.5 Atm can be ignited at temperatures as low as 610K. Hydrogen
mixtures burn with a colorless flame \cite{bi:mc75}.

The total volume of hydrogen in the target is approximately 5.4 $l$.
The volume changes between the liquid state and gas at STP by a factor
of about 850. Thus filling the target would require about 4500 STP
$l$ of hydrogen. The hydrogen target is connected to a 1,000 Gallon
(about 3,800 $l$) recovery tank. The normal running condition for
hydrogen is 25 psia. Thus, the total amount of hydrogen in the system
is about 11000 STP $l$. A similar volume of deuterium will be required.
In addition to this volume, one hydrogen and one deuterium bottle
will be kept in the Hall in order to fill and pump/purge the targets.
These bottles will be placed in a gas rack behind the gas panels.
The large storage tanks are located outside the Hall at the rear of
the couting house.
}

The basic idea behind safe handling of any flammable or explosive
gas is to eliminate oxygen (required for burning) and to prevent exposure
to any energy source that could cause ignition. In the Hall A environment,
the most likely source of oxygen is of course the atmosphere and the
most likely ignition sources are from electrical equipment. Oxygen
is removed from the internal volumes of the system by pumping and
purging the system. Extensive procedures reviewed by an independant
expert are used to perform this task. This task shall only be performed
by system experts.

There are three flammable gas detectors installed (one on top of the
target, one each on top of the hydrogen and deuterium gas panels)
to provide early detection of hydrogen/deuterium leaks. These detectors
are sensitive (and calibrated) over the range from 0 to 50 $\%$ Lower
Explosive Limit (LEL) of hydrogen. The electro-chemical sensors were
manufactured by Crowcon Detection Instruments LTD and the readout
(four channels) was purchased from CEA Instruments, Inc. (The Gas
Master Four System). The readout unit provides two alarm levels per
channel. The low level alarm is tripped at 20 $\%$ of LEL while 40
$\%$ of LEL activates the high level alarm. Each channel has a relay
output for both low and high level alarm states and there is also
a set of common relays for both alarm levels (these common relays
respond to the logic of the sensor
inputs). 

\infolevone{ 
\paragraph{Electrical Installation}

Hall A contains a significant amount of electrical equipment and almost
all of it could serve as an ignition source in the presence of an
explosive oxygen and hydrogen mixture. Extensive efforts have been
made to minimize the dangers from the equipment that is most likely
to come into contact with hydrogen gas. Electrical equipment considered
to be in close contact with hydrogen meets the requirements of NFPA
2 Hydrogen Technologies Code and/or NFPA 55 Compressed Gasses and
Cryogenic Fluids Code as well as NFPA 497. Equipent not meeting these
Code requirements is isolated durring off normal events by either
valve isolation (vacuum turbopumps) or by electrical power trip.

A pressure switch, installed on the scattering chamber, will trip
when the vacuum in the scattering chamber is greater than 1 torr (i.e.
durring an isolation vacuum loss event). This switch deenrgizes the
following systems: vacuum gauge power, fan motor power, and heater
power supplies. The switch also forces pneumatically actuated gate
valves to close isolating the turbo pumps.

There are a number of electrically powered devices associated with
the target gas handling system. All the pressure transducers in the
system are approved for use in a hydrogen atmosphere. The solenoid
valves on the gas panels are explosion-proof and have been dissabled.
The readouts for the pressure transducers are mounted in the target
control equipment racks man meters from the gas panels. All the pressure
transducers have 4-20 mA outputs.

In addition to the electrical devices in the gas handling system,
there are a number of devices inside of or mounted on the scattering
chamber.

All the devices which are in the scattering chamber must have their
power delivered to them by wires in vacuum. The insulation of these
wires must be radiation resistant, so Kapton and glass fiber tubing
insulation has been used where applicable.

The following electrical items are in close proximity to or are actually
in the hydrogen system.
\begin{description}
\item [{\bf Axial Circulation Fan}] The fans which circulate the hydrogen
in the target are AC induction motors and therefore contain no brushes
and are practically immune to sparking. The three phase power for
these fans is delivered to them by 18 gauge stranded copper wire with
Kapton insulation. The maximum current that the fans draw is 5 $A$
for a maximum power consumption of 200 $W$ when pumping liquid hydrogen/deuterium.
The current and voltage drawn by the fans is monitored by the control
system. 
\item [{\bf Fan Motor Tachometer}] The fans have a tachometer which consist
of a coil that views the flux change caused by a permanent magnet
attached to the motor rotor. The tachometer signals are carried on
22 gauge stranded wire with Kapton insulation. This is a low power
signal. The control system monitors the frequency of the fans. 
\item [{\bf High Power Heater}] There is a high-power heater in the pipe
of the loop. The maximum power available is 1500 $W$.. The current
and voltage supplied to this heater are monitored by the control system
and there is a software power maximum enforced on the power setting
of this heater. Internal vacuum connections to the heaters are made
with 18 gauge stranded wire with Kapton insulation. 
\item [{\bf Resistive temperature sensors}] There are six resistive temperature
sensors immersed in each target loop. These resistors provide temperature
measurements of the target fluid. The temperature controllers that
read them use a current of less than 30 $\mu$A to excite them ( they
are excited with a constant voltage which for our resistors is on
the order of 30 mV). The resistors are connected to the outside world
with quad strand 36 gauge phosphor bronze wire with Formvar insulation. 
\item [{\bf Target Lifter}] An AC servo motor provides the power to lift
the target ladder. This motor is powered by three phase 208 $V$ power
and is equipped with fail safe brakes (the brakes are \textbf{released}
by a loss of 24 $V$ DC control voltage) and 50 to 1 gear reducers.
On power up, there is a delay relay that ensures that the motors are
always energized before the brakes are released. 
\item [{\bf Vacuum Pumps}] The scattering chamber is evacuated by two
Leybold 1000 $l/s$ turbo pumps that are backed by a Leybold 65 $cfm$
mechanical pump. The turbo pumps are powered by 120 $V$ AC power
while the backing pump requires three phase 208 $V$ AC power. The
turbo pumps are isolated durring an insulating vacuum failure event
by the use of automatic gate valves. The motors on the backing pumps
are induction motors and approved for use in this environment. (The
JLab fire protection engineer has reviewed this issue). An identical
mechanical pump is used in the pump and purge system of the gas panels.
Both the scattering chamber backing pump and the pump and purge system's
mechanical pump exhaust to the vent line.
\item [{\bf Vacuum Gauges}] The chamber vacuum is monitored by an HP cold
cathode gauge. This gauge is not rated for hydrogen service and is
therfore isolated from the scattering chamber vacuum durring an insulating
vacuum loss event by automatic gate valve. This gauge has a maximum
operating voltage of 4000 $V$ and a maximum current of 133 $\mu$A.
The pressure at the entrance to the roughing pump is measured by a
convectron gauge. 
\end{description}
}

\paragraph{Gas Handling System}

The most important aspect of hydrogen safety is to minimize the possibility
of explosive mixtures of hydrogen and oxygen occurring. Therefore
the gas handling system has been made of stainless steel components
(wherever possible) and as many junctions as possible have been welded.
Flanged connections are made with metal seals where possible. Resonable
measures have been implimented to ensure that the system pressure
does not fall near or below atmospheric pressure.

The pressure in the gas handling system is monitored in numerous places.
Most importantly, the absolute pressure of the target is viewed by
two pressure transducers, one on the fill line, PT127 for H$_{2}$
and PT136 for D$_{2}$, and one on the return line, PT131 for H$_{2}$
and PT140 for D$_{2}$. These pressures are also measured by manual
gauges. The fill line gauges are PI126 for H$_{2}$ and PI135 for
D$_{2}$. The return line gauges are designated PI130, H$_{2}$ and
PI139, D$_{2}$. The gas tanks are viewed with both pressure transducers
(PT133 for hydrogen and PT142 for deuterium) and pressure gauges (PI123
for hydrogen and PI112 for deuterium).

If the pressures significantly deviate either from one another or
from the normal operating pressure, the target operator shall call
the target-expert-on-call. When they differ from one another, it often
is due to a failure of one (or more) of the pressure transducers.
If more than one deviate significantly from the normal operating pressure,
it could be due to temperature change or could be a more serious situation
(i.e. a leak in the system).

The target system is considered by JLab to be a ``Pressure System''.
Thus, the design and construction of the system must meet the requirements
of the the most applicable ASME pressure code. The Codes of Record
for the system are ASME Boiler and Pressure Vessel Code Section VIII
Division 1 and ASME B31.3 Process Piping Code. These Codes have conservative
safety allowances. The system was not originally designed or constructed
to these Codes, however all alterations of the system are in compliance.
Further the relief system has been modified to meet these Code requirements.
All currently used cells and cell blocks also meet the requirements
of the B31.3 Code. The large volume storage tanks located outside
the Hall also meet the requirements of the ASME Boiler and Pressure
Code and bear an ASME nameplate. These tanks are inspected on a regular
basis and currently (or will by time of operation) meet the National
Board Inspection Code requirements.

\paragraph{Target Cells}

The target cells themselves represent the most likely failure point
in the hydrogen system. The outer wall is made of 0.006 $in$ thick
aluminum. The entrance and exit windows are thinner, but no less than
0.004 $in$. There is one 15 $cm$ long cell bolted on to each cell
block. The cell has an outer diameter of 3 inches. The upstream windows
are connected to 0.8 $in$ diameter tubes with flanges which are also
bolted on to the cell block. A vertical flow diverter plays a role
to make the coolant flow in vertical direction to help remove the
beam heating more effectively. The cell and cell block components
have been pressure tested hydro-statically to meet the requirements
of the ASME B31.3 2008 Process Piping Code. The design pressure of
the current cell is 100 psi.


\paragraph{Pressure Relief}

The gas handling and controls systems have been designed to prevent
excessive pressure build up in the system in order to protect the
target cells from rupture. It has been determined that the worst case
pressure load will arise from an insulating vacuum loss. The calculation
of this load was reviewed by a JLab Design Authority not associated
with the target group. The estimated relief load is 350 scfm of hydrogen.
The primary relief path is 2 inch Sch 10s pipe connected to the recovery
tank. There is a separate relief valve on the fill side of the target,
CRV30 for H$_{2}$ and CRV59 for D$_{2}$; this valve is not capable
of handeling the entire relief load from an insulating vacuum loss
event. Overpressure protection of the system (on each loop) is provided
by an ASME relief valve which meets the requirements of the Code.
The capacity of this relief is 1100 scfm for hydrogen. The relief
valves exhaust to the Hall A hydrogen vent line. This line is 2 inch
Sch 10s IPS stainless steel pipe \textasciitilde{}150 ft. long. The
vent line is continuously purged with 1 psig of He gas from the House
Helium supply. The scattering chamber and pump/purge vacuum pumps
are also exhausted to this line. Thus any vented target gas is placed
in an inert environment until it is released outside of Hall A. Additionally,
each gas tank has one relief valve as required by Code.

\infolevone{
The scattering chamber provides secondary containment in the event
of a cell rupture. Therefore, the scattering chamber itself has a
1 psig relief (check valve), VRV01 and a 4 psig rupture disk. Thus,
the scatting chamber internal pressure will not exceed 5 psig. This
relief path is also exhausted to the hydrogen vent line. A series
of valves and controls allow for the safe removal and exhaust of hydrogen
from the scattering chamber should a cell burst.
}

\paragraph{Scattering Chamber Vacuum}

\label{sec:cryo_targ_cmb_falure}

The scattering chamber will be leak checked before service but, the
possibility of vacuum loss cannot be eliminated. A conservative calculation
estimating the relief load on the relief system of each loop has been
performed. This calculation was performed as part of Code and JLab
policy requirements and was reviewed by an independant JLab Design
Authority. This calculation (TGT-CALC-301-010) has been filed in the
Hall A Cryotarget pressure system directory PS-TGT-XX-026. In summary,
this calculation conservatively indiactes that the relief path and
safety devices limit the maximum developed pressure in the cell to
less than the 120 psi for all credible overpressure conditions as
required by ASME B31.3 322.6.3.

\infolevone{
\paragraph{Temperature Regulation}

This is really more an issue of target stability than one of safety.
However, a target with a carefully regulated temperature will presumably
not undergo worrisome pressure changes.

Each target contains six quality temperature measurements with resistive
temperature sensors. The temperature regulation is performed by a
software PID control of a high power heater using one of the quality
temperature measurements as input to the loop. This is a three parameter
control loop (Proportional, Integral and Differential Control or PID).
The PID loop also compensates for the beam heat load durring beam
trip and recovery incidents. This is not a true regulation but rather
one for one replacement of the beam load should the beam disappear
for whatever reason. The beam load is calculated from the target length,
the beam current as read from a current monitor and the target material.

Excursions of the target temperature outside acceptable limits will
cause the control system to take action. Finally the redundancy of
temperature measurements can be used by the control system to pick
up the failure of a sensor or its readout channel. A more complete
discussion of target temperature regulation is available in Reference
\cite{bi:tgts}. 
}

\paragraph{Target Freezing}

Solid hydrogen is more dense than the
liquid phase, so freezing does not endanger the mechanical integrity
of a closed system. The chief hazard is that relief routes out of
the system will become clogged with hydrogen ice, making the behavior
of the system during a warm-up unpredictable. For
this reason, the relief route bypasses the heat exchanger and should
not freeze during any credible senario. 

\infolevone{ The coolant flow through the three target heat exchangers
is connected in parallel for the three target loops. The entire target
system will be operatred so that it represents a constant heat load
on the ESR. For instance, the ESR will deliver a constant mass flow
of helium cryogen at a constant temperature, about 15 K, and the coolant
will be returned at an approximately constant but higher temperature,
usually about 20 K.
The targets are temperature regulated by IOC heater PID loop.}

In the unlikely event that the target temperature drops too low, an
alarm will sound and the target operator shall turn down the corresponding
J-T valve(s) or apply auxilliary heater power. Target temperature
can fall after IOC reboot. After the reboot the high power heater
will be reset to zero before going back to PID control. Although the
time the high power heater is zeor is short (for about 1 minute),
the temperature will drop. To prevent this from happening, an auxiliary
heater is used in parallel to the regular heater. During an IOC reboot,
the auxiliary heater supply will replace the main supply to keep the
temperature from dropping unacceptably. Since 2008, the IOC has been
relocated to the entry laberinth where the radiation exposure has
been minimized. As a result the frequency of IOC reboots has dramatically
decreased.

\paragraph{ODH}
\label{sec:targ:odh}

The total volume of the targets is relatively small, with the entire
scattering chamber containing only 9,000 STP $l$ of target gas when
all three targets are full. As the scattering chamber is located in
the middle of Hall A (i.e. not in a confined area) and the total Hall
A volume is 40,000 m$^{3}$, the ODH hazard is minimal. 

\infolevone{
\paragraph{Controls}

The target controls have been implemented with the EPICS~\cite{EPICSwww}
control system and with hardware very similar to that employed by
the accelerator. The basic control functions reside on a VME based
single board computer or IOC. The graphical interfaces to the control
system use a PC, and also require the Hall A Hewlett Packard, HP,
computer for control (HAC) to be present as well. Power failures will
result in a loss of computer control. As a result of such a failure
the target heat exchanger may freeze and the remainder of the target
may vent through the relief path. The beam will also be tripped durring
such a failure. There is thus, little chance of damage or danger from
the system. Durring a power failure, the target operator shall call
the target-on-call immediately.

The principal functions that the control system performs are: 
\begin{description}
\item [{Pressure/Temperature Monitoring}] The pressure and temperature
are monitored at various places in the system and alarm states are
generated if a sensor returns a value that is outside defined limits.
\item [{Target Lifter}] The target lifting mechanism is controlled by
the computer. This allows one to place the desired target in the beam.
Limit switches and hard stops are installed to ensure the target cannot
move outside the allowable range.
\end{description}
}

\infolevone{
\subsection{Checklist}

Checklist for pre-hall-closing: 
\begin{itemize}
\item Target has completed cool-down, at least one loop has liquid hydroge
with temperature stable at 19K, pressure stable at around 25 psi. 
\item High power heater in PID control for the hydorgen loop. 
\item Loop fan (pump) has been set to non zero value (20-75 Hz) for the
hydrogen loop. 
\item Coolant (ESR) flow and inlet temperature are stable. 
\item All unused loops are filled with over 1 atm gauge of helium gas. 
\item Scattering chamber vacuum is normal (below $10^{-5}$). 
\item target in ``Empty'' position for beam tuning. 
\item Alarm handler is on and all alarm limits are set. 
\item No constant alarms caused by abnormal conditions. 
\item Target-on-call name is written on the whiteboard. 
\end{itemize}
}

\subsection{Responsible Personnel}

The principle contacts for the cryogenic targets are listed in table~\ref{tab:cryotarg:personnel-con}.
Every shift must have a trained target operator whenever the cryogenic
targets contain liquid. These operators are trained by one of the
``experts'' listed in the table and certified by J.P.~Chen or John LeRose.

\begin{namestab}{tab:cryotarg:personnel-con}{Cryo Target: personnel
and contacts}{Cryo target: authorized personnel and contacts. ''W.B''
stands for the white board in Counting House.} \namestabheader{Hall
A Technicians} \TechonCall{Vacuum} \EdFolts{Vacuum} \namestabheader{Hall
A Physicists} \CryotargonCall{Cryotarget} \JianPingChen{Cryotarget} \JohnLeRose{Cryotarget}
\namestabheader{JLab Cryo-Target Group} \DaveMeekins{} \ChristopherKeith{}
\namestabheader{Central Helium Liquefier (CHL) Experts} \CryoonCall{ESR}
\CHLgroup{ESR} \end{namestab}
\end{safetyen}
\clearpage
