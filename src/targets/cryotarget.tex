\chapter[Cryogenic Targets]{Cryogenic Targets
\footnote{
  $CVS~revision~ $Id: cryotarget.tex,v 1.8 2003/12/13 06:23:39 gen Exp $ $ }
\footnote{Authors: J. P. Chen \email{jpchen@jlab.org}}
}

\section{Procedure for Normal Running of the Hall A Cryogenic Targets}
This procedure provides guidelines for the everyday running of the Hall A
cryogenic Hydrogen and Deuterium targets.

\subsection{Introduction }
The Hall A cryotarget system contains three target loops. The top loop (loop 1)
has a single 10 cm ``tuna can'' helium cell, which will be filled with either 
$^3$He or $^4$He gas with pressure up to 15 atm (about 220 PSIA). This loop 
is usually not been used for normal Hydrogen and Deuterium running, in which 
case it will be filled with a little over one atm helium gas. Both the middle 
loop and the bottom loop contain one 15 cm and one 4 cm  ``beer can'' cells.
The middle loop (loop 2) is usually filled with liquid Hydrogen during normal 
operation.
The bottom loop (loop 3) is usually filled with liquid Deuterium during normal
operation. If only the Hydrogen target is used, such as the case for   
the HAPPEX running, loop 3 (the Deuterium loop) 
will be filled with 17 PSIA helium gas to save cooling power.

\par
During the normal operation, the Hydrogen and/or Deuterium target should
have already been liquefied and are in a stable state of 
about 2 degree sub-cooled. 
The normal operating conditions of the targets are given in Table~\ref{tab:target-cryo-param}.
Also listed in Table~\ref{tab:target-cryo-param} are the freezing and boiling temperatures.
These parameters should be reasonably stable provided that the End Station
Refrigerator (ESR) is stable. The
temperature is controlled by a software PID loop with a high power heater (up
to 600 Watts) and a hardware PID loop with a low power heater (up to about 
60 Watts). Both PID loops read the output of one
of the Cernox resistors and adjust the power in the high or low power heater 
appropriately.
The control loops function extremely well and the temperature fluctuations with
steady beam are typically measured in hundredths of degrees. During beam off-
beam on transitions high power fluctuations of a few tenths of a degree are
not uncommon.

\begin{table}{hp}
\begin{center}
\begin{tabular}{|c|c|c|c|c|}
\hline 
Target &
 Temperature (\( ^{\circ }K \))&
 Pressure (PSIA)&Freezing T (\( ^{\circ }K \))&Boiling T (\( ^{\circ }K \))\\
\hline 
H\( _{2} \)&
 19&
 25&13.86&22.24\\
\hline 
D\( _{2} \)&
 22&
 22&18.73&25.13\\
\hline 
\end{tabular}
\end{center}
\caption[Cryo-target: operation conditions]{Normal operation conditions
  of the cryo-target cells}
\label{tab:target-cryo-param}
\end{table}

\infolevltone{

For more information consult the full OSP manual~\cite{HallAosp}.
}

\infolevone{
\section*{Graphical User Interface}

The principal interface with the target is through the
Graphical User Interface (GUI), of the control system.
Every target operator should be familiar with the
short version of the GUI manual. The long version can 
be used to find more details or for reference.
 

\subsection{Alarm Handler}

\begin{safetyen}{10}{5}
It is \emph{mandatory} to have an alarm handler,ALH, running at all times. 
Further,
it is \emph{mandatory} that the alarm handler be visible in all work spaces
on the target control computer.
\end{safetyen}
Even though the  
target safety is ultimately
insured by mechanical measures, the alarm handler can save you lots of 
time, grief and potentially prevent problems with data. 
The ALH will alarm if any of its parameters goes out of normal range. 
Servicing the alarm is the responsibility of the target operator.
At high beam current, the ALH will usually alarm when the beam goes from
on to off or from off to on, since the temperature change is out of normal
range. The ALH can also repeated alarm if there are noisy analog channels. 
\begin{safetyen}{10}{5}
If the AH alarms repeatedly or the cause of the alarm is not clear, the 
target operator should contact the on-call target expert.
\end{safetyen}

\subsection{Target Motion and Fast Raster}

The target motions are interlocked with the machine Fast Shut Down (FSD) 
system. Therefore, it is \emph{mandatory} that you call MCC so that they can
remove beam from the Hall and mask our FSD node \emph{before} using \emph{any}
target motion mechanism.

\par
\begin{safetyen}{10}{5}
When full power beam with tiny beam spot hit the cryotarget, there is a danger 
that the beam can melt the target cell. The fast raster is used to prevent this
from happening. Every time when moving the cryotarget into beam position,
the target operator \emph {must check to make sure} that the faster raster 
is on and has a reasonable size.
\end{safetyen}


\subsection{Cryogenic Consumption}

The ESR is not a bottomless reservoir of helium coolant. Every effort should
be made to keep our consumption within reasonable bounds. This means that heater
overheads should be tens and not hundreds of Watts and that loops which will
be dormant for extended periods should be powered down as much as possible.
If you feel that the cryogenic consumption is too high (or have received complaints
from another ESR user) and are uncertain about the appropriate action contact
the on-call target expert.


\subsection{Checklist}

The Hall A target experts and the JLab target group like to track the state
of the target. To help them in this task the 
target checklist \emph{must be filled
out once per shift} and store in the target checklist binder.

}
\infolevtwo{
\subsection{Target Operators}

One individual on each shift is responsible for target operations. 
This individual
is the designated target operator. To become a target operator, 
one must be trained by one of the target experts and to sit at least
one shift with an already certified target operator. The training usually 
takes place in the Hall A counting house and consists of a guided walk 
through of the control system. 
\par
The target operator must read this document,
the Safety Assessment Document for the Hall A Cryogenic Targets,
and the short version of the GUI manual. The target operator should
be familiar with the GUI system and be able to handle the normal target
loop operation, the cryostat operation and the target motion. 
He/she should also be able to deal with the GUI crash, the IOC crash
and the usual alarms.
\par
After the target operator's training, if he/she feels comfortable
with the normal operation of the cryotargets, he/she should sign
his/her name on the target operator authorization list, indicating that
he/she has read this procedure and has been trained. The target expert
who trained him/her should inform the Hall A staff who is responsible 
for the cryotarget system (J. P. Chen), who will then sign off to authorize
him/her to be a certified target operator. 
\par
The table below lists the qualified target operators and provides space 
for additional entries. The names of all
operators must appear in this table.

\vspace{0.3cm}

\begin{center}
\begin{tabular}{|c|c|c|c|c|}
\hline 
Operator Name&e-mail&
 Date&
 Signature&
 Authorization\\
\hline 
J. P. Chen&jpchen&
 3/21/99&
 --&
 --\\
\hline 
K. McCormick&mccormic&
 3/21/99&
 --&
 --\\
\hline 
G. Rutledge&grutledg&
 3/21/99&
 --&
 --\\
\hline 
R. Suleiman&suleiman&
3/21/99&
--&
--\\
\hline 
&&
&
&
\\
\hline 
&&
&
&
\\
\hline 
&&
&
&
\\
\hline 
&&
&
&
\\
\hline 
&&
&
&
\\
\hline 
&&
&
&
\\
\hline 
&&
&
&
\\
\hline 
&&
&
&
\\
\hline 
&&
&
&
\\
\hline 
&&
&
&
\\
\hline 
&&
&
&
\\
\hline 
&&
&
&
\\
\hline 
&&
&
&
\\
\hline 
&&
&
&
\\
\hline 
&&
&
&
\\
\hline 
&&
&
&
\\
\hline 
&&
&
&
\\
\hline 
&&
&
&
\\
\hline 
&&
&
&
\\
\hline 
&&
&
&
\\
\hline 
&&
&
&
\\
\hline 
&&
&
&
\\
\hline 
&&
&
&
\\
\hline 
&&
&
&
\\
\hline 
&&
&
&
\\
\hline 
&&
&
&
\\
\hline 
&&
&
&
\\
\hline 
&&
&
&
\\
\hline 
&&
&
&
\\
\hline 
&&
&
&
\\
\hline 
&&
&
&
\\
\hline 
&&
&
&
\\
\hline 
&&
&
&
\\
\hline 
&&
&
&
\\
\hline 
&&
&
&
 \\
\hline 
\end{tabular}
\end{center}

\vspace{0.3cm}

}

\begin{safetyen}{10}{10}
\section{Target Experts}
\end{safetyen}

The following table contains the names of the currently recognized target 
experts (who have worked on the Hall A cryotarget system and
have extensive knowledge of the system)
and their pager and/or home numbers

\begin{namestab}{tab:cryotarg:personnel}{Cryo Target: experts}{%
   Cryo target: experts and authorized personnel, with their home numbers}
   \namestabheader{Hall A Physicists} 
   \JianPingChen{h:867-7380}
   \KathyMcCormick{h:640-8062}
   \namestabheader{JLab Cryo-Target Group} 
   \DaveMeekins{h:974-4750}
   \MikellSeely{h:833-7890}
   \ChristopherKeith{h:596-3002}
\end{namestab}

\par
A cryotarget expert will be on call all the time when
a cryotarget is in cooled state.
An on-call cryotarget-expert list will be posted in the 
Hall A Counting House. 



\begin{safetyen}{10}{5}
\section{Safety Assessment}
\label{sec:target-cryo-safety}

The cryogenic hydrogen and deuterium targets present a number of potential
hazards, such as the
fire/explosion hazard of the flammable gas as well as the hazards
connected with
the vacuum vessel and the of handling cryogenic liquids
(ODH and high pressure). 
\infolevltone{
A detailed safety assessment is given in the full OSP manual~\cite{HallAosp}.
}
\end{safetyen}
\infolevone{
In this document the hydrogen target will be referred
to, but the deuterium target is essentially identical and almost all
comments apply to both targets. 
}

\infolevone{
\begin{safetyen}{10}{5}
\subsection{Flammable Gas}

Hydrogen and deuterium are
colorless, odorless gases and hence not easily detected by human senses.
Hydrogen air mixtures are flammable over a large range of relative
concentrations from 4 $\%$ to 75 $\%$ H$_2$ by volume. Detonation
can occur with very low energy input, less than $\frac{1}{10}$
that required by mixtures of air and gasoline. At temperatures above
-250 C hydrogen gas is lighter than (STP) air and hence will rise.
At atmospheric pressure, the ignition temperature is approximately
1000 $^\circ$ F but air H$_2$ mixtures at pressures of 0.2 to 0.5 Atm can be
ignited at temperatures as low as 650 $^\circ$ F. Hydrogen mixtures
burn with a colorless flame \cite{bi:mc75}.

The total volume of liquid hydrogen in the heat exchanger is about
2 $l$. The target cells and their associated plumbing hold
an additional 3.4 $l$. Thus the total volume of hydrogen
in the target is approximately 5.4 $l$. The volume changes between
the liquid state and gas at STP by a factor of about 800.
Thus filling the target
would require about 4,300 STP $l$ of hydrogen. The hydrogen target is 
connected to a 1,000 Gallon (about 3,800 $l$) recovery tank. The normal 
running condition for hydrogen is 26 PSIA. So the total amount needed
to fill the target and the tank is about 10,900 STP $l$. 
For deuterium, the target is about 4,300 STP $l$. The normal running
condition for deuterium is 22 PSIA. So the total volume needed
to fill the tank is about 5,600 $l$.
The total to fill both the target and the tank is 
about 9,900 STP $l$.

The Hall A inventory of hydrogen and deuterium gas is stored
outside the Hall A gas shed, adjacent to the counting house.
The current inventory is two A size cylinders of hydrogen
($\approx$ 6,800 STP $l$ each) and four A size cylinders
of deuterium ($\approx$ 5,000 STP $l$ each). 
One bottle of hydrogen and one deuterium bottle will be kept in the Hall
in order to fill the targets. These bottles will be placed in a gas rack behind
the gas panels. 

The basic idea behind safe handling of any flammable or explosive gas
is to eliminate oxygen (required for burning)
and to prevent exposure to any energy source that could cause ignition.
In the Hall A environment, the most likely source of oxygen is of course the
atmosphere and the most likely ignition sources are from electrical equipment.

\subsubsection{ Electrical Installation}

Hall A contains a lot of electrical equipment and almost all of it
could serve as an ignition source in the presence of an explosive
oxygen and hydrogen mixture. We have made an effort to minimize the dangers
from the equipment that is most likely to come into contact with
hydrogen gas.

There are a number of electrically powered devices associated with the
target gas handling system.
All the pressure transducers in the system are approved
for use in a hydrogen atmosphere. 
The solenoid valves on the gas panels are explosion-proof.
The AC power for the solenoids is
carried by wires which are contained in either hard or flexible conduit.
There are also LEDs on the gas panels that provide an indication as to the
status of the valve solenoids. These are powered by a 24 $V$ DC supply.
The readouts for the pressure transducers are
mounted on the gas panels and the AC power for these readout units
is in conduit. All the pressure transducers have 4-20 mA outputs.

In addition to the electrical devices in the gas handling system,
there are a number of devices inside of or mounted on the scattering
chamber.

All the devices which are in the scattering chamber must have
their power delivered to them by wires in vacuum. The
insulation of these wires should be radiation resistant, so Kapton
has been used where available.

The following electrical items are in close proximity to
or are actually in the hydrogen system.

\begin{description}
\item[{\bf Axial Circulation Fan}] The fans which 
circulate the hydrogen in the target are AC induction motors
and therefore contain no brushes and are practically immune to sparking.
The three phase power for these fans is delivered to them by 18
gauge stranded copper wire with Kapton insulation.
The maximum current that 
the fans draw is 5 $A$ for a maximum power consumption of
200 $W$ when pumping liquid hydrogen/deuterium. The current and voltage drawn
by the fans is monitored by the control system.
\item[{\bf Fan Motor Tachometer}]
The fans have a tachometer which consist of a
coil that views the flux change caused by a permanent magnet attached to 
the motor rotor. The tachometer signals are carried on 22 gauge stranded wire
with Kapton insulation. This is a low power signal. The control system
monitors the frequency of the fans.
\item[{\bf Low Power Heater}] This is a ``hair dryer'' style heater ( it 
resembles the heater elements found in hair dryers and heat guns) that
is immersed in the hydrogen. The heater is made of 0.0179 $in$ diameter
Nichrome wire with a resistance of 1.993 $\Omega$ per foot wrapped
on a G10 carrier board.
The maximum power available to this
heater is 80 $W$. The power for the low power heater is supplied by
a Oxford ITC-502 temperature controller. The heater lead wire is
18 gauge Kapton insulated copper stranded wire. The heaters have a
DC resistance of 20 $\Omega$ and hence will draw a maximum of 2 $A$. 
The power supplied to this heater is monitored by the control system.
\item[{\bf High Power Heater}]There are two kapton enclosed incoloy heater foils
wrapped on the inside wall of each heat exchanger.
The maximum power available to each
heater is  500 $W$. The heater has a DC resistance of
26 $\Omega$ and two heaters in parallel
are driven by a 150 $V$, 7 $A$ power supply.
The current and voltage supplied to this heater are monitored by the control
system and there
is a software power maximum enforced on the power setting
of this heater. 
The heater is connected to
the outside world by 18 gauge stranded wire with Kapton insulation.
\item[{\bf Resistors}] There are two Allen Bradley and four Cernox
resistors immersed in each target loop. These resistors provide temperature
measurements of the target fluid. The temperature controllers that
read them use a current of less than 30 $\mu$A to excite them (
they are excited with a constant voltage which for our resistors is
on the order of 30 mV). The Cernox resistors are connected to the outside world
with quad strand 36 gauge phosphor bronze wire with Formvar insulation.
The Allen Bradley resistors are wired with 30 gauge Kapton insulated 
copper stranded wire.
\item[{\bf Target Lifter}] There are two AC servo motors which provide
the power to lift the target ladder. These motors are powered by three
phase 208 $V$ power and are equipped with fail safe brakes (the brakes are
{\bf released} by a 24 $V$ DC control voltage)
and 50 to 1 gear reducers. On power up, there is a delay relay
that insures that
the motors are always energized before the brakes are released.
\item[{\bf Vacuum Pumps}] The scattering chamber is evacuated by two Leybold
1000 $l/s$ turbo pumps that are  backed by a Leybold 65 $cfm$
mechanical pump. The turbo pumps are powered by 120 $V$ AC power while the
backing pump requires three phase 208 $V$ AC power. The motor on the backing 
pump is explosion proof and approved for use in NEC Class 1, Division 1,
Group D (hydrocarbons {\bf but not} hydrogen) environments. An identical
mechanical pump is used in the pump and purge system of the gas panels.
Both the scattering chamber backing pump and the
pump and purge system's mechanical pump exhaust to the vent line. 
\item[{\bf Vacuum Gauges}] The chamber vacuum is monitored by an
HP cold cathode gauge. This gauge has a maximum operating voltage of 
4000 $V$ and a maximum
current of 133 $\mu$A. The pressure at the entrance to the roughing pump
is measured by a convectron gauge.
\end{description}

\subsubsection{ Flammable Gas Detectors}

There are four flammable gas detectors installed (one on top of the target,
one each on top of the hydrogen and deuterium gas panels, one on top of the gas
tanks)
to provide early detection
of hydrogen/deuterium leaks. These detectors are sensitive (and calibrated)
over the range from 0 to 50 $\%$ Lower Explosive Limit (LEL) of hydrogen.
The electro-chemical sensors were manufactured by Crowcon Detection
Instruments LTD and
the readout (four channels) was purchased from CEA Instruments, Inc. 
(The Gas Master Four System). The readout unit provides two alarm
levels per channel. The low level alarm is tripped at 20 $\%$ LEL
while 40 $\%$ LEL activates the high level alarm.
Each channel has a relay output for both low and high level alarm
states and there is also a set of common relays for both alarm
levels (these common relays respond to the "logical or" of the sensor inputs).
The common relays will be connected to the Fast Shut Down System, FSD,
which removes the beam from the hall by disabling a grid bias at the
injector.
\end{safetyen}
}

\infolevone{
\subsection{ Pressure}
\begin{safetyen}{10}{5}

The most important aspect of hydrogen safety is to minimize the
possibility of explosive mixtures of hydrogen and oxygen occurring.
Therefore the gas handling system has been made of stainless steel
components (wherever possible) and as many junctions as possible have
been welded.

The pressure in the gas handling system is monitored in numerous
places. Most importantly, the absolute pressure of the target is
viewed by two pressure transducers, one on the fill line, PT127
for H$_2$ and PT136 for D$_2$, and one on the
return line, PT131 for H$_2$ and PT140 for D$_2$.
These pressures are also measured by manual gauges. The fill line gauges
are PI126 for H$_2$ and PI135 for D$_2$. The return line
gauges are designated PI130, H$_2$ and PI139, D$_2$. The gas tanks are viewed
with both pressure transducers (PT133 for hydrogen and PT142 for deuterium)
and pressure gauges (PI123 for hydrogen and PI112 for deuterium).

\subsubsection{ Target Cells}

The target cells themselves represent the most likely failure point
in the hydrogen system. The outer walls and downstream window 
of the cells are made of
$\approx$ 0.03 to 0.045 $in$ thick 3004 aluminum (Coors beer cans in a former
incarnation) (all the final ones are above 0.035 $in$). 
There are two cells soldered to each cell block, one 15 $cm$ long
and one 4 $cm$ long. Both cells have an outer diameter of 
approximately 2.5 inches.
The upstream windows of the cells are made from 0.0028 $in$ thick 5052
aluminum.
These windows are soldered to 1.75 $in$ diameter (0.065 $in$ wall)
upstream window tubes which are in turn soldered to the cell block.

Since all the components are made of aluminum it is necessary to plate
them before soldering. The final components were copper
plated before assembly. 

The cell block components have been pressure tested hydro-statically at
Jefferson Lab.
We chose the thinnest beer cans for the pressure burst test.
 Results are listed
in the summary table.
Upstream windows have been tested
to similar pressures. Finally, the entire completed cell block
assemblies were pressurized to 85 PSID with helium gas.
A summary of the testing program to date is presented
in Table \ref{ta:test}.

\begin{table}[htb]
\begin{center}
\begin{tabular}{|c|c|c|c|c|} \hline
Object & P (PSIG) & thickness (in) & size & test method \\ \hline
Can & 55  & 0.003 &short &test (J)ig, (D)estructive \\ \hline
Can & 80 & 0.0035 &short &(J), (D) \\ \hline
Can & 80 & 0.0035 &long &(J), (D) \\ \hline
Can & 60 & 0.0038 &long &(V)acuum, (D) \\ \hline
Can & 85 & 0.0039 &long &(V), (D) \\ \hline
Window & 110 &0.028& &(J),(D) \\ \hline
Window & 125 &0.028& &(J),(D) \\ \hline
Window & 125 &0.028& &(J),(D) \\ \hline
Window & 115 &0.028& &(J),(D) \\ \hline
Window & 147 &0.028& &(J),(D) \\ \hline
Window & 150 &0.028& &(J),(D) \\ \hline
Complete Blocks & 85 &&&(V), Non-Destructive \\ \hline
\end{tabular}
\end{center}
\caption[Cryotarget: Cell Pressure Test Data]{A summary of the early cell block pressure test data.}
\label{ta:test}
\end{table}

\subsubsection{ Pressure Relief}

The gas handling and controls systems have been designed to 
prevent excessive pressure build up in the system in order to
protect the target cells from rupture.

In the event that the pressure in the system begins to rise there are multiple
vent paths to release it. The first line of defense is the recovery tank.
The second line of defense is a small orifice solenoid valve
which is slaved to a pressure transducer. This valve, CSV28 for H$_2$ and
CSV57 for D$_2$, 
is normally controlled by the limit
output of the computer (via a VME based relay) readout of
the pressure transducer that views the target
relief line, PT131 for H$_2$ and PT140 for D$_2$. The valve itself is mounted
in the fill line relief assembly. The separation of the valve from its
controlling pressure gauge should provide some dampening of the response
and the small orifice of the valve also ensures that it will
be able to make pressure adjustments gently if need be.
There is 
a separate relief valve on the fill side of the target, CRV30 for H$_2$ and
CRV59 for D$_2$. This relief is mounted in parallel with the small orifice 
solenoid valve. Right on top of the cryo-can, on the return side of the target,
there is a large size (one $in$) relief valve.
All target pressure reliefs are connected to the
nitrogen vent line of the Hall A superconducting magnets.
This is a 3.5 $in$ diameter 
copper pipe which is filled with nitrogen gas at atmospheric pressure.
Thus any vented target gas is placed in an inert environment until it is
released outside of Hall A. Each gas tank has one relief valve and one rupture
disk (CRV43 and CRD44 for hydrogen, and CRV72 and CRD143 for deuterium).
 
In addition to the reliefs on the gas handling system described above, the
scattering chamber itself has a four-$in$ one PSIG relief, VRV01.
This is the path that the
hydrogen will take in the event of a cell failure.   

The target pressure reliefs are summarized in Table \ref{ta:pre}.

\begin{table}[htb]
\begin{center}
\begin{tabular}{|c|c|c|c|c|} \hline
Name & Target & Location & Diameter ($in$) & Pressure (PSIG) \\ \hline
CSV28 & H$_2$ & FRA & 0.125 & 40 \\ \hline
CRV30 & H$_2$ & FRA & 0.5 & 40 \\ \hline
CRV82 & H$_2$ & RL &  1 & 40 \\ \hline
CRV43 & H$_2$ & TANK & 1 & 55 \\ \hline
CRD44 & H$_2$ & TANK & 1 & 55 \\ \hline
CSV57 & D$_2$ & FRA & 0.125 & 40 \\ \hline
CRV59 & D$_2$ & FRA & 0.5 & 40 \\ \hline
CRV64 & D$_2$ & RL &  1 & 40 \\ \hline
CRV72 & D$_2$ & TANK & 1 & 55 \\ \hline
CRD143 & D$_2$ & TANK & 1 & 55 \\ \hline
CRV35 & He & RL &  1 & 40 \\ \hline
CRV01 &  & SC & 4 & 2 \\ \hline
\end{tabular}
\end{center}
\caption[Cryotarget: Relief Device Summary]{ A summary of the pressure relieving devices on the
hydrogen/deuterium targets and the scattering chamber. FRA
is an abbreviation for Fill Line Relief Assembly. 
and RL is an abbreviation for
Relief Line. SC stands for Scattering Chamber.}
\label{ta:pre}
\end{table}

\subsubsection{ Scattering Chamber Vacuum Failure}
\label{sec:cryo_targ_cmb_falure}

The scattering chamber will be leak checked before service but obviously
the possibility of vacuum loss cannot be eliminated. The most
likely sources of vacuum failure are:

\begin{description}
\item[Spectrometer Windows] Initially the scattering chamber will have two
 aluminum windows, one for each side of the beam line.
\item[Target Cell Failure] This is a multiple loop system. If a target
cell fails, the remaining targets will have their insulating
vacuum spoiled.
\end{description}

The two spectrometer windows are both made from aluminum. Each window
is seven $in$ high and subtends 170 $^\circ$ on the 43 $in$ outer
diameter of the scattering chamber. This window is made of
0.016 $in$ thick 5052 H34 aluminum foil.

The scattering chamber was evacuated (and cycled several times)
with both windows covered by the same 0.016 $in$ material. 
The foil forms regularly spaced vertical ridges when
placed under load. The window had an inter-ridge
spacing of 3 inches.
If the window is treated as a collection
of smaller rectangular windows which have the full vertical height
of 7 inches and the inter-ridge spacing as a width,
then stress formulas predict that the 0.016 $in$
material would reach ultimate stress at a pressure higher than 35 PSI. 
There is a gate valve between the 
scattering chamber and the beam entrance (exit) 
pipe. Both 
valves will be closed automatically in the
event that the chamber vacuum begins to rise and an FSD will be caused
( this is done via a relay output of the scattering
chamber vacuum gauge). If either valve is closed an FSD will result.

In the unlikely event of a catastrophic vacuum failure, it
is important that the relief line of the targets be sized
such that it can handle the mass flow caused by the sudden
expansion of its cryogenic contents due to exposure to the
heat load. A calculation has been performed which
models the response of the system to sudden vacuum failure.
That calculation indicates that the relief plumbing is sized such that the
flow remains subsonic at all times and that the maximum pressure
in the cells remains well below their bursting point.

\end{safetyen}

The calculation was performed  by following methods in 
an internal report from the MIT Bates
laboratory \cite{bi:bates}. The formulas and algorithm in the report were
incorporated in two computer codes and those codes were able to reproduce
results in the report (hence they represent an accurate implementation
of the Bates calculation).

The calculation can be logically broken into two parts. First,
the mass evolution rate is calculated from geometric information
and the properties of both the target material and vacuum spoiling gas.
The principal results of this first stage are the heat transferred
per unit area, q, the boil off time, t$_b$, and the mass evolution rate, w.
Second, the capability of the plumbing to handle the mass flow
is checked. The principle result of this second step is the
maximum pressure in the target cell during the discharge, P$_1$.

The formula involved will not be repeated (readers are referred to 
the Bates report for detail).
The information that was used as input
to the calculation is given in tables \ref{ta:gas}, \ref{ta:geo}
and \ref{ta:pdr}.

For the calculation of the boil off rate the target was split into
two pieces: the cells plus cell block, both aluminum; and the heat exchangers
plus the connecting plumbing, all steel. The mass evolution rates
for the two pieces were then added in order to find the total
mass flow rate.

\begin{table}[htb]
\begin{center}
\begin{tabular}{|c|l|l|l|} \hline
Fluid and Phase & Property & Symbol & Value \\ \hline
Hydrogen/Liquid & Temperature & T(K) & 22 \\ \hline
 & Density & $\rho$ (kg/m$^3$) & 67.67 \\ \hline
 & Specific Heat & C$_p$ (J/(kg K)) & 11520 \\ \hline
 & Enthalpy of Vaporization & H$_v$ J/kg & 428,500 \\ \hline
Hydrogen/Vapor & Temperature & T(K) & 22 \\ \hline
 & Density & $\rho$ (kg/m$^3$) & 2.4991 \\ \hline
 & Viscosity & $\mu$ (kg/(s m)) & 1.29$*$10$^{-6}$ \\ \hline
 & Specific Heat & C$_p$ (J/(kg K)) & 13,550 \\ \hline
 & Thermal Conductivity & k (W/(K m)) & 0.02 \\ \hline
 & Volume Expansivity & $\beta$ K$^{-1}$ & 0.00366 \\ \hline
Air & Temperature & T(K) & 273 \\ \hline
 & Pressure & P (Torr) & 760 \\ \hline
 & Density & $\rho$ (kg/m$^3$) & 1.224 \\ \hline
 & Viscosity & $\mu$ (kg/(s m)) & 1.8$*$10$^{-5}$ \\ \hline
 & Specific Heat & C$_p$ (J/(kg K)) & 1005 \\ \hline
 & Thermal Conductivity & k (W/(K m)) & 0.0244 \\ \hline
 & Volume Expansivity & $\beta$ K$^{-1}$ & 0.00367 \\ \hline
\end{tabular}
\end{center}
\caption[Cryotarget: Gas Properties]{ The properties of the gases used to calculate
the heat transferred to the target during a catastrophic vacuum failure.}
\label{ta:gas}
\end{table}

\begin{table}[htb]
\begin{center}
\begin{tabular}{|c|c|c|c|c|} \hline
Quan- & Cell  & Piping & Heat      & Total \\ 
tity  & Block &        & Exchanger &   
 \\ \hline
D & 2.5 in (0.063 m) & 1.5 in (0.038 m) &7 in (0.1778 m) & \\ \hline
k & 55 W/(K m) & 6.5 W/(K m) & 6.5 W/(K m) & \\ \hline
A & 0.146 m$^2$ & 0.185 (m$^2$) & 0.216 m$^2$ & 0.510 m$^2$ \\ \hline
V & 0.001 m$^3$ & 0.0019 (m$^3$) &0.002 m$^3$ & 0.0054 m$^3$ \\ \hline
x & 0.004 in (0.0001 m) &  0.065 in (0.00165 m) & 0.12 in (0.003 m) & \\ \hline
q & 14903 W/m$^2$ & 10526 W/m$^2$ & 11235 W/m$^2$ & \\ \hline
t$_{b}$ & 26.78 s & 28.29 s & 23.89 s & 26.3 s \\ \hline
w & 0.0038 kg/s & 0.0045 kg/s & 0.0056 kg/s & 0.014 kg/s \\
& & & &  (0.03 lbs/s) \\ \hline 
\end{tabular}
\end{center}
\caption[Cryotarget: Volumes and Geometry]{ The geometric quantities needed for and the results
of calculations of the mass evolution rate after a catastrophic vacuum
failure.} 
\label{ta:geo}
\end{table}

\begin{table}[htb]
\begin{center}
\begin{tabular}{|c|c|c|c|} \hline
Inner Diameter & Length & K (K$_{eff}$) \\ \hline
0.44 in tube & 10 ft & 4.64 (31.5) \\ \hline
0.88 in tube & 10 ft & 2.32 (0.98)\\ \hline
\multicolumn{2}{|c|} {Quantity} & Value \\ \hline
\multicolumn{2}{|l|} {Minor Losses}  & 7.4 \\ \hline
\multicolumn{2}{|l|} {K$^{total}_{eff}$} & 40 \\ \hline
\multicolumn{2}{|l|} {Average Diameter} & 0.71 in \\ \hline
\multicolumn{2}{|l|} {xmax} & 0.890 \\ \hline
\multicolumn{2}{|l|} {w$_{sonic}$} & 0.065 lbs/s \\ \hline
\multicolumn{2}{|l|} {m} & 0.323 \\ \hline
\multicolumn{2}{|l|} {x} & 0.748 \\ \hline
\multicolumn{2}{|l|} {P$_{2}$} & 14.7 PSIA \\ \hline
\multicolumn{2}{|l|} {P$_{1}$} & 58.3 PSIA \\ \hline
\multicolumn{2}{|l|} {P$_{1}$} & 43.6 PSIG \\ \hline
\end{tabular}
\end{center}
\caption[Cryotarget: Relief Line Information]{ Tubing sizes, and other information needed to analyze
relief line response. The mass flow rate was 0.03 lbs/s.} 
\label{ta:pdr}
\end{table}

The calculation of the pressure drop includes all the plumbing up to the
large relief valve. 
The calculation assumes that all the mass flow is carried out the
relief side of the target gas handling system (no flow out of the fill line
reliefs). 
The friction factor for each diameter was taken from a Moody plot.
A typical value was $ f = 0.017$.
The effective K values, K$_{eff}$, were adjusted to the average tube inner 
diameter which was taken to be 0.71 in.
The final K$_{eff}$ value was 40.
The minor losses are from bends, expansions and contractions in
piping. 

The final result shows the cells subjected to 58 PSIA during the boil off,
which is comparable to the 75 PSIA pressure that the assembled cell blocks were
tested at, and is significantly below the tested
pressure of the cell components.

The scattering chamber has a volume of about 2,100 $l$ with perhaps an 
additional 200 $l$ of volume in the bellows and the cryo can.
If one target cell were to rupture and the chamber were unrelieved,
the chamber pressure would rise to about 2 Atm.
It takes approximately 150 seconds to
bring 5 $l$ of 22 $^\circ$ K hydrogen to room temperature
by conductive heat transfer with the scattering chamber walls.
In order that the maximum pressure 
in the chamber stay near one atmosphere, it is necessary to
vent one half of the target mass in approximately one half of the
total expansion time. Therefore the relief valve for the scattering chamber
should be capable of venting about three grams per second
at a low pressure difference (say two PSIG).
If one considers the case where all three targets
fail at once, the vent must be capable of handling three times
that amount. A four $in$ diameter relief valve placed near the
top of the scattering chamber should be capable of handling this rate.
A rise in the chamber vacuum will stop the beam, FSD, and
cause the gate valves on either side of the scattering chamber to
close. 

In the unlikely event that a line which carries helium coolant
were to rupture the four $in$ chamber relief valve is capable
of handling the full coolant flow rate.

\subsection{ Temperature Regulation}

This is really more an issue of target stability than one of
safety. However, a target with a carefully regulated temperature
will presumably not undergo worrisome pressure changes.

Each target contains four quality temperature measurements, two Cernox
resistors and two hydrogen vapor pressure thermometers. The primary
temperature regulation is done with a dedicated temperature controller
(an Oxford ITC-502) which slaves a heater (the "low power heater") 
to the temperature read by one of the Cernox resistors.
This is a three parameter control loop (Proportional, Integral and Differential
Control or PID).

In addition, the return temperature of the target systems coolant
gas is used to regulate the supply of coolant from ESR.

Finally, the heat load from the beam will be compensated in the "active"
target loop by use of the high power heater. This is not a true regulation
but rather a one for one replacement of the beam load should the beam
disappear for whatever reason. The beam load is calculated from the
target length, the beam current as read from a current monitor and
the target material.

Excursions of the target temperature outside acceptable limits will
cause the control system to take action.
Finally the redundancy of temperature measurements can be used by the
control system to pick up the failure of a sensor or its readout channel.
A more complete discussion of target temperature regulation is available
in Reference \cite{bi:tgts}.

\subsubsection{Target Freezing}

\begin{safetyen}{10}{5}
Solid hydrogen is more dense than the liquid phase, so freezing does not
endanger the mechanical integrity of a closed system. The chief hazard is that
relief routes out of the system will become clogged with hydrogen ice,
making the behavior of the system during a warm-up unpredictable.
\end{safetyen} 
When the hydrogen and deuterium targets are in use,
we are using only 15 K coolant. While the hydrogen freezing point
is about 13.8 K, the hydrogen target should not get frozen. 
\begin{safetyen}{10}{5}
The freezing point of deuterium is higher than that of hydrogen
and higher than the temperature of the gas used for
cooling (15 K).There is 
a chance that the deuterium target can freeze.
\end{safetyen}

The coolant flow through the three target heat exchangers is
connected in parallel for the three target loops.
The entire target system will be run so that it
represents a constant heat load on the ESR. For instance,
the ESR will deliver
a constant mass flow of helium cryogen at a constant temperature, about
15 K, and the coolant will be returned at an approximately 
constant but higher temperature, usually about 20 K.

The targets are always temperature regulated by
temperature controllers. Also a high power heater will be in the PID loop
to compensate any large temperature fluctuations to keep the temperature 
constant.
\begin{safetyen}{10}{5}
In the unlikely event that the target temperature drops too low,
an alarm will sound and the target operator will turn down the corresponding
J-T valve(s). 

\subsection{ODH}

The total volume of the targets is relatively small, with the entire
scattering chamber containing only 9,000 STP $l$ of target
fluid when all three targets are full. As the scattering chamber
is located in the middle of Hall A
(i.e. not in a confined area) and the total Hall A volume is 40,000 m$^3$,
the ODH hazard is minimal.
\end{safetyen}

\subsection{ Controls}

The target controls have been implemented with the EPICS~\cite{EPICSwww} control
system and with hardware very similar to that employed by the accelerator.
The basic control functions reside on a VME based single board
computer. The graphical interfaces to the control system use a PC, and
also require the Hall A Hewlett Packard, HP, computer for control (HAC)
to be present as well.

All of the instrumentation for the target is downstairs in Hall A.
Most of the equipment (in fact all of the 120 V AC equipment) is
on an Uninterruptable Power Supply, UPS. The items whose power
is not on UPS are:
\begin{description}
\item[$\bullet$] The scattering chamber vacuum pumps and the gas panel
backing pump
\item[$\bullet$] The target lifting mechanism
\item[$\bullet$] The target circulation fans.
\end{description}
This is a 7 kVA zero switching time
UPS which is dedicated to the target. The PC, HAC and the counting house target
X-terminal are on Uninterruptable Power as well. 
The targets dedicated UPS provides 18 minutes of power at full load
(or 50 min at one half load). The status of the UPS, online or offline,
is read by the control system and after ten minutes the control system
will initiate an orderly shut down of the targets.

The principal functions that the control system performs are:
\begin{description}
\item[Pressure Monitoring] The pressure at various places in the system is
monitored and alarm states are generated if a transducer returns a value
that is outside user defined limits. High pressures will cause the
small orifice solenoid valve to open and cause an FSD.
\item[Temperature Monitoring] The temperature of the target
is read from resistors and vapor pressure bulbs and alarm states are
activated when any temperature sensor returns a value outside
the user defined limits. High temperatures will cause an FSD to occur.
\item[Temperature Regulation] The control system allows the target
temperature to be regulated. In the default operating scenario this regulation
is performed by a stand alone temperature controller.
\item[Solenoid Valve Control] The gas systems have a number of solenoid valves
that must be switched.
\item[J-T Valve Control] The flow of coolant through the heat exchangers
is controlled by a set of J-T valves. These valves control the coolant 
helium flow through the three loop heat exchangers and the precool heat
exchanger.
\item[Circulation Fan Monitoring and Control] The fans which circulate
the target fluid are monitored (current, voltage, frequency). The
voltage supplied to the fans is adjustable and alarm states can be set
on out of range frequency, voltage or current values.
\item[Vacuum Monitoring] The scattering chamber vacuum is monitored by
the control system. Unacceptable values will generate an FSD and close
the upstream and downstream  scattering chamber valves.
\item[Target Lifter] The target lifting mechanism is controlled by
the computer. This allows one to place the desired target in the beam.
\end{description}

\begin{safetyen}{10}{15}
\subsection{ Authorized Personnel}
\end{safetyen}

The principle contacts for the cryogenic targets are listed in 
table~\ref{tab:cryotarg:personnel-con}. Every shift must have a trained target
operator whenever the cryogenic targets contain liquid. These
operators are trained by one of the ``experts'' listed in the
table and certified by J.P.~Chen.

\begin{namestab}{tab:cryotarg:personnel-con}{Cryo Target: personnel and contacts}{%
   Cryo target: authorized personnel and contacts. ''W.B'' stands for the white board
   in Counting House.}
   \namestabheader{Hall A Technicians} 
   \TechonCall{Vacuum}
   \EdFolts{Vacuum}
   \namestabheader{Hall A Physicists} 
   \CryotargonCall{Cryotarget}
   \JianPingChen{Cryotarget}
   \KathyMcCormick{}
   \namestabheader{JLab Cryo-Target Group} 
   \DaveMeekins{}
   \MikellSeely{}
   \ChristopherKeith{}
   \namestabheader{Central Helium Liquefier (CHL) Experts} 
   \CryoonCall{ESR}
   \CHLgroup{ESR}
\end{namestab}

} %infolev
% \newpage
%-----------------------------------------------------------------------
\section[Cryogenic Target Control System User Manual]
{Cryogenic Target Control System User Manual 
}
A short version of 
the cryotarget target control system user manual, written by Kathy 
McCormick, is available at
\url{http://hallaweb.jlab.org/equipment/targets/cryotargets/Halla_tgt.html}.
An updated User's Guide to the Hall A Cryotarget, written by Chris Keith,
is available at \url{https://polweb/guides/atarg/ATARG_MAN.html}.  Other useful
information for cryotarget operators 
is also available at the above web sites.


%\begin{thebibliography}{99}
%% hydrogen props
%\bibitem{bi:mc75} R.D. Mc Carty, "Hydrogen Technology Survey: Thermophysical
%Properties", N76-11297, NBS 1975
%%helium props
%%\bibitem{bi:mc72} R.D. Mc Carty, "Thermophysical Properties of $^4$He from
%%2 to 1500 K with Pressures to 1000 Atmospheres", COM 75-10334, NBS 1972
%%\bibitem{bi:gi67} R.M. Gibbons and D.I. Nathan, " Thermodynamic Data
%%of $^3$He ", Tech Report AFML-TR-67-173, Oct 1967
%\bibitem{bi:bates} W. Schmitt and C. Williamson, "Boiloff Rates of
%Cryogenic Targets Subjected to Catastrophic Vacuum Failure", 
%Bates Internal Report $\#$ 90-02, Sept 1990
%\bibitem{bi:tgts} F. Duncan, K. McCormick, and J. Goity,  
%"A User's Guide to the Hall A Cryotarget Control System", Sept 1997
%\end{thebibliography}

%\end{document}
 
% ===========  CVS info
% $Header: /group/halla/analysis/cvs/tex/osp/src/targets/cryotarget.tex,v 1.8 2003/12/13 06:23:39 gen Exp $
% $Id: cryotarget.tex,v 1.8 2003/12/13 06:23:39 gen Exp $
% $Author: gen $
% $Date: 2003/12/13 06:23:39 $
% $Name:  $
% $Locker:  $
% $Log: cryotarget.tex,v $
% Revision 1.8  2003/12/13 06:23:39  gen
% Septum added. Name tables. Polishing
%
% Revision 1.7  2003/12/05 07:35:03  gen
% WT2 modified. Polishing
%
% Revision 1.6  2003/11/21 18:26:21  gen
% polishing
%
% Revision 1.5  2003/11/17 07:18:21  gen
% CVS record added
%
% Revision 1.4  2003/11/16 07:46:44  gen
% Shrink some large objects
%
% Revision 1.3  2003/11/12 16:08:55  jpchen
% cryotarget chapter
%
% Revision 1.2  2003/11/12 15:38:40  jpchen
% cryotarget chapter
%
% Revision 1.1  2003/09/24 17:08:38  jpchen
% cryotargets and polarized 3He target
%
% Revision 1.3  2003/06/06 17:11:53  gen
% Revision printout changed
%
% Revision 1.2  2003/06/06 17:10:24  gen
% Revision printout changed
%
% Revision 1.1  2003/06/06 17:09:04  gen
% Revision printout changed
%
% Revision 1.2  2003/06/05 23:30:01  gen
% Revision ID is printed in TeX
%
% Revision 1.1.1.1  2003/06/05 17:28:27  gen
% Imported from /home/gen/tex/OSP
%
%  Revision parameters to appear on the output



