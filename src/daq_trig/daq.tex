\chapter[Spectrometer Data Acquisition]{Spectrometer Data Acquisition
\footnote{
  $CVS~revision~ $Id: daq.tex,v 1.6 2003/12/05 06:45:07 gen Exp $ $
}
\footnote{Authors: R.Michaels \url{mailto:rom@jlab.org}}
}


%  Re:  Hall A OPS Manual Chapter for DAQ and Trigger
%Author:  Robert Michaels, Hall A Staff, Jefferson Lab
% Date:  March, 1999
% Updated: Aug, 2003



\par
The Hall A data acquisition uses CODA\cite{CODAwww}
(CEBAF Online Data Acquisition), a toolkit
developed at Jefferson Lab by the Data
Acquisition Group. 
%For general information
%about CODA, see the 
%CODA site\htmladdnormallinkfoot{}{\url{http://coda.jlab.org/}}.
Up to date information about the Hall A DAQ
is kept at 
\htmladdnormallinkfoot{}{\url{http://hallaweb.jlab.org/equipment/daq/daq_trig.html}}.

\par
We typically run with two fastbus crates in
each spectrometer, plus VME crates for scalers.
The fastbus modules are of the following
types:
\begin{list}{\arabic{enumi}.~}{\usecounter{enumi}\setlength{\itemsep}{-0.15cm}}
  \item LeCroy model
    1877 TDCs operating in common--stop with 0.5 nsec
    resolution for our drift chambers and
    straw chambers; 
  \item model 1875 TDCs operating in common--start with 0.1 nsec resolution
      or 0.05 nsec resolution depending on the setup,
      for our scintillators and trigger timing;
  \item model 1881M ADCs for analog signals from scintillators, 
        \Cherenkov{}, and leadglass detectors.
\end{list}
In some run periods the beam position monitors
and raster current were available in a VME system,
but presently they are read out in fastbus.

\par
The trigger supervisor is a custom--made
module built by the data
acquisition group.  Its functions are to
synchronize the readout crates, to administer
the deadtime logic of the entire system, and
to prescale various trigger inputs.  
We have two trigger supervisors,
one in each spectrometer.  This allows us to
run the spectrometers independently if needed.

\par
The public account \mycomp{a-onl} is normally
used for running DAQ and \mycomp{adaq} is used
for running other online software including
ESPACE or the C++ analyzer.
On ``\mycomp{adaq}'' the directory
tree of an experiment is \mycomp{adaq/\$EXPERIMENT}
which is organized in subdirectories of 
various tasks, such as scaler display,
ESPACE, and other online codes, all of
which will be described in sections below.
The trigger management software is run from the
\mycomp{atrig} account and is described in
the Trigger chapter. 


\infolevtwo{

\section{General Computer Information}

\par
In the counting room we have various computers 
for DAQ, analysis, and controls.  The controls
subnet is the responsibility of 
J. Gomez and is documented in another chapter.
The DAQ computer's names are denoted by \mycomp{adaqXN}, where
X = s is SunOS and X = l is for Linux PC. 
adaqs1 is the Compton DAQ computer and is normally
reserved by the Compton group.  \mycomp{adaqs2} and \mycomp{s3}
are for running the spectrometer DAQ or doing
online analysis; we are gradually phasing out the Suns.
adaql1 and l2 are Linux computers for running DAQ
while \mycomp{adaql3,4}... and higher are for analysis.
The Linux PCs are administered mostly by Ole Hansen.

\par
To reboot the Suns,  login as \mycomp{adaq}
and type \mycomp{reboot}. 
To reboot the Linux machines, first hit \mycomp{Ctrl-Alt-F1}
to switch to a text console, then hit \mycomp{Ctrl-Alt-Del} 
to reboot.
If power fails for a prolonged time,
you must shutdown before the UPS fails.


\section{Beginning of Experiment Checkout}

\par
This section describes the 
checkout of DAQ and trigger
needed before an experiment can start.

\begin{enumerate}
\item{First ensure that all the fastbus, VME,
CAMAC, and NIM crates are powered
on. They should
boot up in a functional state, except for
heavily loaded fastbus crates that sometimes
lose their NVRAM.  (If that happens, see notes
in \mycomp{adaq/doc/vmeram.doc}).}

\item{You may download
a default trigger, following the directions in
the trigger chapter.  If the hadron momentum changes
you may need to set a new delay.  A trigger expert
should do the start-of-experiment
trigger checklist.}

\item{Make sure the HV is on for all detectors
and that the values are normal.}

\item{Start the \mycomp{xscaler}
display following the instructions below and
check that the rates from detectors are normal.}
 
\item{Startup \mycomp{runcontrol} (CODA) using the directions below
and start a run.  With the trigger downloaded
and the HV on, you are taking cosmics data, typically at 
a rate of 3 Hz per spectrometer.  
Examine the data using ESPACE or the C++ analyzer.
Compare the plots and printouts to normal values.}

\end{enumerate}

\section{Running CODA}

\par
This section describes how to run CODA for
the spectrometer DAQ.  There are two modes:
(1) The most common is the ``1-Trigger-Supervisor (1-TS)''
mode which uses one trigger supervisor and
is used for coincidence experiments; and
(2) The ``2-Trigger-Supervisor (2-TS)'' mode which
is used for running the two spectrometers
independently.

\par
The 1-TS mode can also handle single--arm
triggers but is about 1/2 the aggregate speed
of the 2-TS mode.  When running the 2-TS
mode, one uses the a-onl account on \mycomp{adaql2} for one
spectrometer and the adev account on \mycomp{adaql1} or \mycomp{adaqs2} 
for the other spectrometer.  
The 1-TS mode normally uses the 
a-onl account on \mycomp{adaql2} only.
The information that follows refers to 
the a-onl account, but the other account is quite similar.

\par
Here is how to start and stop a run.
Normally, when you come on shift, 
runcontrol will be running.  If not,
see the section on ``Cold Start'' below.
To start and stop runs, push the buttons
``Start Run'' and ``End Run'' in the
runcontrol GUI.   To change configurations
use the ``Run Type'' button.  If you have
been running you will first have to push the
``Abort'' button before you can change the 
run type. Typically the configurations
you want are the following.

\begin{itemize} 
\item[~]TWOSPECT -- For running the two spectrometers in
1-TS mode.
\item[~]PEDRUN -- To do a pedestal run in 1-TS mode
\item[~]RIGHTHRS -- For R-arm in 2-TS mode
\item[~]LEFTHRS -- For L-arm in 2-TS mode
\item[~]PEDRUNR -- To do a pulser run for R-arm in 2-TS mode
\item[~]PEDRUNL -- To do a pulser run for L-arm in 2-TS mode
\end{itemize} 

\par 
A note about pedestal runs.  They have the exclusive
purpose of obtaining pedestals used for pedestal
suppression.  For details about what is done
and hints for getting pedestals for ESPACE (which
does not want the PEDRUN result), see \mycomp{/ped/README}.

}

\infolevthree{

\subsection{
 Some Frequently Asked Questions about DAQ}


\begin{itemize} 
\item{ {\it Q: Where is the data ?} \hskip 0.05in  
Use a command ``find\_run 1745'' to find
where run 1745 has been written on disk and MSS.  
The data are first written to disks like
\mycomp{/adaql2/dataN}, N=1,2,3...etc.  Files are 
automatically split, with suffixes .0,.1,.2...etc.
Splitting occurs at 2 Gbytes to avoid problem
of system file size limit.  
Files are archived automatically to tape in the MSS
tape silo.  Two tape copies are made.  Data are
purged from disk automatically.  Users should
{\it never} attempt to copy, move, or erase data.}

\item{ {\it Q: How to adjust prescale factors ? } 
\hskip 0.05in
Edit the file \mycomp{\~/prescale/prescale.dat}.
One common problem is putting typographical
errors here which then leads to no triggers
getting accepted.}

\item{ {\it Q: What is the deadtime ? } \hskip 0.05in
The deadtime is displayed in the datamon
window, which normally is running next to the 
runcontrol window, but if this window is
not up, type \mycomp{datamon} to bring it up.
This window also shows the full-path-name of the file
being written by CODA for the present run.}

\item{ {\it Q: Why is the deadtime so high ?  
(and related)} \hskip 0.1in
Search for answers among the following.
The standard lore is that 30\% deadtime is tolerable,
but you should ask your analysis team to decide.
Sometimes people seeing large deadtimes
have forgotten to observe that the beam is in
pulsed mode.  Another possibility is that the
workstation is overloaded.  The computer used
for CODA should not be used for much else.
Do not attempt to read or write rapidly to the
same physical disk to which CODA is writing.
Sometimes it is observed that the workstation
itself is very sluggish.  
This could be due to a foreign
mounted disk having gone away, and there are
other possible reasons.  If a Cold Start of CODA
doesn't solve the problem, you may 
try rebooting the workstation
(see computer section).  Also, if
the event size changes substantially, e.g. due
to VDC thresholds being turned off (a common mistake), 
the deadtime as a function
of rate will change, especially 
in the regime of high rates.}

\end{itemize}

\subsection{ Quick Resets }

Problems with CODA can usually be solved with a simple
reset.  If not, try a Cold Start (see next section).
Do not waste an hour of beam time on resets; 
if they fail, call an expert.  
The expert claims he can restart CODA 
90\% of the time within 10 minutes.

If a ROC (ReadOut Controller, or crate)
is hung up, reboot by going the workspace
``Components'' and typing \mycomp{reboot}.  If this 
doesn't work, try pressing the reset button 
which is on the ``Crate Resets'' section of the
Hall A General Tools EPICS~\cite{EPICSwww} Gui.  Telnet back into
the ROC to verify its alive.  Then press ``Reset''
in runcontrol, download and start a new run.

\subsection{ Cold Start of CODA}

\par
If CODA is not running, or if it gets hung up,
you can do a cold start.  Frequently a subset of
these steps is sufficient to recover from a hangup,
but it takes some experience to realize the
minimum of steps that
are necessary, so the simplest 
thing is to do them all, which takes a few minutes.

\begin{itemize}
\item{Kill off all CODA processes on the workstation
by typing \mycomp{kcoda}.  This stops runcontrol, the
event builder, and other processes, and allows
for a clean start.  The \mycomp{kcoda} script will then
tell you exactly what to do next.}
\item{Make sure the fastbus and VME crates are
running.  The crates are usually known by 
``ROCnumber-computer-(portserver-port)''
where ROCnumber is the unique number for that
ROC (ReadOut Controller, or crate),
computer is the internet name and the 
portserver-port is the portserver IP and port\#
where to login.
An example might be \hskip 0.05in
ROC4-hallasfi4-(hatsv4,port3) which is
ROC4, a fastbus crate with IP \mycomp{hallasfi4} attached
to the portserver IP \mycomp{hatsv4} at port 3.
You can check if the ROCs
are up by looking on the Components work space
at the telnet session (if it's not logged, 
try to telnet in).
If the ROCs don't talk to runcontrol, you can type
\mycomp{reboot} at the arrow prompt ($\rightarrow$).   If you
don't get this arrow prompt, or if you can't telnet in,
the computer is hung up, so press 
the reset button in the ``Crate Resets'' GUI
available from the EPICS screen for
Hall A General Tools.
After the ROC comes back (2 minutes),
telnet back in to verify it's up.
On rare occassions it is necessary to
power cycle the crate, which requires access. }
\item{ Start runcontrol interactively
by typing \mycomp{runcontrol}.
Actually there may be
more steps prior to starting runcontrol,
depending on the experimental setup.
See the printout from \mycomp{kcoda} for instructions.}
\item{ In runcontrol,
press the ``Connect'' button.  
After ``connect''
wait 10 seconds and press ``Run Types''.  
After configure and before download, 
press the ``Reset'' button in the upper left corner.
Choose the run type from the dialog box
(see section on Running
CODA for descriptions of run types).}
\item{ After you configure and download the Run Type,
you can ``Start Run'' to start a new run.}


\end{itemize}


\subsection{ Recovering from a Reboot of Workstation}

If the workstation from which you are running CODA
was rebooted, here is how to recover DAQ.
Login as the relevant account, which is usually
a-onl for 1-DAQ operation. Passwords for the online
accounts should be available on a paper on the wall
in the counting room, or ask the run coordinator.
In the workspace for ``Components'' telnet into
all the ROCs.  If the x-terms windows are not 
available, type \mycomp{setupxterms}.  Start emacs
for the prescale factors: 
\mycomp{emacs /prescale/prescale.dat}.
Make sure msqld is running in the process list;
it is supposed to start when the computer boots.
Then do a Cold Start (see section above).

}

\infolevone{

\section{Electronic Logbook and Beam Accounting}

Two tools are available for logging information
by the shift workers: \hskip 0.05in
(1) The Electronic Logbook ``halog'', and
\hskip 0.05in
(2) The Hall Beam--Time Accounting Table.

\par
The electronic logbook is a web-based
repository of logbook data.  There are
two ways to make entries:  One can use
the halog GUI (type \mycomp{halog} and make your 
entry), or one may use a script
to insert a file.  
Some data from EPICS
and scalers, among other things, are inserted
automatically into halog on each start-of-run
and each end-of-run.  These data also get
written into files with the run number in
their name in \mycomp{/epics/runfiles}.
Data appear on the web at a certain URL
\htmladdnormallinkfoot{}{\url{http://www.jlab.org/~adaq/halog/html/logdir.html}}.
It is recommended that one software expert
from the experiment be
assigned to modify the logging scripts
as he or she sees fit.

\par 
The Hall Beam--Time Accounting Table is the mechanism
to summarize and record
how the beam time in a shift was spent.  
The shift leader is responsible for
submitting this table at the end of the shift.
When submitted, the data are 
logged in a database
and a summary is e-mailed to various people like the
run coordinators and the hall leader.
When you come on shift, the GUI is probably
already running.  If not, you may start it
by logging onto \mycomp{adaql1} as the
adaq account and type ``\mycomp{bta}''.
It is a fairly obvious GUI,
but there is also some online help.

}

\infolevthree{

\section {Port Servers}
Portservers are devices on the network that
allow access to RS232 ports.  Here is how to 
connect from a computer: \mycomp{telnet hatsv5 2011}
will connect to the portserver at IP \mycomp{hatsv5}
and port 11.  Note, the offset of 2000 is needed.
For dealing with HV, it is best to use a Linux PC
for which the keymap is F1 = PF1 and F2 = PF2.

If another person is connected to a certain port,
you cannot connect.  To bump off another user, login
as root with password available from the paper posted
on the wall of the counting room (or ask run coordinator)
as follows \mycomp{telnet hatsv5} as user = root.
At the prompt, type \mycomp{kill tty=4} to clear port 4,
then \mycomp{exit}.  Now you can \mycomp{telnet hatsv5 2004}.

\begin{table}\centering
\caption[Data Acquisition: Port Servers for DAQ]{
Port Servers for DAQ}
\begin{tabular}{|l|l|l|}  \hline
server IP &  Port & Device \\ \hline
\mycomp{hatsv3}    &   1     &  vt100 Dumb Terminal \\
\mycomp{hatsv3}    &   2     &  ROC1 Lower Fastbus Crate \\
\mycomp{hatsv3}    &   3     &  TS0 Trig. Super. VME Crate \\
\mycomp{hatsv3}    &   4     &  R-arm Upper HV Crate \\
\mycomp{hatsv3}    &   5     &  R-arm Lower HV Crate \\
\mycomp{hatsv3}    &   8     &  ROC2 Upper Fastbus Crate \\
\mycomp{hatsv4}    &   1     &  vt100 Dumb Terminal \\
\mycomp{hatsv4}    &   2     &  ROC3 Lower Fastbus Crate \\
\mycomp{hatsv4}    &   3     &  ROC4 Upper Fastbus Crate \\
\mycomp{hatsv4}    &   5     &  HV Crate   \\
\mycomp{hatsv4}    &   6     &  RICH HV Crate \\
\mycomp{hatsv4}    &   7     &  RICH VME Crate \\
\mycomp{hatsv4}    &   14    & TS1 Trig. Super. VME Crate \\
\mycomp{hatsv5}    &   1     &  vt100 Dumb Terminal \\
\mycomp{hatsv5}    &   2     &  e-P Crate 1 \\
\mycomp{hatsv5}    &   3     &  Moller 1 \\
\mycomp{hatsv5}    &   4     &  Moller 2 \\
\mycomp{hatsv5}    &   8     &  Compton ROC3 \\
\mycomp{hatsv5}    &   9     &  Compton ROC4 \\
\mycomp{hatsv5}    &   10    &  Compton ROC5 \\
\mycomp{hatsv5}    &   11    &  Beamline HV  \\
\mycomp{hatsv5}    &   12    &  e-P Crate 2 \\
\mycomp{hatsv5}    &   13    &  ARC Energy \\
\mycomp{hatsv5}    &   14    &  ROC14 VME Crate \\
\mycomp{hatsv5}    &   15    &  ROC15 VME Crate \\
\mycomp{hatsv12}   &   5     &  Compton ROC1 \\
\mycomp{hatsv12}   &   6     &  Compton ROC2 \\
\mycomp{hatsv15}   &   -     &  2nd Floor Counting Room \\
\mycomp{hatsv9}    &   4     &  Parity DAQ Crate \\
\hline
\end{tabular}
\end{table}

} %infolev
 
% ===========  CVS info
% $Header: /group/halla/analysis/cvs/tex/osp/src/daq_trig/daq.tex,v 1.6 2003/12/05 06:45:07 gen Exp $
% $Id: daq.tex,v 1.6 2003/12/05 06:45:07 gen Exp $
% $Author: gen $
% $Date: 2003/12/05 06:45:07 $
% $Name:  $
% $Locker:  $
% $Log: daq.tex,v $
% Revision 1.6  2003/12/05 06:45:07  gen
% Polishing
%
% Revision 1.5  2003/11/17 06:49:00  gen
% Cosmetic changes
%
% Revision 1.4  2003/08/01 20:34:06  rom
% updated for year 2003
%
% Revision 1.3  2003/06/06 15:38:43  gen
% Revision printout changed
%
% Revision 1.2  2003/06/05 23:30:00  gen
% Revision ID is printed in TeX
%
% Revision 1.1.1.1  2003/06/05 17:28:32  gen
% Imported from /home/gen/tex/OSP
%
%  Revision parameters to appear on the output
