 
\section{Spectrometer Data Acquisition}


%  Re:  Hall A OPS Manual Chapter for DAQ and Trigger
%Author:  Robert Michaels, Hall A Staff, Jefferson Lab
% Date:  March, 1999

% Files needed for the Hall OPS Manual chapter on DAQ and
% Trigger are the following.  

%    1. daq.tex   (the latex file)
%    2. etrig.epsi     (encapsulated postscript figure)
%    3. htrig.epsi         "            "         "
%    4. coinc.epsi         "            "         "


\par
The Hall A data acquisition uses CODA\cite{CODAwww}
(CEBAF Online Data Acquisition), a toolkit
developed at Jefferson Lab by the Data
Acquisition Group.  
%For general information
%about CODA, see the CODA site\htmladdnormallinkfoot{}{\url{http://coda.jlab.org/}}.

\par
We have one fastbus crate in the
electron spectrometer and two fastbus
crates in the hadron spectrometer.
The fastbus modules are of the following
types:
(1) LeCroy model
1877 TDCs operating in common--stop with 0.5 nsec
resolution for our drift chambers and
straw chambers; (2) model 1875 TDCs
operating in common--start with 0.1 nsec resolution
for our scintillators and trigger 
diagnostics; and (3) model 1881M ADCs
for signals from scintillators, 
Cerenkov, and leadglass detectors.
Event--driven readout of our beam position
monitors and raster current is
available from VME systems.

\par
The trigger supervisor is a custom--made
module built by the data
acquisition group.  Its functions are to
synchronize the readout crates, to administer
the deadtime logic of the entire system, and
to prescale various trigger inputs.  
We have two trigger supervisors,
one in each spectrometer.  This allows us to
run the spectrometers independently if needed.

\par
Use the public account ``adev''
for running runcontrol, and use ``adaq''
for other online software including ESPACE.
On ``adaq'' the directory
tree of an experiment is adaq/\$EXPERIMENT
which is organized in subdirectories of 
various tasks, such as scaler display,
ESPACE, and other online codes, all of
which will be described in sections below.
The trigger management software is described in
the Trigger chapter. 

\subsection{General Computer Information}

\par
In the counting room we have various computers 
for DAQ, analysis, and controls.  The controls
subnet (which includes the ``hac'' computer 
among others) is the 
responsibility of J. Gomez and is documented in Chapter 6.
The DAQ computer's names are denoted by adaqXN, where
X=s is SunOS, X=h is HP-UX, and X=l is for Linux PC. 
adaqs1 is the Compton DAQ computer and is normally
reserved by the Compton group.  adaqs2 and s3
are for running the spectrometer DAQ or doing
online analysis.  adaqh2, h3, and h4 are for
running the older version 1.4 of CODA still
used by some setups.  adaql1 and l2 are relatively
fast Linux PCs available for analysis and are 
administered by Ole Hansen.  

\par
To reboot the Suns,  login as ``adaq''
and type ``reboot''.  On HP's adaqh2-4 you cannot
reboot, but must ``shutdown''.  For both HPs and Suns
here is how to shutdown:  Login as ``adaq'' and 
type ``shutdown'', wait several minutes until the 
screen indicates it's safe (on Suns it goes black and
on HPs it will say it's safe), then turn off power.
To reboot the Linux machines, first hit Ctrl-Alt-F1
to switch to a text console, then hit Ctrl-Alt-Del 
to reboot.
If power fails for a prolonged time,
you must shutdown before the UPS fails.

\par
For all computer problems you may
call Robert Michaels (x7410)
unless you happen to know another expert who can solve your problem.
If Michaels is not available, call Ole Hansen  (x7627) or
Javier Gomez (x7498).  


\subsection{Beginning of Experiment Checkout}

\par
This subsection describes the 
checkout of DAQ and trigger
needed before an experiment can start.

\begin{enumerate}
\item{First ensure that all the fastbus, VME,
CAMAC, and NIM crates are powered
on. They should
boot up in a functional state, except for
heavily loaded fastbus crates that sometimes
lose their NVRAM.  (If that happens, call R. Michaels)}

\item{You may download
a default trigger, following the directions in
the trigger chapter.  If the hadron momentum changes
you may need to set a new delay.  A trigger expert
should do the start-of-experiment
trigger checklist.}

\item{Make sure the HV is on for all detectors
and that the values are normal.}

\item{Start the xscaler
display following the instructions below and
check that the rates from detectors are normal.}
 
\item{Startup runcontrol (CODA) using the directions below
and start a run.  With the trigger downloaded
and the HV on, you are taking cosmics data, typically at 
a rate of 3 Hz per spectrometer.  
Examine the data using dataspy,
dhist, and ESPACE as explained below.  Compare the
plots and printouts to normal values.}

\end{enumerate}

\subsection{Running CODA}

\par
This section describes how to run CODA for
the spectrometer DAQ.  There are two modes:
1) The most common is the ``1-Trigger-Supervisor (1-TS)''
mode which uses one trigger supervisor and
is used for coincidence experiments; and
2) The ``2-Trigger-Supervisor (2-TS)'' mode which
is used for running the two spectrometers
independently.

\par
The 1-TS mode can also handle single--arm
triggers but is about 1/2 the aggregate speed
of the 2-TS mode.  When running the 2-TS
mode, one uses the adev account on adaqs2 for the
H-spectrometer and the atrig account on adaqs3 for
the E-spectrometer.  
The 1-TS mode normally uses the 
adev account on adaqs2 only.
The information that follows refers to 
the adev account, but
the atrig account is quite similar.  For example,
for a file like /home/adev/prescale/prescale.dat
there is a corresponding file at
/home/atrig/prescale/prescale.dat.

\par
Here is how to start and stop a run.
Normally, when you come on shift, 
runcontrol will be running.  If not,
see the section on ``Cold Start'' below.
To start and stop runs, push the buttons
``Start Run'' and ``End Run'' in the
runcontrol GUI.   To change configurations
use the ``Run Type'' button.  If you have
been running you will first have to push the
``Abort'' button before you can change the 
run type.  Normally the configurations
you want are the following.

\begin{itemize} 
\item[~]TWOSPECT -- For running the two spectrometers in
1-TS mode.
\item[~]PEDRUN -- To do a pedestal run in 1-TS mode
\item[~]ELECTRON -- For E-arm in 2-TS mode
\item[~]HADRON -- For H-arm in 2-TS mode
\item[~]PEDRUNE -- To do a pulser run for E-arm in 1-TS mode
\item[~]PEDRUNH -- To do a pulser run for H-arm in 1-TS mode
\end{itemize} 

\par 
A note about pedestal runs.  They have the exclusive
purpose of obtaining pedestals used for pedestal
suppression.  For details about what is done
and hints for getting pedestals for ESPACE (which
does not want the PEDRUN result), see /home/adev/
ped/README.)

\subsubsection{
 Some Frequently Asked Questions about DAQ}


\begin{itemize} 
\item{ {\it Q: Where is the data ?} \hskip 0.05in  
Use a command ``find\_run 1745'' to find
where run 1745 has been written on disk and MSS.  
The data are first written to disk.  Files are 
automatically split, with suffixes .0,.1,.2...etc.
Splitting occurs at 2 Gbytes to avoid problem
of system file size limit.  Files are
archived automatically to tape in the MSS
tape silo.  Two tape copies are made.  Data are
purged from disk automatically.  Users should
{\it never} attempt to copy, move, or erase data.}

\item{ {\it Q: How to adjust prescale factors ? } 
\hskip 0.05in
Edit the file /home/adev/prescale/ prescale.dat.
One common problem is putting typographical
errors here which then leads to no triggers
getting accepted.}

\item{ {\it Q: What is the deadtime ? } \hskip 0.05in
The deadtime is displayed in the datamon
window, which normally is running next to the 
runcontrol window, but if this window is
not up, type ``datamon'' to bring it up.
This window also shows the full-path-name of the file
being written by CODA for the present run.}

\item{ {\it Q: Why is the deadtime so high ?  
(and related)} \hskip 0.1in
Search for answers among the following.
The standard lore is that 30\% deadtime is tolerable,
but you should ask your analysis team to decide.
Sometimes people seeing large deadtimes
have forgotten to observe that the beam is in
pulsed mode.  Another possibility is that the
workstation is overloaded.  The computer used
for CODA should not be used for much else.
Do not attempt to read or write rapidly to the
same physical disk to which CODA is writing.
Sometimes it is observed that the workstation
itself is very sluggish.  
This could be due to a foreign
mounted disk having gone away, and there are
other possible reasons.  If a Cold Start of CODA
doesn't solve this, you may 
try rebooting the workstation
(see computer section).  Also, if
the event size changes substantially, e.g. due
to noise conditions, the deadtime as a function
of rate will change, especially 
in the regime of high rates.}


\end{itemize}

\subsubsection{ Cold Start of CODA}

\par
If CODA is not running, or if it gets hung up,
you can do a cold start.  Frequently a subset of
these steps is sufficient to recover from a hangup,
but it takes some experience to realize the
minimum of steps that
are necessary, so the simplest 
thing is to do them
all (it takes a few minutes).

\begin{itemize}
\item{Kill off all CODA processes on the workstation
by typing ``kcoda''.  This stops runcontrol, the
event builder, and other processes, and allows
for a clean start.}
\item{Make sure the fastbus and VME crates are
running.  The crates are named below as 
``computer(CODA-name)'',
where computer is the internet name and the 
CODA-name is the name runcontrol calls it: \hskip 0.05in
hallasfi1 (ROC1), hallasfi2 (ROC2), hallasfi3 (ROC3),
halladaq1 (TS0), halladaq4 (TS1), hallavme2 (ROC15),
and hallavme1 (ROC14).  (Note: for 1-TS mode
you don't use ROC15, and sometimes ROC3 which
has FPP is left off for good reasons.)  
You can check if the ROCs
are up by looking on the Components work space
at the telnet session (if it's not logged, 
try to telnet in).
If the ROCs don't talk to runcontrol, you can type
``reboot'' at the arrow prompt ($\rightarrow$).   If you
don't get this arrow prompt, or if you can't telnet in,
the computer is hung up, so press 
for a full 5 seconds
the labeled green button corresponding
to this ROC in the middle room of the counting
room.  After the ROC comes back (2 minutes),
telnet back in to verify it's up.}
\item{ Start runcontrol interactively
by typing ``runcontrol''.
Iconize and ignore the window from which 
you started runcontrol.}
\item{ In runcontrol,
press the ``Connect'' button.  
After ``connect''
wait 10 seconds and press ``Run Types''.  
You may also press the ``Reset'' button in the upper left
corner.  Indeed sometimes reset is {\it all} 
that's needed to fix CODA.
Choose the run type from the dialog box
(see section on Running
CODA for descriptions of run types).}
\item{ After you configure and download the Run Type,
you can ``Start Run'' to start a new run.}


\end{itemize}

\subsection{Electronic Logbook and Beam Accounting}

Two tools are available for logging information
by the shift workers: \hskip 0.05in
1) The Electronic Logbook ``halog'', and
\hskip 0.05in
2) The Hall Beam--Time Accounting Table.

\par
The electronic logbook is a web-based
repository of logbook data.  There are
two ways to make entries:  One can use
the halog GUI (type ``halog'' and make your 
entry), or one may use a script
to insert a file.  
Some data from EPICS
and scalers, among other things, are inserted
automatically into halog on each start-of-run
and each end-of-run.  These data also get
written into files with the run number in
their name in /home/adev/epics/runfiles.
Data appear on the web at a certain URL%
\htmladdnormallinkfoot{}{\url{http://www.jlab.org/~adaq/halog/html/logdir.html}}.

It is recommended that one software expert
from the experiment be
assigned to modify the logging scripts
as he or she sees fit.

\par 
The Hall Beam--Time Accounting Table is the mechanism
to summarize and record
how the beam time in a shift was spent.  
These data are logged automatically in a database
and are e-mailed to various people like the
run coordinators and the hall leader.
When you come on shift, the GUI is probably
already running.  If not, you may start it
by logging onto adaqs2 as the
adaq account, cd to /home/adaq/ACCOUNT, and
type ``atable''.  It is a fairly obvious GUI,
but there is also a help button
which explains everything.


\subsection {Terminal Servers}
There are four DEC 
server 200/MC terminal servers in Hall A;
one in each of the two spectrometer detector
huts and two near the beamline.
These servers allow you check and to modify the
detector HV, as well as the NVRAM of the 
frontend DAQ computers. 
The servers may be accessed through
dumb terminals connected to any CEBAF terminal server,
e.g. the VT100 style terminal in the
middle room of the counting house.
The table entitled ``Terminal Service for DAQ'' 
shows the ports
connected for spectrometer and beamline DAQ.

\begin{table}\centering
\caption[Data Acquisition: Terminal Service for DAQ]{Terminal Service for DAQ}
\begin{tabular}{|l|l|l|l|}  \hline
Port  &   Service & Device & Function\\ \hline
 1    &   HAC1    & halladaq1  & E-arm Trig Super TS0\\
 2    &   HAC2    & hallasfi1  & E-arm Fastbus ROC1\\
 3    &   HAC9    & halladaq4 & H-arm Trig Super TS1 \\
 4    &   HAC11   & hallasfi2  & H-arm Fastbus ROC2\\
 5    &   HAC12   & hallasfi3  & H-arm Fastbus ROC3\\
 6    &   HAC5    & LeCroy 1450 & E-arm HV control\\
 7    &   HAC6    & LeCroy 1450 & E-arm HV control\\
 8    &   HAC13   & LeCroy 1450  & H-arm HV control\\
\hline
\end{tabular}
\end{table}




 
% ===========  CVS info
% $Header: /group/halla/analysis/cvs/tex/osp/src/daq_trig/daq.tex,v 1.1 2003/06/05 17:28:32 gen Exp $
% $Id: daq.tex,v 1.1 2003/06/05 17:28:32 gen Exp $
% $Author: gen $
% $Date: 2003/06/05 17:28:32 $
% $Name:  $
% $Locker:  $
% $Log: daq.tex,v $
% Revision 1.1  2003/06/05 17:28:32  gen
% Initial revision
%
