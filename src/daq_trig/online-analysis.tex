\section[Online Analysis, Data Checks]{Online Analysis, Data Checks
\footnote{
  $CVS~revision~ $Id: online-analysis.tex,v 1.1 2003/06/06 15:38:43 gen Exp $ $
}
\footnote{Authors: R.Michaels \url{mailto:rom@jlab.org}}
}
 
\par
The following tools are available for checking
data online.


\subsubsection{Scaler Display and Scaler Events}
Scaler rates and values are displayed using
a MOTIF based display called ``xscaler''
written by C. Howell of Duke University.
Normally this is already running on adaqs2 or s3.
It also runs on HP-UX but the SunOS version
is preferred.  If it is not running, login as
adaq and go to the appropriate directory,
which is 
/home/adaq/\$EXPERIMENT/electron/scaler
and
/home/adaq/\$EXPERIMENT/hadron/scaler
for the E-arm and H-arm respectively
where \$EXPERIMENT is an environment
variable like e95001.
Then type ``xscaler'' there.  
{\it Remember} to push the button ``Start''.
The first several pages are the scaler rates and
the next half of the pages are the absolute
scaler counts.  The scalers are cleared at
the beginning of each CODA run.
Scalers are read out at approximately 0.5 Hz 
and injected into the CODA data-stream 
as event type 140.  A file
``scaler\_history.dat'' is maintained which is
a complete history of scaler readings at
the end of each run that ended normally.
For 1-TS mode, this file is in /home/adev/scaler.

\subsection{Analysis using ESPACE}

ESPACE is the main offline software package
for analyzing Hall A experiments, and it is
used for rapid near-online analysis in the
counting room.  ESPACE is
documented in a separate chapter, 
but it is worth mentioning here in a list of
essential tools for checking data.

\subsection{Dataspy and Dhist} 

Dataspy and Dhist are online diagnostic programs
whose purposes are: 
\hskip 0.05in 1) To print
out randomly sampled detector data; and
\hskip 0.05in 
2) To automatically plot to the screen
histograms of online data.
\hskip 0.05in 
Dhist is actually a shell script which runs 
the executable ``dplot''.  
For some online help, simply type
``dataspy'' or ``dplot''.
These codes analyze a random sample of
of {\it raw, uncut} data
in real time from the CODA computer.
The data are distributed on the network by
a server which obtains data from shared memory.

\par 
This paragraph assumes you are on shift
and wish to run dhist.  
While ``dplot'' can run in
several ways, let's be definite:  Log in to
adaqh2 (or s3) as ``adaq'' account.  If dhist
isn't running, type ``dhist''.  Now you will
see a reminder of what directory to go to and
that you should type ``./dhist'' there.

\par 
There are optional interfaces ``dopte'' and 
``dopth'' for turning on/off the histogram
pages.  The dopte(h) interfaces also show an 
alarm status for the histograms by statistical
comparison to a set of reference histograms.
To start these interfaces, type ``dopte'' or
``dopth'' for the E-arm and H-arm respectively.

\par 
dhist makes about 20 pages of plots which pop
up on the workstation screen and remain for
a few seconds in succession.  Each page also
results in a postscript file for printing, and
there is a histogram file dplot.his which one
can view in PAW.    

% ===========  CVS info
% $Header: /group/halla/analysis/cvs/tex/osp/src/daq_trig/online-analysis.tex,v 1.1 2003/06/06 15:38:43 gen Exp $
% $Id: online-analysis.tex,v 1.1 2003/06/06 15:38:43 gen Exp $
% $Author: gen $
% $Date: 2003/06/06 15:38:43 $
% $Name:  $
% $Locker:  $
% $Log: online-analysis.tex,v $
% Revision 1.1  2003/06/06 15:38:43  gen
% Revision printout changed
%
% Revision 1.2  2003/06/05 23:30:00  gen
% Revision ID is printed in TeX
%
% Revision 1.1.1.1  2003/06/05 17:28:32  gen
% Imported from /home/gen/tex/OSP
%
%  Revision parameters to appear on the output
