\documentclass[12pt,letterpaper]{article} 


% -  Define the environments for different information levels (lev=0 - no special environment)
% default: everything is set. It can be redefined with the file currinfolev.tex
% (Comment: is done in such a clumsy way since if-then-else does not work with latex2html) 
\newcommand{\infolevone}[1]{#1}%
\newcommand{\infolevtwo}[1]{#1}%
\newcommand{\infolevthree}[1]{#1}%
\newcommand{\infolevfour}[1]{#1}%

% input the INFO LEVEL flag, say \def\infolevel{4} and redefine the environments
%
\def\infolevel{4}  % INFO LEVEL flag 
%   =0 - overview+safety only, 
%   =1 + procedures, 
%   =2 + components, 
%   =3 + principles of operation, 
%   =4 + performance
\renewcommand{\infolevltfour}[1]{}%
\renewcommand{\infolevltthree}[1]{}%
\renewcommand{\infolevlttwo}[1]{}%
\renewcommand{\infolevltone}[1]{}%
\renewcommand{\infoleveqthree}[1]{}%
\renewcommand{\infoleveqtwo}[1]{}%
\renewcommand{\infoleveqone}[1]{}%
\renewcommand{\infoleveqnull}[1]{}%

% %
\def\infolevel{4}  % INFO LEVEL flag 
%   =0 - overview+safety only, 
%   =1 + procedures, 
%   =2 + components, 
%   =3 + principles of operation, 
%   =4 + performance
% ===========  CVS info
% $Header: /group/halla/analysis/cvs/tex/osp/src/common/infolevel.tex,v 1.2 2003/06/05 23:30:00 gen Exp $
% $Id: infolevel.tex,v 1.2 2003/06/05 23:30:00 gen Exp $
% $Author: gen $
% $Date: 2003/06/05 23:30:00 $
% $Name:  $
% $Locker:  $
% $Log: infolevel.tex,v $
% Revision 1.2  2003/06/05 23:30:00  gen
% Revision ID is printed in TeX
%
% Revision 1.1.1.1  2003/06/05 17:28:33  gen
% Imported from /home/gen/tex/OSP
%
%  Revision parameters to appear on the output

%   =0 - overview+safety only, 
%   =1 + procedures, 
%   =2 + components, 
%   =3 + principles of operation, 
%   =4 + performance

\usepackage{color}
\usepackage{longtable}
% \usepackage{amsfonts}
% \usepackage{amssymb}

\def\Mcol{black}  % Main text color
\def\Scol{red}    % Safety text color
\def\SBcol{red}   % Safety marginbar color
\def\Ccol{magenta}   % Computer input/output color
%\definecolor{Scol}{rgb}{red}    % Safety text color

%begin{latexonly}
% ===    Set a true/false value for PDF hyper marks  
\newif\ifhyprf
%\hyprffalse
\hyprftrue

\newif\ifpdf
\ifx\pdfoutput\undefined
    \pdffalse           % we are not running PDFLaTeX
\else
    \pdfoutput=1        % we are running PDFLaTeX
    \pdftrue
\fi

\ifpdf
  \pdfcompresslevel=9
  \usepackage[pdftex]{graphicx}
%  \usepackage{thumbpdf}
  \definecolor{rltred}{rgb}{0.75,0,0}
  \definecolor{rltgreen}{rgb}{0,0.3,0}
  \definecolor{rltblue}{rgb}{0,0,0.75}
  \definecolor{rltdarkgreen}{rgb}{0.1,0.6,0.1}
  \ifhyprf
     \usepackage[pdftex,
         colorlinks=true,
         urlcolor=rltblue,       % \href{...}{...} external (URL)
         filecolor=rltgreen,     % \href{...} local file
         linkcolor=rltred,       % \ref{...} and \pageref{...}
         citecolor=rltdarkgreen, % citations
         pagebackref,
         pdfpagemode=None,
         pdftitle={OSP Hall A},
         pdfauthor={Hall A},
         pdfsubject={JLab Hall A Operations},
         pdfkeywords={JLab HallA operations safety OSP}]{hyperref}
  \fi
  \usepackage{pdfcolmk}
  \DeclareGraphicsExtensions{.pdf,.png,.jpg}
\else
  \usepackage{graphicx}
  \DeclareGraphicsExtensions{.eps,.epsi,.ps,.eps.gz,.epsi.gz,.ps.gz}
\fi

\usepackage{cite}
\usepackage{comment}
\usepackage{ifthen}
\usepackage{changebar}
% \usepackage{url}
\usepackage{html,hthtml}

\oddsidemargin=0.25in 
\evensidemargin=0.25in 
\topmargin=-0.2in 
\textwidth=6.25in 
\textheight=8.7in
\renewcommand{\textfraction}{0.05} 

%end{latexonly}

\newcommand{\obsolete}[1]{}%
\newcommand{\myhtml}[1]{}%

\begin{htmlonly}
  \usepackage{graphicx}
%  \usepackage{epsf}
  \DeclareGraphicsExtensions{.eps,.epsi,.ps}
  \def\makeglossary{}
  \pagecolor[named]{White}

% -  Define the environments for different information levels (lev=0 - no special environment)

\end{htmlonly}


\newenvironment{safetyen}[2]%  SAFETY environment
       {\color{\SBcol}%
              \marginpar{\rule[-#2mm]{1mm}{#1mm}}%
        \color{\Scol}}%
       {\color{\Mcol}}
%       {\color{red}\bfseries}%
%       {\color{black}\rmfamily}

\newcommand{\dirfig}[0]{figs}
\newcommand{\dircur}[0]{}
\newcommand{\mycomp}[1]{{\color{\Ccol}{\sf #1}}}

 
% ===========  CVS info
% $Header: /group/halla/analysis/cvs/tex/osp/src/common/preamb_common.tex,v 1.2 2003/06/05 23:30:00 gen Exp $
% $Id: preamb_common.tex,v 1.2 2003/06/05 23:30:00 gen Exp $
% $Author: gen $
% $Date: 2003/06/05 23:30:00 $
% $Name:  $
% $Locker:  $
% $Log: preamb_common.tex,v $
% Revision 1.2  2003/06/05 23:30:00  gen
% Revision ID is printed in TeX
%
% Revision 1.1.1.1  2003/06/05 17:28:33  gen
% Imported from /home/gen/tex/OSP
%
%  Revision parameters to appear on the output
 % include the common preambule 
% - Definition for the namestab environment to print a table with names

\newenvironment{namestab}[3]{%
     \newcommand{\mynamtabcap}[0]{%
        \caption[#2]{#3}%
        \label{#1}%
     }%
   \begin{table}[htp]
     \centering
     \small
     \begin{tabular}{|p{114pt}|l|l|l|p{94pt}|p{70pt}|} \hline
         \multicolumn{1}{|c|}{Name (first,last)} & Dept. & \multicolumn{2}{c|}{Call\cite{inst:JLab}} & 
         \multicolumn{1}{c|}{e-mail} & \multicolumn{1}{c|}{Comment} \\ 
         \cline{3-4}   
             &   & Tel & Pager &  &  \\
         \hline 
}%
{    \hline
     \end{tabular}%
     \normalsize%
     \mynamtabcap%
   \end{table}%
}

% A header through all the columns
\newcommand{\namestabheader}[1]{\hline \multicolumn{6}{|c|}{\em #1} \\ \hline}


\newcommand{\myil}[0]{{\color{red}{\em infolevel}}}

% \newcommand{\Cerenkov}{\v{C}erenkov} 
\newcommand{\Cerenkov}{Cherenkov} 
\parindent=2em 
\hyphenation{cebaf} 

\begin{document}
\color{\Mcol}
\pagestyle{headings}

% \pagestyle{empty}
\hypersetup{pageanchor=false}
\begin{titlepage}
\title{{\bf OSP Instructions for Contributors}}

\author{E.Chudakov} 
\setcounter{tocdepth}{3}
\end{titlepage} 
\maketitle 
\hypersetup{pageanchor=true}
 
% \pagenumbering{arabic} 
% \setcounter{page}{1} 
\tableofcontents

%\listoftables

%\listoffigures
\newcommand{\JessieButler}[1]{Jessie Butler & Hall-A & 5544 & & \email{jbutler@jlab.org} & #1 \\ }
\newcommand{\AndrewLumanog}[1]{Andrew Lumanog & Hall-A & 7459 & 327- & \email{andrewl@jlab.org} & #1 \\ }
\newcommand{\JasonGlorioso}[1]{Jason Glorioso & Hall-A & 6258 & 831- & \email{glorioso@jlab.org} & #1 \\ }
\newcommand{\MahlonLong}[1]{Mahlon Long & Hall-A & 6436 & 770- & \email{mlong@jlab.org} & #1 \\ }
\newcommand{\TechonCall}[1]{~ Tech-on-Call & Hall-A & W.B. & &  & #1 \\ }
\newcommand{\CryotargonCall}[1]{~ Cryotarg-on-Call & Hall-A & W.B. & &  & #1 \\ }
\newcommand{\CryoonCall}[1]{~ Cryo-on-Call & via-MCC & 7048 & &  & #1 \\ }
\newcommand{\CHLgroup}[1]{~ CHL-group & Cryo & 7405 & &  & #1 \\ }
\newcommand{\RadCon}[1]{~ Rad-Con & Hall-A & 7236 & &  & #1 \\ }
\newcommand{\HariAreti}[1]{Hari Areti & Accel. & 7187 & 584- & \email{areti@jlab.org} & #1 \\ }
\newcommand{\MarkAugustine}[1]{Mark Augustine & Hall-A & 7103 & 584- & \email{augustin@jlab.org} & #1 \\ }
\newcommand{\MartialAuthier}[1]{Martial Authier & CEA & 4324\cite{inst:CEA} & & \email{mauthier@Cea.Fr} & #1 \\ }
\newcommand{\ToddAverett}[1]{Todd Averett & CWM & 5007 & & \email{averett@jlab.org} & #1 \\ }
\newcommand{\PierreBertin}[1]{Pierre Bertin & IN2P3 & OFF & & \email{bertin@jlab.org} & #1 \\ }
\newcommand{\AlexandreCamsonne}[1]{Alexandre Camsonne & Hall-A & 5064 & 660- & \email{camsonne@jlab.org} & #1 \\ }
\newcommand{\GordonCates}[1]{Gordon Cates & UVA & 6932 & & \email{cates@jlab.org} & #1 \\ }
\newcommand{\JianPingChen}[1]{Jian-Ping Chen & Hall-A & 7413 & & \email{jpchen@jlab.org} & #1 \\ }
\newcommand{\GregSmith}[1]{Greg Smith & Hall-C & 5405 & & \email{smithg@jlab.org} & #1 \\ }
\newcommand{\SilviuCovrig}[1]{Silviu Covrig & Hall-C & 6410 & & \email{covrig@jlab.org} & #1 \\ }
\newcommand{\EugeneChudakov}[1]{Eugene Chudakov & Hall-A & 6959 & 584- & \email{gen@jlab.org} & #1 \\ }
\newcommand{\EvaristoCisbani}[1]{Evaristo Cisbani & INFN & OFF & & \email{cisbani@jlab.org} & #1 \\ }
\newcommand{\FrancescoCusano}[1]{Francesco Cusano & INFN & & &  & #1 \\ }
\newcommand{\NathalieColombel}[1]{Nathalie Colombel & CEA & 8350\cite{inst:CEA} & & \email{ncolombel@Cea.Fr} & #1 \\ }
\newcommand{\ChristopherCurtis}[1]{Christopher Curtis & Hall-A & 7086 & 438- & \email{curtis@jlab.org} & #1 \\ }
\newcommand{\FrancescoCusanno}[1]{Francesco Cusanno & INFN & OFF & & \email{cusanno@jlab.org} & #1 \\ }
\newcommand{\TonyDay}[1]{Tony Day & Hall-A & & &  & #1 \\ }
\newcommand{\PascaleDeck}[1]{Pascale Deck & CEA & 2426\cite{inst:CEA} & & \email{pdeck@Cea.Fr} & #1 \\ }
\newcommand{\AlainDelbart}[1]{Alain Delbart & CEA & 3454\cite{inst:CEA} & & \email{adelbart@Cea.Fr} & #1 \\ }
\newcommand{\AlexandreDeur}[1]{Alexandre Deur & UVA & 7526 & 584- & \email{deurpam@jlab.org} & #1 \\ }
\newcommand{\GaryDezern}[1]{Gary Dezern & Hall-A & 7119 & & \email{dezern@jlab.org} & #1 \\ }
\newcommand{\ChiranjibDutta}[1]{Chiranjib Dutta & Kentucky & OFF & & \email{chiran@jlab.org} & #1 \\ }
\newcommand{\ToddEwing}[1]{Todd Ewing & Hall-A & 6097 & 349- & \email{jtewing@jlab.org} & #1 \\ }
\newcommand{\HeidiFansler}[1]{Heidi Fansler & Hall-A & 6915 & 349- & \email{fansler@jlab.org} & #1 \\ }
\newcommand{\RobertFeuerbach}[1]{Robert Feuerbach & Hall-A & & &  & #1 \\ }
\newcommand{\EdFolts}[1]{Ed Folts & Hall-A & 7857 & & \email{folts@jlab.org} & #1 \\ }
\newcommand{\PeteFrancis}[1]{Pete Francis & Hall-A & 7528 & 289- & \email{francis@jlab.org} & #1 \\ }
\newcommand{\HaiyanGao}[1]{Haiyan Gao & Duke & 5314 & & \email{gao@jlab.org} & #1 \\ }
\newcommand{\RonaldGilman}[1]{Ronald Gilman & Rutgers & 7011 & & \email{gilman@jlab.org} & #1 \\ }
\newcommand{\OleksandrGlamazdin}[1]{Oleksandr Glamazdin & KhIPT & 5441 & & \email{glamazdi@jlab.org} & #1 \\ }
\newcommand{\JavierGomez}[1]{Javier Gomez & Hall-A & 7498 & & \email{gomez@jlab.org} & #1 \\ }
\newcommand{\RickGonzales}[1]{Rick Gonzales & Hall-A & 7198 & 289- & \email{gonzales@jlab.org} & #1 \\ }
\newcommand{\ViktorGorbenko}[1]{Viktor Gorbenko & KhIPT & OFF & & \email{gorbenko@jlab.org} & #1 \\ }
\newcommand{\FrancoisGougnaud}[1]{Francois Gougnaud & CEA & & &  & #1 \\ }
\newcommand{\OleHansen}[1]{Ole Hansen & Hall-A & 7627 & 584- & \email{ole@jlab.org} & #1 \\ }
\newcommand{\ScottHiggins}[1]{Scott Higgins & Hall-A & 7411 & 353- & \email{higgins@jlab.org} & #1 \\ }
\newcommand{\DouglasHiginbotham}[1]{Douglas Higinbotham & Hall-A & 7851 & 584- & \email{doug@jlab.org} & #1 \\ }
\newcommand{\JinHuang}[1]{Jin Huang & MIT & OFF & & \email{jinhuang@jlab.org} & #1 \\ }
\newcommand{\XiaodongJiang}[1]{Xiaodong Jiang & Rutgers & OFF & & \email{jiang@jlab.org} & #1 \\ }
\newcommand{\MauroIodice}[1]{Mauro Iodice & INFN & OFF & & \email{iodice@jlab.org} & #1 \\ }
\newcommand{\JoeKatich}[1]{Joe Katich & CWM & 5332 & & \email{jkatich@jlab.org} & #1 \\ }
\newcommand{\ChristopherKeith}[1]{Christopher Keith & Physics & 5878 & & \email{ckeith@jlab.org} & #1 \\ }
\newcommand{\WolfgangKorsch}[1]{Wolfgang Korsch & UoK & OFF & & \email{korsch@jlab.org} & #1 \\ }
\newcommand{\KevinKramer}[1]{Kevin Kramer & CWM & & &  & #1 \\ }
\newcommand{\BrianKross}[1]{Brian Kross & Hall-A & 7022 & & \email{kross@jlab.org} & #1 \\ }
\newcommand{\RonLauze}[1]{Ron Lauze & Hall-A & 7186 & & \email{lauze@jlab.org} & #1 \\ }
\newcommand{\JohnLeRose}[1]{John LeRose & Hall-A & 7624 & & \email{lerose@jlab.org} & #1 \\ }
\newcommand{\DavidLhuillier}[1]{David Lhuillier & CEA & OFF & & \email{david@jlab.org} & #1 \\ }
\newcommand{\NilangaLiyanage}[1]{Nilanga Liyanage & UVA & 7697 & & \email{nilanga@jlab.org} & #1 \\ }
\newcommand{\MaurizioLucentini}[1]{Maurizio Lucentini & INFN & OFF & & \email{lucentin@jlab.org} & #1 \\ }
\newcommand{\YvesLussignol}[1]{Yves Lussignol & CEA & 2828\cite{inst:CEA} & & \email{lussi@Cea.Fr} & #1 \\ }
\newcommand{\BertManzlak}[1]{Bert Manzlak & Hall-A & 7556 & 897- & \email{manzlak@jlab.org} & #1 \\ }
\newcommand{\JacquesMarroncle}[1]{Jacques Marroncle & CEA & & &  & #1 \\ }
\newcommand{\KathyMcCormick}[1]{Kathy McCormick & Rutgers & & & \email{mccormic@jlab.org} & #1 \\ }
\newcommand{\DaveMeekins}[1]{Dave Meekins & Physics & 5434 & 449- & \email{meekins@jlab.org} & #1 \\ }
\newcommand{\BillMerz}[1]{Bill Merz & Hall-A & 5836 & 584- & \email{merz@jlab.org} & #1 \\ }
\newcommand{\ZeinEddineMeziani}[1]{Zein-Eddine Meziani & Temple & 5282 & & \email{meziani@jlab.org} & #1 \\ }
\newcommand{\RobertMichaels}[1]{Robert Michaels & Hall-A & 7410 & & \email{rom@jlab.org} & #1 \\ }
\newcommand{\JohnMusson}[1]{John Musson & Hall-A & 7441 & & \email{musson@jlab.org} & #1 \\ }
\newcommand{\SirishNanda}[1]{Sirish Nanda & Hall-A & 7176 & & \email{nanda@jlab.org} & #1 \\ }
\newcommand{\DamienNeyret}[1]{Damien Neyret & CEA & OFF & & \email{neyret@jlab.org} & #1 \\ }
\newcommand{\KentPaschke}[1]{Kent Paschke & Syracuse & 6932 & & \email{paschke@jlab.org} & #1 \\ }
\newcommand{\CharlesPerdrisat}[1]{Charles Perdrisat & CWM & 5304 & & \email{perdrisa@jlab.org} & #1 \\ }
\newcommand{\RomanPomatsalyuk}[1]{Roman Pomatsalyuk & KhIPT & OFF & & \email{romanip@jlab.org} & #1 \\ }
\newcommand{\VinaPunjabi}[1]{Vina Punjabi & CWM & 5304 & & \email{punjabi@jlab.org} & #1 \\ }
\newcommand{\BodoReitz}[1]{Bodo Reitz & Hall-A & & &  & #1 \\ }
\newcommand{\YvesRoblin}[1]{Yves Roblin & Accel. & 7105 & & \email{roblin@jlab.org} & #1 \\ }
\newcommand{\GaryRutledge}[1]{Gary Rutledge & CWM & & &  & #1 \\ }
\newcommand{\YiQiang}[1]{Yi Qiang & Duke & 7237 & 584- & \email{yqiang@jlab.org} & #1 \\ }
\newcommand{\ArunSaha}[1]{Arun Saha & Hall-A & & &  & #1 \\ }
\newcommand{\RustySalmons}[1]{Rusty Salmons & Hall-A & & &  & #1 \\ }
\newcommand{\MikellSeely}[1]{Mikell Seely & Hall-A & & &  & #1 \\ }
\newcommand{\JackSegal}[1]{Jack Segal & Hall-A & 7242 & & \email{segal@jlab.org} & #1 \\ }
\newcommand{\DavidSeidman}[1]{David Seidman & Hall-A & 7054 & & \email{seidman@jlab.org} & #1 \\ }
\newcommand{\KarlSlifer}[1]{Karl Slifer & Temple & 6933 & & \email{slifer@jlab.org} & #1 \\ }
\newcommand{\PatriciaSolvignon}[1]{Patricia Solvignon & Temple & OFF & & \email{solvigno@jlab.org} & #1 \\ }
\newcommand{\ScotSpiegel}[1]{Scot Spiegel & Hall-A & 5900 & & \email{spiegel@jlab.org} & #1 \\ }
\newcommand{\DarrellSpraggins}[1]{Darrell Spraggins & Hall-A & OFF & & \email{spraggin@jlab.org} & #1 \\ }
\newcommand{\MarkStevens}[1]{Mark Stevens & Hall-A & 6383 & 584- & \email{stevensm@jlab.org} & #1 \\ }
\newcommand{\VinceSulkosky}[1]{Vince Sulkosky & Hall-A & 5487 & & \email{vasulk@jlab.org} & #1 \\ }
\newcommand{\GerardTarte}[1]{G\'{e}rard Tarte & CEA & 8464\cite{inst:CEA} & & \email{gtarte@Cea.Fr} & #1 \\ }
\newcommand{\MichaelTiefenback}[1]{Michael Tiefenback & Hall-A & 7430 & 438- & \email{tiefen@jlab.org} & #1 \\ }
\newcommand{\PascalVernin}[1]{Pascal Vernin & CEA & OFF & & \email{vernin@jlab.org} & #1 \\ }
\newcommand{\ChristianVeyssiere}[1]{Christian Veyssi\`{e}re & CEA & 9704\cite{inst:CEA} & & \email{cveyssiere@Cea.Fr} & #1 \\ }
\newcommand{\HakobVoskanyan}[1]{Hakob Voskanyan & ErPhI & 6621 & & \email{voskania@jlab.org} & #1 \\ }
\newcommand{\BogdanWojtsekhowski}[1]{Bogdan Wojtsekhowski & Hall-A & 7191 & 584- & \email{bogdanw@jlab.org} & #1 \\ }
\newcommand{\XiaohuiZhan}[1]{Xiaohui Zhan & MIT & OFF & & \email{zhanxh@jlab.org} & #1 \\ }
\newcommand{\JosephZhang}[1]{Joseph Zhang & Hall-A & 5575 & 584- & \email{shukui@jlab.org} & #1 \\ }
\newcommand{\YiZhang}[1]{Yi Zhang & Lanzhou & OFF & & \email{zhangyi@jlab.org} & #1 \\ }
\newcommand{\XiaochaoZheng}[1]{Xiaochao Zheng & ANL & 5433 & & \email{xiaochao@jlab.org} & #1 \\ }
 % list of names

\section{Introduction}
\label{sec:intro}

  The Hall A\cite{Hallawww} OSP (Operation Safety Procedures) document describes  
  the systems, devices and procedures used in Hall A, containing
  the safety issues. The systems include, for example, the HRS
  scintillation counters, the HRS magnets, the Moller polarimeter
  etc. 

  Many Hall A collaboration members contributed to the Operating Safety 
  Procedures (OSP) document. This write-up gives instructions for the
  contributors\cite{OSPgen}.

\section{OSP Document Source}
\label{sec:document}
   
\subsection{Document Format}
\label{sec:format}   

  The source document is in \LaTeX{} format.
  The output OSP document is available in two formats: 
  \begin{list}{$\bullet$}{\setlength{\itemsep}{-0.15cm}}
    \item ~PDF, produced with \mycomp{pdflatex} command; 
    \item HTML, produced with \mycomp{latex2html} command (this has not been used for a long
          time and is basically deprecated). 
  \end{list}
  This makes certain requirements to the source document,
  since both \mycomp{pdflatex} and \mycomp{latex2html} have their own
  peculiarities.

\subsection{Obtaining the Source Package}
\label{sec:obtain}

 The source package contains the source of the text, 
 the pictures and tools to make the document.
 It is kept in the \href{http://www.cvshome.org/}{CVS}\cite{CVSwww} format.
 The repository is located at JLab at: \\
 \mycomp{/group/halla/analysis/cvs/tex/osp} \\
 and is available for those included in the UNIX group \mycomp{a-doc}.
 In order to change the source you have to check out the source
 (make your personal copy), edit the source files and commit
 your changes (write down your changes into the repository). Use
 a JLab machine, with mounted \mycomp{/group}, say \mycomp{jlabl1}, \mycomp{dlha1} etc.
% I checked the procedure only on the Linux platform.
  \begin{list}{$\bullet$}{\setlength{\itemsep}{-0.15cm}}
    \item Declare an environmental variable \mycomp{CVSROOT=/group/halla/analysis/cvs};\\
          You may use files from my home directory: :
        \begin{itemize} 
            \item If in \mycomp{tcsh}, type \mycomp{source /home/gen/.cvs\_osp\_ini.csh}
            \item If in \mycomp{bash}, type \mycomp{source /home/gen/.cvs\_osp\_ini.bsh}
        \end{itemize}
    \item \mycomp{cp /home/gen/.cvswrappers~~ $\tilde{}$/} - provide a CVS wrappers file,
           defining the binary files in the repository;
    \item Let us assume you want to keep the source in the directory \mycomp{$\tilde{}$/tex/},
          type \mycomp{cd~ $\tilde{}$/};
    \item \mycomp{cvs checkout tex/osp} - a source tree directory \mycomp{$\tilde{}$/tex/osp}
          should appear. 
  \end{list}
 
   
\subsection{Platform to Use}
\label{sec:platform}   

  In section~\ref{sec:obtain} it was assumed that a Linux (or a UNIX) platform
  is used. In fact, one can checkout the source from a MS Windows command line.
  I did it as follows:
  \begin{list}{$\bullet$}{\setlength{\itemsep}{-0.15cm}}
    \item \mycomp{cd j:} - go to the UNIX home directory;
    \item \mycomp{cvs -d :local:m:/halla/analysis/cvs checkout osp/cvs};
  \end{list}
  Unfortunately it is all I was able to do on MS Windows.
  The scripts and Makefile did not work - at least without
  a major change. Also, I did not find \mycomp{pdflatex}.

  I recommend to use a Linux%
\footnote{A relatively new version of \href{http://www.tug.org/teTeX/}{tetex} 
  package should be used (at least \mycomp{tetex-1.0.7-47.1} or more recent),
  generally available at JLab Linux computers.},
  
%. I have tested it with \mycomp{tetex-1.0.7-47.1},
%  available for Linux RedHat 7.3. In contrast, the older package
%  installed at RedHat 7.2 gave problems with \mycomp{latex2html}. 
%  By Dec,5 2004, RedHat Enterprise Linux is installed on nearly all JLab
%  computers, equipped with newer versions of  \mycomp{tetex}. One may use 
%of all the JLab public computers, RedHat 7.3 was installed only
%  at the diskless machines \mycomp{dlha1}, \mycomp{dlha2} etc. These machines
%  can be used to make the documents.
  machine to
  check out the code and build the documents.
  There are only a few commands to be typed in the UNIX
  command line - the \mycomp{cvs} commands and the
  \mycomp{make} commands. I suppose that editing and
  browsing can be done in the MS Windows environment,
  if one prefers it. 

\subsection{Source Package Structure}
\label{sec:structure}

  The \mycomp{tex/osp} directory has the following structure:

  \begin{list}{}{\setlength{\itemsep}{-0.15cm}}
    \item \mycomp{CVS/} - CVS data
    \item \mycomp{cvs.template} - a template to add to new files
    \item \mycomp{misc} - miscellaneous documents - not used regularly
    \item \mycomp{scripts} - auxiliary scripts and the makefile
    \item \mycomp{src} - the \LaTeX{} code
  \end{list}

\subsection{OSP Document Information Level}
\label{sec:infolev}

  One can create the OSP document with different levels of
  details (noted as \myil{}), the lower the value, 
  the more slim is the document. All the levels contain the 
  basic safety information for every system. Additional 
  information depends on the level:
  \begin{list}{}{\setlength{\itemsep}{-0.15cm}}
    \item[{\color{red}0}] - a short (1 page) description per system including 
            a picture for the layout if needed;
    \item[{\color{red}1}] - plus a description of the system's components;
    \item[{\color{red}2}] - plus a description of the procedures and pictures, 
               like MEDM windows in PNG format;
    \item[{\color{red}3}] - plus the main principles of the devices' 
               operation;
    \item[{\color{red}4}] - plus photographs and other large pictures.
  \end{list} 


\subsection{Making the OSP Document}
\label{sec:make}

  In order to produce a document%
   at a certain \myil{}, say {\color{red}4},
  one has to create a work directory \mycomp{run}:
  \begin{list}{}{\setlength{\itemsep}{-0.15cm}}
    \item \mycomp{cd tex/osp} - go to the top package directory;
    \item \mycomp{./scripts/config osp 4} - create the work directory \mycomp{run}.
  \end{list}

  At first the PDF document should be built, the HTML document 
  can be built afterwords:
  \begin{list}{}{\setlength{\itemsep}{-0.15cm}}
    \item \mycomp{cd run} 
    \item \mycomp{make pdf} - a file \mycomp{osp.pdf} is created;
    \item \mycomp{make html} - a directory \mycomp{osp\_html} is created.
  \end{list}
  The created documents should be copied to the proper
  destination be hand.

\subsection{Making other Documents}
\label{sec:makeother}

  This write-up can be produces similar to the OSP document:
  \begin{list}{}{\setlength{\itemsep}{-0.15cm}}
    \item \mycomp{./scripts/config doc 0} - create the work directory \mycomp{run}.
    \item \mycomp{cd run} 
    \item \mycomp{make pdf} - a file \mycomp{doc.pdf} is created;
  \end{list}
 
\subsection{Updating Your Source}
\label{sec:updatesource}

  The repository can change at any moment, when people commit their changes.
  In order to update your source (match it with the repository) of, for example. directory \mycomp{run}, you should: :
  \begin{list}{}{\setlength{\itemsep}{-0.15cm}}
    \item \mycomp{cd src/hrs} 
    \item \mycomp{cvs update -A} 
  \end{list}

  Other useful commands, helping to compare your source with the repository:
  \begin{list}{}{\setlength{\itemsep}{-0.15cm}}
     \item \mycomp{cvs status} 
     \item \mycomp{cvs diff} 
  \end{list}
  
\section{Editing the OSP Source}
\label{sec:edit}

\subsection{The \LaTeX{} Files}
\label{sec:latexfiles}

  The \LaTeX{} files are stored in subdirectories of the directory
  \mycomp{src}:
  \begin{list}{}{\setlength{\itemsep}{-0.15cm}}
    \item \mycomp{beamline/} - the beam line equipment;
    \item \mycomp{bibl/} - the bibliography, in the BIBTEX format;
    \item \mycomp{checklist/} - checklist templates;
    \item \mycomp{common/} - general style and preamble files;
    \item \mycomp{controls/} - the slow controls;
    \item \mycomp{daq\_trig/} - the DAQ and trigger;
    \item \mycomp{doc/} - this document (not used for the OSP);
    \item \mycomp{hrs/} - the HRS arms;
    \item \mycomp{hrs\_det/} - the HRS detectors;
    \item \mycomp{introduction/} - the introduction and access description;
    \item \mycomp{offline/} - the offline analysis;
    \item \mycomp{osp/} - the top file and the title;
    \item \mycomp{people/} - a list of the personnel;
    \item \mycomp{targets/} - the targets' description.
  \end{list}

 For example, the directory \mycomp{hrs/} contains:
  \begin{list}{}{\setlength{\itemsep}{-0.15cm}}
    \item \mycomp{AlignmentOps.tex} - alignment section;
    \item \mycomp{all.tex} - a file containing a list of \LaTeX{} files in
           the directory to be included, and some other definitions;
    \item \mycomp{figs/} - a directory for the files with pictures;
    \item \mycomp{hrs-1999.tex} - HRS section;
    \item \mycomp{nmr-1999.tex} - NMR section.
    \item \mycomp{vacuum.tex}   - vacuum elements in the hall.
  \end{list}
 
 All the directories used, apart from the ``top'' one \mycomp{osp/},
 contain a file \mycomp{all.tex} for steering and a subdirectory
 \mycomp{figs/} for the pictures.

\subsection{Pictures and Formats}
\label{sec:pictures}

  The pictures for all the sections in a given directory are stored
  in a subdirectory \mycomp{figs/}. The following formats are accepted:
  
  \begin{list}{}{\setlength{\itemsep}{-0.15cm}}
    \item \mycomp{pdflatex} - PDF, PNG, JPEG\footnote{
           I notices a limitation for the JPEG picture size, but
           do not know exactly what the limit is. A 1800x1300 picture
           did not work, while a 900x1200 picture did. One can use, say 
           \mycomp{xv} to truncate or squeeze pictures.}
    \item \mycomp{latex2html} - Encapsulated PostScript, PNG, JPEG 
  \end{list}
  The picture (say, layout) can be presented by either one
  file (\mycomp{layout.png} or \mycomp{layout.jpg}), or by two
  files: \mycomp{layout.eps} and \mycomp{layout.pdf}.
  For the Encapsulated PS, \mycomp{.epsi} files can be also used.
  In fact, \mycomp{.ps} files can be used, if they contain
  the \mycomp{BoundingBox} command. The tools to convert
  the formats are:
  \begin{list}{}{\setlength{\itemsep}{-0.15cm}}
    \item \mycomp{ps2epsi} - PS $\rightarrow$ EPSI;
    \item \mycomp{epstopdf} - EPS $\rightarrow$ PDF;
    \item \mycomp{pstops} - rectify a PS file;
    \item \mycomp{eps2eps} - rectify an EPS file.
  \end{list}

  The last two commands are needed sometimes, to
  make a file ``more standard'' and usable for other
  format converting programs\footnote{EPS files produced by
  Adobe Illustrator had to be processed by \mycomp{eps2eps},
  otherwise the \mycomp{BoundingBox} command was not seen by
  the \mycomp{epstopdf} command}.
  
  The computer screen windows, like the MEDM ones, can be easily
  grabbed using the command \mycomp{import~~tmp.png} and clicking
  with the mouse on the appropriate window. A file \mycomp{tmp.png}
  will be created.

  In \LaTeX{}, a command as
  {\color{blue}
  \begin{verbatim}
    \includegraphics*[angle=XX,width=\texwidth]{xxxx}
  \end{verbatim}
  } \noindent
  should be used. No extension (as \mycomp{.eps}) should be
  given in the file name. Neither the directory name
  \mycomp{figs/} should be given.
  It is better to use angle=0, since
  \mycomp{latex2html}
    contained a bug for calculating
    the limits of a rotated picture, reducing the picture to a line
    at 90 and 270$^{\circ}$. If there is a need to rotate
    the picture in PDF, do as here:
  {\color{blue}
  \begin{verbatim}
%begin{latexonly}
\includegraphics[angle=90.0,height=20cm]{xxxx}
%end{latexonly}
\html{
\includegraphics[angle=0,height=20cm]{xxxx}
}%
  \end{verbatim}%
  }\noindent%
  The first and third lines are recognized as commands (not comments)
  by the package {\color{blue}\verb|html.sty|}.

  If is easy to rotate an PS/EPS picture using \mycomp{ghostscript}, in
  order to avoid rotating it in \LaTeX{}. A script from the OSP package
  can be used to make a -90$^\circ$ rotation: \\
\mycomp{osp/scripts/psrotat input.ps output.ps}\\

\subsection{Selecting the Information Levels}
\label{sec:selinfolev}

  As explained in section~\ref{sec:infolev}, one can attribute to a
  part of the document a certain information level.
  This part will appear on the output if the document is made
  with the information level flag \myil{} not less than this level.
  For \myil{}=0 only the unmarked pieces will appear on the output.
  The following commands can be used:
  {\color{blue}
   \begin{verbatim}
\infolevone{ ... }
\infolevtwo{ ... }
\infolevthree{ ... }
\infolevfour{ ... }
  \end{verbatim}%
  }\noindent
  The selected parts can be long and can contain sections, figures etc.

  In a slim document, produced at a low information level,
  one may want to say something like ``For details see the full document
  ... URL ...''. Let us assume that the missing information belongs
  to the information level 4. Then, in order not to print this sentence
  for the documents at the level 4 (since these ``details'' are
  present), one should use:
  {\color{blue}
   \begin{verbatim}
\infolevltfour{For details see the full document ...URL...}
  \end{verbatim}%
  }\noindent
  The sentence will be printed only if the current information level 
  is less than 4.
  There are similar commands for different information levels
  of the information required:
  {\color{blue}
   \begin{verbatim}
\infolevltone{...}
\infolevlttwo{...}
\infolevltthree{...}
  \end{verbatim}%
  }\noindent

\subsection{Selecting the Safety Information}
\label{sec:safety}

  A section or a sentence dedicated to the safety issues
  should be marked as a \LaTeX{} environment:
  {\color{blue}
   \begin{verbatim}
\begin{safetyen}{10}{5}
  ... Safety First! ...
\end{safetyen}
  \end{verbatim}%
  }\noindent
  The text will be marked by a different color and a bar
  will be place on the right margin, all in order to attract 
  attention to the safety issues.
  The first argument (10)
  indicates the length (in mm) of the bar, while the second on (5),
  indicates the vertical offset of the top edge of the bar.
  One may need to adjust these values after looking
  at the made document. The bar is not transferred to the next page,
  therefore one should not make it too long. The example from
  above will look as: \\
\begin{safetyen}{10}{5}
  ... Safety First! ...
\end{safetyen}

\subsection{Hyper References}
\label{sec:hyperref}

  PDF allows to have hyper references both internal to the document
  (citations, sections, figures etc.) and external ones (WWW). In the latter
  case one can configure \mycomp{acroread} or \mycomp{xpdf}
  to use a WWW browser\footnote{
    Use the \mycomp{Edit/Preference} button
   on \mycomp{acroread}. For \mycomp{xpdf} (version 1.00-5)
   you may need to copy \mycomp{/etc/xpdfrc}
   to \mycomp{$\tilde{}$/.xpdfrc} and edit it, uncommenting the line
   \mycomp{\#urlCommand}     \mycomp{firefox -remote 'openURL(\%s)'}
   and commenting out the previous line.
   Typically, \mycomp{xpdf/acroread} tries to use an existing WWW browser
   window and does not open a new window.
  }.
  The \LaTeX{} references {\color{blue}\verb|\ref{...}|} to sections, 
  figures etc. automatically become hyper references in PDF and HTML.
  For the external references one can use:
  {\color{blue}
   \begin{verbatim}
\url{http://www.jlab.org/~moller/}
\href{http://www.jlab.org/~moller/}{M{\o}ller page}
  \end{verbatim}%
  }\noindent
  In the former case the whole URL will appear on the output,
  in the latter only the second argument will appear,
  but clicking the mouse button on both will produce the same
  results. No escape sequences for the special characters are 
  needed.

\subsection{Names of Authorized Personnel}
\label{sec:personnel}
Since the OSP document contains many lists with the names and coordinates 
of the authorized personnel, a special environment has been defined
in order to minimize typing and make these lists look standardized.

The names are stored in a file \mycomp{src/people/names.txt} in text
format as:
{\color{blue}
  \begin{verbatim}
Areti            Hari           JLab
Chen             Jian-Ping      JLab 
Spraggins        Darrell        JLab
Saha             Arun           JLab
Veyssi\`{e}re    Christian       CEA   9704  cveyssiere@Cea.Fr   
  \end{verbatim}%
}\noindent
The first 3 fields are obligatory, while the 4-th and 5-th field
are optional. The latter indicate the ``outside'' phone number in the home institution 
(last 4 digits) and the e-mail to be used, different from the JLab e-mail. The name fields
allow special \LaTeX{} characters as shown in the last line. In case
an ``outside'' telephone number is given, one has to provide an appropriate entry
in the \mycomp{src/bibl/jlab\_manuals.bib}, as:
{\color{blue}
  \begin{verbatim}
@manual{inst:CEA,
ORGANIZATION = "{CEA Saclay, F-91191 Gif-sur-Yvette, France}",
NOTE = "{Telephone number, from US: 9-011-33-1-6908-XXXX}",
URL = "http://www-dapnia.cea.fr/Sphn/"
}
  \end{verbatim}%
}\noindent
The ``outside'' telephone number printed in the document will contain a reference to the
institution entry.
While making the document,
the file \mycomp{src/people/names.txt} is automatically processed and a file
\mycomp{src/people/names.tex} is created and included. This file contains
a separate command for every name in the list, which should be used
in the text for referencing to this name. The JLab coordinates of each person
are automatically obtained using the command \mycomp{staff}. Therefore,
the names' spelling should match the ``staff'' database. 

A table of the authorized personnel may look like:
{\color{blue}
  \begin{verbatim}
\begin{namestab}{tab:beam:personnel_example}{Example: Beam line: authorized personnel}{%
   Example: Beam Line: authorized personnel}
  \namestabheader{Hall A Physicists}
  \ArunSaha{\em 1-st Contact}
  \JianPingChen{\em 2-nd Contact}
  \namestabheader{Liaisons from Accelerator Division}
  \HariAreti{..to Physics Div.}
  \DarrellSpraggins{..to Hall-A}
  \namestabheader{Remote Experts}
  \ChristianVeyssiere{Electronics}
\end{namestab}
  \end{verbatim}%
}\noindent

The command {{\color{blue}\verb|begin{namestab}|}} has 3 variables:
\begin{list}{\arabic{enumi}.~}{\usecounter{enumi}\setlength{\itemsep}{-0.15cm}}
  \item the label for the table; 
  \item the caption for the table list;
  \item the caption.
\end{list}
A name is referred to by a command consisting of the first and last names.
A parameter may contain some additional information. The command
{{\color{blue}\verb|begin{namestab}|}} inserts a horisontal line 
with the comment given, into the table. The example above
is printed in Table \ref{tab:beam:personnel_example}.
\begin{namestab}{tab:beam:personnel_example}{Example: Beam line: authorized personnel}{%
   Example: Beam Line: authorized personnel}
  \namestabheader{Hall A Physicists}
  \ArunSaha{\em 1-st Contact}
  \JianPingChen{\em 2-nd Contact}
  \namestabheader{Liaisons from Accelerator Division}
  \HariAreti{..to Physics Div.}
  \DarrellSpraggins{..to Hall-A}
  \namestabheader{Remote Experts}
  \ChristianVeyssiere{Electronics}
\end{namestab}

Use the existing OSP document as the source of other examples.

\subsection{Other Commands and Environments}
\label{sec:otherenv}

  I suggest to use a special style for the computer
  commands or names:
  {\color{blue}
   \begin{verbatim}
\mycomp{ .... }
  \end{verbatim}%
  }\noindent
  At the moment it uses the {\color{blue}\verb|\sf|} font
  and the magenta color.

  For the e-mail references use a command as {\color{blue}\verb|\email{gen@jlab.org}|}
  which would produce \email{gen@jlab.org}.

  In order to comment out a big piece of the text, instead
  of commenting out each line one can use 
  a command:
  {\color{blue}
   \begin{verbatim}
\obsolete{ .... }
  \end{verbatim}%
  }\noindent

\subsection{New \LaTeX{} files}
\label{sec:newlatex}

  Creating a new \LaTeX{} file in an existing directory,
  one has to add its name to the appropriate \mycomp{all.tex}
  and add the contents of the file \mycomp{cvs.template}
  to the end of the new file. 
  It is also recommended to include another template \mycomp{cvs.id1.template}
  which contains two footnotes, in the name of the first section.
  The first footnote
  shows the dynamic version number which will change automatically, along
   with the date of the last change and other information. This footnote
   should not be modified by hand. The second footnote contains the name
   of the author of this file and should be edited accordingly.

   New files must be added to the repository and then committed:
   see the CVS instructions.
  \ref{sec:commit}.

\subsection{Changing the Repository}
\label{sec:commit}

  After completing and debugging your changes, you may commit them
  to the CVS repository by:\\
  \begin{list}{}{\setlength{\itemsep}{-0.15cm}}
    \item \mycomp{cvs add {\em file}} - for files which do not exist yet in the repository,
                                        for all new files;
    \item \mycomp{cvs commit -m \"{}some comments\"{}~ {\em file}} - commit the updated file.
          The comments will be shown in the file history and should explain the reason for the
          change, say: ``several typos corrected'', or ``A new target chamber, installed in May 2007 is described''.
  \end{list}

\section{Conclusion}
\label{sec:conclusion}

  One can use the file \mycomp{src/beamline/moller.tex} as an
  example for the used \LaTeX{} commands and extensions. \\

  I have checked the procedure only on the Linux platforms.
%  Although CVS for MS Windows exists, it is perhaps easier
%  to login to a Linux
%  machine from Windows, and do all the work in Linux.
%  Browsing the output with \mycomp{acroread} can be done
%  in Windows. 

  In case you can not use \LaTeX{},
  please do not send me documents in proprietary MS formats, like Word,
  but rather send just a simple text file and pictures, which I can
  convert to \LaTeX{} and incorporate. 


\bibliographystyle{unsrtmod}
\bibliography{bibl/jlab_manuals,bibl/doc}


\end{document} 
 
