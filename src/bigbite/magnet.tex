%
% Update 20 February 2008
%	- referenced old OSP
%       - added note about hand held field measurement units
%
% Update 15 December 2005
%
% 	- changed LCW pressure from 50 psi to 100 psi
%       
%       - changed wording in \subsection{Energizing the BigBite Magnet}
%         to clarify what should be done if there is an unauthorized
%         entry or a hazard noted
%

%--- Description of components

\chapter[BigBite Magnet]{BigBite Magnet
\footnote{Author: D. W. Higinbotham \email{doug@jlab.org}}
}

\section{Overview}
The BigBite magnet, the key component to several Hall~A approved experiments, was 
commissioned to 550~A under TOSP PHY-04-014 and subsequently to 
800~A under TOSP PHY-05-015. 
The purpose of this document is to describe the hazards 
and safety procedures for operating this magnet.  These procedures include running 
the magnet during an experiment and making field measurements.  
The BigBox power supply being used with BigBite was commissioned during 
Hall A experiment E99-114 (TSOP PHY-02-003) and this document presents 
the same procedures for safely operating the supply.  
This BigBite dipole magnet document is an update to the expired OSP PHY-05-001
and OSP PHY-08-002. 
under which the BigBite dipole was run for the past several years.
This document does not attempt to describe the function or operation of the dipole or
power supply. 

\section{Description of Magnet}

Hall~A BigBite experiments will make use of a large-acceptance dipole magnet 
to deflect charged particles into the various BigBite detector packages.  
The BigBox power supply will be used to energize this magnet.  
The power supply has previously been tested and was successfully 
used during the Hall~A RCS experiment (E99-114).
The control software of the supply runs from the HAC computer with the BigBox GUI.  
The first BigBite experiment (E01-015) ran the magnet at an excitation 
of 0.92~T which required 518~A with the maximum current limited to
550~A.  Subsequently, after the successful completion of a high current 
commissioning under TOSP PHY-05-015, the limited was raised to 800~A.

\section{Authority and Responsibility}

\subsection{BigBox Power Supply}

Only authorized personnel may enable the power supply as per the operating
guidelines described in Section~\ref{guideline-commissioning}.  To become authorized one must:

\begin{itemize}

\item{Read and understand this document.}

\item{Read and understand Chapter 6440 of the Jefferson Lab EH\&S Manual on Static Magnet Fields.}

\item{Complete JLab Lock and Tag Training (SAF104)}

\item{Complete NFPA-70E Electrical Safety Training (SAF603)}

\item{Complete training on power supply operation by authorized Hall A technical staff.}

\item{Obtain an authorizing signature from the BigBite contact person, Douglas Higinbotham, on the
attached signature sheet.}

\end{itemize}

\subsection{Energizing the BigBite Magnet}

Once the BigBox power supply has been enabled, the BigBite magnet can either be
controlled locally or with the BigBox GUI.
Shift workers will only be allowed to control the magnet via the BigBox GUI and only after
they have read and signed the COO of the experiment for which they are taking shift.  
One shall immedately turn off the magnet via the BigBox GUI or locally at the BigBox supply 
if someone unauthorized and/or unknown is seen entering the magnet area or any hazard, e.g. leaking low conductivity water,
is identified.

\subsection{Magnetic Field Measurements}

With the written permission of the Hall~A work coordinator, physics users may make field measurements of the
area around the magnet.
The guidelines for safely performing these
measurements can be found in Section~\ref{guideline-mapping}.  
During these measurements, the current to the magnet may be changed using the BigBox GUI.
Once the measurements have been completed, authorized personnel secure the system 
as per Section~\ref{off}.

\section{Location of Equipment}

All required equipment is located in Hall A.  During operation, the BigBite magnet 
will be located near the pivot area and the BigBox power supply is
located near the Hall A control racks.

\section{Hazard Analysis}

The hazards associated with the magnet and power supply are electrical, magnetic, and fire.

\noindent{\textbf{Electrical:}}	The power supply has a maximum output current of 1050~A 
at a voltage of 250~V and thus presents a potentially lethal hazard.  
A hazard also arises from the power bus on the magnet itself. 

\noindent{\textbf{Magnetic:}}	The magnet produces a central field of 0.92~T at 518~A.  
As the magnet has a return yoke and a front field clamp, the external field is much smaller 
than the central field.   Although the magnetic field is primarily confined to the magnet 
gap, fringe fields are strong enough to accelerate unsecured metal objects in the vicinity of
the magnet.  In addition these fields may present a particularly large hazard 
to individuals using a pacemaker.
An additional hazard arises due to the close proximity of the magnet to the target area where
an unsecured metal object could destroy the scattering chamber.

\noindent{\textbf{Fire:}}  There exists a potential fire hazard with high current power supplies.  
\section{Hazard Mitigation}

\noindent{\textbf{Electrical:}}
Access to the power supply or magnet can only be made after following
``Lockout/Tagout Procedures'' 
as described in Chapter 6110 of the Jefferson Lab EH\&S manual and the {\it{Hall~A power
supply test and maintenance}} safety procedure.  
When working on
the power supply, the responsible people will follow the guidelines in the 
electrical safety chapter of the EH\&S manual.  
Before being energized, the magnet's exposed current bus must be covered
to mitigate the shock hazard.
The power supply bus must be covered and all doors secured.
Also, to keep the current of the BigBite magnet within operational limits, 
the over-current circuit in the BigBox power supply should be set to
no more than 800~A.  

\noindent{\textbf{Magnetic:}}
The possible presence of high magnetic fields will be 
indicated by standard Jefferson Lab signs and by a 
flashing beacon.  The area surrounding the magnet will be 
roped off whenever it is possible that the magnet
will be energized.  The ropes will be at a distance from the 
magnet such that the fringe fields are less than
500~{$\mu$}T (5~G) at the maximum allowed current of 800~A.  
This should be roughly one meter, but should be checked as soon as
possible once the magnet has been energized.  Personnel with ferromagnetic 
implants and those wearing 
electronic medical devices are not allowed inside the roped off area.  
Due to the large magnet gap size, personnel
working inside the roped-off area should be aware of the possible presence of a magnetic fringe field, as well as
a high field in the magnet gap.  

\noindent{\textbf{Fire:}}
The magnet coils are protected from over-heating by 
Klixon devices installed on the magnet and interlocked to the power supply that will shut off the
power supply in case of the coil over-heating.

\section{Operating Guidelines}


\subsection{Testing the BigBite Magnet After Installation}
\label{guideline-commissioning}

Once the BigBite magnet has been installed and connected to the BigBox power supply, it should be
tested to ensure that it is working properly.  

\begin{itemize}
\item{At least two qualified persons must be working on the task together.}
\item{Rope off the area around the magnet.}
\item{Install protective covers as needed over the target windows and the spectrometer
sieve slit.  Check with the Hall~A work coordinator to ensure proper covers are used.}
\item{Sweep the area inside the ropes for magnetic material.  Make sure that the area is clean,
and that no foreign objects are in or near the aperture of the magnet or the inside of the stay-clear zone.
All such materials must be removed and placed outside of the ropes.}
\item{Make sure all protective barricades, signs and beacons are in place to warn of possible exposure 
to magnetic and electrical hazards.}
\item{Verify all covers on energized conductors on the magnet are securely in place.}
\item{Verify all power supply doors and cabinets are closed and locked.}
\item{Check that the cooling water is turned on.  Valves on the magnet and on individual cooling paths must all
be open.}
\item{Verify that water flow is present.  The flow switches on the supply and return lines must be open and the 
the supply pressure must be verified to be 50~psi greater than the return pressure.}
\item{Turn on the flashing beacons.}
\item{Remove the administrative lock on the power supply disconnect switch.  Make sure the Jefferson Lab's
Lockout/Tagout procedures, as described in Chapter 6110 of the Jefferson Lab EH\&S manual are followed.
Make sure your Lockout/Tagout training is up-to-date, you have been trained on the operation of the power
supply and magnet and that you have been authorized by Douglas Higinbotham.}
\item{Enable main power on the power supply and ramp output current at the rate of approximately 10~A per
second to 50~A.  Check that all controls and safety features are operational then continue to ramp at the
rate of 10~A per second to the maximum current.}
\item{Enter record of the successful test into a Hall A electronic log book.}
\end{itemize}

\subsection{Enabling the BigBite Magnet For Physics}

\begin{itemize}
\item{At least two persons must be working on the task together.}
\item{Rope off the area around the magnet.}
\item{Sweep the area inside the ropes for magnetic material.  Make sure that the area is clean,
and that no foreign objects are in or near the aperture of the magnet or the inside of the stay-clear zone.
All such materials must be removed and placed outside of the ropes.}
\item{Make sure all protective barricades, signs and beacons are in place to warn of possible exposure
to magnetic and electrical hazards.}
\item{Verify all covers on energized conductors on the magnet are securely in place.}
\item{Verify all power supply doors and cabinets are closed and locked.}
\item{Check that the cooling water is turned on.  Valves on the magnet and on individual cooling paths must all
be open.}
\item{Verify that water flow is present --- checking that the differential pressure is greater than 50 psi and 
look at flow switches (inlet pressure should be greater than 100~psi).}
\item{Turn on the flashing beacons.}
\item{Remove the administrative lock on the power supply disconnect switch.  Make sure the Jefferson Lab's
Lockout/Tagout procedures, as described in Chapter 6110 of the Jefferson Lab EH\&S manual are followed.
Make sure your Lockout/Tagout training is up-to-date, you have been trained on the operation of the power
supply and magnet and that you have been authorized by Douglas Higinbotham.}
\item{Enable main power on the power supply and check the current can be set with the BigBox GUI by ramping
the magnet to 50~A.}
\item{Set the magnet to 0~A and submit a electronic log entry that magnet is ready
and that Hall~A shift workers now can control the magnet via the BigBox GUI.}
\end{itemize}

\subsection{Magnetic Field Measurements}
\label{guideline-mapping}
With the Hall~A work coordinator's written authorization, a map of the magnet's fringe
field can be made.  During magnetic field measurements, the covers should be on the
scattering chamber.  All work in the vicinity of the magnet must conform 
to the practice described in Chapter
6440 of the Jefferson Lab EH\&S manual.  In particular 
all workers must respect the limits shown in the table
``exposure limits for static magnetic fields'' of that chapter and reproduced below.  

{\tiny
\begin{tabular}{ccc}
				& & \\
	& Routine 8 Hour Average & Maximum Allowable \\
Ferromagnetic implant and & Routine Exposure & 0.5~mT (5 G) or \\
electronic medical device & Not Recommended  & as determined by a physician \\
wearers                   & &  \\
				& & \\
Regular Employees	& Whole Body: 60~mT (600~G)     & Whole Body: 2~T (20,000 G) \\  
	                & Limbs: 600~mT (6000~G)	& Limbs: 5~T (50,000 G) \\
				& & \\
\end{tabular}
}

The first field measurements should be made around the 1~meter perimeter to make sure the fringe is less than 5~G at
this location.  Measurements closer to the magnet and in the gap of the magnet can
be made as long as the limits in the table are adhered to and a description of the planned measurement points
has been presented to Douglas Higinbotham for approval.
A hand-held Lake Shore field measurement device is avaliable from Douglas Higinbotham 
for establishing the 5~G perimeter
and making measurements to 0.1~G precision.


\subsection{Turning Off The BigBox Supply}
\label{off}

The magnet power supply should always be locked out when the magnet is not going 
to be used for any extended period.

\begin{itemize}
\item{Ramp output current to zero and turn off main power at the power supply.}
\item{Apply the administrative lock to the power supply.}
\item{Turn off the flashing beacons.}
\end{itemize}

\subsection{Before Moving BigBite}

Before moving the BigBite magnet (e.g. changing the angle of the spectrometer which must
be done locally), the power supply must be turned off as per Section~\ref{off}.   

\clearpage
\section{Authorized Personnel}

Only Hall A technical staff are authorized to enable the main power supply and require
proper electrical safety training.  Shift workers are only allowed to control 
the BigBite power supply via a GUI.  

