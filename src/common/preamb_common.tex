
% -  Define the environments for different information levels (lev=0 - no special environment)
% default: everything is set. It can be redefined with the file currinfolev.tex
% (Comment: is done in such a clumsy way since if-then-else does not work with latex2html) 
\newcommand{\infolevone}[1]{#1}%
\newcommand{\infolevtwo}[1]{#1}%
\newcommand{\infolevthree}[1]{#1}%
\newcommand{\infolevfour}[1]{#1}%

\newcommand{\infolevltone}[1]{#1}%
\newcommand{\infolevlttwo}[1]{#1}%
\newcommand{\infolevltthree}[1]{#1}%
\newcommand{\infolevltfour}[1]{#1}%

\newcommand{\infoleveqnull}[1]{#1}%
\newcommand{\infoleveqone}[1]{#1}%
\newcommand{\infoleveqtwo}[1]{#1}%
\newcommand{\infoleveqthree}[1]{#1}%
\newcommand{\infoleveqfour}[1]{#1}%

\newcommand{\ifpdfhref}[1]{}%

% input the INFO LEVEL flag, say \def\infolevel{4} and redefine the environments
%
\def\infolevel{4}  % INFO LEVEL flag 
%   =0 - overview+safety only, 
%   =1 + procedures, 
%   =2 + components, 
%   =3 + principles of operation, 
%   =4 + performance
\renewcommand{\infolevltfour}[1]{}%
\renewcommand{\infolevltthree}[1]{}%
\renewcommand{\infolevlttwo}[1]{}%
\renewcommand{\infolevltone}[1]{}%
\renewcommand{\infoleveqthree}[1]{}%
\renewcommand{\infoleveqtwo}[1]{}%
\renewcommand{\infoleveqone}[1]{}%
\renewcommand{\infoleveqnull}[1]{}%

% %
\def\infolevel{4}  % INFO LEVEL flag 
%   =0 - overview+safety only, 
%   =1 + procedures, 
%   =2 + components, 
%   =3 + principles of operation, 
%   =4 + performance
% ===========  CVS info
% $Header: /group/halla/analysis/cvs/tex/osp/src/common/infolevel.tex,v 1.2 2003/06/05 23:30:00 gen Exp $
% $Id: infolevel.tex,v 1.2 2003/06/05 23:30:00 gen Exp $
% $Author: gen $
% $Date: 2003/06/05 23:30:00 $
% $Name:  $
% $Locker:  $
% $Log: infolevel.tex,v $
% Revision 1.2  2003/06/05 23:30:00  gen
% Revision ID is printed in TeX
%
% Revision 1.1.1.1  2003/06/05 17:28:33  gen
% Imported from /home/gen/tex/OSP
%
%  Revision parameters to appear on the output

%   =0 - overview+safety only, 
%   =1 + procedures, 
%   =2 + components, 
%   =3 + principles of operation, 
%   =4 + performance

\usepackage{color}
\usepackage{longtable}
% \usepackage{amsfonts}
% \usepackage{amssymb}

\def\Mcol{black}  % Main text color
\def\Scol{red}    % Safety text color
\def\SBcol{red}   % Safety marginbar color
\def\Ccol{magenta}   % Computer input/output color
%\definecolor{Scol}{rgb}{red}    % Safety text color

%begin{latexonly}
% ===    Set a true/false value for PDF hyper marks  
\newif\ifhyprf
%\hyprffalse
\hyprftrue

\newif\ifpdf
\ifx\pdfoutput\undefined
    \pdffalse           % we are not running PDFLaTeX
\else
    \pdfoutput=1        % we are running PDFLaTeX
    \pdftrue
\fi

\ifpdf
  \pdfcompresslevel=9
  \usepackage[pdftex]{graphicx}
%  \usepackage{thumbpdf}
  \definecolor{rltred}{rgb}{0.75,0,0}
  \definecolor{rltgreen}{rgb}{0,0.3,0}
  \definecolor{rltblue}{rgb}{0,0,0.75}
  \definecolor{rltdarkgreen}{rgb}{0.1,0.7,0.1}
  \ifhyprf
     \usepackage[pdftex,
         colorlinks=true,
         urlcolor=rltblue,       % \href{...}{...} external (URL)
         filecolor=rltgreen,     % \href{...} local file
         linkcolor=rltred,       % \ref{...} and \pageref{...}
         citecolor=rltdarkgreen, % citations
         pagebackref,
         pdfpagemode={UseOutlines},
        ]{hyperref}
     \hypersetup{
         pdftitle={OSP Hall A},
         pdfauthor={Hall A},
         pdfsubject={JLab Hall A Operations},
         pdfkeywords={JLab HallA operations safety OSP}
       }
     \renewcommand{\ifpdfhref}[1]{#1}%
  \fi
  \usepackage{pdfcolmk}
  \DeclareGraphicsExtensions{.pdf,.png,.jpg}
\else
  \usepackage{graphicx}
  \DeclareGraphicsExtensions{.eps,.epsi,.ps,.eps.gz,.epsi.gz,.ps.gz}
\fi

\usepackage{cite}
\usepackage{comment}
\usepackage{ifthen}
\usepackage{changebar}
\usepackage{url}
% \usepackage{html,hthtml}

\oddsidemargin=0.25in 
\evensidemargin=0.25in 
\topmargin=-0.2in 
\textwidth=6.25in 
\textheight=8.7in
\renewcommand{\textfraction}{0.05} 

%end{latexonly}

\newcommand{\obsolete}[1]{}%
\newcommand{\myhtml}[1]{}%

%\begin{htmlonly}
%  \usepackage{graphicx}
%%  \usepackage{epsf}
%  \DeclareGraphicsExtensions{.eps,.epsi,.ps}
%  \def\makeglossary{}
%  \pagecolor[named]{White}
%
%% -  Define the environments for different information levels (lev=0 - no special environment)
%
%\end{htmlonly}


\newenvironment{safetyen}[2]%  SAFETY environment
       {\color{\SBcol}%
              \marginpar{\rule[-#2mm]{1mm}{#1mm}}%
        \color{\Scol}}%
       {\color{\Mcol}}
%       {\color{red}\bfseries}%
%       {\color{black}\rmfamily}

\newcommand{\dirfig}[0]{figs}
\newcommand{\dircur}[0]{}
\newcommand{\mycomp}[1]{{\color{\Ccol}{\sf #1}}}

\newcommand{\email}[1]{\href{mailto:#1}{#1}}
 
